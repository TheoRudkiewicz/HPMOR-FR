
\chapter{Hiérarchies de dominance}

NdT : J'utilise le mot anglais \emph{sentient}  à la façon de Guy Abadia dans l'Étoile et le Fouet.
\par\noindent\rule{\textwidth}{0.4pt}
Toute J.K. Rowling suffisamment avancée est indiscernable de la magie.
\par\noindent\rule{\textwidth}{0.4pt}
"\emph{Ça ressemble au genre de chose que j'ai l'habitude de faire, non ?"} 
\par\noindent\rule{\textwidth}{0.4pt}
C'était un vendredi matin à l'heure du petit déjeuner. Harry reprit une énorme bouchée de sa tartine puis essaya de rappeler à son cerveau que dévorer son petit déjeuner ne l'aiderait pas à aller aux donjons plus vite. Ils avaient de toute façon une heure d'étude entre le petit déjeuner et le début du cours de potions.

Mais des donjons ! À Poudlard ! L'imagination de Harry esquissait déjà les gouffres, les ponts étroits, les appliques de torches et les champs de mousse lumineuse. Y aurait-il des rats ? Y aurait-il des \emph{dragons}  ?

"Harry Potter," dit une voix légère située derrière lui.

Harry regarda par-dessus son épaule et se retrouva face à un Ernie Macmillan élégamment habillé de robes à bordures jaunes et à l'air un petit peu inquiet.

"Neville pense que je devrais te prévenir," dit Ernie à voix basse. "Je pense qu'il a raison. Fais attention au professeur de Potions aujourd'hui. Les Poufsouffle plus âgés nous ont dit que le professeur Rogue peut être vraiment méchant avec ceux qu'il n'aime pas, et il n'aime pas la plupart des gens qui ne sont pas Serpentard. Si tu lui réponds... de ce que j'ai entendu, ça pourrait vraiment aller mal. Garde juste la tête baissée et ne lui donne aucune raison de te remarquer."

Il y eut une pause tandis que Harry absorbait cela, et il leva ensuite ses sourcils (Harry aurait aimé pouvoir lever un seul de ses sourcils, comme Spock, mais il n'y était jamais parvenu). "Merci," dit Harry. "Tu m'as peut-être évité beaucoup d'ennuis."

Ernie hocha la tête, et fit demi-tour pour retourner à la table des Poufsouffle.

Harry continua de manger sa tartine.

C'est environ quatre bouchées plus tard que quelqu'un dit "Excuse-moi," et Harry se retourna et vit un Serdaigle plus âgé à l'air un peu inquiet -

Peu après, Harry finissait sa troisième assiette de tranches de bacon (Il avait appris à beaucoup manger au petit déjeuner. Il pouvait toujours manger léger au déjeuner s'il décidait de ne pas utiliser le Retourneur de Temps). Et il entendit encore une autre voix derrière lui, qui disait : "Harry ?"

"Oui," dit Harry d'un ton las, "j'essaierai de ne pas attirer l'attention du professeur Rogue -"

"Oh, ça c'est sans espoir," dit Fred.

"Carrément sans espoir," dit George.

"Alors on a demandé aux elfes de maison de te faire un gâteau," dit Fred.

"On mettra une bougie dessus pour chaque point que tu feras perdre à Serdaigle," dit George.

"Et on aura une fête pour toi à la table des Gryffondor pendant le déjeuner," dit Fred.

"On espère que ça te déridera après le cours," conclut George.

Harry avala sa dernière bouchée de bacon et fit demi-tour. "Très bien," dit Harry. "Je n'allais pas poser cette question après avoir vu le professeur Binns, vraiment pas, mais si le professeur Rogue est \emph{si}  terrible que ça, pourquoi n'a-t-il pas été renvoyé ?"

"Renvoyé ?"

"Tu veux dire remercié ?"

"Oui," dit Harry. "C'est ce qu'on fait aux mauvais enseignants. On les renvoie. Et ensuite on engage un meilleur enseignant. Vous n'avez pas de syndicats ou de titularisations par ici ?"

Fred et George fronçaient les sourcils exactement comme ces anciens d'une tribu de chasseurs-cueilleurs pourraient froncer les sourcils si vous essayiez de leur apprendre les fonctions mathématiques.

"Je ne sais pas," dit Fred après un moment "je n'y avais jamais pensé."

"Moi non plus," dit George.

"Ouais," dit Harry, "j'entends ça souvent. On se verra au déjeuner les gars, et ne m'en voulez pas s'il n'y a aucune bougie sur ce gâteau."

Fred et George rirent tous les deux, comme si Harry avait dit quelque chose de drôle, puis ils s'inclinèrent et repartirent vers Gryffondor.

Pendant qu'il mangeait son cupcake, Harry repensa au pire enseignant qu'il avait rencontré de sa vie, le professeur d'Histoire, M. Binns. Le professeur Binns était un fantôme. Vu ce que Hermione avait dit au sujet des fantômes, il semblait peu probable qu'ils aient conscience d'eux-mêmes. Il n'y avait pas de découverte célèbre faite ni de production originale faite par des fantômes, peu importe qui ils avaient été de leur vivant. Les fantômes avaient généralement du mal à se souvenir du siècle actuel. Hermione avait dit qu'ils étaient comme des portraits accidentels, imprimés dans la matière environnante par l'éclat d'énergie psychique qui accompagnait soi-disant la mort soudaine d'un sorcier.

Harry avait croisé quelques enseignants stupides lors de ses incursions ratées dans l'éducation moldue normale - son père avait été beaucoup plus pointilleux lorsqu'il avait été question de choisir des étudiants en doctorat pour lui servir de précepteur, bien sûr - mais le cours d'Histoire était le premier où il rencontrait un enseignant qui n'était littéralement pas sentient.

Et ça se voyait. Harry avait laissé tombé au bout de cinq minutes et avait commencé à lire un manuel. Lorsqu'il était devenu clair que "professeur Binns" ne verrait pas d'objection à cela, Harry avait fouillé dans sa bourse et récupéré des boules quiès.

Les fantômes n'avaient-ils pas besoin d'un salaire ? Était-ce là l'explication ? Ou était-il littéralement impossible de renvoyer qui que ce soit de Poudlard \emph{même s'ils mouraient}  ?

Et maintenant, il semblait que le professeur Rogue passait son temps à être absolument horrible envers tous ceux qui n'étaient pas Serpentard, et ça n'était jamais \emph{venu à l'esprit}  de quelqu'un de résilier son contrat.

Et le Directeur avait mis le feu à un poulet.

"Excuse-moi," dit une voix inquiète située derrière lui.

"Je vous jure," dit Harry sans se retourner, "cet endroit est presque huit et demi pour cent aussi pitoyable que ce que Papa dit au sujet d'Oxford."
\par\noindent\rule{\textwidth}{0.4pt}
Harry donna un coup de pied aux couloirs de pierre, l'air vexé, furieux et agacé à la fois.

"Des donjons !" siffla Harry. "Des \emph{donjons}  ! Ce ne sont pas des donjons ! C'est une cave ! Une \emph{cave !} "

Quelques-une des filles de Serdaigle lui jetèrent des regards étranges. Les garçons étaient déjà habitués.

Il semblait que le fait que l'étage où se situait le cours de Potions soit au sous-sol et qu'il y fasse plus froid que dans le reste du château suffisait à le faire appeler "les donjons".

À \emph{Poudlard}  ! À \emph{Poudlard !}  Harry avait attendu toute sa vie, et maintenant il allait \emph{encore}  attendre et s'il y avait un seul endroit \emph{sur Terre}  doté de donjons décents, ça devait être Poudlard ! Harry allait-il devoir construire son propre château si jamais il voulait voir une seule petite abysse sans fond ?

Peu de temps après, ils arrivèrent à la salle des Potions, et Harry se réjouit considérablement.

La salle des Potions avait d'étranges créatures préservées flottant dans d'immenses bocaux sur des étagères qui couvraient chaque centimètre carré d'espace mural entre les placards. Harry avait à présent assez lu pour pouvoir identifier certaines de ces créatures, tel le Zabriskan Fontema. Et bien que l'araignée de cinquante centimètre \emph{ressemblât}  à une Acromantula, elle était beaucoup trop petite pour en \emph{être}  une. Il aurait bien essayé de demander à Hermione, mais elle n'avait pas semblé désireuse de regarder dans les directions qu'il pointait du doigt.

Harry regardait une grande boule de poussière dotée d'yeux et de pieds quand l'assassin surgit dans la pièce.

Ce fut la première pensée qui traversa l'esprit de Harry quand il vit le professeur Severus Rogue. Il y avait chez lui quelque chose de silencieux et de mortel, dans la façon dont il rôdait entre les pupitres des enfants. Ses robes étaient mal entretenues, ses cheveux sales et gras. Il y avait quelque chose qui rappelait Lucius, même s'ils ne se ressemblaient absolument pas, et on avait le sentiment que là où Lucius vous aurait éliminé avec une parfaite élégance, cet homme se serait contenté de vous tuer.

"Asseyez-vous," dit le professeur Severus Rogue. "Maintenant."

Harry et les quelques autres enfants qui étaient restés debout à discuter se précipitèrent vers les pupitres les plus proches. Harry avait prévu d'être à côté de Hermione mais il se retrouva assis au pupitre vide à côté de Justin Finch-Fletchley (c'était une Double session, Serdaigle et Poufsouffle) ce qui le plaçait deux pupitres à gauche de Hermione.

Severus s'assit au bureau du professeur, et sans la moindre transition ni la moindre introduction, dit : "Hannah Abbott."

"Présente," dit Hannah d'une voix assez tremblante.

"Susan Bones."

"Présente."

Et ça continua, personne n'osant dire un mot plus haut que l'autre, jusqu'à :

"Ah, oui. Harry Potter. Notre nouvelle... \emph{célébrité} ."

"La célébrité est présente, \emph{monsieur} ."

La moitié de la classe tressauta, et quelques uns parmi les plus malins eurent l'air de songer à courir hors de la salle avant qu'elle ne soit détruite.

Severus sourit d'anticipation et appela le prochain nom de sa liste.

Harry soupira mentalement. C'était arrivé bien trop vite pour qu'il puisse y faire quoi que ce soit. Tant pis. Cet homme ne l'aimait visiblement déjà pas, quelle qu'en soit la raison. Et maintenant que Harry y réfléchissait, il valait beaucoup mieux que le professeur de Potions s'en prenne à \emph{lui } plutôt qu'à, disons, plutôt qu'à Neville ou à Hermione. Harry était bien plus capable de se défendre. Ouaip, ça valait probablement mieux.

Lorsque l'appel fut terminé, Severus balaya la classe du regard. Ses yeux étaient aussi vides qu'une nuit sans étoiles.

"Vous êtes ici," dit Severus d'une voix douce que les étudiants à l'arrière eurent du mal à entendre, "pour apprendre la science subtile et l'art exact de la fabrication de potions. Vu le peu d'agitation idiote de baguette que nous aurons à faire ici, beaucoup parmi vous auront du mal à croire que nous faisons de la magie. Je ne m'attends pas à ce que vous compreniez la beauté du chaudron frémissant doucement et de ses fumées chatoyantes, le pouvoir délicat des liquides se glissant dans les veines du corps humain," il avait dit ça d'une voix caressante et emplie de jubilation, "ensorcelant l'esprit, piégeant les sens," ça devenait de plus en plus terrifiant. "je peux vous apprendre à mettre la célébrité en bouteille, à brasser la gloire, et même à mettre la mort dans un flacon - si vous n'êtes pas une bande d'idiots aussi désespérants que ceux à qui je dois d'habitude enseigner."

Severus sembla remarquer l'air de scepticisme sur le visage de Harry, ou du moins ses yeux bondirent soudain à l'endroit où Harry était assis.

"Potter !" lança le professeur de Potions. "Qu'obtiendrai-je si j'ajoutais de la racine en poudre d'asphodèle à une infusion d'armoise ?"

Harry cligna des yeux. "Était-ce dans \emph{Breuvages et Potions Magiques}  ?" dit-il. "Je viens de finir de le lire, et je ne me souviens pas de quoi que ce soit utilisant de l'armoise -"

La main de Hermione se leva et Harry lui jeta un regard qui la fit lever sa main encore plus haut.

"Tut, tut," dit Severus d'une voix de velours. "La célébrité ne fait clairement pas tout."

"Vraiment," dit Harry. "Mais vous venez de nous dire que vous nous apprendrez à mettre la célébrité en bouteille. Alors dites-moi, \emph{comment}  cela fonctionne-t-il exactement ? On la boit et on devient une célébrité ?"

Les trois quarts de la classe tressautèrent.

La main de Hermione redescendait lentement. Eh bien, ça n'était pas surprenant. Elle était peut-être son rival, mais elle n'était pas le genre de fille à continuer de jouer une fois devenu clair que le professeur essayait délibérément d'humilier Harry.

Harry essayait de toute ses forces de garder son sang-froid. La première réponse qui lui était venue à l'esprit était 'Abracadabra'.

"Essayons encore," dit Severus. "Potter, où iriez-vous si je vous disais d'aller me chercher un Bézoard ?"

"Ce n'est pas non plus dans le manuel," dit Harry, "mais j'ai lu dans un livre moldu qu'un trichinobézoard est une masse de cheveux solidifiée trouvée dans un estomac humain, et dans le passé, les Moldus croyaient que ça pouvait guérir tous les poisons -"

"Faux," dit Severus. "Un Bézoard se trouve dans l'estomac d'une chèvre, il n'est pas fait de cheveux, et il guérira de la plupart des poisons, mais pas de tous."

"Je n'ai pas \emph{dit}  que c'est ce qui se passerait, j'ai dit que c'était ce que j'avais lu dans un livre moldu -"

"Personne ici se s'intéresse à vos \emph{pathétiques}  livres moldus. Dernier essai. Potter, quelle est la différence entre le napel et le tue-loup ?"

C'en était trop.

"Vous savez," dit Harry d'un ton glacial, "dans un de mes \emph{fascinants}  livres moldus est décrite une étude dans laquelle des gens parvenaient à se donner l'air très intelligent en posant des questions au sujet de faits quelconques que eux seuls connaissaient. Apparemment, les spectateurs remarquaient seulement que ceux qui posaient la question savaient et que ceux qui essayaient d'y répondre ne savaient pas, et ils n'arrivaient pas à prendre en compte l'injustice profonde du jeu. Alors, professeur, pouvez-vous me dire combien d'électrons se trouvent sur l'orbite externe d'un atome de carbone ?"

Le sourire de Severus s'agrandit. "Quatre," dit-il. "Mais c'est une information inutile que personne ne devrait s'embêter à noter. Et pour votre gouverne, Potter, l'asphodèle et l'armoise font une potion de sommeil si puissante qu'elle est connue sous le nom de Goutte du Mort-Vivant. En ce qui concerne le napel et le tue-loup, il s'agit de la même plante, qui est aussi appelée aconit, comme vous le sauriez si vous aviez lu \emph{Mille Herbes et Champignons} . Vous pensiez ne pas avoir besoin d'ouvrir le livre avant de venir, hein, Potter ? Tous les autres, vous devriez prendre note afin de ne pas être aussi ignorants que lui." Severus marqua une pause, l'air plutôt content de lui. "Et ce sera... cinq points ? Non, disons dix points retirés à Serdaigle pour impertinence."

Hermione s'étrangla, et de nombreux autres firent de même.

"Professeur Severus Rogue," mordit Harry. "Je n'ai connaissance d'aucun acte que j'aurais pu commettre pour mériter votre hostilité. S'il existe un problème dont je ne suis pas au courant, je suggère que nous -"

"Taisez-vous, Potter. Dix points en moins pour Serdaigle. Les autres, ouvrez votre livre page 3."

Il n'y eut qu'une légère sensation de brûlure à l'arrière de la gorge de Harry, et aucune humidité dans ses yeux. Si pleurer n'était pas une stratégie efficace pour détruire un professeur de Potions, alors il était inutile de pleurer.

Lentement, Harry s'assit bien droit sur sa chaise. Tout son sang semblait avoir été évacué et remplacé par de l'azote liquide. Il savait qu'il avait essayé de garder son calme mais il ne pouvait pas se rappeler pourquoi.

"Harry," chuchota frénétiquement Hermione, deux pupitres plus loin, "arrête, s'il te plaît, ce n'est pas grave, on ne le comptera pas -"

"Vous parlez en classe, Granger ? Trois -"

"Donc," dit une voix plus froide que le zéro absolu, "comment s'y prend-t-on pour déposer une plainte formelle au sujet d'un professeur abusif ? Faut-il en parler à la Directrice Adjointe, écrire une lettre au conseil d'administration... pourriez-vous m'expliquer comment cela fonctionne ?"

La classe était complètement figée.

"Retenu pour un mois, Potter," dit Severus, son sourire encore plus large.

"Je refuse de reconnaître votre autorité en tant qu'enseignant et ne me plierai à aucune retenue donnée par vous."

Des gens s'arrêtèrent de respirer.

Le sourire de Severus disparut. "Alors vous serez -" il s'interrompit.

"Renvoyé, alliez-vous dire ?" Harry, lui, souriait maintenant d'un air subtil. "Mais vous avez semblé douter de votre capacité à mettre à exécution la menace, ou craindre les conséquences que cela aurait entraînées. Moi, en revanche, je ne doute ni ne crains la perspective de trouver une école où se trouveraient des professeurs moins abusifs. Ou peut-être devrais-je engager des précepteurs, comme j'ai l'habitude de le faire, et prendre des cours à ma vitesse d'apprentissage maximale. J'ai assez d'argent pour ça dans mon coffre-fort. Quelque chose en rapport avec un certain Seigneur des Ténèbres que j'aurais vaincu. Mais il y \emph{a}  des enseignants à Poudlard que j'aime bien, donc je pense que ce sera plus simple si je trouve plutôt un moyen de me débarrasser de vous."

"Vous débarrasser de moi ?" dit Severus, souriant à son tour d'un air subtil. "Quelle idée amusante. Et comment comptez-vous faire cela, Potter ?"

"J'ai cru comprendre qu'il y a eu un certain nombre de plaintes vous concernant venant d'élèves et de leurs parents," c'était une conjecture, mais elle était assez prudente, "ce qui ne nous laisse qu'une seule question : pourquoi n'êtes-vous pas déjà parti ? Poudlard est-elle financièrement trop en difficulté pour s'offrir un vrai professeur de Potions ? Si c'est le cas, je pourrais participer. Je suis sûr qu'ils trouveraient un meilleur enseignant s'ils offraient le double de votre salaire actuel."

Deux rayons de glace généraient un hiver glacial au travers de la salle.

"Vous découvrirez," dit doucement Severus, "que le conseil d'administration ne sera pas le moins du monde réceptif à votre offre."

"Lucius..." dit Harry. "\emph{Voilà}  pourquoi vous êtes encore ici. Peut-être devrais-je en discuter avec Lucius. Je crois qu'il désire me rencontrer. Je me demande si j'ai quelque chose qu'il recherche ?"

Hermione secoua frénétiquement la tête. Harry le remarqua du coin de l'œil, mais son attention était toute sur Severus.

"Vous êtes un garçon insensé," dit Severus. Il ne souriait maintenant plus du tout. "Vous n'avez rien qui importe plus à Lucius que mon amitié. Et même si c'était le cas, j'ai d'autres alliés." Sa voix devint dure. "Et je commence à trouver très peu probable que vous n'ayez pas été réparti à Serpentard. Comment êtes-vous parvenu à rester hors de ma Maison ? Ah, oui, parce que le Choixpeau Magique a dit qu'il \emph{rigolait} . Pour la première fois de l'histoire. De quoi \emph{parliez-vous}  vraiment avec le Choixpeau, Potter ? Aviez-vous quelque chose qu'il recherchait ?"

Harry fit face au regard froid de Severus puis se souvint que le Choixpeau l'avait mis en garde contre le fait de regarder des gens dans les yeux pendant qu'il pensait à - Harry baissa les yeux et fixa le bureau de Severus.

"Vous semblez étrangement réticent à me regarder dans les yeux, Potter !"

Un choc de compréhension soudaine - "Alors c'est contre \emph{vous}  que le Choixpeau Magique me mettait en garde !"

"Quoi ?" dit la voix de Severus, l'air sincèrement surprise, même si Harry ne voyait bien sûr pas son visage.

Harry se leva de son bureau.

"Asseyez-vous, Potter," dit une voix en colère venant d'un endroit qu'il ne regardait pas.

Harry l'ignora, et regarda le reste de la classe. "Je n'ai pas l'intention de laisser un enseignant si peu professionnel me gâcher ma scolarité à Poudlard," dit Harry avec un calme mortel. "Je pense que je vais prendre congé de ce cours, et soit engager un précepteur pour m'enseigner les Potions pendant que je suis ici, ou si le conseil d'administration est vraiment bouché, apprendre pendant l'été. Si un ou une d'entre vous décide qu'il ne souhaite pas être malmené par cet homme, mes sessions vous seront ouvertes."

"\emph{Asseyez-vous, Potter}  !"

Harry traversa la pièce et saisit la poignée.

Elle ne tourna pas.

Harry se retourna lentement et entraperçut Severus qui souriait méchamment avant de se souvenir de regarder ailleurs.

"Ouvrez cette porte."

"Non," dit Severus.

"Vous me faites me sentir menacé," dit une voix si glacée qu'elle ne ressemblait pas du tout à celle de Harry, "et c'est une erreur."

La voix de Severus s'esclaffa. "Et que comptez-vous faire à ce sujet, petit garçon ?"

Harry fit six longues enjambées en s'éloignant de la porte, jusqu'à se tenir à côté de la dernière rangée de pupitres.

Puis Harry se tint très droit et leva sa main d'un mouvement terrible, les doigts prêts à claquer.

Neville cria et plongea sous son pupitre. D'autres enfants se ratatinèrent ou levèrent leur bras instinctivement pour se protéger.

"\emph{Harry, non !} " cria Hermione. "Quoi que tu aies voulu faire, ne le fais pas!"

"Êtes-vous tous devenus \emph{fous}  ?" aboya la voix de Severus.

Lentement, Harry rabaissa sa main. "Je n'allais pas lui faire de mal, Hermione," dit Harry, sa voix redevenue un peu plus normale. "J'allais juste faire exploser la porte."

Quoique, se souvint Harry, on n'était pas censé Métamorphoser des choses qui allaient être brûlées, ce qui voulait dire que remonter dans le temps et demander à Fred et à George de Métamorphoser une quantité précautionneusement mesurée d'explosifs ne serait peut-être pas une si bonne idée...

"\emph{Silencio} ," dit la voix de Severus.

Harry essaya de dire "Quoi ?" mais découvrit qu'aucun son ne sortait.

"Tout ceci est ridicule. Je pense que je vous ai laissé vous causer assez d'ennuis pour une journée, Potter. Vous êtes l'élève le plus perturbateur et le plus indiscipliné que j'ai jamais vu, et je ne me rappelle pas combien de points Serdaigle a pour le moment mais je suis sûr que je peux tous les liquider. Dix points de moins pour Serdaigle. Dix points de moins pour Serdaigle. Dix points de moins pour Serdaigle ! Cinquante points de moins pour Serdaigle ! Maintenant asseyez-vous et regardez le reste de la classe suivre le cours !"

Harry mit sa main dans sa bourse et essaya de dire "marqueur", mais bien sûr aucun son ne sortit. Pendant un bref moment, cela l'arrêta ; puis l'idée lui vint d'épeler M-A-R-Q-U-E-U-R avec des mouvements de doigt, ce qui fonctionna. B-L-O-C, et il avait du papier. Harry marcha jusqu'à un pupitre vide, un autre que celui où il avait initialement été assis, et il griffonna un bref message. Il déchira cette feuille de papier, mit le marqueur et le bloc-notes dans une poche de ses robes afin d'avoir accès à eux plus rapidement, et tint son message, non pas face à Rogue, mais face au reste de la classe.

JE PARS

QUELQU'UN D'AUTRE

A-T-IL BESOIN DE SORTIR ?

"Vous êtes dément, Potter," dit Severus d'un ton de froid mépris.

À part ça, personne d'autre ne parla.

Harry fit une courbette ironique en direction du bureau du professeur, marcha jusqu'à un mur, ouvrit grand la porte d'un placard d'un mouvement fluide, y entra, et referma la porte derrière lui.

Puis il y eut le son étouffé de quelqu'un claquant des doigts ; puis plus rien.

Dans la salle, les étudiants se regardaient avec peur et perplexité.

Le visage du professeur de Potions était maintenant enragé. Il traversa la pièce en de terribles enjambées et ouvrit grand la porte du placard.

Le placard était vide.
\par\noindent\rule{\textwidth}{0.4pt}
Une heure plus tôt, Harry tendait l'oreille depuis l'intérieur du placard. Il n'y avait ni son à l'extérieur ni raison de prendre des risques inutiles.

C-A-P-E, épelèrent ses doigts.

Une fois invisible, il entrouvrit la porte très lentement, avec beaucoup de précautions, et jeta un coup d'œil dehors. Personne ne semblait se trouver dans la salle.

La porte n'était pas fermée.

Ce n'est qu'une fois hors du lieu dangereux et dans les couloirs, invisible et en sûreté, qu'une partie de la colère s'évacua et qu'il se rendit compte de ce qu'il venait de faire.

Ce qu'il venait de faire.

Le visage invisible de Harry était figé dans une expression d'horreur absolue.

Il venait de se mettre à dos un professeur, trois ordres de grandeur au-dessus de ce qu'il avait jamais fait auparavant. Il avait menacé de quitter Poudlard et aurait peut-être tenu sa parole. Il avait perdu tous les points de Serdaigle, puis il avait utilisé le Retourneur de Temps...

Son imagination lui montra ses parents lui criant dessus après qu'il ait été renvoyé, le professeur McGonagall déçue, et c'était juste trop douloureux et il ne pouvait pas le supporter et il \emph{ne pouvait pas imaginer comment se tirer de là}  -

La pensée que Harry s'autorisa à avoir était que si se mettre en colère l'avait mis dans le pétrin, alors peut-être qu'une fois en colère il trouverait une issue ; les choses semblaient étrangement plus claires quand il était énervé.

Et la pensée que Harry ne s'autorisa pas à avoir était qu'il ne pouvait tout simplement pas faire face au futur s'il n'était pas en colère.

Il ravala donc ses pensées et se souvint de la brûlante humiliation -

\emph{Tut, tut. La gloire ne fait clairement pas tout.} 

\emph{Dix points de moins pour Serdaigle pour impertinence.} 

Le froid apaisant se déversa à nouveau dans ses veines comme une vague revenant d'un récif, et Harry expira.

Bien. De nouveau sain d'esprit.

En fait, il se sentait assez déçu que son lui non-furieux se soit écroulé comme ça et n'ai rien voulu d'autre que de se tirer du pétrin. Le professeur Severus Rogue était un problème pour \emph{tout le monde} . Harry-normal avait oublié ça et avait souhaité trouver un moyen de se protéger \emph{lui} . Et laisser toutes les autres victimes aller se faire pendre ? La question n'était pas comment se protéger lui, la question était comment détruire le professeur de Potions.

\emph{Alors c'est ça, mon côté obscur ? En voilà un terme chargé de préjugés, mon côté clair a l'air bien plus égoïste et lâche, sans parler de sa tendance à être confus et paniqué.} 

Et maintenant qu'il pensait clairement, l'action suivante était également limpide. Il s'était déjà donné une heure pour se préparer, et pouvait en obtenir cinq de plus si nécessaire...
\par\noindent\rule{\textwidth}{0.4pt}
Minerva McGonagall attendait dans le bureau du Directeur.

Dumbledore était assis dans son trône rembourré, derrière son bureau, habillé de cinq couches de robes solennelles couleur lavande. Minerva était assise dans une chaise à côté de lui, et de l'autre côté de Dumbledore se trouvait Severus, qui était dans une autre chaise. Face aux trois se trouvait un tabouret en bois.

Ils attendaient Harry Potter.

\emph{Harry,}  pensa Minerva avec désespoir, \emph{vous aviez promis de ne pas mordre d'autres enseignants !} 

Et dans son esprit, elle put clairement voir le visage en colère de Harry et sa réponse outragée : \emph{J'ai dit que je ne mordrais personne qui ne m'aurait pas mordu en premier !} 

Il y eut un coup à la porte.

"Entrez !" dit Dumbledore.

La porte s'ouvrit grand, et Harry Potter entra. Minerva faillit manquer d'air. Le garçon semblait calme, serein, et avoir un contrôle total de lui-même.

"Bonj-" la voix de Harry s'interrompit soudain. Sa mâchoire s'affaissa.

Minerva suivit le regard de Harry, et elle vit qu'il regardait Fumseck posé sur son perchoir doré. Fumseck battait ses ailes rouge-or vif au rythme du tremblement d'une flamme, et il pencha sa tête, offrant au garçon un salut mesuré.

Harry pivota et fixa Dumbledore.

Dumbledore lui fit un clin d'œil.

Minerva sentit qu'elle venait de rater quelque chose.

Une incertitude apparut brutalement sur le visage de Harry. Son calme vacilla. La peur apparut dans ses yeux, puis la colère, et le garçon était à nouveau calme.

Un frisson parcourut l'épine dorsale de Minerva. Quelque chose n'allait pas.

"Asseyez-vous, s'il vous plaît," dit Dumbledore. Son visage était de nouveau sérieux.

Harry s'assit.

"Donc, Harry," dit Dumbledore. "J'ai entendu un compte-rendu de ce qui s'est passé aujourd'hui, donné par le professeur Rogue. Voudriez-vous s'il vous plaît me dire en vos termes ce qui s'est passé ?"

Harry jeta un coup d'oeil dédaigneux et rapide en direction de Severus. "Ce n'est pas compliqué," dit le garçon, un léger sourire sur le visage. "Il a essayé de me malmener tout comme il a malmené tous les non-Serpentard de l'école depuis le jour où Lucius vous l'a imposé. Pour ce qui est des détails, je sollicite une conversation privée avec vous. Après tout, on peut difficilement s'attendre à ce qu'un étudiant faisant part d'un comportement abusif de la part d'un professeur parle avec franchise en présence de ce même professeur."

Cette fois, Minerva ne put s'empêcher s'étouffer.

Severus se contenta de rire.

Et le visage de Dumbledore devint grave. "M. Potter," dit le Dumbledore, "on ne parle pas d'un professeur de Poudlard ainsi. J'ai peur que vous soyez victime d'un terrible malentendu. Le professeur Severus Rogue a ma totale confiance, et sert Poudlard sous mes ordres, pas sous ceux de Lucius Malfoy."

Il y eut un silence pendant quelques instants.

Lorsque le garçon parla à nouveau, sa voix était glaciale. "Y a-t-il quelque chose qui m'aurait échappé ?"

"Un certain nombre de choses, M. Potter," dit le Directeur. "Pour commencer, vous devriez comprendre que le but de cette réunion est de déterminer votre punition pour les événements de ce matin."

"Cet homme a terrorisé votre école pendant des années. J'ai parlé aux élèves et ai commencé à assembler des témoignages afin de m'assurer qu'il y ait de quoi commencer une campagne dans les journaux afin de rallier les parents contre lui. Certains des étudiants les plus jeunes ont pleuré lorsqu'ils m'ont raconté leur histoire. J'ai presque pleuré quand je les ai entendues ! \emph{Vous avez laissé cet agresseur libre ? Vous avez fait ça à vos étudiants ? Pourquoi ?} "

Minerva ravala une boule dans sa gorge. Elle avait - pensé cela, parfois, mais elle n'avait jamais tout à fait -

"M. Potter," dit le directeur, sa voix à présent sévère, "cette réunion n'est pas à propos du professeur Rogue. Elle est à propos de vous et de votre mépris pour la discipline de cette école. Le professeur Rogue a suggéré, et j'y ai consenti, que trois mois de retenue seraient appropriés -"

"Rejetés," dit Harry d'un ton glacial.

Minerva était sans voix.

"Ce n'est pas une requête, M. Potter," dit le Directeur. Toute la force de son regard était concentrée sur le garçon. "C'est votre puniti-"

"Vous m'expliquerez pourquoi vous avez laissé cet homme faire du mal aux enfants placés sous votre garde, et si votre explication n'est pas satisfaisante, alors je commencerai ma campagne dans les journaux et \emph{vous } enserez la cible."

Le corps de Minerva fut ébranlé par la force de ce coup, de ce \emph{lèse-majesté}  absolu.

Même Severus avait l'air choqué.

"Ceci, Harry, serait extrêmement malavisé," dit lentement Dumbledore. "Je suis la principale pièce opposant Lucius sur le plateau de jeu. Que vous fassiez une chose pareille le renforcerait grandement, et je ne pensais pas que c'était le camp que vous aviez choisi."

"Cette conversation devient privée," dit Harry. Sa main fit un rapide mouvement en direction de Severus. "Faites-le sortir."

Dumbledore secoua la tête. "Harry, ne vous ai-je pas dit que Severus Rogue a ma totale confiance ?"

Le visage du garçon révéla à quel point cela venait de le choquer. "Les actes de cet homme vous rendent vulnérables ! Je ne suis pas le seul qui pourrait commencer une campagne dans les journaux contre vous ! C'est insensé ! Pourquoi faites-vous cela ?"

Dumbledore soupira. "Je suis navré, Harry. Ça a à voir avec des choses que vous n'êtes pas, pour l'instant, prêt à entendre."

Le garçon fixa Dumbledore. Puis il se tourna pour regarder Severus. Puis à nouveau Dumbledore.

"C'\emph{est}  de la folie," dit lentement le garçon. "Vous ne lui avez pas serré la bride parce que vous pensez qu'il fait \emph{partie du motif} . Que Poudlard a besoin d'un professeur de Potions maléfique pour être une vraie école de magie, tout comme elle a besoin d'un fantôme pour enseigner l'Histoire."

"Ça ressemble au genre de chose que j'ai l'habitude de faire, non ?" dit Dumbledore en souriant.

"Inacceptable," dit catégoriquement Harry. Son visage était maintenant froid et sombre. "Je n'accepterai ni ces brutalisations ni ces abus. J'avais envisagé de nombreuses façon de traiter ce problème, mais je vais rendre la chose simple. Soit cet homme part, soit je pars."

Minerva glapit de nouveau. Quelque chose de bizarre s'agita dans les yeux de Severus.

Le regard de Dumbledore devenait froid lui aussi. "L'expulsion, M. Potter, est la menace ultime qui puisse être utilisée contre un élève. Elle n'est habituellement pas utilisée comme menace par un étudiant contre le Directeur. Poudlard est la meilleure école de magie du monde, et il n'est pas donné à tout le monde d'être éduqué ici. Avez-vous l'impression que Poudlard ne pourrait pas se passer de vous ?"

Et Harry resta assis là, un léger sourire sur le visage.

Une horreur soudaine s'abattit sur Minerva. Harry n'allait quand même pas -

"Vous oubliez," dit Harry, "que vous n'êtes pas le seul capable de déceler les motifs. \emph{Ceci devient confidentiel. Maintenant faites-le -"}  Harry fit un nouveau geste en direction de Severus, puis s'arrêta au milieu de sa phrase et de son geste.

Minerva pouvait le voir sur le visage de Harry, le moment où il s'était souvenu.

Après tout, elle le lui avait dit.

"M. Potter," dit le Directeur, "une fois de plus, Severus Rogue a ma confiance totale."

"Vous lui avez dit," murmura le garçon, "pauvre idiot."

Dumbledore ne réagit pas à l'insulte. "Dit quoi ?"

"Que le Seigneur des Ténèbres est en vie."

"\emph{Nom de Merlin mais de quoi parlez-vous, Potter ?} " s'écria Severus sur une multitude de tons de pure stupéfaction et d'outrage.

Harry lui jeta un rapide coup d'œil, souriant de façon sinistre. "Oh, alors on est \emph{bien}  un Serpentard," dit Harry, "je commençais à me demander."

Et ils furent silencieux.

Dumbledore parla enfin. Sa voix était faible. "Harry, de \emph{quoi}  parlez-vous ?"

"Je suis navrée, Albus," murmura Minerva.

Severus et Dumbledore se tournèrent pour la regarder.

"Le professeur McGonagall ne me l'a pas dit," dit la voix de Harry, rapidement, et moins calmement qu'il ne l'avait été jusqu'à maintenant. "J'ai deviné. Je vous l'ai dit, je peux voir les motifs moi aussi. J'ai deviné, et elle a contrôlé sa réaction exactement comme Severus l'a fait. Mais son contrôle a été à un cheveu de la perfection, et je me suis rendu compte que sa réaction n'était pas authentique mais contrôlée."

"Et je lui ai dit," dit Minerva, sa voix un peu tremblante, "que vous, moi, et Severus étions les seuls à savoir."

"Ce qu'elle m'a concédé afin de m'empêcher de me promener partout en posant des questions à tout le monde, ce que j'avais menacé de faire si elle refusait de parler," dit Harry. Le garçon gloussa brièvement. "J'aurais vraiment dû coincer l'un de vous seul à seul et vous dire qu'elle m'avait tout dit, pour voir si vous auriez trahi quelque chose. Ça n'aurait probablement pas marché, mais ça aurait valu le coup d'essayer." Le garçon sourit à nouveau. "La menace est toujours là et je compte être \emph{pleinement}  mis au courant un jour ou l'autre."

Severus la regardait avec le mépris le plus profond. Minerva releva le menton et soutint le regard. Elle savait qu'elle le méritait.

Dumbledore s'enfonça dans son trône rembourré. Jamais Minerva ne l'avait vu avec des yeux aussi froids, pas depuis le jour où son frère était mort. "Et vous menacez de nous abandonner à Voldemort si nous ne nous soumettons pas à vos souhaits."

La voix de Harry était tranchante comme un rasoir. "Je suis au regret de vous informer que vous n'êtes pas le centre de l'univers. Je ne menace pas d'abandonner l'Angleterre magique. Je menace de \emph{vous}  abandonner. Je ne suis pas un humble petit Frodon. C'est \emph{ma } quête, et si vous voulez en faire partie, vous jouerez selon \emph{mes}  règles."

Le visage de Dumbledore était toujours froid. "Je commence à douter de votre aptitude à être le héros, M. Potter."

Le regard que lui renvoya Harry était tout aussi glacé. "Je commence à douter de votre aptitude à être mon Gandalf, \emph{M. Dumbledore} . Au moins Boromir était une erreur plausible. Qu'est-ce que ce \emph{Nazgul}  fait dans ma Communauté ?"

Minerva était totalement perdue. Elle regarda Severus, pour voir s'il arrivait à suivre, et elle vit qu'il avait détourné son visage hors du champ de vision de Harry et qu'il souriait.

"Je suppose," dit lentement Dumbledore, "que de votre point de vue, c'est une question raisonnable. Donc, M. Potter, si le professeur Rogue vous laisse tranquille à l'avenir, cela va-t-il être la dernière fois que ce problème survient, ou vais-je vous retrouver ici chaque semaine, armé d'une nouvelle exigence ?"

"\emph{Me}  laisser tranquille ?" la voix de Harry était outragée. "Je ne suis pas sa seule victime et je ne suis certainement pas le plus vulnérable ! \emph{Avez-vous oublié à quel point les enfants sont sans défense ? À quel point ils peuvent souffrir ?}  À l'avenir, Severus traitera \emph{tous}  les étudiants de Poudlard avec la courtoise appropriée à un professionnel, ou vous vous trouverez un autre professeur de Potions, ou vous vous trouverez un autre héros !"

Dumbledore commença à rire. Un rire chaud et plein d'humour, à gorge déployée, comme si Harry venait d'exécuter une danse comique devant lui.

Minerva n'osait pas bouger. Elle jeta un coup d'œil rapide et vit que Severus était tout aussi immobile.

Le visage de Harry devint encore plus froid. "Vous me comprenez mal, Directeur, si vous pensez que c'est une plaisanterie. Ce n'est pas une requête. C'est votre punition."

"M. Potter -" dit Minerva. Elle ne savait même pas ce qu'elle comptait dire. Elle ne pouvait simplement pas laisser passer ça.

D'un geste, Harry lui intima de se taire, et il continua de parler à Dumbledore. "Et si cela vous semble malpoli," dit Harry, sa voix un peu moins dure, "sachez que ça ne semblait pas moins malpoli quand vous me l'avez dit. Vous ne diriez pas une chose pareille à quelqu'un que vous considéreriez comme un véritable être humain et non pas comme un enfant subordonné, et je vous traiterai avec la même courtoisie que celle avec laquelle vous me traitez -"

"Oh, en effet, en grand effet, si j'ai jamais reçu une punition, c'est bien celle-là ! Bien \emph{sûr}  que vous êtes ici, à me faire chanter pour sauver vos camarades, pas pour vous sauver vous-même ! Je ne comprends pas comment j'ai pu imaginer qu'il en soit autrement !" Dumbledore riait de plus belle. Il frappa trois fois du poing sur la table.

Harry sembla plus incertain. Son visage se tourna vers Minerva, s'adressant à elle pour la première fois. "Excusez-moi," dit Harry. Sa voix semblait vacillante. "A-t-il besoin de prendre ses médicaments, ou quelque chose dans le genre ?"

"Ah..." Minerva ne savait pas ce qu'elle pouvait bien dire.

"Bon," dit Dumbledore. Il essuya les larmes qui s'étaient amoncelées dans ses yeux. "Excusez-moi. Je suis navré de cette interruption. Continuez avec le chantage, s'il vous plaît."

Harry ouvrit la bouche puis la referma. Il semblait à présent légèrement chancelant. "Ah... il devra aussi arrêter de lire dans l'esprit des élèves."

"Minerva," dit Severus, sa voix meurtrière, "tu -"

"Le Choixpeau Magique m'a mis en garde," dit Harry.

"\emph{Quoi}  ?"

"Je ne peux pas en dire plus. Bref, je pense que c'est tout. J'ai terminé."

Silence.

"Et maintenant ?" dit Minerva, quand il devint clair que personne d'autre n'allait parler.

"Et maintenant ?" dit Dumbledore en écho. "Allons, le héros gagne, bien sûr."

"\emph{Quoi ?} " dirent Severus, Minerva et Harry.

"Eh bien, il semble certainement nous sommes coincés," dit Dumbledore, souriant gaiement. "Mais Poudlard \emph{a}  besoin d'un professeur de Potions maléfique, ou ce ne serait pas une véritable école de magie, n'est-ce pas ? Alors que diriez-vous si le professeur Rogue était horrible uniquement envers les étudiants de cinquième année et plus ?"

"\emph{Quoi}  ?" dirent-ils tous les trois.

"Si ce sont les victimes les plus vulnérables qui vous inquiètent. Peut-être que vous avez raison, Harry. Peut-être que \emph{j'ai}  oublié, au fil des décennies, ce que c'est que d'être un enfant. Alors faisons un compromis. Severus continuera de décerner injustement des points à Serpentard et de faire régner une discipline molle sur sa Maison, et il sera horrible envers les étudiants non-Serpentard de cinquième année et plus. Pour les autres, il sera effrayant, mais il ne les malmènera pas. Il promettra de ne lire l'esprit des étudiants que lorsque leur sécurité rendra cet acte nécessaire. Poudlard aura son professeur de Potions maléfique, et les victimes les plus vulnérables, comme vous le dites, seront en sécurité."

Minerva McGonagall était plus choquée qu'elle ne l'avait jamais été de sa vie. Elle jeta un regard incertain à Severus, dont le visage était devenu complètement neutre, comme s'il ne pouvait pas décider de l'expression qu'il était censé afficher.

"Je suppose que c'est acceptable," dit Harry. Il avait un voix étrange.

"Je suis tout à fait en faveur de cette idée," dit lentement Minerva. Elle était tellement en faveur que son cœur cognait follement sous ses robes. "Mais que pourrons-nous bien dire aux élèves ? Ils ne se sont peut-être pas posé de questions pendant que Severus était... horrible envers tout le monde, mais -"

"Harry peut dire aux autres étudiants qu'il a découvert un terrible secret concernant Severus et qu'il a fait un peu de chantage," dit Dumbledore. "C'est vrai après tout ; il a découvert que Severus lisait les esprits, et il nous a certainement fait du chantage."

"C'est de la folie !" explosa Severus.

"Bwah ha ha !" dit Dumbledore.

"Ah..." dit Harry d'un ton incertain. "Et si quelqu'un me demande pourquoi les cinquième année et plus se sont fait avoir ? Je ne leur en voudrais pas s'ils devenaient furieux, et ce n'est pas vraiment moi qui ai décidé que -"

"Dites-leur," dit Dumbledore, "que ce n'est pas vous qui avez suggéré le compromis, que c'est tout ce que vous pouviez obtenir. Puis refusez d'en dire plus. Cela aussi sera vrai. C'est un art, vous l'apprendrez avec la pratique."

Harry hocha lentement la tête. "Et les points qu'il a enlevés à Serdaigle ?"

"Ils ne doivent pas être rendus."

C'est Minerva qui avait dit ça.

Harry la regarda.

"Je suis désolée, M. Potter," dit-elle. Elle \emph{était}  désolée, mais ça devait être fait. "Votre mauvais comportement \emph{doit}  avoir quelques conséquences, ou cette école va tomber en morceaux."

Harry haussa les épaules. "Acceptable," dit-il. "Mais à l'avenir, Severus ne s'attaquera pas à mes liens avec ma Maison en m'ôtant des points, et il ne gâchera pas mon temps précieux avec des retenues. S'il se trouve que mon comportement mérite une correction, il fera part de ses préoccupations au professeur McGonagall."

"Harry," dit Minerva, "continuerez-vous de vous soumettre à la discipline de cette école, ou serez-vous maintenant au-dessus de la loi, comme Severus l'était ?"

Harry la regarda. Quelque chose de chaleureux apparut brièvement dans ses yeux avant d'être écrasé. "Je continuerai d'être un étudiant ordinaire avec tout membre du personnel n'étant ni fou ni maléfique, du moment qu'ils ne sont pas victimes de pressions venant d'autres étant l'un ou l'autre." Harry jeta un bref coup d'œil à Severus, puis se retourna vers Dumbledore. "Laissez Minerva seule, et en sa présence je serai un étudiant normal de Poudlard. Pas de privilèges spéciaux ni d'immunités."

"Magnifique," dit sincèrement Dumbledore. "Parlé comme un vrai héros."

"Et," dit-elle, "M. Potter doit s'excuser en public pour ses actes d'aujourd'hui."

Harry lui jeta un autre coup d'œil. Celui-ci était un peu sceptique.

"La discipline de cette école a été fortement mise à mal par vos actes, M. Potter," dit Minerva. "Elle doit être restaurée."

"Professeur McGonagall, je pense que vous surestimez grandement l'importance de ce que vous appelez la discipline de l'école, comparé au fait d'avoir l'Histoire enseignée par un professeur vivant ou à celui de ne pas torturer vos étudiants. Il semble beaucoup plus sage, moral et important de maintenir la hiérarchie et de faire respecter ses règles quand on est en haut à faire la loi que quand on est en bas, et si nécessaire je peux citer des études à ce sujet. Je pourrais continuer sur ce sujet pendant des heures mais je m'arrêterai là."

Minerva secoua la tête. "M. Potter, vous sous-estimez l'importance de la discipline parce que vous n'en avez pas vous-même besoin -" Elle s'interrompit. Elle n'avait pas bien formulé sa phrase, et Severus, Dumbledore et même Harry la regardaient d'un air bizarre. "Pour apprendre, je veux dire. Tous les enfants ne peuvent pas apprendre en l'absence d'une autorité. Et ce sont les autres enfants qui en souffriront, M. Potter, s'ils vous voient comme un exemple à suivre."

Les lèvres de Harry formèrent un sourire tordu. "Le premier et le dernier recours est toujours la vérité. La vérité est que je n'aurais pas dû me mettre en colère, je n'aurais pas dû déranger la classe, je n'aurais pas dû faire ce que j'ai fait, et j'ai montré le mauvais exemple à tout le monde. La vérité est aussi que Severus Rogue s'est comporté d'une façon ne convenant pas à un professeur de Poudlard, et que dorénavant il fera plus attention aux sentiments blessés des étudiants en quatrième année et en-dessous. Nous pourrions tous deux nous lever et dire la vérité. Je peux vivre avec ça."

"Dans vos rêves, Potter !" cracha Severus.

"Après tout,", dit Harry, souriant de façon sinistre, "si les élèves voient que les règles sont pour \emph{tout le monde} ... pour les professeurs aussi, pas seulement pour les pauvres étudiants impuissants qui ne tirent rien du système à part de la souffrance... voyons, les effets positifs sur la discipline scolaire devraient être \emph{prodigieux} ."

Il y eut une brève pause, puis Dumbledore gloussa. "Minerva pense que vous avez raison, bien plus que vous ne devriez avoir le droit de l'être."

Le regard de Harry s'éloigna de Dumbledore dans un tressaillement, et dériva vers le sol. "Lisez-\emph{vous son}  esprit ?"

"On confond souvent le bon sens et la Legilimancie," dit Dumbledore. "Je parlerai de cette affaire avec Severus, et aucune excuse publique ne sera exigée de vous à moins qu'il ne s'excuse lui aussi. Et je déclare maintenant cette affaire classée, au moins jusqu'au déjeuner." Il s'interrompit. "Mais, Harry, j'ai peur que Minerva ne souhaite discuter d'une autre affaire vous. Et ce n'est pas le résultat d'une quelconque pression de ma part. Minerva, s'il te plaît ?"

Minerva se leva de sa chaise et faillit tomber. Il y avait trop d'adrénaline dans son sang, son cœur battait trop vite.

"Fumseck," dit Dumbledore, "accompagne-la, s'il te plaît."

"Je n'ai -" commença-t-elle à dire.

Dumbledore lui jeta un regard, et elle devint silencieuse.

Le phénix s'envola dans la pièce comme une langue de flamme aurait soudain bondit, et il atterrit sur son épaule. Elle sentit la chaleur à travers ses robes et à travers son corps.

"Suivez-moi, M. Potter," dit-elle, fermement cette fois, et ils franchirent la porte.
\par\noindent\rule{\textwidth}{0.4pt}
Ils se tenaient sur les escaliers rotatifs, descendant en silence.

Minerva ne savait pas quoi dire. Elle ne savait pas qui était cette personne debout à côté d'elle.

Et Fumseck commença à chanter.

C'était tendre et doux, le son qu'aurait fait un feu de cheminée s'il avait eu une mélodie, et il se déversa dans l'esprit de Minerva, apaisant, rassurant, relaxant ce qu'il touchait...

"\emph{Qu'} est-ce que c'est que \emph{ça}  ?" murmura Harry à côté d'elle. Sa voix était instable, vacillante, aux tons changeants.

"La chanson du phénix," dit Minerva, pas tout à fait consciente de ce qu'elle disait, son attention entièrement dirigée vers l'étrange musique douce. "Elle guérit, elle aussi."

Harry se détourna d'elle, mais elle entraperçut un éclair de souffrance sur son visage.

La descente semblait prendre longtemps, ou peut-être était-ce seulement la musique qui semblait prendre longtemps, et lorsqu'ils passèrent par le trou où s'était trouvée une gargouille, elle tenait fermement la main de Harry dans la sienne.

Alors que la gargouille revenait à sa place, Fumseck quitta son épaule pour venir flotter face à Harry.

Harry regardait Fumseck avec l'air hypnotisé de quelqu'un regardant la flamme changeante d'un incendie.

"Que dois-je faire, Fumseck ?" murmura Harry. "Je n'aurais pas pu les protéger si je n'avais pas été en colère."

Les ailes du phénix continuèrent de battre, et il continua à flotter sur place. Il n'y avait pas d'autre son que le battement des ailes. Puis il y eut un flash, comme un feu qui aurait brièvement resplendi avant de se rendormir, et Fumseck n'était plus là.

Ils clignèrent des yeux, comme s'ils venaient de se réveiller, ou peut-être comme s'ils venaient de se rendormir.

Minerva baissa les yeux.

Le jeune et intelligent visage de Harry la regardait.

"Les phénix sont-ils des gens ?" dit Harry. "Je veux dire, sont-ils assez intelligents pour être considérés comme des gens ? Pourrais-je parler avec Fumseck si je savais comment faire ?"

Minerva cligna des yeux de façon appuyée. Puis elle cligna à nouveau. "Non," dit Minerva, la voix chancelante. "Les phénix sont des créatures faites d'une puissante magie. Cette magie donne à leur existence un sens qu'aucun animal ne pourrait avoir. Ils sont le feu, la lumière, la guérison et la renaissance. Mais en fin de compte, non."

"Où pourrais-je en obtenir un ?"

Minerva se pencha et le prit dans ses bras. Elle n'avait pas prévu de le faire, mais elle ne semblait pas avoir vraiment le choix.

Lorsqu'elle se releva, elle découvrit qu'elle avait du mal à parler. Mais il fallait qu'elle sache. "Que s'est-il passé aujourd'hui, Harry ?"

"Moi non plus, je ne connais pas les réponses aux questions importantes. Mais pour le moment je préférerais vraiment ne pas y penser."

Minerva prit à nouveau sa main dans la sienne, et ils parcoururent le reste du chemin en silence.

C'était un court trajet, puisque le bureau de l'adjointe était naturellement proche de celui du directeur.

Minerva s'assit derrière son bureau.

Harry s'assit devant son bureau.

"Donc," murmura Minerva. Elle aurait donné quasiment n'importe quoi pour ne pas faire ça, ou pour ne pas être celle qui devait le faire, ou pour le faire n'importe quand, mais pas maintenant. "C'est une question de discipline scolaire. Dont vous n'êtes pas exempté."

"À savoir ?" dit Harry.

Il ne savait pas. Il n'avait pas encore compris. Elle sentit sa gorge se serrer. Mais il y avait un travail à accomplir et elle ne s'y déroberait pas.

"M. Potter," dit le professeur McGonagall, "faites-moi s'il vous plaît voir votre Retourneur de Temps."

Toute la paix du phénix disparut instantanément du visage de Harry et Minerva eut l'impression qu'elle venait de le poignarder.

"\emph{Non !} " dit Harry. Sa voix fut prise de panique. "J'en ai besoin, je ne pourrai pas étudier à Poudlard, je ne pourrai pas dormir !"

"Vous pourrez dormir," dit-elle. "Le Ministère a délivré les coques de protections pour votre Retourneur de Temps. Je l'enchanterai pour qu'il s'ouvre seulement entre 21h et minuit."

Le visage de Harry se tordit. "Mais - mais je -"

"M. Potter, combien de fois avez-vous utilisé le Retourneur de Temps depuis lundi ? Combien d'heures ?"

"Je..." dit Harry. "Attendez, laissez-moi compter -" Il regarda sa montre.

Minerva eut une montée de tristesse. C'était bien ce qu'elle pensait. "Ce n'était pas seulement deux fois par jour, alors. Je pense que si j'interrogeais vos camarades de dortoir, je découvrirais que vous avez eu du mal à rester debout assez longtemps pour aller vous coucher à une heure raisonnable, et que vous vous êtes levé de plus en plus tôt chaque matin. Correct ?"

Le visage de Harry disait tout ce qu'elle avait besoin de savoir.

"M. Potter," dit-elle avec gentillesse, "il y a des élèves à qui on ne peut confier des Retourneurs de Temps, parce qu'ils deviennent drogués. Nous leur donnons une potion qui rallonge leur cycle de sommeil de la durée nécessaire, mais ils finissent par utiliser le Retourneur de Temps pour bien plus que de la présence en cours. Et nous devons alors les reprendre. M. Potter, vous avez fait du Retourneur de Temps votre solution à tout, souvent de façon idiote. Vous l'avez utilisé pour récupérer un Rapeltout. Vous avez disparu d'un placard devant les autres étudiants au lieu de revenir en arrière après être sorti et d'être venu me chercher moi ou quelqu'un d'autre pour que nous venions ouvrir la porte."

Vu la tête de Harry, il n'avait pas pensé à ça.

"Et plus important," dit-elle, "vous auriez dû simplement suivre le cours du professeur Rogue. Et regarder. Et partir à la fin du cours. Comme vous l'auriez fait si vous n'aviez pas possédé un Retourneur de Temps. Il existe certains élèves à qui on ne peut confier un Retourneur de Temps. Vous êtes l'un d'eux. J' en suis navrée."

"Mais j'en ai \emph{besoin}  !" lâcha Harry. "Et si des Serpentard me menacent et que je dois m'échapper ? Il me garde en \emph{sécurité}  -"

"Tous les autres étudiants de ce château courent les même risques, et je vous assure qu'ils survivent. Aucun étudiant n'est mort dans ce château depuis près de cinquante ans. M. Potter, vous allez me rendre le Retourneur de Temps et vous allez me le rendre maintenant."

Le visage de Harry était déchiré par la souffrance, mais il sortit le Retourneur de Temps de sous ses robes et le lui donna.

Minerva sortit une des coques protectrices qui avaient été envoyées à Poudlard de son bureau. Elle clippa la coque autour du sablier rotatif du Retourneur de Temps, puis elle déposa sa baguette sur la coque pour compléter l'enchantement.

"\emph{Ce n'est pas juste}  !" glapit Harry. "J'ai sauvé la moitié de Poudlard du professeur Rogue aujourd'hui, est-il juste que je sois puni pour ça ? J'ai vu votre regard, vous \emph{détestiez}  ce qu'il faisait !"

Minerva resta silencieuse pendant quelques instants. Elle enchantait.

Lorsqu'elle eut fini et qu'elle releva le regard, elle sut que son visage était dur. Peut-être qu'elle avait tort. Mais peut-être que c'était ce qu'il fallait faire. Il y avait un enfant obstiné face à elle, et ça ne voulait \emph{pas}  dire que l'univers était en morceaux.

"\emph{Juste} , M. Potter ?" dit-elle d'un ton moqueur. "J'ai dû faire \emph{deux rapports}  au Ministère pour utilisation public d'un Retourneur de Temps, \emph{deux jours de suite}  ! Soyez \emph{extrêmement}  reconnaissant qu'on vous ait autorisé à conserver le Retourneur de Temps, même sous une forme restreinte ! Le Directeur a fait un appel par cheminée pour plaider votre cause personnellement, et si vous n'étiez pas le Survivant, même ça n'aurait pas suffit !"

Harry la regarda, la bouche grande ouverte.

Elle savait qu'il voyait le visage du professeur McGonagall en colère.

Les yeux de Harry s'emplirent de larmes.

"Je suis, désolé," chuchota-t-il, sa voix maintenant étouffée et brisée. "Je suis désolé, de vous avoir, déçue..."

"Je suis désolée aussi, M. Potter," dit-elle d'un ton dur, et elle lui tendit le nouveau Retourneur de Temps Restreint. "Vous pouvez partir."

Harry fit demi-tour et fuit son bureau en sanglotant. Elle entendait ses pieds traîner dans le couloir, puis le son fut interrompu par celui d'une porte qui se refermait.

"Je suis désolée aussi, Harry," murmura-t-elle dans la pièce silencieuse. "Je suis désolée aussi."
\par\noindent\rule{\textwidth}{0.4pt}
Quinze minutes après le début du déjeuner.

Personne ne parlait à Harry. Certains des Serdaigle lui jetaient des regards de colère, d'autres de sympathie, et quelques-uns parmi les plus jeunes le regardaient même avec admiration, mais personne ne lui parlait. Même Hermione n'avait pas essayé de s'approcher.

Fred et George étaient venus avec précaution. Ils n'avaient rien dit. L'offre était claire, et elle était optionnelle. Harry leur avait dit qu'il viendrait pour le dessert, pas avant. Ils avaient hoché la tête et étaient vite partis.

C'était probablement l'absence totale d'expression sur le visage de Harry qui avait cet effet.

Les autres pensaient probablement qu'il contrôlait sa colère ou sa consternation. Ils savaient qu'il avait été appelé dans le bureau du Directeur parce qu'ils avaient vu le professeur Flitwick venir le chercher.

Harry essayait de ne pas sourire, parce que s'il souriait, il commencerait à rire, et s'il commençait à rire, il ne s'arrêterait pas avant que les gentils monsieurs en veste blanche ne viennent l'emporter.

C'en était trop. C'en était vraiment trop. Harry était presque passé du Côté Obscur, son côté obscur avait fait des choses qui semblaient démentes rétrospectivement, son côté obscur avait gagné une victoire impossible qui était peut-être réelle mais qui était peut-être une pure fantaisie de la part d'un Directeur fou, son côté obscur avait protégé ses amis. Ils ne pouvait plus le supporter. Il avait besoin que Fumseck chante à nouveau. Il avait besoin d'utiliser le Retourneur de Temps, de s'en aller passer une heure au calme pour récupérer, mais ce n'était plus possible et la perte était comme un trou dans son existence, mais il ne pouvait pas y penser parce qu'alors il pourrait commencer à rire.

Vingt minutes. Tous les élèves qui allaient venir au déjeuner étaient déjà là, et presque aucun n'était parti.

Le battement d'une cuillère résonna dans la Grande Salle.

"Si je pouvais avoir votre attention," dit Dumbledore. "Harry Potter aimerait partager quelque chose avec nous."

Harry prit une profonde inspiration et se leva. Il marcha jusqu'à la Grande Table, et tous les yeux étaient braqués sur lui.

Harry se retourna et regarda les quatre tables.

Ça devenait de plus en plus difficile de ne pas sourire, mais Harry vida son visage de toute expression alors qu'il récitait son bref discours appris par cœur.

"La vérité est sacrée," dit Harry d'une voix sans timbre. "Une de mes possessions que je chéris le plus est un bouton sur lequel il est écrit : 'Dis la vérité, même si ta voix tremble.' Voici donc la vérité. Souvenez-vous en. Je ne le dis pas parce qu'on m'y oblige, mais parce que c'est vrai. Ce que j'ai fait durant le cours du professeur Rogue était sot, stupide, puéril, et une enfreinte inexcusable aux règles de Poudlard. J'ai dérangé la classe et j'ai privé mes camarades d'un temps d'étude irremplaçable. Tout ça parce que je ne suis pas parvenu à contrôler mon tempérament. J'espère qu'aucun d'entre vous ne suivra jamais mon exemple. Je compte certainement essayer de ne plus jamais le suivre."

Un grand nombre des étudiants qui regardaient Harry arboraient maintenant une expression grave et mécontente, comme on aurait pu en voir lors d'une cérémonie marquant la perte d'un champion tombé au combat. L'expression était quasi unanime chez les plus jeunes Gryffondor.

Jusqu'à ce que Harry lève sa main.

Il ne la leva pas haut. Ça aurait pu sembler prétentieux. Il ne la leva certainement pas vers Severus. Harry leva simplement sa main au niveau de sa poitrine, et claqua doucement des doigts, un geste qui fut plus vu qu'entendu. Il était possible que la majorité de la Grande Table ne l'ait même pas remarqué.

Ce qui ressemblait à un geste de défiance déclencha des sourires chez les plus jeunes étudiants et chez les Gryffondor, ainsi que des ricanements hautains chez les Serpentard, et des grimaces et des airs inquiets chez tous les autres.

Harry maintint son expression neutre. "Merci," dit-il. "C'est tout."

"Merci, M. Potter," dit le Directeur. "Et maintenant, le professeur Rogue aimerait lui aussi partager quelque chose avec vous."

Severus se leva avec fluidité de sa chaise à la Grande Table. "On a porté à mon attention," dit-il, "que mes propres actes ont en partie provoqué la colère certes inexcusable de M. Potter, et dans la discussion qui s'en est suivie, je me suis rendu compte que j'avais oublié avec quelle facilité les sentiments des jeunes et des immatures peuvent être blessés -"

Il y eut le son de nombreuses personnes s'étouffant en même temps.

Severus continua comme s'il n'avait rien entendu. "Le cours de Potion est un lieu dangereux, et je pense toujours que la discipline la plus stricte est nécessaire, mais je ferai dorénavant plus attention à la... fragilité émotionnelle... des élèves en quatrième année et moins. Ma déduction de points de Serdaigle tient toujours, mais j'annule la retenue de M. Potter. Merci."

Il y eut un unique applaudissement venant de Gryffondor, et plus vite que la lumière la baguette de Severus était dans sa main et \emph{"Silencio}  !" fit taire le contrevenant.

"J'exigerai toujours la discipline et le respect dans \emph{tous}  mes cours," dit froidement Severus, "quiconque essayant de me chahuter le regrettera."

Il se rassit.

"Merci aussi !" dit le directeur Dumbledore avec joie. "Continuons !"

Et Harry, toujours neutre, commença à redescendre jusqu'à son siège à la table Serdaigle.

Il y eut une explosion de conversation. Deux mots étaient clairement identifiables au début. Le premier était un "Qu'est -" initial, commençant de nombreuses phrases telles que "Qu'est ce qui s'est passé -" ou "Qu'est-ce que c'était que -". L'autre était "\emph{Récurvite!} ", au fur et à mesure que les élèves nettoyaient leurs robes et la nappe de la nourriture tombée et des boissons crachées.

Certains élèves pleuraient ouvertement. Le professeur Chourave pleurait aussi.

À la table Gryffondor, où attendait un gâteau décoré de cinquante et une bougies, Fred murmura, "Je pense qu'on est peut-être un peu dépassés, là, George."

Et depuis ce jour, peu importe ce que Hermione essaierait de dire aux gens, ce serait une légende communément acceptée de Poudlard que Harry Potter pouvait faire survenir absolument n'importe quoi en claquant des doigts.

