
\chapter{Poser les mauvaises questions}

Elen sila J.K. Rowling omentielvo.

MISE À JOUR : Ne paniquez pas. Je jure solennellement qu'il existe une explication logique, prévue, et conforme au canon à tout ce qui se passe dans ce chapitre. C'est un puzzle, vous êtes censé essayé de le résoudre, et sinon lisez juste le prochain chapitre.
\par\noindent\rule{\textwidth}{0.4pt}
"\emph{C'est une des énigmes les plus simples que j'ai jamais entendues."} 
\par\noindent\rule{\textwidth}{0.4pt}
Dès que Harry ouvrit les yeux, dans le dortoir des garçons de première année de Serdaigle, le matin de sa première journée complète à Poudlard, il sut que quelque chose n'allait pas.

Tout était calme.

\emph{Bien trop}  calme.

Ah, mais oui... il y avait un charme de Sourdinam lancé sur le dossier de son lit, contrôlable par un petit curseur glissant, qui était la seule raison pour laquelle quiconque pouvait jamais espérer s'endormir à Serdaigle.

Harry se redressa et regarda autour de lui, s'attendant à voir les autres se lever -

Le dortoir, vide.

Les lits, froissés et défaits.

Le soleil, entrant dans la chambre depuis un angle plutôt élevé.

Et son réveil-matin mécanique toujours en marche, mais avec l'alarme désactivée.

On l'avait apparemment autorisé à dormir jusqu'à 9h52 du matin. En dépit de ses meilleurs efforts pour synchroniser sont rythme de sommeil de 26 heures avec l'arrivée à Poudlard, il n'avait réussi à s'endormir qu'aux alentours de 1h du matin. Il avait prévu de se lever à 7h du matin avec les autres élèves ; il pouvait supporter un léger manque de sommeil le premier jour, du moment qu'il recevait un fortifiant magique quelconque avant le lendemain. Mais maintenant il avait raté le petit déjeuner. Et son premier cours à Poudlard, Herbologie, avait débuté une heure et vingt-deux minutes auparavant.

La colère se réveillait en lui, lentement, très lentement. Oh, quel bon petit tour. Éteindre son alarme. Augmenter le Sourdinam. Et laisser M. Gros Bonnet Harry Potter rater son premier cours et être réprimandé pour son lourd sommeil.

Quand Harry découvrirait qui avait fait ça...

Non, ça n'aurait pu être fait qu'avec la coopération des douze autres garçons du dortoir de Serdaigle. Ils avaient tous vu sa forme endormie sous les draps. Ils l'avaient tous laissé dormir jusqu'à après le petit déjeuner.

La colère s'écoula hors de lui et fut remplacée par de la confusion et par l'horrible sentiment d'avoir été blessé. Ils l'\emph{aimaient} . Avait-il cru. La nuit dernière, il pensait qu'ils l'aimaient. \emph{Pourquoi...} 

Alors que Harry se levait de son lit, il aperçut un bout de papier attaché au dossier de son lit, face vers l'extérieur.

Le papier disait :

\emph{Mes chers Serdaigles,} 

\emph{Ce fut une journée particulièrement longue. Merci de me laisser dormir tard et ne vous en faites pas pour mon petit déjeuner. Je n'ai pas oublié le premier cours.} 

\emph{Bien à vous,} 

\emph{Harry Potter.} 

Et Harry se tint là, figé, tandis que de l'eau glacée commençait à s'écouler le long de ses veines.

Le papier portait son écriture, tracée avec son critérium.

Et il ne se souvenait pas l'avoir écrit.

Et... Harry plissa les yeux pour mieux voir le papier. Et à moins qu'il ne soit en train de se l'imaginer, les mots "Je n'ai pas oublié" étaient écrits d'une façon différente, comme si il essayait de se dire quelque chose... ?

Avait-il \emph{su}  qu'il allait être Oublietté ? Était-il resté debout tard, avait-il commis un crime quelconque ou exercé une activité secrète, avant de... mais il ne \emph{connaissait}  pas le sort Oubliettes... quelqu'un d'autre avait-il... hein...

Une pensée lui vint. Si il \emph{avait}  su qu'il allait être Oublietté...

Toujours en pyjama, Harry fit le tour de lit, parvint à sa malle, appuya son pouce contre le loquet, récupéra sa bourse, y plongea sa main et dit : "Note à moi-même."

Et un autre morceau de papier apparut dans sa main.

Harry le prit et le fixa. Lui aussi portait son écriture.

La note disait :

\emph{Cher Moi,} 

\emph{S'il te plaît, joue à ce jeu. Tu ne peux y jouer qu'une seule fois dans ta vie. C'est une opportunité irremplaçable.} 

\emph{Code de reconnaissance 927, je suis une pomme de terre.} 

\emph{Bien à toi,} 

\emph{Toi.} 

Harry hocha lentement la tête. "Code de reconnaissance 927, je suis une pomme de terre" était en effet le message qu'il avait mis au point - quelque années plus tôt, alors qu'il regardait la télévision - de façon à ce que seul lui le connaisse. Au cas où il devrait déterminer si une copie de lui-même était vraiment \emph{lui} , ou quelque chose du genre. Juste au cas où. Soyez prêts.

Harry ne pouvait pas faire \emph{confiance}  au message, il aurait pu y avoir d'autres sorts impliqués. Mais ça éliminait la possibilité d'une simple plaisanterie. Il avait certainement écrit cela ; et il ne se rappelait certainement pas l'avoir écrit.

En regardant le papier, Harry avisa de l'encre visible au travers de la feuille.

Il la retourna.

L'autre côté disait :


\begin{center}\emph{INSTRUCTIONS POUR LE JEU} \end{center}



\begin{center}\emph{tu ne connais pas les règles du jeu} \end{center}



\begin{center}\emph{tu ne sais pas ce qui est en jeu} \end{center}



\begin{center}\emph{tu ne connais pas le but du jeu} \end{center}



\begin{center}\emph{tu ne sais pas qui contrôle le jeu} \end{center}



\begin{center}\emph{tu ne sais pas comment se termine le jeu} \end{center}



\begin{center}\emph{Tu démarres avec 100 points.} \end{center}



\begin{center}\emph{Commence.} \end{center}


Harry regarda longuement les "instructions". Ce côté n'était pas manuscrit : l'écriture était parfaitement régulière, et donc artificielle. On aurait dit que le message avait été écrit par une Plume à Paroles, comme celle qu'il avait achetée pour prendre dictée.

Il n'avait \emph{absolument aucune idée}  de ce qui était en train de se passer.

Bon... l'étape numéro un était de s'habiller et de manger. Et peut-être d'inverser l'ordre de ces actions. Son estomac lui paraissait plutôt vide.

Il avait raté le petit déjeuner, bien sûr, mais il était Prêt à cette éventualité car il l'avait visualisée à l'avance. Harry mit sa main dans sa bourse et dit "Barre énergétique", s'attendant à obtenir la boîte de barres énergétiques qu'il avait achetées avant de partir à Poudlard.

Ce qui apparut n'avait pas la consistance d'une boîte de barres énergétiques.

Lorsque Harry ramena sa main à l'intérieur de son champ de vision, il vit deux petites barres en sucre - loin d'être suffisantes pour un repas - attachées à une note, et la note était écrite de la même écriture que les instructions du jeu.

La note disait :

TENTATIVES ECHOUEES : -1 POINT

POINTS ACTUELS : 99

ETAT PHYSIQUE : ENCORE FAIM

ETAT MENTAL : CONFUS

"Gleehhhhh" dit la bouche de Harry sans qu'aucune forme d'intervention ou de décision consciente de sa part n'ai lieu.

Il resta là environ une minute.

Une minute plus tard, ça n'avait \emph{toujours}  aucun sens et il n'avait \emph{toujours}  aucune idée de ce qui se passait et son cerveau n'avait même pas \emph{commencé}  à s'accrocher à la moindre \emph{hypothèse} , comme si ses mains mentales étaient encastrées dans des balles en caoutchouc et qu'il ne pouvait rien saisir.

Son estomac, qui avait ses propres priorités, suggéra un petit test expérimental.

"Ah..." dit Harry à la pièce vide. "J'imagine que je ne pourrais pas dépenser un point et récupérer ma boîte de barres énergétiques ?"

Il n'y eut que du silence.

Harry mit sa main dans la bourse et dit : "Boîtes de barres énergétiques."

Une boîte qui semblait avoir la bonne forme apparut dans sa main...mais elle était trop légère, et elle était ouverte, et elle était vide, et la note qui y était attachée disait :

POINTS DEPENSES : 1

POINTS ACTUELS : 98

TU AS OBTENU : UNE BOITE DE BARRES ENERGETIQUES

"J'aimerais dépenser un point et \emph{vraiment}  obtenir les \emph{barres énergétiques} ," dit Harry.

Silence à nouveau.

Harry plaça sa main dans la bourse et dit "barres énergétiques."

Rien ne vint.

Harry haussa les épaules avec désespoir et se rendit au cabinet de toilettes qu'on lui avait attribué, situé près de son lit, afin de prendre ses robes de sorcier du jour.

Sur le sol du cabinet, sous ses robes, se trouvaient les barres, et une note :

POINTS DEPENSES : 1

POINTS ACTUELS : 97

TU AS OBTENU : 6 BARRES DE GOUTER

TU PORTES ENCORE : PYJAMAS

NE MANGE PAS ALORS QUE TU PORTES TES PYJAMAS

TU AURAS UNE PENALITE PYJAMAS

\emph{Et maintenant je sais que celui ou celle qui contrôle ce jeu est dingue} .

"Je devine que ce jeu est contrôlé par Dumbledore," dit Harry haut et fort. Peut-être que \emph{cette fois}  il pouvait établir un nouveau record de vitesse de compréhension.

Silence.

Mais Harry commençait à comprendre la méthode ; la note serait au prochain endroit où il regarderait. Alors Harry regarda sous son lit.

HA ! HA HA HA HA HA !

HA HA HA HA HA HA !

HA ! HA ! HA ! HA ! HA ! HA !

DUMBLEDORE NE CONTROLE PAS LE JEU

MAUVAISE SUPPOSITION

TRES MAUVAISE SUPPOSITION

-20 POINTS

ET TU PORTES ENCORE TES PYJAMAS

C'EST TON QUATRIEME COUP

ET TU PORTES ENCORE TES PYJAMAS

PENALITE PYJAMAS : -2 POINTS

POINTS ACTUELS : 75

Ouaip, Harry était foutu. C'était son premier jour d'école et si on éliminait Dumbledore, il ne connaissait personne qui puisse être aussi fou que ça.

Son corps plus ou moins en pilote automatique, Harry assembla un ensemble de robes et de sous-vêtements, ouvrit le niveau caverne de sa malle (il était quelqu'un de très pudique et quelqu'un aurait pu entrer dans le dortoir), s'habilla, et remonta les escaliers pour ranger ses pyjamas.

Harry marqua une pause avant d'ouvrir le tiroir du cabinet qui contenait ses pyjamas. Si la méthode fonctionnait encore...

"Comment puis-je gagner plus de points ?" dit Harry haut et fort.

Puis il tira le tiroir.

LES OPPORTUNITES DE FAIRE LE BIEN SONT PARTOUT

MAIS LES TENEBRES SONT LA OU LA LUMIERE DEVRAIT ETRE

COUT DE LA QUESTION : 1 POINT

POINTS ACTUELS: 74

JOLIS SOUS-VETEMENTS

C'EST TA MERE QUI LES A CHOISIS ?

Harry broya la note dans sa main, le visage écarlate et brûlant. L'injure de Draco lui revint. \emph{Fils de sang-de-bourbe}  -

Il en savait à présent assez pour ne pas le dire tout haut. Il recevrait probablement une Pénalité de Profanité.

Harry s'équipa de sa bourse en peau de Moke et de sa baguette. Il éplucha l'emballage d'une de ses barres énergétiques et le jeta dans la corbeille de la chambre, où il tomba par-dessus une Grenouille en Chocolat à peine entamée, une enveloppe froissée et du papier d'emballage rouge et vert. Il mit les autres barres dans sa bourse.

Il balaya le lieu du regard dans une tentative ultime, désespérée, et en définitive futile, de trouver des indices.

Puis Harry quitta le dortoir en mangeant, à la recherche des donjons de Serpentard. Du moins c'est ce à quoi il \emph{pensait}  que la note faisait allusion.

Essayer de naviguer les couloirs de Poudlard était comme... probablement \emph{pas}  aussi terrible que de se promener dans une peinture d'Escher, c'était le genre de chose que vous disiez pour l'effet rhétorique plutôt que parce que c'était vrai.

Peu de temps après, Harry se dit qu'en fait, une peinture d'Escher aurait des avantages et des inconvénients comparé à Poudlard. Inconvénients : pas de d'orientation cohérente de la gravitation. Avantages : au moins les escalier ne bougeaient pas \emph{PENDANT QU'ON ÉTAIT ENCORE DESSUS} .

Harry avait initialement grimpé quatre escaliers pour atteindre son dortoir. Après avoir descendu pas moins de douze escaliers sans arriver en vue des donjons, Harry avait conclut que (1) une peinture d'Escher serait \emph{du gâteau}  en comparaison, (2) il était incroyablement \emph{plus haut}  dans le château que lorsqu'il était parti, et (3) il était si \emph{parfaitement}  perdu qu'il n'aurait pas été surpris si, en regardant par la prochaine fenêtre, il avait vu deux lunes dans le ciel.

Le plan de secours A avait été de s'arrêter et de demander son chemin, mais il semblait y avoir une pénurie aigüe de promeneurs, comme si ces gueux assistaient aux cours comme ils étaient censés le faire, ou quelque chose du genre.

Plan de secours B...

"Je suis perdu," dit Harry haut et fort. "Le, euh, l'esprit de Poudlard pourrait-il m'aider ?"

"Je ne pense pas que ce château ait un esprit," remarqua une vieille femme desséchée depuis l'un des portraits placés sur le mur. "Une vie, peut-être, mais pas un esprit."

Il y eut une brève pause.

"Êtes-vous -" dit Harry, puis il se la ferma. À bien y réfléchir, non, il n'allait PAS demander au portrait si elle était pleinement consciente, au sens d'être conscient de sa propre conscience.

"Je suis Harry Potter," dit sa bouche, plus ou moins en pilote automatique. Et plus ou moins en pilote automatique, Harry tendit sa main au tableau.

La femme dans le tableau baissa la yeux vers la main et Harry et leva les sourcils.

La main redescendit lentement jusqu'au flanc de Harry.

"Désolé," dit Harry, "je suis un peu nouveau ici."

"J'avais remarqué, jeune aigle. Où essayez-vous d'aller ?"

Harry hésita. "Je ne suis pas vraiment sûr," dit-il.

"Alors peut-être y êtes-vous déjà."

"Ben, quel que soit l'endroit où \emph{j'essaie}  d'aller, je ne pense pas que ce soit \emph{ici} ..." Harry se la ferma, se rendant soudain compte qu'il avait vraiment l'air d'un idiot. "Laissez-moi réessayer. Je joue à ce jeu dont je ne connais pas les règles -" Ça ne marchait pas vraiment non plus. "Bon, troisième essai. Je recherche des opportunités de faire le bien pour gagner des points, et tout ce que j'ai c'est cet indice sibyllin parlant des ténèbres se trouvant là où la lumière devrait être, alors j'essayais de descendre mais il semble qu'au lieu de ça je vais vers le haut..."

La vieille dame dans le portrait le regardait d'un air plutôt sceptique.

Harry soupira. "Ma vie tend à être un peu curieuse."

"Serait-il correct de dire que vous ne savez pas où vous essayez d'aller ni même pourquoi vous essayez d'y aller ?"

"\emph{Entièrement}  correct."

La vieille femme hocha la tête. "Je ne suis pas sûre qu'être perdu dans le château soit le plus important de vos problèmes, jeune homme."

"Vrai, mais à la différence de mes problèmes les plus importants, c'est un problème que je peux apprendre à résoudre et \emph{waoh}  cette conversation est en train de devenir une métaphore de l'existence humaine, je ne m'en étais pas rendu compte jusqu'à cet instant."

La dame observa Harry avec appréciation. "Vous \emph{êtes}  un excellent jeune aigle. Pendant un instant j'ai commencé douter. Eh bien, dans ce cas, en règle générale, si vous ne faites que tourner à gauche, vous finirez forcément par descendre."

Ça avait l'air familier, mais Harry n'arrivait pas à se souvenir où il avait entendu ça auparavant. "Euh... vous semblez être quelqu'un de très intelligent. Ou l'image de quelqu'un de très intelligent... quoi qu'il en soit, avez-vous entendu parler d'un mystérieux jeu auquel on ne peut jouer qu'un fois et dont on ne vous dira pas les règles ?"

"La vie," répondit immédiatement la dame. "C'est une des énigmes les plus simples que j'ai jamais entendu."

Harry cligna des yeux. "Non," dit-il lentement. "Je veux dire que j'ai eu une vraie note et tout ça, disant que je devais jouer au jeu mais qu'on ne me dirait pas les règles, et quelqu'un me laisse ces petits bouts de papier me disant combien de points j'ai perdu pour avoir brisé les règles, une pénalité de moins deux points pour port de pyjamas par exemple. Connaissez-vous qui que ce soit ici à Poudlard qui soit assez fou et puissant pour faire quelque chose comme ça ? À part Dumbledore, bien sûr ?"

La dame soupira. "Je ne suis qu'une image, jeune homme. Je me souviens de Poudlard telle que c'était - pas tel que c'est. Tout ce que je puis vous dire, c'est que si c'était une énigme, la réponse serait que le jeu est la vie, et que bien que nous ne décidions pas de toutes les règles nous-mêmes, c'est toujours nous-mêmes qui nous décernons ou nous ôtons des points. Si ce n'est pas une énigme mais la réalité - alors je ne sais pas."

Harry s'inclina profondément devant l'image. "Merci, ma dame."

La dame lui fit une révérence. "J'aimerais pouvoir dire que je me souviendrai de vous avec grande affection," dit-elle, "mais je ne me souviendrai probablement pas du tout de vous. Adieu, Harry Potter."

Il s'inclina à nouveau en guise de réponse, et commença à descendre les escaliers les plus proches.

Quatre virages à gauche plus tard il se retrouva face à un corridor qui s'arrêtait abruptement en un large monticule de rochers - comme si il y avait eu un éboulement, sauf que les murs et le plafond étaient intacts et faits de pierre de château assez normale.

"Très bien," dit Harry au vide qui l'entourait, "Je laisse tomber. Je demande un autre indice. Comment puis-je aller là où j'ai besoin d'aller ?"

"Un indice ! Un indice, dis-tu ?"

La voix exaltée venait d'un tableau non loin, celui-ci étant le portrait d'un homme d'âge moyen portant les robes roses les plus voyantes que Harry avait jamais vues ou imaginées. Dans le portrait, il portait un vieux chapeau pointu tombant avec un poisson dessus (pas un dessin de poisson, dites-vous bien, mais un poisson).

"Oui !" dit Harry. "Un indice ! Un indice, dis-je ! Mais pas seulement \emph{n'importe quel}  indice, je recherche un indice \emph{spécifique} , c'est pour un jeu auquel je joue -"

"Oui, oui ! Un indice pour le jeu ! Vous êtes Harry Potter, n'est-ce pas ? Je suis Cornelion Flubberwalt ! Erin le Consort me l'a dit, et Lord Weaselnose lui avait dit, et ... j'oublie qui le lui avait dit à lui. Mais c'était un message que \emph{je}  devais vous donner ! \emph{Moi}  ! Personne ne s'est préoccupé de moi depuis, je ne sais pas depuis combien de temps, peut-être depuis toujours, j'ai été coincé ici dans ce satané corridor inutile - un indice ! J'ai votre indice ! Ça vous coûtera trois points ! Le voulez-vous ?"

"Oui ! Je le veux !" Harry se rendait compte qu'il devrait peut être garder son sarcasme sous contrôle mais il ne semblait pas capable de s'en empêcher.

"Les ténèbres peuvent être trouvées entre les salles d'études vertes et la classe de Métamorphose de McGonagall ! C'est l'indice ! Et bouge-toi, tu es plus lent qu'un sac d'escargot ! Moins dix points pour lenteur ! Maintenant tu as 61 points ! C'était le reste du message !"

"Merci," dit Harry. Il commençait à vraiment traîner à ce jeu. "Euh... j'imagine que vous ne savez pas d'où le message \emph{provenait}  ?"

"Il a été dit par une voix creuse qui émanait d'un trou dans l'air lui-même, un trou qui s'ouvrait sur une abysse flamboyante ! C'est ce qu'ils m'ont dit !"

Harry n'était alors plus sûr que ce soit le genre de chose au sujet de laquelle il aurait dû être sceptique ou le genre chose qu'il aurait dû prendre pour argent comptant. "Et comment puis-je trouver la démarcation entre les salles d'études vertes et la classe de Métamorphose ?"

"Faites juste demi-tour et allez à gauche, à droite, en bas, en bas, à droite, à gauche, à droite, en haut et à nouveau à gauche, vous ferez face à une grande salle d'étude verte et si vous entrez marchez tout droit jusqu'au côté opposé vous verrez un grand corridor courbe qui va à une intersection et sur le côté droit de cette intersection vous trouverez un grand couloir droit qui va à la salle de Métamorphose !" L'image de l'homme d'âge moyen s'interrompit. "Du moins les choses étaient ainsi quand \emph{j'étais}  à Poudlard. Nous \emph{sommes}  un lundi d'une année impaire, n'est-ce pas ?"

"Criterié et feuille de papium," dit Harry à sa bourse. "Euh, annule ça, critérium et feuille de papier." Il leva les yeux. "Vous pourriez répéter ça ?"

Après s'être perdu deux fois de plus, Harry eut l'impression qu'il commençait à comprendre les règles de bases de navigation des couloirs changeants de Poudlard, c'est à dire : \emph{demande ton chemin aux tableaux} . Si ça renvoyait à une leçon de vie incroyablement profonde il n'arrivait pas à deviner de laquelle il s'agissait.

La salle d'étude verte était un espace étonnamment plaisant. La lumière du soleil ruisselait depuis des fenêtres aux vitrages vert représentant des dragons dans des scènes calmes et pastorales. Il y avait des chaises qui avaient l'air extrêmement confortables et des tables qui semblaient parfaitement adaptées à l'étude en compagnie d'un à trois amis.

Harry ne pouvait pas \emph{vraiment } continuer tout droit et sortir par la porte de l'autre côté. Il y avait des \emph{rayonnages de livres}  dans le mur, et il devait y aller et lire quelques uns des titres pour ne pas perdre son droit d'être appelé un Verres. Mais il le fit rapidement, conscient de l'accusation de lenteur, puis sortir de l'autre coté.

Il descendait le "grand corridor courbe" lorsqu'il entendit le cri d'une voix de jeune garçon.

En de moments pareils, Harry avait une excuse pour sprinter à fond sans se préoccuper de conserver son énergie ou de faire les exercices d'échauffement adaptés ou de s'inquiéter de percuter quelque chose, une course frénétique et soudaine qui s'arrêta presque aussi soudainement lorsqu'il faillit dépasser un groupe de six Poufsouffles de première année...

... qui étaient blottis les uns contre les autres et avaient l'air plutôt effrayés et semblaient désespérément vouloir faire quelque chose mais sans savoir exactement quoi, ce qui avait probablement quelque chose à voir avec le groupe de cinq Serpentards plus âgés qui avaient l'air d'entourer un autre jeune garçon.

Harry devint soudain plutôt coléreux.

"\emph{Excusez moi !"}  cria Harry de toutes ses forces.

Ça n'était peut-être pas forcément nécessaire. Les gens le regardaient déjà. Mais ça permettait certainement de figer la situation.

Harry dépassa le groupe de Poufsouffles et se dirigea vers les Serpentards.

Ils le regardèrent avec des expressions allant de la rage à l'amusement à la délectation.

Une partie du cerveau de Harry hurlait, paniqué, que c'étaient des garçons bien plus grands et bien plus âgés, qui pourraient l'aplatir.

Une autre partie disait sèchement que quiconque vu pendant qu'il aplatissait le Survivant aurait une \emph{montagne}  d'ennuis, en particulier si il faisait partie d'une bande de Serpentards plus âgés et que sept Poufsouffles l'avaient vu, et que les chances qu'ils lui causent des dommages permanents face à des témoins étaient quasi nulles. La seule véritable arme que les garçons plus âgés avaient contre lui était sa propre peur, si il le permettait.

Et Harry vit que le garçon qu'ils avaient piégé était Neville Londubat.

Bien sûr.

Voilà qui réglait les choses. Harry avait décidé de s'excuser humblement auprès de Neville, et ça voulait dire que Neville était \emph{sien} , comment osaient-ils ?

Harry tendit la main, attrapa Neville par le poignet et le \emph{tira brusquement}  depuis le centre du cercle formé par les Serpentards ; le garçon, choqué, trébucha alors que Harry le tirait et presque dans le même mouvement se projetait lui-même par le passage ainsi ouvert.

Et Harry se tint au centre des Serpentards, là où Neville s'était tenu, le regard levé vers les garçons bien plus vieux, bien plus grands et bien plus forts.

"Bonjour," dit Harry. "Je suis le Survivant."

Il y eut une pause plutôt gênante. Personne ne semblait savoir comment la conversation était censée évoluer.

Les yeux de Harry glissèrent jusqu'au sol où il vit quelques livres et papiers éparpillés. Oh, le vieux jeu où vous laissiez le garçon essayer de ramasser ses livres puis les faisiez tomber à nouveau. Harry ne se souvenait pas avoir été lui-même l'objet de ce jeu, mais il avait une bonne imagination et cette imagination le rendait furieux. Eh bien lorsque la situation serait réglée ce serait assez simple pour Neville de revenir et de ramasser ses affaires, du moment que les Serpentards restaient trop concentrés sur Harry pour faire quoi que ce soit aux livres.

Malheureusement, on avait remarqué que ses yeux s'étaient égarés. "Ooh," dit le plus grand des garçons, "on voulait les p'tits bouquins -"

"La ferme," dit Harry froidement. \emph{Garde-les déséquilibrés. Ne fais pas ce à quoi ils s'attendent. N'ai pas un comportement qui les encourage à te malmener. } "Cela fait-il partie d'un plan incroyablement malin vous permettant d'obtenir un avantage futur, ou est-ce autant une inutile disgrâce du nom de Salazar Serpentard que ça en a -"

Le plus grand des garçons poussa Harry Potter avec force, et ce dernier s'étala hors du cercle de Serpentards sur le dur sol de Poudlard.

Et les Serpentards rirent.

Harry se leva dans un mouvement qui lui sembla être horriblement lent. Il ne savait pas encore comment utiliser sa baguette, mais il n'y avait aucune raison de laisser cela l'arrêter, au vu des circonstances.

"Je voudrais payer \emph{autant de points que nécessaire}  pour me débarrasser de cette personne," dit Harry, pointant son doigt vers le plus grand des Serpentards.

Puis Harry leva son autre main, dit "Abracadabra," et claqua des doigts.

Au son d'\emph{Abracadabra} , deux des Poufsouffles hurlèrent, y compris Neville, trois autres Serpentards se jetèrent désespérément loin de la direction vers laquelle pointait le doigt de Harry, et le plus grand des Serpentards fit un pas chancelant en arrière avec un air choqué. Une grande éclaboussure rouge décorait son visage, son cou et sa poitrine.

Harry ne s'était \emph{pas}  attendu à ça.

Lentement, le plus grand des Serpentards porta la main à sa tête et décolla le plat de tarte à la cerise dont il venait d'être drapé. Il tint le plat dans sa main un moment en le regardant, puis il le laissa tomber au sol.

Ce n'était probablement pas le meilleur moment possible pour qu'un des Poufsouffles commence à rire, mais c'est exactement ce qu'un des Poufsouffles était en train de faire.

Puis Harry remarqua la note en-dessous du plat.

"Attends," dit Harry, et il s'élança pour récupérer la note. "Cette note est pour moi je pense -"

"\emph{Toi,} " grogna le plus grand des Serpentards, "\emph{toi, tu, vas -"} 

"\emph{Regardez} -moi ça !" cria Harry, brandissant la note devant le Serpentard plus âgé. "Franchement, \emph{regardez}  ça ! Pouvez-vous croire que je doive payer 30 points pour la livraison et l'utilisation d'une pauvre tarte ? 30 points ! J'y perds, même après avoir secouru un innocent en détresse ! Et une facture de stockage ? Des coûts de transport ? Des frais logistiques ? Comment peut-on payer des \emph{frais logistiques}  pour une \emph{tarte}  ?"

Il y eut une de ces pauses gênantes. Harry eut des pensées mortelles envers celui des Poufsouffles qui ne pouvait s'empêcher de glousser, cet idiot allait le mettre dans le pétrin.

Harry fit un pas en arrière et jeta son meilleur regard mortel aux Serpentards. "Maintenant partez ou je continuerai à rendre votre existence de plus en plus surréaliste jusqu'à ce que vous vous exécutiez. Laissez-moi vous prévenir... que se frotter à \emph{ma}  vie aura tendance à rendre la \emph{vôtre} ... \emph{légèrement épouvantable} . Vu ?"

Dans un terrible mouvement, le plus grand des Serpentards fit jaillir sa baguette et la pointa vers Harry, et au même moment une nouvelle tarte le frappait sur la tête, celle-ci à la myrtille.

La note sur cette tarte était plutôt grande et clairement lisible. "Tu devrais peut-être lire la note sur cette tarte," remarqua Harry. "Je pense que c'est pour toi cette fois-ci."

Le Serpentard leva lentement la main, regarda le plat à tarte, le retourna dans bruit de succion collant qui fit tomber encore plus de myrtille par terre, et lut une note qui disait :


\begin{center}\MakeUppercase{AVERTISSEMENT}\end{center}



\begin{center}\MakeUppercase{AUCUNE} MAGIE NE SERA UTILISEE SUR LE CONCURRENT\end{center}



\begin{center}PENDANT QUE LE JEU EST EN COURS\end{center}



\begin{center}TOUTE AUTRE INTERFERENCE AU JEU\end{center}



\begin{center}\MakeUppercase{SERA} REPORTEE AUX AUTORITES DU JEU\end{center}


L'expression de pure perplexité sur le visage du Serpentard était un chef d'œuvre. Harry songea qu'il commençait peut-être à aimer le Contrôleur du Jeu.

"Écoute," dit Harry, "tu veux qu'on s'arrête là ? Je pense que les choses commencent à échapper à notre contrôle par ici. Et si tu retournais à Serpentard et que je retournais à Serdaigle et que nous nous calmions tous un peu, d'accord ?"

"J'ai une meilleure idée," siffla le plus grand des Serpentards. "Et si tu te cassais accidentellement tous tes doigts ?"

"Comment, au nom de Merlin, pourrais-tu mettre en scène un accident crédible après avoir fait cette menace devant une douzaine de personnes, espèce \emph{d'idiot}  -"

Le plus grand des Serpentards tendit sa main vers celle de Harry, lentement, délibérément, et Harry se figea, le partie de son cerveau qui s'était rendue compte de l'âge et de la force du garçon parvenant enfin à se faire entendre, criant : \emph{MAIS QU'EST CE QUE JE SUIS EN TRAIN DE FAIRE ?} 

"Attends !" dit l'un des autres Serpentards, sa voix soudain paniquée. "Arrête, tu ne devrais pas le faire pour de vrai !"

Le plus grand des Serpentards l'ignora, prit fermement la main droite de Harry dans sa main gauche, et prit l'index de Harry dans sa main droite.

Harry regarda le Serpentard droit dans les yeux. Une partie de Harry hurlait que ce n'était pas censé avoir lieu, que ce n'était pas permis, que les adultes ne laisseraient jamais une chose pareille arriver \emph{pour de vrai}  -

Lentement, le Serpentard commença à tordre son doigt en arrière.

\emph{Il n'a pas encore vraiment cassé mon doigt et c'est indigne de moi de ne serait-ce que tressaillir avant qu'il le fasse. Jusque là, ce n'est qu'une autre tentative pour provoquer la peur.} 

"Arrête !" dit le Serpentard qui avait auparavant protesté. "Arrête, c'est une très mauvaise idée !"

"Je suis plutôt d'accord," dit une voix de glace. La voix d'une femme plus âgée.

Le plus grand des Serpentard relâcha la main de Harry et fit un bond en arrière comme si elle était brûlante.

"Professeur Chourave !" s'écria l'un des Poufsouffle, d'un ton plus heureux que Harry n'avait jamais entendu de sa vie.

Alors qu'il pivotait, une petite femme boulotte se faufila dans son champ de vision. Elle avait des cheveux gris hirsute et bouclés, et ses vêtements étaient couverts de poussière. Elle pointa un doigt accusateur en direction des Serpentards. "Expliquez-vous," dit-elle. "Que faites-vous avec mes Poufsouffles et..." elle le regarda, "mon excellent étudiant, Harry Potter."

\emph{Oh oh. C'est vrai ça, c'était SA classe qu'il avait ratée ce matin.} 

"Il a menacé de nous tuer !" lâcha l'un des autres Serpentard, celui qui avait demandé à ce qu'ils s'arrêtent.

"Quoi ?" dit Harry, le visage vide d'expression. "Certainement \emph{pas}  ! Si je comptais vous tuer je ne commencerais pas par faire des menaces publiques !"

Un troisième Serpentard ne put s'empêcher de rire puis s'arrêta de façon abrupte lorsque les autres garçons lui jetèrent des regards mortels.

Le Professeur Chourave avait adopté une expression plutôt sceptique. "Et de quelle menace mortelle s'agirait-il exactement ?"

"Le Sort de Mort ! Il a feint d'utiliser le Sort de Mort sur nous !"

Le Professeur Chourave se tourna et regarda Harry. "Oui, une menace vraiment terrible venant d'un garçon de onze ans. Mais tout de même pas quelque chose que vous devriez \emph{jamais } rêver de feindre, Harry Potter."

"Je ne connais même pas les \emph{mots}  du Sort de Mort," dit promptement Harry. "Et je n'ai sorti ma baguette à aucun moment."

C'était maintenant à Harry que le Professeur Chourave jetait un regard sceptique. "J'imagine que ce garçon s'est jeté deux tartes sur \emph{lui-même}  alors."

"Il n'a \emph{pas}  utilisé sa baguette !" lâcha l'un des jeunes Poufsouffles. "Je ne sais pas non plus comment il a fait, il a claqué des doigts, et il y avait une tarte !"

"Vraiment," dit le Professeur Chourave après une pause. Elle tira sa propre baguette. "Je ne l'exigerai pas vu que vous semblez être la victime, mais accepteriez-vous que j'examine votre baguette pour vérifier ça ?"

Harry sortit sa baguette. "Qu'est ce que je -"

"\emph{Priori Incantatem} ," dit Chourave. Elle fronça les sourcils. "C'est étrange, votre baguette semble n'avoir jamais été utilisée."

Harry haussa les épaules. "C'est le cas à vrai dire, je n'ai eu ma baguette et mes manuels qu'il y a quelques jours."

Chourave hocha la tête. "Alors nous avons un clair cas de magie accidentelle de la part d'un garçon qui se sentait menacé. Et les règles disent clairement que vous ne serez pas tenu pour responsable. En ce qui \emph{vous } concerne..." elle se tourna vers les Serpentards. Ses yeux descendirent délibérément vers les livres de Neville, étalés au sol.

Il y eut un long silence pendant lequel elle regarda les cinq Serpentards.

"Trois points ôtés de Serpentard, \emph{chacun} ," dit-elle enfin. "Et six de \emph{lui} ," pointant le garçon couvert de tarte. "Ne touchez plus \emph{jamais}  à mes Poufsouffles, ni à mon étudiant Harry Potter. Maintenant \emph{partez} ."

Elle n'eut pas à se répéter ; les Serpentards firent demi-tour et s'en furent très rapidement.

Neville alla ramasser ses livres. Il semblait pleurer mais un petit peu seulement. Ç'aurait pu être l'effet différé du choc, ou ç'aurait pu être parce que les autres garçons l'aidaient.

"Merci \emph{beaucoup} , Harry Potter," lui dit le Professeur Chourave. "Sept points à Serdaigle, un pour chaque Poufsouffle que vous avez protégé. Et je ne dirai rien de plus."

Harry cligna des yeux. Il s'était attendu à quelque chose ressemblant à une conférence sur l'importance de rester à l'écart des ennuis, et une réprimande plutôt sévère pour avoir raté son tout premier cours.

Peut-être qu'il aurait \emph{dû}  aller à Poufsouffle. Chourave était cool.

"\emph{Récurvite} ," dit Chourave au fatras de tartes qui était sur le sol, qui disparut promptement.

Et elle partit, marchant le long du couloir qui menait à la salle d'étude verte.

"Comment as-tu \emph{fait}  ça ?" siffla l'un des garçons de Poufsouffle dès qu'elle était partie.

Harry sourit avec suffisance. "Je peux faire survenir tout ce que je veux juste en claquant des doigts."

Les yeux du garçon s'agrandirent. "\emph{Vraiment ?} "

"Non," dit Harry. "Mais quand vous raconterez cette histoire à tout le monde, assurez-vous de la partager avec Hermione Granger, en première année à Serdaigle, elle a une anecdote que vous trouverez amusante." Il n'avait aucune idée de ce qui se passait, mais il n'allait pas laisser passer une opportunité de contribuer à sa légende grimpante. "Oh, et qu'est ce que c'était que cette histoire à propos du Sort de Mort ?"

Le garçon lui jeta un étrange regard. "Tu ne sais vraiment pas ?"

"Si je le savais je ne poserais pas la question."

"Les mots pour le Sort de Mort sont," le garçon avala sa salive, et sa voix devint un murmure, et il tint ses mains loin de ses flancs comme pour rendre très clair le fait qu'il ne tenait pas de baguette, "\emph{Avada Kedavra} ."

\emph{Évidemment.} 

Harry ajouta cela à sa liste croissante de choses à ne jamais dire Papa, le Professeur Michael Verres-Evans. C'était déjà assez de parler du fait que vous étiez la seule personne à avoir survécu au terrible Sort de Mort sans avoir à admettre que le Sort de Mort était "Abracadabra."

"Je vois," dit Harry après une pause. "Eh bien c'est la dernière fois que je dis \emph{ça}  avant de claquer des doigts." Bien que ça \emph{ait}  produit un effet qui pourrait être tactiquement utile.

"\emph{Pourquoi}  as-tu -"

"Éduqué par les Moldus. Les Moldus pensent que c'est une blague et que c'est drôle. Je suis sérieux, c'est ce qui s'est passé. Désolé mais pourrais-tu me rappeler ton nom ?"

"Je suis Ernie Macmillan," dit le Poufsouffle. Il tendit sa main, et Harry la serra. "Honoré de te rencontrer."

Harry s'inclina légèrement. "Ravi de te rencontrer, oublie les 'honoré' et autres."

Puis les autres garçons firent foule autour de lui et il y eut un déluge de présentations.

Lorsqu'ils en eurent fini, Harry avala sa salive. Ça allait être très difficile. "Euh... si vous voulez bien m'excuser... j'ai quelque chose à dire à Neville -"

Tous les yeux se tournèrent vers Neville, qui fit un pas en arrière, l'air appréhensif.

"J'imagine," dit Neville d'une petite voix, "que tu vas dire que j'aurais dû être plus brave -"

"Oh, non, rien de ce genre !" dit hâtivement Harry. "Rien à voir avec \emph{ça} . C'est juste, euh, quelque chose que le Choixpeau Magique m'a dit -"

Soudain les autres garçons eurent l'air \emph{très}  intéressés, mis à part Neville, qui avait l'air encore \emph{plus}  appréhensif.

Il semblait y avoir quelque chose bloquant la gorge de Harry. Il savait qu'il devrait juste le dire, mais c'était comme si il avait avalé une grande brique qui bouchait maintenant le passage.

C'était comme si Harry devait prendre manuellement le contrôle de ses lèvres et produire chaque syllabe individuellement, mais il parvint à le dire. "Je, suis, dés,olé." Il exhala et prit une profonde inspiration. "Pour ce que j'ai fait, euh, l'autre jour. Tu... tu n'a pas à être chic ou quoi que ce soit. Je comprendrai si tu me détestes. Je ne suis pas en train d'essayer d'avoir l'air cool en m'excusant ou de te forcer à accepter mon excuse. Ce que j'ai fait était mal."

Il y eut une pause.

Neville serra ses livres contre sa poitrine. "Pourquoi as-tu fait ça ?" dit-il d'une voix fluette et tremblante. Il cligna comme pour retenir des larmes. "Pourquoi est-ce que \emph{tout le monde}  me fait ça, même le Survivant ?"

Harry se sentit soudain plus petit qu'il ne s'était jamais senti. "Je suis désolé," dit Harry à nouveau, sa voix maintenant enrouée. "C'est juste que... tu avais l'air \emph{tellement}  effrayé, c'était comme un signe au-dessus de ta tête disant 'victime', et je voulais te montrer que les choses ne tournaient pas \emph{toujours}  mal, que parfois les monstres donnent du chocolat... je me suis dit que si je te montrais ça, tu te rendrais compte qu'il n'y avait pas tellement de quoi avoir peur -"

"Mais il y \emph{a}  de quoi avoir peur," chuchota Neville, "tu l'as vu aujourd'hui !"

"Ils n'auraient rien fait de mal devant des témoins. Leur arme principale est la peur. C'est pour ça qu'ils \emph{te}  prennent pour cible, parce qu'ils peuvent voir que tu as peur. Je voulais que tu aies moins peur... te montrer que la peur est pire que son objet... ou c'est ce que je me suis dit, mais le Choixpeau Magique m'a dit que je me mentais à moi-même et que j'avais fait ça parce que c'était amusant. Donc c'est pour ça que je m'excuse -"

"Tu m'as fait mal," dit Neville. "A l'instant. Quand tu m'as attrapé et m'a tiré loin d'eux." Neville tendit son bras et indiqua l'endroit où Harry l'avait saisi. "J'aurai peut-être un bleu ici plus tard, tellement tu as tiré fort. En fait, tu m'as fait plus mal qu'aucun des Serpentards ne l'avait fait en me poussant."

"\emph{Neville !} " siffla Ernie. "Il essayait de te \emph{sauver } !"

"Je suis désolé," murmura Harry. "Quand j'ai vu ça je me suis juste... vraiment mis en colère..."

Neville le regarda calmement. "Alors tu m'as éjecté avec force et tu t'es mis là où j'étais et tu as dit 'Bonjour, je suis le Survivant'."

Harry acquiesça.

"Je pense que tu seras vraiment cool un jour," dit Neville. "Mais pour l'instant tu ne l'es pas."

Harry avala le nœud soudain apparu dans sa gorge et s'en fut. Il continua le long du corridor jusqu'à l'intersection, puis tourna à gauche dans un couloir et continua de marcher aveuglément.

Qu'était-il \emph{censé}  faire ? Ne jamais se mettre en colère ? Il n'était pas certain qu'il aurait pu faire quoi que ce soit sans se mettre en colère, et alors qui sait ce qui serait arrivé à Neville et à ses livres. Et puis, Harry avait lu assez de livres de fantasy pour savoir comment \emph{ça}  se terminait. Il essaierait de réprimer sa colère, et il échouerait et elle continuerait de jaillir. Et après ce long voyage de découverte de soi il apprendrait à la fin que sa colère était une partie de lui et que c'était seulement en l'acceptant qu'il pourrait apprendre à l'utiliser avec sagesse. \emph{Star Wars}  était le seul univers dans lequel la réponse \emph{était}  que vous deviez vraiment vous séparer de toute émotion négative, et quelque chose chez Yoda avait toujours poussé Harry à haïr le petit crétin vert.

Donc le plan qui faisait clairement gagner du temps était de zapper le voyage de découverte de soi et d'aller directement au moment où il se rendait compte qu'en acceptant que sa colère était une partie de lui il pourrait garder le contrôle.

Le problème, c'était qu'il ne se \emph{sentait}  pas perdre le contrôle quand il était en colère. La rage froide lui donnait le sentiment qu'il était \emph{en pleine possession de ses moyens} . Ce n'est que lorsqu'il revenait sur les événements \emph{dans leur ensemble}  qu'ils semblaient avoir...échappé à son contrôle, de façon incompréhensible.

Il se demanda quelle importance le Contrôleur de Jeu attachait à ce genre de chose, et si ça lui avait fait perdre ou gagner des points. Harry sentait qu'il avait perdu pas mal de points, et il était certain que la vieille dame dans le tableau lui aurait dit qu'à ce sujet, la seule opinion qui comptait était la sienne.

Et Harry se demandait aussi si le Contrôleur du Jeu avait envoyé le Professeur Chourave. C'était logique : la note avait menacé d'avertir les autorités du Jeu, et le Professeur Chourave était arrivé. Peut-être que le Professeur Chourave \emph{était}  le Contrôleur du Jeu, la \emph{Directrice de la Maison Poufsouffle}  était la \emph{dernière}  personne que quiconque aurait soupçonne, ce qui devait la mettre presque en haut de la liste de Harry. Il avait aussi lu un ou deux mystères policiers.

"Alors, comment je m'en sors dans le jeu ?" dit Harry haut et fort.

Une feuille de papier vola au-dessus de sa tête, comme si quelqu'un l'avait jetée depuis son dos - Harry se retourna, mais il n'y avait personne ici - et quand Harry se retourna de nouveau, la note était posée au sol.

La note disait :

POINTS POUR LE STYLE : 10

POINTS POUR LA REFLEXION : - 3.000.000

BONUS DE POINTS MAISON SERDAIGLE : 70

POINTS ACTUELS : - 2.999.871

TOURS RESTANTS : 2

"\emph{Moins trois millions de points ?} " dit Harry au couloir vide d'un ton indigné. "Ça me semble excessif ! Je veux faire appel auprès des Autorités du Jeu ! Et comment puis-je regagner trois millions de points dans les deux prochains tours ?"

Une autre note vola au-dessus de sa tête.

APPEL : REJETE

POSER LES MAUVAISES QUESTIONS : - .000 POINTS

POINTS ACTUELS : - .871

TOURS RESTANTS : 1

Harry abandonna. Avec un tour restant il ne pouvait plus que donner tout ce qu'il avait, même si ça n'était pas grand chose. "Ma réponse est que le jeu représente la vie."

Une dernière feuille de papier vola au-dessus de sa tête, et il y avait écrit :


\begin{center}TENTATIVE ECHOUEE\end{center}



\begin{center}ECHOUEE ECHOUEE ECHOUEE\end{center}



\begin{center}POINTS ACTUELS : MOINS L'INFINI\end{center}



\begin{center}\MakeUppercase{TU AS PERDU LE JEU}\end{center}



\begin{center}INSTRUCTION FINALE :\end{center}



\begin{center}\emph{vas au bureau du Professeur McGonagall} \end{center}


La dernière ligne était de la main de Harry.

Harry regarda la dernière ligne un moment, puis haussa les épaules. Très bien. Ce serait le bureau du Professeur McGonagall. Si \emph{elle}  était le Contrôleur de Jeu...

D'accord, honnêtement, Harry n'avait absolument aucune idée de ce qu'il ressentirait si McGonagall était le Contrôleur de Jeu. Son esprit était totalement vide. C'était, littéralement, inimaginable.

Quelques portraits plus tard - ce n'était pas un long voyage, le bureau du Professeur McGonagall n'était pas loin de sa classe de Métamorphose, du moins pas les Lundis des années impaires - Harry se tint hors de la porte de son bureau.

Il frappa.

"Entrez," dit le Professeur d'une voix étouffée.

Il entra.

