
\chapter{Biais égocentrique}

[NdT : Ceci est une \textbf{traduction}  de \emph{Harry Potter and the Methods of Rationality} . Je ne suis pas l'auteur de cette fanfic ! Allez sur mon profil pour l'original en anglais.]
\par\noindent\rule{\textwidth}{0.4pt}
Et il faut que tu saches que, malheureusement, si tu veux découvrir ce qu'est J.K. Rowling, tu devras l'explorer toi-même.

Préambules scientifiques : Luosha fait remarquer que la théorie de l'empathie du chapitre 27 (nous utiliserions notre cerveau pour simuler ceux des autres) n'est pas exactement un fait scientifique avéré. Les données pointent dans cette direction, mais nous n'avons pas suffisamment analysé la circuiterie cérébrale pour pouvoir le prouver. De même, les formulations intemporelles de la mécanique quantique (auxquelles il a été fait allusion dans le chapitre 28) sont si élégantes que je serais choqué d'apprendre que la théorie finale inclut la notion de temps ; mais ces formulations ne sont pour le moment pas non plus établies.
\par\noindent\rule{\textwidth}{0.4pt}
Dernièrement, il y avait eu un malaise dans l'estomac de Hermione à chaque fois qu'elle avait entendu les autres élèves parler d'elle et de Harry. Alors qu'elle prenait une douche le matin même, elle avait surpris une conversation entre Morag et Padma, et ça avait été la goutte d'eau qui avait fait déborder le vase.

Elle commençait à se dire que débuter une rivalité entre elle et Harry Potter avait été une terrible erreur.

Si elle était juste restée \emph{loin}  de lui, alors elle aurait pu être Hermione Granger, la plus grande des étoiles intellectuelles de Poudlard, capable de gagner plus de points pour Serdaigle que qui que ce soit d'autre. Elle n'aurait certes pas été aussi célèbre que le Survivant, mais elle aurait pu attribuer sa renommée uniquement à \emph{elle-même} .

Au lieu de cela, le Survivant avait une rivale dans le domaine scolaire, et son nom se trouvait être Hermione Granger.

Pire que ça, elle avait eu un rendez-vous galant avec lui.

L'idée de vivre une Romance avec Harry lui avait semblé être une bonne idée au premier abord. Elle avait lu des livres de ce genre, et s'il y avait quelqu'un à Poudlard qui méritait d'être l'objet de son attention, c'était évidemment Harry Potter. Intelligent, drôle, célèbre, parfois effrayant...

Alors elle avait forcé Harry Potter à avoir un rendez-vous avec elle.

Et maintenant, c'était \emph{elle}  l'objet de \emph{son}  attention.

Ou pire, l'un des plats présents sur son menu du soir.

Elle avait été prendre une douche le matin même, et elle avait entendu des ricanements venir de l'extérieur juste avant d'allumer l'eau. Et elle avait entendu Morag dire que cette née-Moldue ne serait probablement pas assez combative pour pouvoir vaincre Ginevra Weasley, et Padma avait alors spéculé sur le fait que Harry Potter pourrait aussi décider qu'il les voulait \emph{toutes les deux} .

C'était comme si elles ne comprenaient pas que les FILLES avaient différents plats pour leur menu du soir et que les GARÇONS se battaient pour elles.

Mais pour être honnête, ce n'était pas cela qui lui faisait le plus mal. C'était surtout que lorsqu'elle obtenait un 19 à l'un des contrôles du professeur McGonagall, la grande nouvelle n'était pas que Hermione Granger avait eu la meilleure note de la classe, c'était que la rivale de Harry Potter avait eu trois points de plus que lui.

Si vous vous approchiez trop du Survivant, vous deveniez une partie de sa légende.

Vous n'aviez pas la vôtre.

Et la pensée lui était venue, celle qui disait qu'elle aurait juste dû partir loin de tout cela, mais ç'aurait été trop triste.

Elle voulait néanmoins \emph{recouvrir}  ce qu'elle avait accidentellement abandonné en se faisant connaître comme étant la rivale de Harry. Elle voulait de nouveau être une personne à part entière, pas la troisième jambe de Harry. Était-ce trop demander ?

C'était un piège d'où il était difficile de s'extirper une fois qu'on y était tombé. Peu importe que vos notes soient extraordinaires, même si vous faisiez quelque chose qui mérite une annonce spéciale au dîner cela voulait juste dire que vous n'aviez pas renoncé à rivaliser avec Harry Potter.

Mais elle pensait avoir trouvé un moyen.

Quelque chose qui ne donnerait \emph{pas}  l'impression qu'elle tirait sur l'autre extrémité de la scie de Harry.

Ce serait difficile.

Cela irait contre sa nature.

Il lui faudrait vaincre quelqu'un de très méchant.

Et elle devrait demander l'aide de quelqu'un d'encore \emph{plus}  méchant.

Hermione leva sa main, prête à frapper à cette terrible porte.

Elle hésita.

Hermione se rendit compte qu'elle se comportait de façon \emph{idiote}  et leva sa main un peu plus haut.

Elle essaya à nouveau de frapper.

Sa main resta loin de parvenir à atteindre la porte.

Et la porte s'ouvrit quand même en grand.

"Allons," dit l'araignée, lovée dans sa toile, "était-ce si difficile que ça de perdre un seul point Quirrell, mademoiselle Granger ?"

Hermione n'avait pas bougé, la main levée, les joues rosissantes. Ça \emph{avait}  été très difficile.

"Eh bien, mademoiselle Granger, j'aurais pitié de vous," dit le méchant professeur Quirrell. "Considérez qu'il est déjà perdu. Voilà, je vous ai débarrassé d'un choix difficile. N'êtes-vous pas reconnaissante ?"

"Professeur Quirrell," parvint-elle à dire d'une voix qui couinait un peu, "j'ai beaucoup de points Quirrell, n'est-ce pas ?"

"Tout à fait," répondit-il, "mais un de moins que précédemment. Atroce n'est-ce pas ? Imaginez un peu, si je n'apprécie pas la raison de votre venue, vous pourriez en perdre cinquante autres. Peut-être que je les enlèverai un... par un... par un..."

Les joues de Hermione devenaient encore plus rouges. "Vous êtes vraiment méchant, personne ne vous a jamais dit ça ?"

"Mademoiselle Granger," dit très sérieusement le professeur Quirrell, "il peut être dangereux d'offrir ce genre de compliments à des gens qui ne l'ont pas véritablement mérité. Le receveur pourrait se sentir intimidé et indigne de telles louanges, et il pourrait décider de faire quelque chose pour se hisser à leur hauteur. Alors, mademoiselle Granger, de quoi souhaitiez-vous me parler ?"
\par\noindent\rule{\textwidth}{0.4pt}
Un jeudi après-midi, peu après l'heure du déjeuner, Hermione et Harry étaient confortablement installés dans le petit recoin d'une bibliothèque, entourés d'un \emph{Sourdinam}  leur permettant de discuter librement. Harry était allongé sur le ventre, ses genoux posés au sol, tête entre les mains, ses pieds battant nonchalamment la mesure. Hermione occupait une chaise rembourrée bien trop grande pour elle, comme si elle avait été le cœur d'un bonbon fourré au chocolat.

Harry avait suggéré qu'ils commencent par lire seulement les \emph{titres}  de tous les livres de la bibliothèque, et Hermione pourrait ensuite lire les tables des matières.

Hermione avait trouvé cette idée brillante. Elle n'avait jamais fait une chose pareille dans une bibliothèque.

Malheureusement, ce plan comportait une légère faille.

À savoir : ils étaient tous les deux Serdaigle.

Hermione lisait un livre intitulé \emph{Mnémotechniques magiques} .

Harry lisait un livre intitulé \emph{Le sorcier sceptique} .

Ils s'étaient tous les deux dit que c'était une exception qu'ils feraient juste cette fois, et aucun d'eux ne s'était encore rendu compte qu'il serait impossible pour aucun d'eux de jamais finir la lecture de tous les titres, et ce quel que soit l'effort qu'ils y mettraient.

Le calme de leur petit recoin fut brisé par deux mots.

"Oh \emph{non} ," dit soudain Harry à voix haute, d'une voix qui laissait penser que les mots lui avaient été arrachés.

Puis le calme reprit.

"Il n'a pas fait \emph{ça} ," dit Harry de cette même voix.

Puis elle entendit Harry glousser de façon incontrôlable.

Hermione leva les yeux de son livre.

"Très bien," dit-elle, "qu'y a-t-\emph{il}  ?"

"Je viens de découvrir pourquoi il ne faut jamais parler à un Weasley de leur rat de famille," dit Harry. "C'est \emph{vraiment}  horrible et je ne devrais pas en rire et je suis quelqu'un d'affreux."

"Oui," répondit Hermione d'un ton guindé, "en effet. Dis moi."

"D'accord, d'abord le contexte. Il y a tout un chapitre dans ce livre sur les théories conspirationnistes qui concernent Sirius Black. Tu te souviens bien de qui il s'agit ?"

"Bien sûr," dit Hermione. Sirius Black était un traître, un ami de James Potter qui avait révélé à Voldemort l'emplacement des parents de Harry.

"Eh bien il s'avère qu'il y a eu un certain nombre de, disons d'\emph{irrégularités}  liées à la venue de Black à Azkaban. Il n'a pas eu de procès, et le sous-secrétaire d'État en poste à l'époque où les Aurors ont arrêté Black n'était autre que Cornelius Fudge, l'actuel ministre de la Magie."

Cela semblait un peu suspect à Hermione, et elle le fit savoir à Harry.

Harry fit mine de hausser les épaules, toujours allongé au sol, face à son livre. "Des choses suspectes se produisent en permanence, et si on est un adepte des théories conspirationnistes, on peut toujours trouver \emph{quelque chose} ."

"Mais \emph{pas de procès}  ?" dit Hermione.

"C'était juste après la défaite du Seigneur des Ténèbres," dit Harry d'un ton sérieux. "Les choses étaient alors chaotiques, et lorsque les Aurors remontèrent jusqu'à Black, il se tenait là, riant dans une ruelle, du sang jusqu'au chevilles, avec vingt témoins oculaires pour raconter comment il avait tué un ami de mon père nommé Peter Pettigrew ainsi que douze passants. Je ne dis pas que j'approuve le fait que Black n'ait pas eu de procès. Mais parlons de sorciers, alors ce n'est pas beaucoup plus suspect que, je ne sais pas, que le genre de chose que les gens mentionnent lorsqu'ils débattent au sujet de l'identité de l'assassin de John F. Kennedy. Enfin bref, Sirius Black est le Lee Harvey Oswald des sorciers. Il y a toutes sortes de théories sur la personne qui aurait \emph{vraiment}  trahi mes parents, et l'une des théories les plus populaires est Peter Pettigrew, et c'est là que les choses commencent à se compliquer."

Hermione écoutait, fascinée. "Mais comment tu vas de ça au \emph{rat}  \emph{de compagnie } des Weasley -"

"Attends," dit Harry, "j'y viens. Alors, après la mort de Pettigrew, il fut révélé qu'il avait été un espion pour le compte des gentils - pas un agent double, juste quelqu'un qui farfouillait et découvrait des secrets. Il avait été doué dans le domaine depuis son adolescence, et même à Poudlard il avait la réputation de pouvoir découvrir toutes sortes de secrets. La théorie est donc que Pettigrew est devenu un Animagus non déclaré quand il était encore à Poudlard, un Animagus petit, capable de courir partout et d'écouter des conversations. Le problème principal étant que les gens capables d'être des Animagus sont rares et qu'il est encore plus rare d'y parvenir étant encore adolescent, donc les théories disent bien sûr que mon père et Black étaient eux aussi des Animagus non déclarés. Et dans cette théorie conspirationniste, Pettigrew a tué les douze témoins lui-même puis il s'est transformé en sa petite forme Animagus, et il s'est enfuit. Donc Michael Shermer dit qu'il y a quatre problèmes avec cette théories. Un, Black était le seul à savoir où se trouvaient mes parents," (la voix de Harry devint un peu rauque lorsqu'il prononça cette phrase), "deux, Black était à la base un suspect bien plus probable que Pettigrew, il y a une rumeur selon laquelle Black aurait délibérément essayé de faire tuer un élève lorsqu'il était à Poudlard, et il venait de cette famille Sang Pur vraiment méchante, Bellatrix Black était littéralement sa cousine. Trois, Black était vingt fois plus fort que Pettigrew en magie de combat, même s'il n'était pas aussi malin que lui. Le duel entre eux aurait été comme un duel entre le professeur Quirrell et le professeur Chourave. Pettigrew n'aurait probablement pas eu une chance de sortir sa baguette et encore moins de falsifier toutes les preuves dont cette théorie a besoin. Et quatre, Black se tenait au milieu de la rue en \emph{riant} ."

"Mais le \emph{rat}  -" dit Hermione.

"Oui," dit Harry. "Eh bien, pour résumer, Bill Weasley a décidé que le rat de compagnie de son petit frère Percy était la forme Animagus de Pettigrew -"

La mâchoire de Hermione se décrocha.

"Oui," dit Harry, "on ne s'attendrait pas à ce que le maléfique Pettigrew vive une vie furtive et triste sous la forme du rat de compagnie d'une famille de sorciers ennemis, on l'imaginerait soit chez les Malfoy, soit plus probable encore quelque part aux Caraïbes après un peu de chirurgie esthétique. Quoi qu'il en soit, Bill met son petit frère Percy KO, puis il assomme le rat, s'en saisit, et il envoie tous ces messages d'urgences par chouette -"

"Oh \emph{non}  !" dit Hermione comme si les mots lui avaient été arrachés.

"- et parvient on ne sait comment à ameuter Dumbledore, le ministre de la Magie et le chef des Aurors -"

"Il n'a pas fait \emph{ça}  !" dit Hermione.

"Et bien sûr lorsqu'ils arrivent ils se disent qu'il est fou, mais ils jettent quand même \emph{Veritas Oculum} , juste pour être sûrs, et que découvrent-ils ?"

Elle en serait \emph{morte} . "Un rat."

"Tu gagnes un bonbon ! Alors ils ont traîné le pauvre Bill Weasley jusqu'à Sainte Mangouste, et il s'avéra que c'était une crise de schizophrénie assez standard, ça arrive à certaines personnes, en particulier aux jeunes homme lorsqu'ils approchent de l'âge où on entre habituellement à l'université. Le mec était convaincu qu'il avait quatre-vingt-dix-sept ans et qu'il était mort et qu'il avait remonté le temps jusqu'à son lui plus jeune en prenant le train. Et il répondit parfaitement bien aux neuroleptiques et il est de nouveau normal et tout va bien maintenant, sauf que les gens ne parlent plus beaucoup des théories conspirationnistes au sujet de Sirius Black et qu'on ne parle jamais aux Weasley de leur rat de compagnie."

Hermione gloussait de façon incontrôlable. C'était vraiment horrible et elle n'aurait pas dû en rire et elle était une personne affreuse.

"Ce que je ne comprends \emph{pas} ,", dit Harry après que leurs gloussements se soient éteints, "c'est \emph{pourquoi}  Black pourchasserait Pettigrew au lieu de courir aussi vite que possible. Il devait savoir que les Aurors étaient à ses trousses. Je me demande s'ils ont arraché la raison de son comportement à Black avant de l'emmener à Azkaban ? Tu vois, c'est pour ça que les gens qui sont absolument forcément coupables devraient quand même passer par le système judiciaire et avoir des procès."

Hermione dut admettre que c'était vrai.

Peu après, Harry en eut fini avec son livre, tandis que Hermione n'était encore qu'à la moitié du sien - c'était un livre bien plus difficile que celui de Harry, mais elle se sentit quand même gênée. Puis elle dut remettre \emph{Mnémotechniques magiques}  sur son étagère et se forcer à quitter la bibliothèque, car c'était l'heure de faire face aux cours le plus redouté de Poudlard, VOL SUR BALAIS.

Harry lui colla aux pieds tandis qu'elle s'y rendait même si son cours à lui ne commencerait pas avant une heure et demie, comme un avion de chasse escortant un triste petit avion à hélices en chemin vers ses funérailles.

Le garçon lui dit au revoir d'une petite voix pleine d'empathie, et elle marcha sur les champs herbeux du Destin.

Et il y eut moult glapissements et quasi-chutes et d'horribles buissons de la mort et le sol totalement au \emph{mauvais}  endroit et le soleil dans ses yeux et Morag qui la frôlait et Mandy qui croyait être \emph{subtile}  à être toujours suffisamment près d'elle pour l'attraper si elle tombait et elle \emph{savait}  que les autres élèves se moquaient d'elles mais elle ne dit jamais rien à Mandy parce qu'elle n'avait pas envie de mourir.

Dix millions d'années plus tard, le cours se termina, et elle fut de nouveau sur le sol, là où serait sa place jusqu'au jeudi suivant. Elle avait parfois des cauchemars dans lesquels on était toujours jeudi.

\emph{Pourquoi}  tout le monde devait-il apprendre à faire ça alors qu'ils allaient juste transplaner ou passer par des cheminées une fois adultes ? C'était pour Hermione un mystère total et profond. Aucun adulte n'avait vraiment besoin de voler sur des balais, c'était comme d'être obligé de jouer à la balle au prisonnier en éducation physique.

Au moins Harry avait la décence d'avoir honte de son talent dans le domaine.
\par\noindent\rule{\textwidth}{0.4pt}
Quelques heures plus tard et elle était dans une salle d'étude de Poufsouffle avec Hannah, Susan, Leanne et Megan. Le professeur Flitwick, étonnamment timide pour un enseignant, lui avait demandé si elle pourrait peut-être aider les quatre élèves pour leurs devoirs de charmes et sortilèges, même si elles n'étaient pas Serdaigles, et Hermione s'était senti si fière qu'elle en avait presque pris feu.

Elle prit une feuille de parchemin, répandit un peu d'encre à la surface de celle-ci, la déchira en quatre morceaux, les chiffonna et les jeta sur la table.

\emph{Elle}  aurait pu y arriver en se contentant de les chiffonner, mais le reste les faisait d'autant plus ressembler à des détritus, et c'était utile quand quelqu'un pratiquait le sort de nettoyage pour la première fois.

Hermione se concentra sur son ouïe et dit : "Allez, essayez."

"\emph{Everto} ."

"\emph{Everto} ."

"\emph{Everto} ."

"\emph{Everto} ."

Elle ne pensa pas avoir bien saisi toutes les erreurs. "Pourriez-vous toutes réessayer ?"

Une heure plus tard, Hermione avait conclu que (1) Leanne et Megan semblaient manquer de rigueur mais si on leur demandait de continuer à s'entraîner elles obéiraient, (2) Hannah et Susan étaient concentrées et motivées, à tel point qu'il fallait sans cesse leur dire de \emph{ralentir}  et de se \emph{détendre}  et de \emph{réfléchir}  au lieu faire autant \emph{d'efforts } - c'était étrange de se dire que ces deux-là seraient bientôt à \emph{elle}  - et (3) elle aimait aider les Poufsouffle, toute la salle d'étude avait une atmosphère très enjouée.

Lorsqu'elle se rendit au dîner, elle trouva le Survivant qui lisait un livre en attendant de l'escorter. Elle se sentit flattée, et un peu inquiète aussi, car Harry ne semblait vraiment parler à \emph{personne}  à par elle.

"Savais-tu qu'il y a une quatrième année de Poufsouffle qui est un Métamorphomage ?" dit Hermione tandis qu'il se dirigeaient vers la Grande Salle. "Elle peut rendre ses cheveux vraiment rouges, comme un panneau stop pas comme un Weasley, et la fois où elle s'est renversée du thé chaud dessus elle s'est transformée en un garçon brun jusqu'à ce qu'elle arrive à reprendre le contrôle d'elle-même."

"Vraiment ? Cool," dit Harry, l'air un peu distrait. "Euh, Hermione, juste histoire de vérifier, tu sais que c'est demain le dernier jour pour s'inscrire aux armées du professeur Quirrell ?"

"Oui," dit Hermione. "Les armées du maléfique professeur Quirrell." Sa voix était un peu colérique, même si Harry ne savait bien sûr pas pourquoi.

"Hermione," dit Harry, la voix exaspérée, "il n'est pas maléfique. Il est un peu sombre et très Serpentard. Ce n'est pas pareil que d'être \emph{maléfique} ."

Harry Potter avait trop de mots pour désigner les choses, c'était ça son problème. Il s'en serait mieux sorti en divisant le monde entre Bon et Mauvais. "Le professeur Quirrell m'a fait venir devant toute la classe et il m'a dit de \emph{tirer sur quelqu'un}  !"

"Il avait raison," dit Harry, le visage grave. "Je suis désolé Hermione, mais il avait raison. Tu aurais dû \emph{me}  tirer dessus, ça ne m'aurait pas dérangé. Tu ne peux pas apprendre la magie de combat si tu ne peux pas t'entraîner contre de vrais adversaires en utilisant de vrais sorts. Et maintenant tu te débrouille en lutte, non ?"

Hermione n'avait que douze ans, et bien qu'elle sache qu'il avait tort, elle ne savait pas l'exprimer, elle n'arrivait pas à trouver la phrase qui aurait convaincu Harry.

Le professeur Quirrell avait choisi une jeune fille et l'avait fait venir devant tout le monde, et il lui avait ordonné d'ouvrir le feu sans sommation sur un de ses camarades.

Ça n'avait pas \emph{d'importance}  qu'il ait eu raison au sujet du fait que c'était quelque chose qu'elle devait apprendre à faire.

Le professeur McGonagall n'aurait jamais fait ça.

Le professeur Flitwick n'aurait jamais fait ça.

Peut-être même que le professeur Rogue n'aurait jamais fait ça.

Le professeur Quirrell était \emph{maléfique} .

Mais elle n'arrivait pas à trouver les bons mots, et elle savait que Harry ne la croirait jamais.

"Hermione, j'ai parlé aux élèves plus âgé," dit Harry. "Le professeur Quirrell est peut-être le \emph{seul}  professeur compétent qu'on aura dans nos sept ans à Poudlard. Tout le reste, on pourra l'apprendre plus tard. Si on veut étudier la Défense, on doit le faire \emph{cette année} . Les élèves qui s'inscrivent pour les activités du soir vont énormément apprendre, bien plus que ce que le ministre pense que les élèves de première année doivent apprendre - tu savais qu'on va apprendre le Patronus ? En \emph{janvier}  ?"

"Le \emph{Patronus}  ?" dit Hermione, sa voix grimpant d'une octave sous l'effet de la surprise.

Ses livres disaient que c'était une des magies les plus pures que l'on connaissait, une arme contre les plus sombres des créatures, lancée avec les émotions les plus positives. Elle ne se serait pas attendue à ce que le professeur Quirrell l'enseigne - ou qu'il s'arrange pour le faire enseigner, puisque Hermione ne pouvait pas l'imaginer capable de jeter le sort lui-même.

"Oui," dit Harry. "Les élèves n'apprennent généralement pas le Patronus avant leur cinquième année, ou même plus tard ! Mais le professeur Quirrell dit que l'organisation du ministère a été faite par des veracrasses doués de paroles et que la capacité de lancer le Patronus dépend plus des émotions que de la force magique. Le professeur Quirrell dit qu'il pense que la plupart des élèves en font beaucoup moins qu'ils ne pourraient en faire et qu'il va le prouver cette année."

C'était le ton de vénération habituel que Harry employait quand il parlait du professeur Quirrell, et Hermione grinça des dents et continua de marcher.

"Je me suis déjà inscrit, en fait," dit Hermione d'une petite voix. "Ce matin. Pour tout, comme tu avais dit."

\emph{Quand le vin est tiré, il faut le boire} , disait le proverbe.

Et puis elle ne voulait pas \emph{perdre} , et si elle voulait gagner, il lui faudrait apprendre.

"Alors tu \emph{feras}  partie des armées ?" la voix de Harry était soudain enthousiaste. "Hermione, c'est génial ! J'ai déjà ma liste de soldats, mais je suis sûr que le professeur Quirrell me laissera en ajouter ou en échanger -"

"Je ne vais pas rejoindre ton armée." La voix de Hermione était acérée. Elle savait que c'était une supposition raisonnable de la part de Harry, mais ça l'énervait \emph{quand même} .

Harry cligna des yeux. "Sûrement pas celle de Malfoy. Alors tu veux être dans la troisième armée? Même si on ne sait pas encore \emph{qui}  est le général ?" Harry avait l'air surpris et légèrement blessé, et elle ne pouvait pas lui en vouloir, même si bien sûr elle lui en \emph{voulait} , puisque de fait, tout était de sa faute. "Mais pourquoi pas la mienne ?"

"Réfléchis-y," lâcha-t-elle, "et peut-être que tu arriveras à comprendre !"

Et elle accéléra le pas, laissant Harry bouche bée derrière elle.
\par\noindent\rule{\textwidth}{0.4pt}
"Professeur Quirrell," dit Draco de sa voix la plus formelle, "je me dois de protester contre la nomination de Hermione Granger au poste de troisième général."

"Oh," dit le professeur Quirrell, confortablement installé dans sa chaise, un air nonchalant et détendu sur le visage. "Protestez donc, M. Malfoy."

"Granger n'est pas apte à tenir ce poste," dit Draco.

Le professeur Quirrell tapota d'un doigt sur sa joue, l'air pensif. "Mais si, elle l'est. Avez-vous d'autres protestations ?"

"Professeur Quirrell," dit Harry Potter, qui se tenait à côté de Draco, "avec tout le respect que je dois aux nombreux talents scolaires de mademoiselle Granger et aux points Quirrell qu'elle a fort justement obtenu lors de vos cours, sa personnalité ne correspond pas à la tâche de commandeur militaire."

Draco avait été soulagé quand Harry avait accepté de l'accompagner dans le bureau du professeur Quirrell. Ce n'était pas \emph{seulement}  que Harry était un le plus grand des chouchou qui soit aux yeux du professeur Quirrell. Draco avait aussi commencé à s'inquiéter de la possibilité que Harry soit \emph{réellement}  ami avec Granger, ça faisait un bout de temps maintenant et Harry ne l'avait \emph{toujours pas}  attaquée... mais voilà qui était mieux.

"Je suis d'accord avec M. Potter," dit Draco. "La nommer général va transformer tout cela en une farce."

"Sévèrement formulé," dit Harry, "mais je ne peux faire autrement que de convenir que M. Malfoy a raison. Pour être direct, professeur Quirrell, Hermione Granger a environ autant d'intention de tuer qu'un bol de raison mûrs."

"Cela n'est pas un fait dont je n'aurais pu me rendre compte moi-même," répondit le professeur Quirrell avec douceur, "vous ne me dites rien que je ne sais déjà."

C'était à Draco de dire quelque chose mais il semblait que la conversation venait soudain d'attraper le hoquet. Cette réponse n'avait \emph{pas}  fait partie des possibilités que lui et Harry avaient envisagées lorsqu'ils s'étaient creusés les méninges avant de venir ici. Que \emph{fallait-il}  faire après que le professeur vous ait dit qu'il savait tout ce que vous saviez et qu'il allait quand même commettre une erreur évidente ?

Le silence continua un moment.

"Préparez-vous quelque chose ?" dit lentement Harry.

"Tout ce que je fais doit-il être un complot ?" dit le professeur Quirrell. "Ne puis-je jamais créer du chaos pour le plaisir de le faire ?"

Draco faillit s'étrangler.

"Pas dans votre cours de magie de combat," dit Harry, armé d'une calme certitude. "En d'autres lieux peut-être, mais pas ici."

Le professeur Quirrell leva lentement ses sourcils.

Harry lui rendit son regard avec assurance.

Draco frissonna.

"Eh bien dans ce cas," dit le professeur Quirrell, "il semble qu'aucun de vous n'a envisagé une très simple question. Qui pourrais-je nommer à la place de mademoiselle Granger ?"

"Blaise Zabini," dit Draco sans hésiter.

"D'autres suggestions ?" dit le professeur Quirrell, l'air très amusé.

\emph{Anthony Goldstein et Ernie Macmillan}  vinrent à l'esprit de Draco avant que son bon sens ne prenne le dessus et n'exclue tout Sang-de-Bourbe et tout Poufsouffle, peu importe leur agressivité lors des duels. Au lieu de cela, Draco se contenta de dire : "Qu'est ce qui ne va pas avec Zabini ?"

"Je vois..." dit lentement Harry.

"Moi \emph{pas} ," dit Draco. "Pourquoi pas Zabini ?"

Le professeur Quirrell regarda Draco. "Parce que, M. Malfoy, peut importe les efforts qu'il y mettra, il ne sera jamais à votre hauteur, ni à celle de M. Potter."

Le choc stupéfia Draco. "Vous ne pouvez pas croire que \emph{Granger}  va -"

"Il parie sur elle," dit doucement Harry. "Ce n'est pas garanti. Les chances ne sont même pas bonnes. Elle ne ne nous tiendra probablement jamais tête, et même si elle le fait, elle pourrait mettre des mois à apprendre. Mais elle est la seule dans notre année à avoir la moindre de chance de devenir capable de nous battre."

Les mains de Draco se tordirent mais ne devinrent pas des poings. Se faire passer pour votre partisan pour ensuite se retirer était une tactique de sapage classique, donc Harry Potter \emph{était } avec Granger et \emph{cela}  impliquait que -

"Mais professeur," enchaîna naturellement Harry, "je suis inquiet à l'idée que Hermione soit \emph{malheureuse}  à ce poste de général. Je parle en tant qu'ami à présent, professeur Quirrell. La compétition nous fait peut-être du bien à Draco et à moi, mais ce que vous demandez d'elle ne lui fera pas de \emph{bien}  !"

\emph{Au temps pour moi} , songea Draco.

"Votre amitié pour Hermione Granger vous fait honneur," dit sèchement le professeur Quirrell. "Tout particulièrement parce que vous parvenez à être en même temps ami avec Draco Malfoy. En voilà une belle prouesse."

Harry eut soudain l'air un peu nerveux, ce qui voulait dire qu'il se sentait probablement \emph{beaucoup}  plus nerveux, et Draco jura en son for intérieur. Bien sûr que Harry n'allait pas tromper le professeur Quirrell.

"Et je doute que mademoiselle Granger apprécie votre préoccupation amicale," dit le professeur Quirrell. "C'est elle qui m'a demandé le poste, M. Potter, ce n'est pas moi qui suis allé la voir."

Harry se tut pendant un moment. Puis il jeta à Draco un rapide coup d'œil d'excuse et de mise en garde mêlées, disant à la fois \emph{Désolé, j'ai fait de mon mieux}  et \emph{On ne devrait pas insister} .

"Quant à la possibilité qu'elle soit malheureuse," continua le professeur Quirrell, un léger sourire naissant sur ses lèvres, "je suspecte qu'il lui sera bien plus facile de supporter les rigueurs de son poste qu'aucun de vous deux ne pourrait l'imaginer, et qu'elle vous tiendra tête bien plus tôt que vous ne le pensez."

Harry et Draco manquèrent d'oxygène sous le coup de l'horreur.

"Vous n'allez pas la \emph{conseiller}  quand même ?" dit Draco, absolument effaré.

"Je ne me suis jamais inscrit à un combat contre \emph{vous}  !" dit Harry.

Le sourire sur les lèvres du professeur Quirrell s'élargit. "Je lui ai à vrai dire en effet \emph{offert}  quelque suggestions pour ses premières batailles."

"\emph{Professeur Quirrell}  \emph{!} " dit Harry.

"Oh, ne vous en faites pas," dit le professeur Quirrell. "Elle a rejeté mon offre. Comme je m'y attendais."

Les yeux de Draco se rétrécirent.

"Allons, M. Potter," dit le professeur Quirrell, "personne ne vous a-t-il jamais dit qu'il était mal élevé de fixer les gens ?"

"Vous n'allez pas secrètement l'aider d'une \emph{autre } façon ?" dit Harry.

"Ferais-je une chose pareille ?" dit le professeur Quirrell.

"Oui," dirent Harry et Draco à l'unisson.

"Je suis blessé par votre manque de confiance. Eh bien dans ce cas, je promets de ne pas aider le général Granger à votre insu de quelque façon que ce soit. Et je suggère maintenant que vous vous préoccupiez de vos affaires militaires. Novembre approche à grand pas."
\par\noindent\rule{\textwidth}{0.4pt}
Ils émergeaient du bureau du professeur Quirrell, et Draco vit les implications avant que la porte ne se soit totalement refermée.

Harry avait un jour dédaigneusement parlé des "trucs politiques".

Et c'était le seul espoir de Draco.

Pourvu qu'il ne se rende pas compte, pourvu qu'il ne se rende pas compte...

"On devrait juste attaquer Granger d'abord et la faire sortir du tableau," dit Draco. "Après l'avoir écrasée, nous pourrons avoir notre combat sans être distraits."

"Ça ne me semble pas très équitable," dit Harry d'une voix égale.

"Qu'est-ce que \emph{tu } en as à faire ?" dit Draco. "C'est bien ta rivale ?" Alors, avec juste la bonne touche de soupçon dans la voix, "ne me dis pas que tu as commencé à \emph{vraiment}  bien l'aimer, après avoir été son rival tout ce temps..."

"Les fondateurs m'en gardent," dit Harry. "Que veux-tu que je te dise, Draco ? J'ai simplement un sens de la justice instinctif. Granger aussi, tu sais. Elle a une très bonne maîtrise des notions de bien et de mal, et elle va probablement commencer par attaquer le mal. Tu sais, il faut vraiment le chercher pour s'appeler 'Malfoy'."

\emph{BON SANG !} 

"Harry," dit Draco d'un ton blessé et peut-être un peu supérieur, "ne veux tu pas te battre de façon \emph{équitable}  contre moi ?"

"Tu veux dire plutôt que de t'attaquer après que tu aies déjà perdu quelques une de tes troupes en vainquant Granger ?" dit Harry. "Oh, je ne sais pas. Peut-être qu'une fois que je me serais ennuyé de gagner j'essaierai la méthode 'équitable'."

"Peut-être qu'elle t'attaquera \emph{toi} ," dit Draco. "\emph{Tu}  es son rival."

"Mais je suis son \emph{ami}  rival," dit Harry, un méchant sourire dessiné sur le visage. "Je lui ai offert un beau cadeau d'anniversaire et tout. Ça ne serait pas bien vu d'aller saboter son ami rival comme ça."

"Et que dis-tu de l'idée de saboter les chances que ton \emph{ami}  ait un combat équitable ?" dit Draco avec colère. "Je pensais qu'on était amis !"

"Laisse-moi reformuler," dit Harry. "\emph{Granger}  ne saboterait pas un ami rival. Mais c'est parce qu'elle a l'intention de tuer d'un bol de raisons mûrs. \emph{Toi} , tu le ferais. Tu le ferais \emph{carrément} . Et devine quoi : moi aussi."

\emph{BON SANG !} 
\par\noindent\rule{\textwidth}{0.4pt}
Si cela avait été une pièce de théâtre, il y aurait eu une musique dramatique.

Impeccablement vêtu de ses robes à bordures vertes, parfaitement coiffé de ses cheveux blond vénitien, le héros faisait face au méchant.

Enfoncé dans une chaise en bois simple, ses dents de lapins clairement visible, des restes de châtaignes tombant le long de ses cheveux, le méchant faisait face au héros.

Nous étions mercredi 30 octobre, et la première bataille aurait lieu ce dimanche.

Draco se tenait dans le bureau du général Granger, une pièce de la taille d'une petite salle de classe (Draco ne savait pas exactement \emph{pourquoi}  les bureaux des généraux étaient si grands. Une chaise et un bureau lui auraient suffit. Et en premier lieu, il n'était même pas certain que les généraux aient besoin d'un bureau puisque tous leurs soldats savaient ou les trouver. À moins que le professeur Quirrell n'ait délibérément organisé les choses afin de leur fournir une marque de statut, auquel cas Draco était tout à fait en faveur de l'idée).

Granger était assise sur la seule chaise de la salle, comme si cela avait été un trône, et elle se situait loin à l'autre de bout de la pièce, face au mur où se trouvait la porte. Une longue table rectangulaire s'étirait entre eux, d'un bout de la pièce à l'autre, et quatre petite tables circulaires étaient réparties aux quatre coins, mais il y avait seulement cette unique chaise, tout au fond. La pièce avait des fenêtres le long d'un des murs, et un rayon de soleil venait toucher le dessus des cheveux de Granger pour lui faire une couronne lumineuse.

Ça aurait été bien si Draco avait pu s'avancer lentement. Mais il y avait une table qui lui barrait le chemin, Draco aurait dû faire le tour par la diagonale, et il était impossible de le faire d'une façon théâtrale et empreinte de dignité. Cela avait-il été délibéré ? S'il s'était agit de son père, ça l'aurait sûrement été ; mais c'était Granger, donc probablement pas.

Il n'avait nulle part où s'asseoir, et Granger ne s'était pas non plus levée.

Draco se garda bien d'exprimer le moindre outrage.

"Eh bien, M. Draco Malfoy," dit Granger une fois qu'il fut arrivé jusqu'à elle, "vous avez sollicité une audience et j'ai eu la grâce de vous l'octroyer. Quelle est votre requête ?"

\emph{Viens me rendre visite au manoir Malfoy, mon père et moi voudrions te montrer quelques sorts intéressants.} 

"Potter, votre rival, m'a fait une offre," dit Draco, arborant un air sérieux. "Il n'était pas gêné à l'idée de perdre contre moi, mais se serait sentit humilié par votre victoire. Il veut donc joindre ses forces aux miennes et vous éradiquer immédiatement, pas seulement lors de notre première bataille mais à chacune d'entre elles. Si je n'accepte pas, Potter souhaite que je reste inactif ou que je vous harcèle pendant qu'il lancera une attaque de toute ses forces en début de bataille."

"Je vois," dit Granger, l'air surprise. "Et vous offrez de m'aider contre lui ?"

"Bien sûr," enchaîna immédiatement Draco. "Je n'ai pas trouvé que ce qu'il comptait vous faire était très juste."

"Voyons, mais c'est très gentil de votre part, M. Malfoy," dit Granger. "Je suis navrée la façon dont je vous ai parlé plus tôt. Nous devrions être amis. Puis-je vous appeler Draky ?"

Des sirènes d'alarmes se mirent à sonner dans la tête de Draco, mais il y avait une \emph{chance}  qu'elle soit sincère...

"Bien sûr," dit Draco, "si je peux vous appeler Hermy."

Draco fut presque certain d'avoir vu son visage tressaillir.

"Quoi qu'il en soit," dit Draco, "je pensais que ça donnerait une bonne leçon à Potter si nous l'attaquions tous les deux et l'éradiquions."

"Mais ne serait-ce pas injuste envers M. Potter ?" dit Granger.

"Je pense que ce serait très juste," dit Draco. "Il comptait déjà vous infliger cette trahison."

Granger le regardait d'un air sévère qui aurait peut-être pu l'intimider s'il avait été un Poufsouffle et pas un Malfoy. "Vous pensez que je suis assez stupide, n'est-ce pas M. Malfoy ?"

Draco eut un sourire charmeur. "Non, mademoiselle Granger, mais je me suis dit que je devrais au moins vérifier. Alors, que voulez-vous ?"

"Êtes-vous en train d'essayer de me \emph{soudoyer}  ?" dit Granger.

"Tout à fait," dit Draco. "Puis-je juste vous filer Gallion et vous voir combattre Potter plutôt que moi jusqu'à la fin de l'année ?"

"Non," dit Granger, "mais vous pouvez m'offrir dix Gallions et me voir vous combattre de façon équitable au lieu que je me concentre sur vous."

"Dix Gallions représentent une forte somme," dit prudemment Draco.

"Je ne savais pas que les Malfoys étaient pauvres," dit Granger.

Draco fixa Granger.

Il commençait à trouver cette situation étrange.

Cette réponse en particulier ne ressemblait pas à ce que cette fille en particulier avait l'habitude de dire.

"Eh bien," dit Draco, "on ne devient pas riche en jetant son argent par les fenêtres, vous savez."

"Je ne sais pas si vous savez ce qu'est un dentiste, M. Malfoy, mais mes parents sont \emph{dentistes} , et toute somme inférieure à dix Gallions ne mérite pas que je m'y attarde."

"Trois Gallions," dit Draco, plus pour sonder le terrain qu'autre chose.

"Non," dit Granger. "Pas si vous voulez un combat équitable pour tous, et je ne pense pas qu'un Malfoy désire un combat équitable moins qu'il ne désire dix Gallions."

Draco commençait à trouver cette situation \emph{très}  étrange.

"Non," dit Draco.

"Non ?" dit Granger. "Cette offre a une date d'expiration, M. Malfoy. Êtes-vous certain de vouloir risquer une année entière pendant laquelle vous serez lamentablement écrasé par le Survivant ? Ce serait plutôt gênant pour la maison Malfoy, vous ne pensez pas ?"

C'était un argument très persuasif, un argument très difficile à rejeter, mais vous ne deveniez pas riche en dépensant son argent quand votre cœur vous disait que c'était un piège.

"Non," dit Draco.

"À dimanche," dit Granger.

Draco fit demi-tour et sortit du bureau sans prononcer un mot de plus.

Ça ne s'était \emph{pas bien}  passé...
\par\noindent\rule{\textwidth}{0.4pt}
"Hermione," dit Harry avec patience, "nous sommes \emph{censés}  fomenter l'un contre l'autre. Tu pourrais même me trahir et ça ne voudrait rien dire hors du champ de bataille."

Hermione secoua sa tête. "Ça ne serait pas gentil, Harry."

Il soupira. "Je ne pense pas que tu te mettes dans le bain comme il faut."

\emph{Ça ne serait pas gentil} . Elle l'avait vraiment dit. Hermione ne savait pas si elle devait se sentir insultée par ce que Harry pensait d'elle ou si elle devait être inquiétée par la possibilité qu'elle avait \emph{vraiment}  l'air d'une telle sainte-nitouche le reste du temps.

Il était probablement temps de changer de sujet.

"Enfin bref, est-ce que tu fais quelque chose de particulier demain ?" dit Hermione. "C'est -"

Sa voix s'arrêta brutalement lorsqu'elle se rendit compte.

"Oui, Hermione," dit Harry d'une voix un peu pincée, "quel jour sommes-nous ?"
\par\noindent\rule{\textwidth}{0.4pt}
\emph{Interlude}  :

Il avait été un temps ou le 31 octobre avait été Halloween en Angleterre magique.

Maintenant, c'était le jour de Harry Potter.

Harry avait refusé toutes les offres, mêmes celles du ministre Fudge qui lui auraient vraiment offert des faveurs politiques dans l'avenir et qu'il aurait \emph{vraiment}  dû accepter en serrant les dents. Mais pour Harry, le 31 octobre serait à jamais le jour du Le Seigneur des Ténèbres a Tué mes Parents. Il aurait dû y avoir une petite cérémonie commémorative, discrète et digne, mais s'il y en avait eu une Harry n'avait pas été invité.

Poudlard prit un jour férié pour célébrer l'occasion. Même les Serpentard ne portèrent pas de noir hors de leur dortoir. Il y avait des événements spéciaux et des plats inhabituels et les enseignants regardaient ailleurs si quelqu'un courait dans les couloirs. C'était le dixième anniversaire après tout.

Harry passa la journée dans sa malle afin de ne gâcher la journée de personne, mangeant des barres énergétiques à la place de ses repas, relisant quelques uns de ses livres de science-fiction les plus tristes (et pas de fantasy), écrivant une lettre à Maman et Papa qui fut plus longue que celles qu'il envoyait d'habitude.

