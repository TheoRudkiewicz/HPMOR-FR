
\chapter{Compromis Tabous, Après coup 3  Distance}

\emph{[NdT : Reprise de la traduction après 10 mois d'interruption ! Les parutions devraient être rapides avec un } \emph{\textbf{minimum de 1 chapitre tous les 4 jours} } \emph{ jusqu'au chapitre 96 (le dernier paru en anglais). Si vous avez oublié ce qui est en train de se passer, relire les chapitres 80 et 81 vous donnera les événements principaux, mais 82-84 sont aussi importants.]} 

\emph{Ce chapitre a été en grande partie revu le 16 décembre 2012. Le changement principal commence à mi-chemin – cherchez le mot "trivial" pour le trouver. [NdT : ce chapitre a été traduit après l'application du changement mentionné]} 

Lent et ardu, le long escalier qui menait jusqu'au sommet de Serdaigle. De l'intérieur, il semblait monter en ligne droite, mais on pouvait voir de l'extérieur qu'il devait en toute logique monter en spirale. On ne pouvait atteindre le sommet de la tour Serdaigle qu'en suivant cette ascension, sans prendre de raccourci, une marche de pierre après l'autre ; elles défilaient sous les chaussures de Harry, poussées par ses jambes de plus en plus fatiguées.

Il s'était assuré de la sécurité de Hermione jusqu'à ce qu'elle se couche.

Puis il était demeuré dans la salle commune de Serdaigle suffisamment longtemps pour récolter quelques signatures qui pourraient par la suite être utiles à Hermione. Peu d'élèves avaient signé ; les sorciers n'avaient pas été entraînées à penser selon la méthode scientifique Moldue du \emph{prouves-le ou tais-toi} , aussi appelée \emph{prends un risque et fais une prédiction ou arrêtes de prétendre que tu crois à ta théorie} . La plupart d'entre eux n'avaient rien trouvé d'\emph{incongru}  à l'idée d'être trop nerveux pour signer un contrat disant que Hermione se réservait le droit de les faire payer jusqu'à la fin de leurs vies s'ils avaient eu tort tout en affichant ouvertement leur confiance en sa culpabilité. Mais le simple fait d'avoir demandé ces signatures constituerait une démonstration suffisante, après révélation de la vérité, si jamais quelqu'un soupçonnait à nouveau Hermione d'agissements obscurs. Au moins, elle n'aurait pas à traverser cette épreuve \emph{deux fois} .

Harry avait ensuite rapidement quitté la salle commune car tous les sentiments cléments et généreux qu'il avait ressentis à force de raisonnement s'étaient faits de plus en plus difficile à garder en tête. Il songeait parfois que la séparation la plus profonde de sa personnalité n'avait rien à voir avec son côté obscur mais qu'il s'agissait plutôt de la division entre le Harry altruiste et clément du Raisonnement Abstrait et le Harry frustré et en colère du Moment Présent.

La plate-forme circulaire située au-dessus de la tour Serdaigle n'était pas le point le plus haut de Poudlard, mais comme cette dernière saillait à l'écart du corps principal du château, on ne pouvait l'observer depuis la tour d'Astronomie. C'était donc un endroit tranquille où réfléchir pour qui en avait un terrible besoin. Un endroit où peu d'autres élèves s'aventuraient – il y avait d'autres recoins intimes, si vous ne recherchiez que l'intimité.

Les torches de Poudlard, loin en-dessous, illuminaient la nuit. La plate-forme elle-même offrait peu d'obstacles : le escaliers donnaient sur un trou nu dans le sol plutôt que sur une porte dressée. D'ici, les étoiles étaient en cet instant aussi visibles qu'elles pouvaient l'être depuis la Terre.

Le garçon s'assit au centre de la plate-forme, insouciant de ses robes qui auraient pu être tâchées, et laissa reposer sa tête sur le sol dallé de pierre afin que, exception faite de quelques créneaux rocheux à moitié visibles à la lisière de sa perception et d'un éclat de croissant de Lune, la réalité devienne cette lumière stellaire.

Les points de lumière sur sombre velours scintillèrent, tremblèrent puis revinrent ; une beauté différente de leur solide éclat lors de la Nuit Silencieuse.

Harry fixait le vide, son esprit occupé par d'autres choses.

\emph{Aujourd'hui commence ta guerre contre Voldemort...} 

Dumbledore avait dit cela après l'Incident lors du Sauvetage de Bellatrix d'Azkaban. Cela avait été une fausse alarme, mais la phrase exprimait bien le sentiment.

Sa guerre avait commencé deux nuits plus tôt, et Harry ne savait même pas contre \emph{qui} .

Dumbledore pensait que c'était Voldemort, revenu des morts, qui jouait son premier coup contre le garçon qui l'avait auparavant vaincu.

Le professeur Quirrell avait mis en place des barrières de détection autour de Draco car il craignait que le directeur fou de Poudlard essaie de faire tomber Harry pour la mort du fils de Lucius.

Ou le professeur Quirrell avait tout orchestré et c'était pour \emph{ça}  qu'il avait su où trouver Draco. Severus Rogue pensait que le professeur de Défense de Poudlard était un suspect évident, et même \emph{le}  suspect évident.

Et Severus Rogue lui-même pouvait aussi bien être fiable qu'être tout le contraire.

\emph{Quelqu'un}  avait fait une déclaration de guerre à Harry, son premier coup avait été de prendre à la fois Draco et Hermione, et ce n'était que d'un cheveu que Harry avait sauvé Hermione.

On ne pouvait pas appeler cela une victoire. Draco avait été soustrait à Poudlard, et sans être fatal, il n'était pas dit que ce fut réversible ou que Draco serait le même à son retour. L'Angleterre magique pensait maintenant que Hermione avait commis une tentative de meurtre, ce qui pouvait la conduire à choisir d'agir de façon sensée et de partir. Harry avait sacrifié toute sa fortune pour éviter cette perte et une telle carte ne pouvait être jouée qu'une seule fois.

Quelque pouvoir inconnu s'était abattu sur lui, et même si le coup avait été partiellement dévié, il avait quand même fait \emph{très mal} .

Au moins son côté obscur ne lui avait rien demandé en échange du sauvetage de Hermione. Peut-être parce que son côté obscur \emph{n'était pas}  une voix imaginaire comme celle de Poufsouffle ; Harry \emph{imaginait}  peut-être que sa partie Poufsouffle était dotée de ses propres désirs, différents des siens, mais son côté obscur n'était pas comme ça. Son " côté obscur ", du mieux que Harry puisse en juger, constituait un autre état d'être que Harry \emph{était}  parfois. Pour l'instant Harry n'était pas en colère et tenter de demander ce que désirait le " Harry obscur " revenait à laisser sonner un téléphone dans le vide. L'idée elle-même semblait un peu étrange : pouvait-on devoir quelque chose à quelqu'un qu'on était parfois ?

Harry regarda les étoiles placées par le hasard, ces lumières scintillantes et éparses que les cerveaux humains ne pouvaient s'empêcher de regrouper en constellations.

Et puis il y avait cette promesse que Harry avait faite.

Draco aiderait Harry à réformer Serpentard. Et Harry se ferait ennemi de celui ou celle qui, selon le jugement de Harry en sa capacité de rationaliste, avait tué Narcissa Malfoy. Si jamais elle s'était salie les mains, si elle avait vraiment été brûlée vive, si le tueur n'avait pas été dupé – c'étaient là toutes les conditions que Harry se souvenait avoir posées. Il aurait probablement dû les écrire, ou encore mieux, ne jamais faire une promesse qui nécessite autant d'exceptions.

Il existait des échappatoires plausibles, pour le genre de personne qui se laisserait rationaliser que des échappatoires existaient. Dumbledore n'avait pas \emph{vraiment } avoué. Il n'avait pas juste déclaré que c'était lui. Il y avait des raisons plausibles à un tel comportement de la part d'un Dumbledore coupable. Mais c'était \emph{aussi}  ce qu'on se serait attendu à voir si quelqu'un d'autre avait brûlé Narcissa et que Dumbledore s'en était attribué le mérite.

Harry secoua la tête et plaqua un côté de ses cheveux puis l'autre contre le sol de pierre. Il avait toujours une dernière échappatoire, Draco pouvait à tout moment le libérer de sa promesse. Il pourrait au moins décrire la situation et discuter des options possibles avec lui lorsqu'il le verrait à nouveau. Cela ne semblait pas très probable – mais l'idée d'en parler honnêtement satisfaisait la partie de Harry qui exigeait que les promesses soient tenues. Même si cela ne faisait que repousser les choses à plus tard, c'était mieux que de prendre un homme bon pour ennemi.

\emph{Mais}  \emph{Dumbledore } est-il \emph{bon ? d} emanda la voix de Poufsouffle. \emph{Si Dumbledore a brûlé vif quelqu'un – l'idée n'était-elle pas que les bons peuvent tuer mais jamais en faisant souffrir ?} 

\emph{Peut-être qu'il l'a tuée instantanément } dit Serpentard, \emph{et qu'il a mentit à Lucius sur le fait qu'elle était en encore vie. Mais...s'il y avait la } moindre\emph{ possibilité que les Mangemorts aient pu magiquement vérifier la façon dont Narcissa était morte... et si se faire prendre à mentir aurait mis en danger les familles du côté gentil...} 

\emph{Attention à ce que l'on rationalise habilement} , le prévint Gryffondor.

\emph{Tu devrais t'attendre à ce que cela ait des effets sur ta réputation, sur la façon dont les gens te traitent} , dit Poufsouffle. \emph{Si tu décides qu'il y a de bonnes raisons de brûler vive une femme, l'un des effets collatéraux prévisibles est que les gentils décideront que tu as franchi une limite et que l'on doit t'arrêter. Dumbledore aurait dû s'y attendre. Il n'a pas le droit de s'en plaindre.} 

\emph{Ou peut-être qu'il attend de nous qu'on soit plus intelligents, } dit Serpentard\emph{. Maintenant que nous connaissons ce pan de la vérité – peu importe les détails exacts du reste de l'histoire – pouvons-nous vraiment croire que Dumbledore est une personne horrible qui devrait être notre ennemi ? Au milieu d'une horrible guerre sanglante, Dumbledore a immolé } un\emph{ civil ? Ce n'est répréhensible qu'en suivant les critères des bandes dessinées, pas en suivant n'importe quel critère historique réaliste.} 

Harry regarda le ciel nocturne et pensa à l'Histoire.

Dans la vraie vie, dans les vraies guerres...

Pendant la seconde guerre mondiale, il y avait eu un projet de saboter le programme d'armement nucléaires Nazi. Des années plus tôt, Leo Szilard, la première personne à comprendre la possibilité d'une réaction de fission en chaîne, avait convaincu Fermi de ne pas publier la découverte de ce dernier : le graphite purifié était un modérateur de neutrons peu cher et efficace. Fermi avait voulu publier pour le bien du grand projet international de la Science, qui dépassait le nationalisme. Mais Szilard avait persuadé Rabi et Fermi s'était incliné face à la majorité de leur petite conspiration à trois. Et ainsi, des années plus tard, le seul modérateur de neutrons connu des Nazis avait été le deutérium.

La seule source de deutérium contrôlée par les Nazis avait été une installation en Norvège occupée et avait été détruite à coups de bombes et de sabotages, ce qui avait occasionné la mort de vingt-quatre civils.

Les Nazis avaient voulu livrer le deutérium déjà raffiné vers l'Allemagne à bord d'un ferry civil Norvégien, le \emph{SS Hydro} .

Knut Haukelid et ses assistant avaient été découverts par le gardien de nuit du ferry alors qu'ils montaient à bord pour le saboter. Haukelid avait dit au gardien qu'ils fuyaient la Gestapo et le gardien les avait laissés passer. Haukelid avait songé à prévenir le gardien de nuit mais cela aurait menacé la mission, si bien qu'il lui avait juste serré la main. Et le navire civil avait sombré dans la partie la plus profonde du lac avec huit Allemands morts, sept membres d'équipage morts et trois civils innocents morts. Certains des sauveteurs Norvégiens avaient songé que les soldats allemands présents auraient dû être laissés à leur noyade mais cette idée n'avait pas prévalu, et les survivants allemands avaient été sauvés. Et cela avait marqué la fin du programme d'armement nucléaire Nazi.

Ceci impliquait que Knut Haukelid avait tué des innocents, dont un, le gardien de nuit du navire, qui avait été quelqu'un de \emph{bien} . Quelqu'un qui s'était efforcé d'aider Haukelid, qui avait pris des risques pour lui par bonté de cœur, pour de pures raisons morales, et on l'avait noyé. Plus tard, à la froide lumière de l'Histoire, il était apparu que les Nazis n'avaient après tout jamais été proches d'obtenir des armes nucléaires.

Et Harry n'avait jamais rien lu qui puisse suggérer que Haukelid avait mal agi.

C'était la guerre, dans le monde réel. En termes de dégâts totaux et de degré d'innocence des victimes, les actes de Haukelid avaient été considérablement \emph{pires}  que les agissements potentiels de Dumbledore envers Narcissa Malfoy ou que ce que Dumbledore avait peut-être fait afin que Voldemort ait vent de la prophétie et attaque les parents de Harry.

Si Haukelid avait été un héros de bandes dessinées, il aurait trouvé un moyen d'évacuer tous les civils du ferry puis il aurait directement attaqué les soldats allemands...

...plutôt que de laisser un seul innocent mourir...

...mais Haukelid n'avait pas été un superhéros.

Et Albus Dumbledore non plus.

Harry ferma les yeux et déglutit avec force, frappé par une soudaine sensation d'étouffement. Il était soudain très clair que, à côté d'un Harry désireux de vivre les idéaux des Lumières, Dumbledore avait vraiment \emph{participé à une guerre} . Les idéaux non-violents étaient faciles si l'on était un scientifique et que l'on vivait dans la bulle de \emph{Protego}  lancée par les policiers et les soldats dont on avait le luxe de remettre les actes en question. Albus Dumbledore semblait avoir commencé avec des idéaux au moins aussi exigeants que ceux de Harry, si ce n'est encore plus ; et il n'avait pas traversé la guerre sans tuer d'ennemis et sacrifier d'alliés.

\emph{Es-tu tellement meilleur que Haukelid et Dumbledore, Harry Potter, que tu seras capable de combattre sans provoquer une seule perte ? Même dans le monde des super-héros, la seule raison pour laquelle un super-héros comme Batman a } l'air \emph{de réussir, c'est que les lecteurs ne remarquent que la mort des Personnages Importants dont on connaît le Nom, pas quand le Joker tue un passant anonyme quelconque pour démontrer sa méchanceté. Batman n'est pas moins meurtier que le Joker si l'on prend en compte toutes les vies qu'il aurait pu sauver en le tuant. C'est ce que l'homme nommé Maugrey essayait de dire à Dumbledore, et Dumbledore a ensuite semblé regretter d'avoir mis si longtemps à changer d'avis. Vas-tu vraiment essayer de suivre la voie des super-héros, de ne jamais sacrifier un seul de tes pions ni de tuer un seul ennemi ?} 

Épuisé, Harry détourna son attention du dilemme présent l'espace d'un instant et rouvrit ses yeux pour admirer l'hémisphère de nuit qui n'attendait aucune décision de sa part.

Non loin des limites de son champ de vision, le pâle et blanc croissant de Lune dont cette lumière était partie il y a une seconde et quart, à une distance d'environ trois cent soixante-quinze mille kilomètres dans l'espace de simultanéité terrestre.

Au-dessus et sur le côté, Polaris, l'étoile du nord ; la première que Harry avait apprise à identifier en suivant la frontière de la Grande Ourse. C'était en fait un système à cinq étoiles avec une supergéante centrale située à 434 années-lumières de la terre. C'était la première 'étoile' que Harry avait apprise de son père, il y a si longtemps qu'il ne pouvait se souvenir de l'âge qu'il avait alors eu.

Le pâle brouillard était la voie lactée, tant de milliards de lointaines étoiles qu'elles devenaient une rivière indistincte, surface d'une galaxie de cent mille années lumières de diamètre. Si Harry avait ressenti le moindre émerveillement lorsqu'on lui avait dit cela pour la \emph{première}  \emph{fois} , il avait été trop jeune pour pouvoir s'en souvenir, quelques années après.

Au centre de la constellation d'Andromède, son étoile, qui était plutôt une galaxie. La plus proche de la Voie Lactée, à 2,4 millions années lumières, et qu'on estimait abriter mille milliards d'étoiles.

Comparé à de tels nombres, " l'infini " pâlissait, parce que " l'infini " était indéfini, vide. Penser à des étoiles " infiniment " lointaines était bien moins effrayant que d'essayer de calculer combien de mètres il y avait dans 2,4 millions d'années-lumières. 2,4 millions d'années lumières fois 31 millions de secondes par an, fois un photon en déplacement à trois cent mille mètres par seconde...

Il était étrange de songer que de telles distances n'étaient peut-être \emph{pas}  assez lointaines pour être inatteignables. La magie avait été relâchée dans l'univers, et avec elle des choses telles que les Retourneurs de Temps et les balais volants. Un sorcier avait-il jamais essayé de mesurer la vitesse d'un Portoloin ou d'un phénix ?

Et la compréhension humaine de la magie n'avait certainement \emph{rien à voir}  avec ses lois sous-jacentes. Que pouvait-on faire avec la magie, une fois qu'on la comprenait \emph{réellement}  ?

Un an plus tôt, Papa était allé à l'université nationale d'Australie, à Canberra, en tant que conférencier invité, et il avait emmené Maman et Harry. Et ils avaient tous visités le musée national d'Australie car il s'était avéré qu'il n'y avait quasiment rien d'autre à faire à Canberra. Les présentoirs en verre avaient montré des lance-pierres fabriqués par les aborigènes Australiens – comme de grands chausses-pieds en bois, mais polis, creusés et ornés avec une attention minutieuse. Pendant les 40000 ans depuis que des humains anatomiquement modernes avaient émigrés de l'Asie vers l'Australie, personne n'avait inventé l'arc et la flèche. Cela permettait vraiment d'apprécier à quelle point l'idée de Progrès n'était \emph{pas}  évidente. Pourquoi songeriez-vous même que l'idée d'Invention était importante si tous vos contes héroïques parlaient de guerriers ou de grands défenseurs et ne parlaient pas de Thomas Edison ? Comment quiconque aurait pu soupçonner, en creusant son lance-pierre avec une minutieuse attention, qu'un jour les humains inventeraient des fusées et l'énergie nucléaire ?

Auriez-vous pu lever les yeux vers le ciel, vers l'étincelante lumière solaire, et déduire que l'univers contenait des sources de pouvoir plus fortes qu'un simple feu ? Auriez-vous compris que si les lois fondamentales de la physique le permettaient, un jour les humains puiseraient dans les mêmes énergies que le Soleil ? Même si rien que vous puissiez imaginer faire avec un lance-pierre ou avec une outre tressée – aucun déplacement à travers la savane, aucun gibier que vous puissiez chasser – n'aurait permit de l'accomplir, ne serait-ce qu'en imagination ?

Ce n'était pas comme si les Moldus modernes s'étaient en quoi que ce soit approchés des limites énoncées par la physique moldue. Et pourtant, comme des chasseurs-cueilleurs conceptuellement attachés à leurs lances-pierres, la plupart des Moldus vivaient dans un monde défini par les limites de ce que l'on pouvait faire avec des voitures et des téléphones. Même si les lois de la physique moldue permettaient explicitement des choses telles que la nanotechnologie moléculaire ou le processus de Penrose destiné à extraire de l'énergie de trous noirs, la plupart des gens enregistraient ça là où leur cerveau enregistrait les contes de fées et le contenu des livres d'Histoire, bien loin de leurs réalités personnelle : \emph{En un lieu et un temps lointains, si lointains. } Ce n'était donc pas surprenant de voir le monde magique vivre dans un univers conceptuellement limité – non pas par les lois fondamentales de la magie que personne ne connaissait de toute façon – mais seulement par les règles superficielles des charmes et sortilèges connus. On ne pouvait pas observer la façon dont la magie était utilisée aujourd'hui et ne \emph{pas}  repenser au musée national d'Australie une fois qu'on se rendait compte de ce que l'on avait sous les yeux. Même si la première idée de Harry avait été fausse, d'une façon ou d'une autre il était toujours inconcevable que les lois \emph{fondamentales}  de l'univers contiennent un cas particulier pour des lèvres humaines formant la phrase 'Wingardium Leviosa'. Et pourtant même cette maîtrise tâtonnante de la magie suffisait à faire des choses qu'un professeur de physique moldu aurait déclaré être \emph{pour toujours impossibles } : le Retourneur de Temps, de l'eau invoquée à partir de rien par \emph{Aguamenti} . Quelles étaient les possibilités \emph{ultimes}  si les lois sous-jacentes de l'univers permettaient à un enfant de onze ans armé d'un bâton de transgresser quasiment toutes les contraintes de la version moldue de la physique ?

Comme un chasseur-cueilleur levant les yeux vers le Soleil et devinant que l'univers devait avoir une structure qui donnait naissance à l'énergie nucléaire...

On en venait à se demander si vingt mille millions de millions de millions de mètres n'était pas une distance si longue que ça, après tout.

Il y avait un pas après le Raisonnement Abstrait que Harry pouvait faire avec assez de temps pour préparer son environnement et son esprit ; quelque chose qui dépassait le Harry du Raisonnement Abstrait tout comme ce dernier dépassait le Harry du Moment Présent. En regardant les étoiles, on pouvait essayer d'imaginer ce que les lointains descendants de l'humanité penseraient de son dilemme – dans cent millions d'années, quand les étoiles auraient opérés d'immenses mouvements galactiques, auraient atteint des positions entièrement nouvelles, quand toutes les constellations seraient éparpillées. Un théorème fondamental de la probabilité énonçait que si vous saviez ce que votre réponse serait suite à votre rencontre avec une information encore à venir, vous deviez adopter cette réponse immédiatement. Que si vous \emph{connaissiez}  votre destination, vous y étiez déjà. Et par analogie si ce n'est par théorème, que si vous pouviez deviner ce que les descendants de l'humanité penseraient de quelque chose, vous deviez immédiatement choisir cette opinion comme étant la meilleure actuellement disponible.

Vue d'ici, l'idée de tuer les deux tiers du Magenmagot était bien moins attrayante qu'elle ne l'avait été quelques heures plus tôt. Et même si cela s'avérait \emph{nécessaire} , même si vous aviez la certitude que c'était le meilleur choix possible pour l'Angleterre magique et que la Trame de l'Histoire en souffrirait si vous ne le faisiez pas... même inévitables, les morts d'êtres sentients demeureraient une tragédie. Un élément de plus parmi les tourments de la Terre ; la Très Ancienne Terre d'où tout était venu, en un lieu et un temps lointains, si lointains.

\emph{Il n'est pas comme Grindelwald. Il n'y a plus rien d'humain en lui. Lui, tu dois le détruire. Garde ta furie pour cela, et pour cela seulement -} 

Harry secoua légèrement la tête, inclinant quelque peu les étoiles qu'il pouvait voir, allongé sur la pierre, les yeux vers le haut, l'extérieur et l'à venir du temps. Même si Dumbledore avait raison, même si le véritable ennemi était absolument fou et maléfique... dans cent million d'années la forme de vie connue sous le nom de Lord Voldemort ne serait probablement pas très différente de tous les autres enfants hagards de l'Ancienne Terre. Quoi qu'il se soit infligé, aussi horriblement irrévocables que ces noirs rituels paraissent à la simple aune humaine, il ne serait pas incurable dans cent millions d'années. Le tuer, même s'il \emph{fallait}  le faire pour sauver d'autres vies, ne serait qu'une mort de plus, une source de tristesse de plus pour les êtres sentients à venir. Comment pouvait-on regarder les étoiles et penser autrement ?

Harry regarda les lumières vacillantes de l'Éternité et se demanda ce que les enfants des enfants de ses enfants penseraient de ce que Dumbledore avait peut-être-fait à Narcissa.

Mais même en essayant de poser la question sous cet angle, de demander ce que les descendants de l'humanité penseraient, on était limité à son propre savoir, pas au leur. La réponse venait toujours de soi, et l'on pouvait toujours se tromper. Si on ne connaissait pas soi-même la centième décimale de pi, alors on ne savait pas comment les enfants de ses enfants de ses enfants la calculeraient, aussi trivial que soit ce fait.
\par\noindent\rule{\textwidth}{0.4pt}
Lentement – il était resté allongé là à regarder les étoiles plus longtemps qu'il ne l'avait prévu – Harry redressa son buste. Puis il poussa sur ses pieds sous les protestations de ses muscles et marcha jusqu'au bord de la plate-forme de pierre située au faîte de la tour de Serdaigle. Les créneaux de pierre qui entouraient le bord n'étaient pas hauts, pas assez hauts pour offrir une sécurité. Ils étaient clairement des marqueurs plutôt que des rambardes. Harry ne s'approcha pas trop près du bord ; il était vain de prendre des risques. En baissant les yeux vers le domaine de Poudlard, il sentit l'étourdissement prévisible, l'affliction chancelante appelée vertige. Il semblait que son cerveau était alarmé parce que le sol plus bas était très \emph{distant} . Il était peut-être à cinquante mètres de distance.

Il semblait que la leçon à en tirer était que les choses devait être \emph{incroyablement proches } avant que le cerveau puisse les comprendre suffisamment bien pour produire de la peur.

Rare était le cerveau capable d'éprouver une sensation forte au sujet de quoi que ce soit qui ne fut pas proche dans l'espace, proche dans le temps, proche de soi, à portée de main...

Auparavant, Harry s'était imaginé que se rendre à Azkaban nécessiterait des plans, la coopération d'une confédération d'adultes. Des Portoloins, des balais volants, un sortilège d'invisibilité. Quelque moyen d'atteindre les niveaux inférieurs sans que les Aurors ne le remarquent afin de pouvoir creuser son chemin jusqu'à la fosse centrale où les ombres de la Mort attendaient.

Et cela avait suffit à éloigner l'idée dans le futur, séparée d'un \emph{maintenant}  à l'abri.

Ce n'était pas avant aujourd'hui qu'il s'était rendu compte que ce pourrait être aussi simple que de trouver Fumseck et que de dire au phénix qu'il était temps.

Ses souvenirs remontaient à nouveau, des souvenirs qu'il ne pouvait jamais oublier très longtemps. Même si les pierres sous ses pieds n'étaient pas lisse comme du métal, même si le ciel éclairé par la Lune s'étendait au-dessus de lui, il lui était pourtant très simple de s'imaginer pris au piège dans un long couloir de métal éclairé d'une orange lueur.

La nuit était calme, assez calme pour que les souvenirs deviennent clairement audibles.

" Non, je ne voulais pas, ne meurs pas s'il te plaît ! "

" Non, je ne voulais pas, ne meurs pas s'il te plaît ! "

" Ne l'emmène pas, non non non - "

Les mots se brouillèrent et Harry s'essuya les yeux du revers de sa manche.

Si cela avait été \emph{Hermione}  derrière porte -

Si Hermione avait été placée à Azkaban, Harry aurait appelé le phénix, s'y serait rendu, et aurait brûlé jusqu'au dernier des Détraqueurs, peu importe à quel point cela aurait été fou, peu importe ce qu'il aurait voulu accomplir d'autre dans sa vie. C'était juste... c'était... c'était juste ainsi.

Et la femme qui \emph{était}  derrière la porte – n'y avait-il pas quelqu'un, quelque part, pour qui elle aussi était précieuse ? N'était-ce que la distance entre Harry et elle qui empêchait son cerveau de ressentir le besoin d'aller à Azkaban et de la sauver \emph{peu importe les conséquences}  ? Qu'aurait-il fallu, pour qu'il s'y sente contraint ? Aurait-il fallu qu'il connaisse son visage ? Son nom ? Sa couleur préférée ? Aurait-il ressenti le besoin d'aller à Azkaban sauver Tracey Davis ? Aurait-il été contraint de s'y rendre pour sauver le professeur McGonagall ? Maman et Papa... la question ne se posait même pas. Et cette femme avait dit être une mère. Combien de personnes avaient souhaité avoir le pouvoir de briser Azkaban ? Combien de prisonniers d'Azkaban rêvaient chaque nuit d'un sauvetage miraculeux ?

\emph{Aucun. C'est une pensée heureuse.} 

Peut-être \emph{devait-il}  raser Azkaban. Tout ce qu'il avait à faire c'était de trouver Fumseck et de lui dire que le temps était venu. Visualiser le centre de la fosse des Détraqueurs comme il l'avait vue de son balais volant et laisser le phénix l'y emmener. Lancer le Véritable Patronus à bout portant et au diable ce qui se passerait ensuite.

Il n'avait qu'à aller trouver Fumseck.

C'était peut-être aussi simple que de penser à la flamme, que de faire appel à l'oiseau de feu en son cœur...

Une étoile brilla dans la nuit.

Le temps que les yeux de Harry se réorientent vers ce point, mûs par un réflexe entraîné par des pluies de météores, une autre partie de lui se voyait surprise que le phénomène astronomique soit encore présent ; un faible étoile dont la la lueur croissait lentement. Il y eut un autre moment surpris pendant lequel Harry se demanda ce qu'il voyait, pas un météore, mais une nova ou une supernova – pouvait-on les \emph{voir}  briller comme ça ? La première étape d'une nova était-elle censée avoir cette couleur jaune-orange ?

Puis la nouvelle étoile bougea de nouveau, elle sembla grandir et s'intensifier. Elle semblait soudain plus \emph{proche } au lieu d'être si lointaine que la distance en devenait sans importance. Comme lorsque ce que vous croyiez être une étoile se révélait être un avion, une forme éclairée dont on pouvait réellement voir la forme...

...non, pas un avion...

La compréhension sembla se répandre depuis la poitrine de Harry par vagues de picotements et de sueur prête à émerger.

...un oiseau.

Un cri perçant fendit la nuit et fit écho sur les toits de Poudlard.

La créature approchante formait une traînée de feu à la suite de son vol, elle versait de ses plumes des flammes dorées aux mouvements d'étincelles à chaque battement de ses puissantes ailes. Et alors qu'elle décrivait une large courbe pour venir flotter à quelques pas de Harry, alors même que les flammes entourant son passage diminuaient, la créature ne sembla pas plus obscure, pas moins brillante, comme si quelque invisible Soleil l'illuminait de ses rayons.

De grandes ailes étincelantes comme un crépuscule, des yeux tels de perles incandescentes, éclatantes d'un feu d'or et de détermination.

Le bec du phénix s'ouvrit et laissa jaillir un grand croassement que Harry comprit comme s'il s'était agit d'un mot :

\emph{VIENS !} 

Sans même s'en rendre compte, le garçon s'éloigna en chancelant du rebord, les yeux toujours braqués sur le phénix, tout son corps tremblant et tendu, ses poings serrés puis relâchés, il reculait, il s'éloignait.

Le phénix croassa à nouveau, un son désespéré, plaintif. Il ne fut pas compris comme un mot, cette fois, mais comme une émotion, un écho de tout ce que Harry avait ressentit au sujet d'Azkaban, de chaque tentation d'\emph{agir} , de juste \emph{faire}  quelque chose, du besoin de désespéré de faire quelque chose \emph{maintenant} , de ne plus repousser, tout cela dans le cri d'un oiseau.

\emph{Allons-y. Il est temps.}  La voix qui parlait venait de l'intérieur de Harry, pas du phénix ; de si profond en lui qu'il ne pouvait lui donner un nom comme 'Gryffondor'.

Tout ce qu'il avait à faire était d'avancer, de toucher les serres du phénix, il serait alors là où il devait être, là où il ne cessait de songer devoir être, au fond de la fosse centrale d'Azkaban. Harry pouvait voir l'image dans son esprit, brillante d'un insoutenable clarté, l'image de lui-même souriant d'un soulagement soudain alors qu'il rejetait toutes ses peurs et \emph{choisissait}  -

" Mais je...", murmura Harry, sans même se rendre compte de ce qu'il disait. Il leva ses mains tremblantes pour essuyer les larmes qui avaient jailli de ses yeux, face au phénix qui flottait devant lui à grands battements d'ailes. " Mais je... j'ai d'autres personnes que je dois aussi sauver, d'autres choses que je dois faire... "

L'oiseau de feu laissa échapper un cri perçant et le garçon fléchit comme s'il avait reçu un coup. Ce n'était pas un ordre, ce n'était pas une objection, c'était le \emph{savoir} ...

Les couloirs éclairés par la faible lueur orange.

Il sentait comme un désir étouffant dans sa poitrine, celui de juste \emph{agir}  et d'en avoir fini. Il mourrait peut-être, mais s'il ne mourrait pas il se sentirait à nouveau \emph{propre} . Il aurait des principes qui seraient autres chose que des excuses pour l'inaction. C'était \emph{sa}  vie. Son choix d'en faire usage, s'il le souhaitait. Il pourrait le faire le faire quand il le voudrait...

...s'il n'était pas quelqu'un de bien.
\par\noindent\rule{\textwidth}{0.4pt}
Le garçon demeura sur le toit, ses yeux braqués sur deux points de feu. Les étoiles avaient peut-être eu le temps de changer de constellation pendant qu'il était resté ici, tourmenté par cette décision...

…qui ne voulait pas...

...changer.

Les yeux passèrent un instant sur les étoiles au-dessus de lui puis il regarda le phénix.

" Pas encore, " dit le garçon d'une voix à peine audible. " Pas encore. J'ai encore trop à faire. Reviens plus tard, s'il te plaît, lorsque j'en aurai trouvé d'autres capables de lancer le Véritable Patronus – dans six mois, peut-être... "

Sans un mot, sans un son, une sphère de feu entoura la forme de l'oiseau, dans des craquements et des éclats de veines blanches et pourpres, comme si ce qui se trouvait en son centre devait être consumé, et lorsque le feu se dissipa dans une fumée grise, il n'y avait plus de phénix.

Il n'y eut plus que du silence au sommet de la tour Serdaigle. Le garçon rabaissa ses mains de ses oreilles, ne s'arrêtant que pour essuyer ses joues humides.

Lentement, il se retourna...

Puis s'écria et fit un bon tel qu'il tomba presque de la tour, bien que ce faux pas aurait été sans conséquence avec un tel sorcier présent.

" Ainsi en fut-il, " dit Albus Dumbledore presque dans un soupir. " Ainsi en fut-il. " Fumseck était sur son épaule et regardait là où s'était trouvé l'autre phénix d'un regard aviaire indéchiffrable.

" \emph{Que faites-vous ici ? "} 

" Ah ? " dit le vieil homme depuis le coin opposé de la plate-forme. " J'ai ressentit la présence d'une créature inconnue de Poudlard alors je suis venu voir, bien sûr. " Lentement, la main tremblante du vieux sorcier monta afin d'enlever les lunettes en demi-lunes et son autre main essuya ses yeux et son front en usant de sa manche. " Je n'osais... je n'osais pas parler... je savais, je savais qu'entre tous ce choix devait être le tien... "

Une étrange appréhension commençait à se répandre en Harry, à sourdre en lui comme une sensation de nausée.

" Que tout dépendait de cela, " dit Albus Dumbledore toujours de ce presque-soupir, " je le savais. Mais quel choix menait vers les ténèbres, cela, je n'ai pu le deviner. Au moins, le choix était tien. "

" Je ne... " dit Harry, et sa voix s'arrêta.

Une terrible hypothèse, qui gagnait en crédibilité...

" Le phénix vient, " dit le vieux sorcier. " À ceux qui se battront, à ceux qui agiront même au prix de leur vie, le phénix vient. Les phénix ne sont pas sages, Harry, ils ne connaissent aucun moyen de nous juger autre que d'être témoins de ce choix. Je pensais courir à ma mort lorsque le phénix m'a mené à Grindelwald. J'ignorais que Fumseck me soutiendrait, me soignerait et resterait à mes côtés... " la voix du vieux sorcier chevrota un instant. " Cela reste tu... il te faut comprendre, Harry, pourquoi cela reste tu... si l'intéressé savait, le phénix ne pourrait juger. Mais à toi, Harry, je peux maintenant le dire, car le phénix ne vient qu'une fois. "

Le vieux sorcier traversa le sommet de la tour Serdaigle vers un garçon enraciné dans une horreur croissante, croissante et absolue.

\emph{Lors de mon duel face à Grindelwald je ne pouvais vaincre, seulement le combattre de longues heures durant jusqu'à ce qu'il s'effondre d'épuisement, et j'en serais ensuite mort s'il n'y avait eu Fumseck...} 

Harry ne sut qu'il s'était mis à parler qu'après que le murmure lui ait échappé -

" Alors j'aurais \emph{pu} ... "

" Aurais-tu ? " dit le vieux sorcier, d'une voix aux accents bien plus âgés qu'à l'habitude. " Voilà maintenant trois fois qu'un phénix est venu pour l'un de mes élèves. Une l'a renvoyé, et son deuil, je pense, l'a brisée. Et le dernier était un cousin de ta jeune amie Lavande Brown, et il... " la voix du vieux sorcier se brisa. " Il n'est pas revenu, pauvre John, et il n'a sauvé aucun de ceux qu'il souhaitait sauver. On dit, chez les rares érudits des légendes sur les phénix, que pas un sur quatre ne revient de son épreuve. Et même si tu avais survécu – la vie qu'il te faut vivre, Harry James Potter-Evans-Verres – les choix que tu dois faire, le chemin que tu dois suivre – toujours devoir entendre le cri du phénix – qui peut dire que cela ne t'aurait pas rendu fou ? " Le vieux sorcier releva à nouveau sa manche et se la passa une fois de plus sur le visage. " La compagnie de Fumseck m'apportait plus de joie avant que je combatte Voldemort. "

Le garçon ne semblait pas écouter, ses yeux étaient entièrement sur l'oiseau rouge et or perché sur l'épaule du vieux sorcier. " Fumseck ? " dit le garçon d'une voix tremblante. " Pourquoi est-ce que tu ne me regardes pas, Fumseck ? "

Le phénix leva la tête pour regarder le garçon avec curiosité puis se retourner à nouveau vers son maître.

" Tu vois ? " dit le vieux sorcier. " Il ne te rejette pas. Fumseck ne s'intéresse peut-être plus à toi de cette façon particulière à présent, et il sait... " le sorcier sourit sèchement, " que tu n'es pas exactement loyal envers son maître. Mais celui qu'un phénix approche... ne peut être celui qu'un phénix déteste. " La voix du sorcier redevint un soupir. " On n'a jamais vu d'oiseau sur l'épaule de Godric Gryffondor. Même si ce n'est pas révélé parmi ses secrets, je pense qu'il a renvoyé son phénix avant de choisir le rouge et l'or pour couleurs. Peut-être la culpabilité le poussa-t-il à faire plus que ce qu'il aurait osé poursuivre autrement. Ou peut-être cela lui a-t-il enseigné l'humilité et un respect pour la faillibilité humaine, pour l'échec... " le vieux sorcier inclina la tête. " J'ignore véritablement si ton choix fut sage. J'ignore véritablement si c'était la bonne chose à faire, ou la mauvaise. Si je l'avais su, Harry, j'aurais parlé. Mais je... " la voix de Dumbledore se brisa alors. " je ne suis rien d'autre qu'un jeune garçon idiot devenu un vieil homme idiot, et je ne possède aucune sagesse. "

Harry n'arrivait pas à respirer, la nausée semblait emplir et déborder de son corps depuis son estomac solidifié. Il était soudainement et atrocement certain qu'il avait échoué, d'un échec définitif, ce soir là...

Le garçon pivota et courut jusqu'au bord de la tour Serdaigle. " Reviens ! " sa voix se brisa et devint un glapissement. " \emph{Reviens ! } "
\par\noindent\rule{\textwidth}{0.4pt}
\emph{Après-coup final :} 

Elle s'éveilla dans un hoquet d'horreur, elle s'éveilla un cri muet sur les lèvres et aucun mot ne vint, elle ne pouvait comprendre ce qu'elle avait vu,\emph{ elle ne pouvait comprendre ce qu'elle avait vu...} 

" Quelle heure est-il ? " murmura-t-elle.

Son réveil-matin en or et serti de joyaux lui murmura en retour : " Environ onze heures. Rendors-toi. "

Ses draps et sa chemise de nuit étaient inondés de sueur, elle prit sa baguette à côté de son oreiller et se sécha avant d'essayer de se rendormir et de finir par y parvenir.

Sybille Trelawney se rendormit.

Dans la Forêt Interdite, un centaure éveillé par une appréhension incertaine ne cessait de parcourir le ciel nocturne du regard et, n'y ayant trouvé que des questions et aucune réponse, d'un repliement de ses nombreuses jambes, Firenze se rendormit.

Dans les terres lointaines d'Asie magique, une vieille sorcière nommée Fan Tong, en plein repos diurne, dit à son anxieux arrière-arrière petit-fils qu'elle allait bien, que ce n'avait été qu'un cauchemar, et elle se rendormit.

Dans un pays où les moldus ne recevaient jamais aucune lettre, une petite fille trop jeune pour avoir un nom à elle se fit bercer dans les bras de sa mère agacée mais aimante jusqu'à ce qu'elle arrête de pleurer et se rendorme.

Aucun d'eux ne dormit bien.

