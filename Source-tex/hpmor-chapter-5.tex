
\chapter{L'Erreur Fondamentale d'Attribution}

J.K. Rowling a ses yeux braqués sur vous. Pouvez-vous sentir leur présence ? Elle lit votre esprit avec ses Rayons Rowling.
\par\noindent\rule{\textwidth}{0.4pt}
"Hermione, il n'a que onze ans."\\"Toi aussi."\\"Moi je ne compte pas."
\par\noindent\rule{\textwidth}{0.4pt}
Le magasin de Moke était un petit magasin pittoresque (certains auraient même dit mignon) confortablement installé derrière un étal de légumes, lui-même derrière un magasin de gants magiques se trouvant sur une route adjacente au Chemin de Traverse. C'était décevant mais le propriétaire n'était pas un vieil homme mystérieux et desséché, juste une jeune femme à l'air nerveux et portant des robes jaunes passées. Elle tenait pour le moment une Super Bourse en Peau de Moke QX31 dont l'avantage majeur était qu'il avait une Ouverture Élargissante ainsi qu'un Charme d'Extension Indétectable : vous pouviez y mettre de grands objets, mais le volume total était tout de même limité.

Harry avait \emph{insisté}  pour venir ici tout de suite, avant toute autre chose - insisté autant que possible sans pour autant éveiller de soupçons chez McGonagall. Harry avait quelque chose qu'il devait mettre dans la bourse aussi vite que possible. Ce n'était pas le sac de Gallions que McGonagall l'avait autorisé à retirer de Gringotts. C'étaient tous les autres Gallions que Harry avec subrepticement fourrés dans sa poche après être accidentellement tombé dans un tas de pièces d'or. Cela \emph{avait}  été un véritable accident, mais Harry n'était pas du genre à passer outre une opportunité... même si le geste avait été grandement impulsif. Depuis lors Harry avait malhabilement transporté le sac de Gallions autorisé contre sa poche de pantalon, afin que tout tintement semble venir de là où il fallait.

Ce qui ne répondait pas à la question de comment il avait pouvoir mettre les \emph{autres}  pièces dans la Peau sans se faire prendre. Les pièces d'or étaient peut-être siennes, mais elles \emph{étaient}  tout de même volées – auto-volées ?

Harry releva les yeux vers la Super Bourse en Peau de Moke QX31 posée sur le comptoir face à lui. "Puis-je l'essayer un moment ? Pour vérifier qu'elle fonctionne, euh, de façon fiable ?" Il agrandit ses yeux et arbora l'expression d'un garçonnet innocent et enjoué.

Bien sûr, après que Harry ait mis le sac de Gallions dans la Peau, y ait plongé sa main, ait murmuré "sac d'or" et ait extrait le sac, le tout dix fois de suite, McGonagall s'éloigna et détourna la tête pour regarder d'autres objets du magasin, et le propriétaire se mit à regarder sa montre.

Harry déposa le sac d'or dans la peau avec sa main \emph{gauche}  ; sa main \emph{droite}  sortit de sa poche, tenant fermement quelques unes des pièces d'or ; il plongea cette même main dans la peau, y déposa les Gallions, et (en murmurant "sac d'or") récupéra le sac original. Puis le sac vint dans sa main \emph{gauche} , puis fut à nouveau déposé dans la peau, et la main \emph{droite}  de Harry replongea dans sa poche...

McGonagall lui jeta un coup d'oeil, mais Harry parvint à ne pas broncher ni à se figer, et elle n'eut pas l'air de remarquer quoi que ce soit. Mais nous n'étiez jamais \emph{vraiment}  sûr, avec les adultes dotés d'un sens de l'humour. Il fallut trois itérations pour finir le travail, et Harry estima qu'il avait réussi à se voler à peu près trente Gallions.

Harry releva le bras, essuya un peu de sueur de son front, et exhala : "Je voudrais celle-ci, s'il vous plaît."

Quinze Gallions plus léger (apparemment le double du prix d'une baguette de sorcier) et une Super Bourse en Peau de Moke QX31 plus lourd, Harry et McGonagall se frayèrent un chemin jusqu'à la porte. La poignée de la porte se transforma en une main et les salua tandis qu'ils partaient, extrudant son bras d'une façon qui écoeura légèrement Harry.

Puis, malheureusement...

"Êtes-vous \emph{vraiment}  Harry Potter ?" souffla le vieil homme, une larme énorme coulant le long de sa joue. "Vous ne mentiriez pas à ce sujet tout de même ? Mais j'ai entendu des rumeurs comme quoi vous n'aviez pas \emph{vraiment}  survécu au Sortilège de la Mort et que c'est pour cela que personne n'avait plus entendu parler de vous depuis."

... il semblait que le sort de déguisement de McGonagall n'était pas parfaitement efficace contre les praticiens magiques expérimentés.

McGonagall avait posé une main sur l'épaule de Harry et l'avait tiré dans la ruelle la plus proche au moment où elle avait entendu "Harry Potter ?" Le vieil homme les avait suivi, mais au moins personne d'autre ne semblait avoir entendu.

Harry étudia la question. \emph{Était} -il vraiment Harry Potter ? "Je ne sais que ce qu'on m'en a dit," dit Harry. "Ce n'est pas comme si je me souvenais de ma naissance." Il se passa la main sur le front. "J'ai eu cette cicatrice pour aussi longtemps que je me souvienne, et on m'a dit que mon nom était Harry Potter pour aussi longtemps que je me souvienne. Mais," dit pensivement Harry, "si il y a déjà assez de raisons de postuler l'existence d'une conspiration, il n'y a pas de raison pour laquelle ils ne trouveraient pas un autre orphelin sorcier et l'élèveraient en lui faisant croire qu'\emph{il}  était Harry Potter -"

D'exaspération, McGonagall se masqua le visage. "Vous ressemblez exactement à votre père James dans sa première année à Poudlard, sauf que vous avez les yeux de votre mère, Lily. Et je puis démontrer en me basant uniquement sur votre \emph{personnalité}  que vous êtes \emph{sans aucun doute}  lié au Fléau de Gryffondor."

"\emph{Elle } pourrait faire partie de la conspiration," observa Harry.

"Non," chevrota le vieil homme. "Elle a raison. Vous avez les yeux de votre mère."

"Hmmmm". Harry fronça les sourcils. "Je suppose que \emph{vous}  pourriez en faire partie aussi -"

"Assez, M. Potter." dit McGonagall.

Le vieil homme leva une main comme si il allait toucher Harry, mais la laissa retomber. "Je suis juste content que vous soyez en vie," murmura-t-il. "Merci, Harry Potter. Merci pour ce que vous avez fait...je vous laisse tranquille maintenant."

Et le battement de sa cane sur la pavé s'éloigna lentement, hors de la ruelle et le long du Chemin de Traverse.

McGonagall jeta un coup d'oeil autour d'elle, une expression tendue et sinistre sur le visage. Automatiquement, Harry regarda à son tour autour de lui. Mais la ruelle semblait vide de tout sauf de vieilles feuilles, et depuis l'entrée menant au Chemin de Traverse on ne pouvait voir que des passants à la démarche pressée.

McGonagall se détendit enfin. "Ce n'était pas bien joué," dit-elle d'une voix basse. "Je sais que vous n'avez pas l'habitude de tout ça, M. Potter, mais les gens se soucient de vous. S'il vous plaît, soyez gentil avec eux."

Harry regarda ses chaussures. "Ils ne devraient pas", dit-il avec une nuance d'amertume. "Se soucier de moi, je veux dire."

"Vous les avez sauvés de Vous-Savez-Qui," dit McGonagall. "Comment pourraient-ils y être indifférents ?"

Harry regarda McGonagall et soupira. "J'imagine qu'il n'y a aucune chance que si je dis \emph{Erreur d'attribution fondamentale}  vous ayez la moindre idée de ce dont je parle."

McGonagall secoua la tête "Non, mais expliquez-moi, je vous en prie."

"Eh bien..." dit Harry, essayant de trouver comment bien décrire cette section particulière de la science Moldue. "Supposez que vous arriviez au travail et que vous voyez votre collègue donner des coups de pied dans son bureau. Vous vous dites 'que cette personne doit être colérique !'. Votre collègue pense au fait que quelqu'un l'a poussé contre un mur alors qu'il se rendait au travail, puis lui a crié dessus. Il se dit que \emph{n'importe qui}  serait en colère suite à ça. Lorsque nous regardons les autres, nous voyons des traits de personnalités qui expliquent leur comportement, mais lorsque nous nous observons nous-mêmes, nous voyons des circonstances qui expliquent notre comportement. Les histoires personnelles des gens ont un sens de leur point de vue, de l'intérieur, mais nous ne voyons pas les histoires personnelles des gens flottant derrière eux dans les airs. Nous ne les voyons que dans une situation, et ne voyons pas le comportement qu'ils auraient dans une autre. L'erreur d'attribution fondamentale est donc que nous expliquons par des traits permanents et durables ce qui serait mieux expliqué par des circonstances et par un contexte." Il y avait d'élégantes expériences qui confirmaient ce fait, mais Harry ne comptait pas en arriver là.

Les sourcil de McGonagall s'élevèrent. "Je pense que je comprends..." dit-elle lentement. "Mais qu'est-ce que ça a à voir avec vous ?"

Harry donna un coup de pied dans le mur de brique de la ruelle, suffisamment fort pour se faire mal au pied. "Les gens pensent que je les ai sauvés de Vous-Savez-Qui parce que je suis une sorte de grand guerrier de la Lumière."

"Celui au pouvoir capable de vaincre le Seigneur des Ténèbres..." murmura McGonagall, une ironie dans la voix que Harry ne comprit alors pas.

"Oui," dit Harry, la voix divisée entre la frustration et la contrariété, "comme si j'avais détruit le seigneur des Ténèbres parce que je possède une sorte de caractéristique permanente de type destructeur-de-Seigneur-des-Ténèbres. J'avais quinze mois à l'époque ! Je ne \emph{sais}  pas ce qui s'est passé, mais je \emph{devine}  que ça a quelque chose à voir avec, comme on dit, des circonstances environnementales contingentes. Et certainement rien à avoir avec ma personnalité. Les gens ne se soucient pas de \emph{moi} , ils ne font même pas attention à \emph{moi} , ils veulent serrer la main à une \emph{mauvaise explication} ." Harry s'arrêta, et regarda McGonagall. "Savez-\emph{vous}  ce qui s'est vraiment passé ?"

"J'\emph{ai}  formé une conjecture..."dit McGonagall. "Après vous avoir rencontré."

"Oui ?"

"Vous avez triomphé contre le Seigneur des Ténèbres en étant plus épouvantable que \emph{lui} , et avez survécu au Sortilège de la Mort en étant plus horrible que la Mort."

"Ha. Ha. Ha." Harry donna un nouveau coup de pied dans le mur.

McGonagall gloussa. "Allons chez Madame Malkin. Je pense que vos habits Moldus attirent peut-être l'attention."

Ils rencontrèrent deux sympathisants de plus sur le chemin.

McGonagall s'arrêta devant la porte des Robes de Madame Malkin. C'était une devanture véritablement ennuyeuse, principalement faite de briques rouges comme des briques ordinaires, et des fenêtres de verre montraient des robes noires unies. Pas de robes brillantes ou changeantes ou tournantes ou radiantes d'étranges rayons qui paraîtraient traverser votre chemise et vous chatouiller. Juste des robes noires unies - ou du moins c'était tout ce que vous pouviez voir à travers la fenêtre. La porte était grande ouverte, comme pour dire qu'il n'y avait ici aucun secret et rien à cacher."

"Je vais m'absenter quelques minutes pendant qu'on prend vos mesures pour les robes," dit McGonagall. "Cela vous convient-il ?"

Harry hocha la tête. Il détestait le shopping de vêtements avec une ardente passion et ne pouvait pas blâmer McGonagall si elle partageait son aversion.

McGonagall tapota le front de Harry avec sa baguette. "Vous devrez être clairement visible aux sens de Madame Malkin, j'enlève donc le sort d'Obfuscation."

"Euh..." dit Harry. Cela le tracassait un peu.

"Je suis allé à Poudlard avec Madame Malkin," dit McGonagall. "Même alors, elle était l'une des personne les plus \emph{composées}  que je connaisse. Elle ne hausserait pas un sourcil si Vous-Savez-Qui lui-même entrait dans son magasin." La voix de McGonagall semblait provenir d'un souvenir, et son ton était approbateur. "Madame Malkin ne vous embêtera pas, et elle ne laissera personne d'autre vous embêter."

"Où allez-\emph{vous}  ?" s'enquit Harry. "Juste au cas où, vous savez, quelque chose se passait."

McGonagall donna un regard dur et sceptique à Harry. "Je vais \emph{là} ," dit-elle pointant du doigt un bâtiment en face qui arborait le dessin d'un tonnelet de bois, "pour m'offrir un verre, dont j'ai désespérément besoin. \emph{Vous}  allez voir vos mesures prises pour les robes, \emph{rien d'autre} . Je vais revenir vous surveiller \emph{bientôt } et je m'\emph{attends}  à voir le magasin de Madame Malkin toujours debout et pas en feu de \emph{quelque façon que ce soit} ."

Madame Malkin était une vielle femme animée qui ne pipa mot lorsqu'elle vit la cicatrice sur son front, et jeta un regard sévère à une assistante lorsque celle-ci sembla être sur le point de dire quelque chose. Madame Malkin exhiba un ensemble de pièces de tissus animées qui se contorsionnaient et semblaient servir de mètre ruban, et commença à travailler.

A côté de Harry, un jeune garçon pâle au visage pointu et aux cheveux blond-blanc \emph{supercools } semblait être dans la phase finale d'un processus similaire. L'un des deux assistants de Malkin examinait le garçon au cheveux blancs avec attention, ainsi que la robe aux motif en damier qu'il portait ; à l'occasion elle touchait un coin de la robe avec sa baguette, et la robe se relâchait ou se resserrait.

"Bonjour," dit le garçon, "Poudlard, toi aussi ?"

Harry pouvait prédire où cette conversation allait le mener, et dans une demi seconde de frustration il décida que c'en était assez.

"Grands dieux," murmura Harry, "ce n'est pas possible." Il laissa ses yeux se dilater. "Votre... nom, monsieur ?"

"Draco Malfoy," dit Draco Malfoy, l'air un peu perplexe.

"C'\emph{est}  vous ! Draco Malfoy. Je - Je ne pensais pas avoir un jour l'honneur, monsieur." Harry aurait aimé pouvoir faire sortir des larmes de ses yeux. Les autres commençaient généralement à pleurer à ce moment là de la conversation.

"Oh," dit Draco d'un ton légèrement confus. Puis ses lèvres s'étirèrent en un sourire suffisant. "Il est agréable de rencontrer quelqu'un qui connaît sa place."

L'une des assistantes, celle qui semblait avoir reconnu Harry, fit un bruit de gloussement étouffé.

Harry continua son murmure. "Je suis ravi de vous rencontrer M. Malfoy. Juste ineffablement ravi. Et aller à Poudlard la même année que vous ! Mon coeur se pâme."

Oups. Cette dernière partie avait peut-être été un peu étrange, comme si il flirtait avec Draco.

"Et mon propre coeur est illuminé de constater que je puis m'attendre à être traité avec le respect dû à la famille Malfoy," renvoya Malfoy avec un sourire similaire à l'un de ceux que le plus haut des rois pourrait octroyer au plus bas de ses sujets, si ce sujet était honnête en dépit de sa pauvreté.

Eh... Mince, Harry avait du mal à inventer sa prochaine réplique. Eh bien, tout le monde \emph{voulait}  serrer la main de Harry Potter, alors - "Lorsque mes vêtements seront apprêtés, monsieur, accepterez-vous de me serrer la main ? Rien d'autre ne saurait parachever ce jour, non, ce mois, et à vrai dire, ma vie entière."

Draco le foudroya du regard. "Je pense que vous demandez de ma personne une familiarité bien déplacée ! Qu'avez vous jamais fait pour la famille Malfoy qui vous donne droit à une pareille requête ?"

\emph{Oh, je vais tellement essayer cette routine sur la prochaine personne qui essaie de me serrer la main} . Harry inclina sa tête. "Non, non monsieur, je comprends. Je suis désolé de vous l'avoir demandé. Je devrais plutôt me sentir honoré de nettoyer vos bottes."

"En effet," lâcha Draco. Son visage dur s'éclaira plus ou moins. "Cela étant dit, votre souhait est compréhensible. Dites moi, dans quelle Maison pensez-vous être trié ? Je suis destiné à Serpentard bien sûr, comme mon père Lucius avant moi. Et pour vous, je devine la Maison Poufsouffle, ou peut-être la Maison Elfe."

Harry fit un sourire penaud. "Professeur McGonagall dit que je suis la personne la plus Serdaigle qu'elle ait jamais vu ou dont elle ait jamais entendu parler dans des légendes, à tel point que Rowena elle-même me dirait de sortir plus, quoi que \emph{cela}  veuille dire, et que je finirai sans aucun doute à Serdaigle si le Choixpeau magique ne crie pas d'horreur si fort qu'aucun d'entre nous ne peut comprendre ce qu'il dit, fin de citation."

"Wow," dit Draco, l'air légèrement impressionné. Il fit une sorte de soupir mélancolique. "Votre flatterie était excellente, ou du moins je le pensais - vous réussiriez à Serpentard aussi. C'est généralement devant mon père que les gens s'aplatissent. J'\emph{espère}  que les autres Serpentards me lècheront les bottes maintenant que je suis à Poudlard... j'imagine donc que c'est bon signe."

Harry toussa. "A vrai dire, désolé, mais je ne sais absolument pas qui tu es."

"\emph{Non mais franchement}  !" dit Draco, violemment déçu. "Pourquoi ferais-tu ça ?" les yeux de Draco s'élargirent dans un élan de suspicion soudain. "Et comment peux-tu ne \emph{pas}  connaître les Malfoys ? Et quels sont ces \emph{vêtements } que tu portes ? Tes parents sont-ils \emph{Moldus}  ?"

"Deux des mes parents sont morts", dit Harry. Il éprouva un pincement au coeur. Lorsqu'il le formulait ainsi - "Mes deux autres parents sont des Moldus, et ce sont eux qui m'ont éduqué."

"\emph{Quoi}  ?" dit Draco. "Qui \emph{es} -tu ?"

"Harry Potter, ravi de te faire ta connaissance."

"\emph{Harry Potter ?"}  haleta Draco. "\emph{Le}  Harry -" et le garçon s'arrêta brusquement.

Il y eut un bref silence.

Puis, avec un enthousiasme éclatant : "Harry Potter ? \emph{Le}  Harry Potter ? Mon dieu, j'ai toujours voulu te rencontrer !"

L'assistante qui s'occupait de Draco émit un son qui donnait l'impression qu'elle s'étranglait mais elle continua son travail, soulevant les bras de Draco pour enlever la robe damée avec attention.

"Tais toi," suggéra Harry.

"Pourrais-je avoir un autographe ? Non, attends, d'abord je veux une photo avec toi !"

"Tais\emph{toi} tais\emph{toi} tais\emph{toi} "

"Je suis juste tellement, inexprimablement \emph{enchanté}  de te rencontrer !"

"Prends feu et meurs."

"Mais tu es Harry Potter, le glorieux sauveur du monde magique, vainqueur du Seigneur des Ténèbres ! Le héros de tous, Harry Potter ! J'ai toujours voulu devenir comme toi quand je serai plus grand pour pouvoir vaincre des Seigneur des Ténèbres moi aus-"

Draco s'interrompit au beau milieu de sa phrase. Son visage de pétrifia dans une expression d'horreur absolue.

Grand, cheveux blancs, froidement élégant dans des robes noires de la meilleure des qualités. Une main enserrant une cane à poignée d'argent qui par vertu d'être dans cette main prenait l'apparence d'une arme mortelle. Ses yeux considérèrent la pièce avec le calme d'un éxécuteur, d'un homme pour qui tuer n'était pas douloureux, ni même délicieusement interdit, mais une activité aussi routinière que respirer.

C'était l'homme qui avait, à cet instant, nonchalamment franchi le seuil de la porte ouverte.

"Draco," dit l'homme, d'une voix basse et très en colère, "\emph{qu'es-tu}  en train de \emph{dire}  ?"

En une demi seconde de panique compatissante, Harry formula un plan de secours.

"Lucius Malfoy !" haleta Harry Potter. "\emph{Le}  Lucius Malfoy ?"

L'un des assistants de Malkin dut détourner le regard et contempler le mur.

Des yeux froid et meurtiers le considéraient. "Harry Potter."

"Je suis tellement, tellement honoré de vous rencontrer !"

Les yeux noirs s'élargirent, et la surprise choquée remplaça la menace mortelle.

"Votre fils m'a \emph{tout}  dit de vous," continua Harry avec grande animation, sachant à peine ce qui sortait de sa bouche, essayant juste de parler le plus vite possible. "Mais bien sûr je savais tout de vous bien avant cela, tout le monde vous connaît, Lucius Malfoy ! Le lauréat le plus honoré de Serpentard, j'ai moi-même pensé à aller à Serpentard juste parce que j'ai entendu que vous y étiez enfant -"

"\emph{Que dites-vous, M. Potter ?" } dit un quasi-cri depuis l'extérieur du magasin, et le Professeur McGonagall déboula une seconde plus tard.

Il y avait une telle horreur sur son visage que la bouche de Harry s'ouvrit automatiquement, puis se bloqua sur rien-à-dire.

"Professeur McGonagall !" s'écria Draco. "Est-ce vraiment vous ? Mon père m'a tellement parlé de vous, j'ai pensé à me faire trier à Gryffondor afin de -"

"\emph{Quoi } ?" hurlèrent Lucius Malfoy et le Professeur McGonagall parfaitement à l'unison, debout l'un à coté de l'autre. Leurs têtes pivotèrent symmétriquement et ils se regardèrent, puis ils s'éloignèrent l'un de l'autre comme si ils interprétaient une danse synchronisée.

Il y eut une grande agitation alors que Lucius s'emparait de Draco et le traînait hors du magasin.

Puis tout fut silencieux.

McGonagall regarda le petit verre de vin qu'elle avait en main. Il était horizontal, oublié dans sa galopade, et seules quelques gouttes d'alcool s'y accrochaient encore.

McGonagall s'avança dans le magasin jusqu'à ce qu'elle fit face à Madame Malkin.

"Madame Malkin," dit McGonagall d'une voix calme. "Que s'est-il passé ici ?"

Madame Malkin la regarda silencieusement pendant quatre secondes, puis elle craqua. Elle tomba contre le mur, riant plus qu'elle ne respirait, ce qui fit craquer ses deux assistantes, et l'une d'entre elles tomba sur ses mains et ses genoux, prise d'un fou rire hystérique.

McGonagall se retourna lentement et regarda Harry avec une expression froide. "Je vous laisse seul pendant cinq minutes. Cinq minutes, M. Potter, dixit cette horloge."

"Je ne faisais que plaisanter", protesta Harry, alors que les rires hystériques continuaient non loin.

"\emph{Draco Malfoy a dit face à son père qu'il souhaitait être trié à Gryffondor ! } Plaisanter \emph{ne suffit pas } à en \emph{venir}  là !" McGonagall pausa, respirant lourdement. "Quelle partie de 'se faire prendre ses mesures' avez-vous interprété comme voulant dire \emph{s'il vous plaît jetez un sort de Confusion à l'univers entier !} "

"Il était dans un contexte situationnel où ses actions avaient un sens de son point de vue -"

"Non. N'expliquez pas. Je ne veux pas savoir ce qui s'est passé ici. Jamais. Il y a certaines choses que je ne suis pas censée savoir, et c'est une de ces choses. Quelle que soit la force de chaos de démoniaque qui vous habite, elle est \emph{contagieuse} , et je ne veux pas finir comme ce pauvre Draco Malfoy, cette pauvre madame Malkin, ou ses deux pauvres assistantes."

Harry soupira. Il était clair que le Professeur McGonagall n'était pas d'humeur à prêter l'oreille à des explications raisonnables. Il regarda Madame Malkin, qui s'appuyait toujours contre le mur, et ses deux assistantes, qui étaient maintenant \emph{toutes deux}  à genoux, et finalement son propre corps entouré de mètre ruban.

"Je n'ai pas encore fini avec les mesures," dit Harry avec gentillesse. "Pourquoi ne retourneriez-vous pas prendre un autre verre ?"

