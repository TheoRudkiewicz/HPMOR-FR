
\chapter{Échec, partie 1}

Le Seigneur des Ténèbres riait.

Le rire fou du professeur de Défense venait de nulle part, si fluté, si terrible ; c'était le rire de Voldemort à présent, le rire du Seigneur des Ténèbres, sans dissimulation, sans retenue.

L'esprit de Harry était en piteux état. Son regard ne quittait pas l'endroit où s'était trouvé Albus Dumbledore. Le sentiment d'horreur en lui était trop immense pour qu'il puisse comprendre, pour qu'il puisse réfléchir. Son esprit n'avait cesse d'essayer de remonter le cours de temps, d'annuler la réalité, mais ce genre de magie n'existait pas et la réalité demeura.

Il avait perdu, il avait perdu Dumbledore. Il n'y aurait pas de seconde chance, et cela signifiait que la guerre était aussi perdue.

Et le Seigneur des Ténèbres continuait de rire.

"Ah, ah hah, ah hah hah ha ! Professeur Dumbledore, ah, professeur Dumbledore, quelle fin appropriée à notre jeu !" Un autre éclat de rire. "Le mauvais sacrifice, même à la fin, car la pièce que tu as tout abandonné pour sauver était déjà entre mes mains ! Le mauvais piège depuis le début, car j'aurais pu abandonner ce corps n'importe quand ! Ah, hahahahaha, aha ! Tu n'as jamais appris la ruse, pauvre vieil imbécile."

"Vous…" une voix émergea de la gorge de Harry. "Vous…"

"Ahahahaha ! Mais oui, petit enfant, si tu m'as accompagné dans cette aventure, c'est pour être mon otage, c'était le seul but de ta présence ici. Ha, hahahaha ! Il te manque des décennies pour pouvoir jouer à ce jeu contre le véritable Tom Jedusor, petit." Le Seigneur des Ténèbres ôta sa capuche et sa tête devint visible ; il commença à enlever le reste de la Cape. "Et maintenant, petit, \emph{tu m'as aidé, bien aidé, et il est donc temps de resssussciter ton amie enfant-fille. Pour tenir parole.} " Le sourire du Seigneur des Ténèbres était froid, très froid. "J'imagine que tu as des doutes ? Rends-toi bien compte que je pourrais te tuer sur-le-champ, car il n'y a plus de directeur à Poudlard pour s'en rendre compte. Doutes de moi tant que tu veux, mais souviens toi de ça." Sa main tenait à nouveau le pistolet. "Maintenant viens avec moi, enfant insensé."

Et ils partirent.

Ils repartirent par la porte qui menait à la salle des potions, et le Seigneur des Ténèbres banni le feu pourpre à nouveau embrasé d'un mouvement de baguette. Ils traversèrent les chambres où s'était trouvé l'Épouvantard, les pièces d'échec en morceaux, et celle aux clés, calcinée. Le Seigneur des Ténèbres lévita jusqu'à la trappe et Harry monta tant bien que mal à sa suite le long de l'escalier de feuilles en spirales, les tiges du Filet du Diable comme apeurées sous ses pieds. Le Survivant faisait de son mieux pour ne pas éclater en sanglots, et ses motifs de pensée obscurs ne l'aidaient pas ; peut-être parce que Voldemort n'avait jamais connu la culpabilité.

Ils dépassèrent l'immense Inferi à trois têtes qui, d'un murmure du Seigneur des Ténèbres, s'écroula sur la trappe et redevint un cadavre.

Ils dépassèrent Severus Rogue, qui montait la garde, et qui leur dit à tous deux que c'était ce qu'il faisait, et qu'ils devaient partir, sans quoi il leur enlèverait des points de Maison.

Le Seigneur des Ténèbres dit "\emph{Hyakuju montauk} " sans marquer d'arrêt et agita vivement sa baguette ; Severus vacilla avant de se redresser, comme mort, et de se placer à nouveau devant la porte.

"Qu'est-ce que…" dit Harry en le suivant, "qu'est-ce que vous avez…"

"Seulement mon devoir envers mon fidèle serviteur. Je ne le tuerai pas, comme je te l'ai promis." Le Seigneur des Ténèbres rit à nouveau.

"Les otages…" dit Harry. Il avait du mal à garder une voix neutre. "Les élèves, vous avez dit que vous alliez annuler ce qui va les tuer…"

"\emph{Oui. Cessse de t'en ssoucier. Nous le ferons en ssortant.} "

"En sortant ?"

"Nous partons, petit." Le Seigneur des Ténèbres souriait toujours.

Ces nouvelles inquiétudes furent noyées dans un océan d'émotions négatives.

Le Seigneur des Ténèbres consultait maintenant ce qu'il avait appelé la Carte de Poudlard. Les lignes manuscrites à sa surface semblaient se déplacer à mesure qu'ils avançaient. Une partie de l'esprit de Harry, qui avait envisagé des plans au cas où ils croiseraient une patrouille d'Aurors (que le Seigneur des Ténèbres pouvait tuer ou Oublietter en un instant) perdit espoir elle aussi.

Ils descendirent le grand escalier du deuxième étage sans rencontrer personne.

Le Seigneur des Ténèbres tourna dans un couloir, inconnu de Harry, et descendit d'autres escaliers. Alors qu'ils descendaient d'étages en étages, des torches remplacèrent les fenêtres. Ils étaient à présent dans les donjons Serpentard.

Devant eux, la silhouette d'une personne habillée en robes de Poudlard apparu.

Le Seigneur des Ténèbres continua d'avancer vers cette personne.

Harry le suivit.

Une Serpentard en sixième ou septième année attendait à l'angle d'un mur décoré d'une belle gravure de Salazar Serpentard maniant sa baguette contre ce qui ressemblait à un géant recouvert de stalactites. La sorcière ne commenta ni la posture droite du professeur Quirrell ni la présence de Harry ni celle d'une arme à feu dans la main du professeur de Défense. Si elle avait le regard vide, Harry n'arrivait pas à le voir.

Le Seigneur des Ténèbres plongea une main dans ses robes, en sortit une Noise et la lui jeta. "Klaudia Alicja Tabor, voici mes ordres : porte cette Noise au cercle magique que je t'ai montré sous les gradins de Quidditch et place-la au centre de celui-ci. Puis fais-toi oublier les six dernières heures."

"Oui, seigneur," dit la sorcière avant de s'incliner et de partir.

"Je pensais…" dit Harry. "Je pensais que vous aviez besoin de la Pierre pour…"

Le Seigneur des Ténèbres souriait toujours ; il n'avait jamais cessé de le faire. "Je n'ai pas dit ça en Fourchelangue, petit. Tout ce que j'ai dit en Fourchelangue, c'est que j'avais mis en place un plan destiné à tuer des élèves et que je l'interromprais si j'obtenais la Pierre. Le reste était en langue humaine. J'aurais aussi interrompu le sacrifice de la Forteresse de Sang si je n'avais pas obtenu la Pierre, du moment que j'étais libre de mes mouvements et toujours sous couverture. Les élèves de Poudlard sont une importante ressource que j'ai passé longtemps à entraîner." Puis le Seigneur des Ténèbres siffla à l'adresse du mur : "\emph{Ouvre-toi.} "

Les yeux de Harry virent le petit serpent placé en haut à gauche de la gravure lorsque les murs pivotèrent lentement vers l'arrière et révélèrent l'entrée d'un immense tunnel circulaire. De la mousse poussait le long de ses parois et une odeur de poussière et de moisi s'en dégageait ; l'intérieur était aussi recouvert de multiples couches de toiles d'araignées.

"Des araignées…" murmura le Seigneur des Ténèbres. Il soupira, et l'espace d'un bref instant ressembla plus au professeur Quirrell.

Le Seigneur des Ténèbres s'avança dans l'immense tuyau ; les toiles brûlaient devant lui. Harry, ne voyant pas d'autre possibilité, le suivit.

Le tuyau forma un Y une fois, puis une autre. Le Seigneur des Ténèbres alla à gauche, puis à droite.

Le tuyau déboucha sur un mur de métal. "\emph{Ouvre-toi,} " siffla le Seigneur des Ténèbres, et le métal se fêla avant de se replier sur lui-même.

Le tuyau donnait au beau milieu d'un long tunnel de pierre.

"Nous allons marcher un moment," dit le Seigneur des Ténèbres. "Avais-tu d'autres questions pour moi, petit enfant ?"

"Je… aucune ne me vient à l'esprit… pour l'instant…"

Un autre rire froid lui répondit et ils s'avancèrent dans le tunnel, prenant à droite.

Il ignora, et il ne sut jamais, combien de temps ils marchèrent ; la lumière des toiles en train de brûler était trop faible pour éclairer sa montre mécanique, et il n'avait pas pensé à regarder l'heure avant d'entrer. Il eut l'impression qu'ils marchèrent le long de kilomètres et de kilomètres de souterrain.

L'esprit de Harry essaya lentement de se remettre une dernière fois. Il était fort probable que ce soit la dernière, s'il avait raison en songeant que le Seigneur des Ténèbres comptait le tuer ensuite… mais le Seigneur des Ténèbres avait dit qu'il ressusciterait Hermione, ce qui semblait dans ce cas inutile… était-ce simplement qu'il tenait une promesse qu'il n'aurait autrement pas pu faire en Fourchelangue… mais pourquoi ne pas avoir juste tué Harry…

\emph{Sérieux,}  dit la dernière partie de son cerveau encore en état de marche à l'intention de toutes les autres, \emph{ce serait vraiment le moment d'avoir une idée, une idée que le Seigneur des Ténèbres n'a pas encore eue, quelque chose qu'on peut faire avec notre bourse ou notre baguette ou notre Retourneur de Temps, quelque chose que le professeur Quirrell ne pense pas qu'on peut faire… réfléchis, réfléchis, s'il te plaît trouve quelque chose. Ne tombe pas en panne maintenant, même si tu as peur, même si on n'a jamais vraiment, vraiment fait face à la mort, pas au sens de n'avoir plus qu'une heure à vivre, CE N'EST PAS LE MOMENT DE TOMBER EN PANNE…} 

L'esprit de Harry resta vide.

\emph{Supposons} , dit la dernière partie encore debout, \emph{supposons que nous avons gagné, ou au moins qu'on s'en est sorti vivant. Si quelqu'un T'ANNONÇAIT LE FAIT que tu t'en étais sorti vivant, ou même que tu avais gagné, que tu avais réussi à tout arranger, que penses-tu qu'il se serait passé…} 

\emph{Procédure invalide} , murmura Serdaigle, \emph{l'univers ne fonctionne pas comme ça, on va juste mourir.} 

\emph{Quelqu'un se rend compte qu'on est absent,}  se dit Poufsouffle, \emph{et Maugrey Fol-Œil arrive avec une équipe d'Aurors pour nous sauver. Je pense que c'est le moment d'admettre qu'on est pas plus compétents que les autorités.} 

\emph{Ce qui nous sauve doit venir de } nous, dit la dernière voix. \emph{Sinon ça ne sert à rien d'y réfléchir.} 

\emph{Deuxième problème} , dit Gryffondor. \emph{Harry n'est pas absent, il est au match de Quidditch, et tout le monde peut le voir. Le professeur Quirrell a pensé à ça aussi, c'est pour ça qu'il a envoyé un faux message. Troisième problème. Je ne pense pas que Maugrey Fol-Œil et une équipe d'Aurors puissent battre le Seigneur des Ténèbres, et certainement pas avant qu'il ne nous tue. Je ne suis pas sûr que tout le DJM réuni peut battre le Seigneur des Ténèbres s'il s'y met sérieusement, et Dumbledore est parti. Quatrième problème. Le match de Quidditch s'est déroulé sans encombre, et c'est probablement la seule raison pour laquelle le professeur Quirrell a été prêt à exécuter un plan aussi compliqué.} 

\emph{Dans un autre genre,}  se hasarda Serpentard, \emph{peut-être que le professeur Quirrell va demander à quelqu'un d'autre de nous faire perdre la mémoire. Légilimancie, Imperius, Confundus, qui sait ; nous ne sommes pas un Occlumens parfait. Et il aurait alors un lieutenant intelligent - enfin, plus ou moins intelligent - à son service. Ça pourrait aussi expliquer pourquoi le professeur Quirrell avait tant envie de nous dire des secrets, puisqu'il savait qu'on allait perdre la mémoire. C'est aussi une raison de sortir de l'enceinte de Poudlard, pour qu'il puisse appeler Bellatrix et qu'elle fasse le travail…} 

\emph{Tout ce raisonnement est invalide et je refuse d'y participer} , dit Serdaigle.

\emph{Quelles magnifiques dernières paroles} , dit la dernière voix. \emph{Maintenant ferme-la et réfléchis.} 

Harry marchait sur un sol de pierre brute. Ses chaussures s'enfonçaient parfois dans de la mousse ou glissaient presque sur une surface courbe. Les neurones dans son cerveau, qui continuaient de s'allumer, imaginaient des voix qui se parlaient, qui se criaient dessus, pendant que celui qui Écoutait demeurait anesthésié par l'horreur et la honte.

Gryffondor et Poufsouffle débattaient d'un plan suicide consistant à foncer sur le pistolet du Seigneur des Ténèbres ou à avaler le petit caillou au doigt de Harry. Est-ce qu'il valait mieux pour le monde que le Seigneur des Ténèbres n'ait pas Harry en esclavage mental ? Si le Seigneur des Ténèbres allait gagner, peut-être qu'il valait mieux que ça se fasse rapidement.

Et la dernière voix continuait de parler à travers tout ceci ; même au plus profond de l'échec, la dernière voix demeurait. \emph{Qu'est-ce que le Seigneur des Ténèbres a toujours dit en langue humaine et jamais en Fourchelangue ? Est-ce qu'on s'en souvient ? N'importe quoi, juste un mot ?} 

C'était trop lointain, trop lointain alors même que cela avait eu lieu aujourd'hui. Le Seigneur des Ténèbres lui avait dit à l'instant en Fourchelangue qu'il était temps de ressusciter Hermione, puis avait poursuivi en langue humaine, et Harry pouvait à peine se souvenir de ce qui venait d'être dit. Avant ça… avant ça il y avait eu le Cercle de Dissimulation, quand le professeur Quirrell avait sifflé que la barrière exploserait s'il la touchait. Et le professeur de Défense avait dit en langue humaine de ne pas enlever la Cape, de ne pas traverser le cercle, il avait dit en langue humaine que la résonnance pourrait ensuite le toucher lui mais que Harry serait déjà mort. Il avait dit en langue humaine que si Harry touchait sa magie et que le professeur Quirrell ne savait pas comment mettre fin à la résonnance, cela les tuerait tous les deux…

\emph{Supposons que cela ne nous tue pas tous les deux} , dit la dernière voix. \emph{À Godric's Hollow, la nuit d'Halloween, le corps du Seigneur des Ténèbres a brûlé et on n'a eu qu'une cicatrice sur le front. Supposons que la résonnance entre nous est plus dangereuse pour le Seigneur des Ténèbres que pour nous. Et si, pendant tout ce temps, on avait pu tuer le Seigneur des Ténèbres en fonçant vers lui et en touchant sa peau ? Et notre cicatrice se remettrait à saigner, mais ce serait tout. La sensation de 'non, arrête' nous vient du pire souvenir du Seigneur des Ténèbres, de son erreur à Godric's Hollow, elle ne s'applique peut-être pas au Survivant.} 

Une petite note d'espoir monta.

Monta, et fut écrasée.

\emph{Le Seigneur des Ténèbres peut juste jeter sa baguette} , répondit Serdaigle d'une voix morne. \emph{Le professeur Quirrell peut prendre sa forme d'Animagus. Même s'il meurt, le Seigneur des Ténèbres possèdera quelqu'un, reviendra, et il torturera nos parents pour nous punir.} 

\emph{On pourrait retrouver nos parents à temps,}  dit la dernière voix. \emph{On pourrait les mettre à l'abri. On pourrait prendre la Pierre Philosophale au Seigneur des Ténèbres en tuant son corps actuel tout de suite, et cette pierre pourrait être le noyau d'une armée de résistance.} 

Le Seigneur des Ténèbres avançait dans le couloir de pierre. Il avait toujours l'arme en main. Il était au moins à quatre mètres de Harry.

\emph{Si on lui fonce dessus, il nous sentira venir à travers la résonnance,}  dit Poufsouffle. \emph{Il volera rapidement vers l'avant, il peut faire ça avec ses enchantements pour balai. Il volera vers l'avant, il se retournera et il nous tirera dessus. Il connaît la résonnance, il y a déjà pensé. Ce n'est pas le genre de possibilité qui lui aura échappé. Il est prêt, et il attend.} 

\emph{Pour reprendre l'idée précédente,}  dit la dernière voix. \emph{Supposons qu'on puisse lancer des sortilèges sur le professeur Quirrell sans problème mais que lui ne peut pas nous en lancer.} 

\emph{Pourquoi est-ce que ça serait vrai ?}  demanda Serdaigle. \emph{En fait, on a des raisons de croire que c'est faux. À Azkaban, on a eu l'impression que notre tête se fendait en deux quand l'Avada Kedavra du professeur Quirrell a touché notre Patronus…} 

\emph{Supposons que c'était}  sa \emph{magie qui partait en vrille. Peut-être si on lui avait lancé un simple Luminos, rien ne se serait produit.} 

\emph{Mais pourquoi ?}  dit Serdaigle. \emph{Pourquoi supposer ça ?} 

\emph{Parce} , pensa Harry, \emph{ça explique pourquoi le professeur Quirrell ne m'a pas}  mis en garde \emph{à ce sujet à Azkaban. Parce que je ne me souviens pas qu'il m'ait dit en Fourchelangue que souffrirais si jamais j'essayais de lui lancer un sortilège. Il aurait pu me prévenir de ça, mais il ne l'a pas fait, alors qu'il m'a donné beaucoup d'autres avertissements. L'absence de preuve est une faible preuve d'absence.} 

Il y eut un temps mort, pendant lequel Harry considéra cette idée.

\emph{Nous n'avons pas notre baguette} , dit Serdaigle.

\emph{Peut-être qu'on la récupérera plus tard} , pensa la dernière voix.

\emph{Mais quand bien même,}  pensa Harry, et le désespoir gris revint, \emph{le Seigneur des Ténèbres connait l'existence de la résonnance. Il a déjà pensé à tous les usages que je pourrais en faire, il a déjà prévu toutes ses réactions. J'ai fait cette erreur dès le début. Je n'ai pas respecté l'intelligence du Seigneur des Ténèbres, je n'ai pas pensé qu'il savait peut-être tout ce que je sais, qu'il pouvait voir tout ce que je vois, qu'il avait déjà pris tout cela en compte.} 

\emph{Alors,}  dit la dernière voix, \emph{si on a gagné, c'est qu'on l'a eu avec un truc qu'il ignore.} 

\emph{Des Détraqueurs,}  proposa Gryffondor.

\emph{Le Seigneur des Ténèbres} sait \emph{qu'on peut détruire, repousser et peut-être contrôler les Détraqueurs,} , répondit Serdaigle. \emph{Il ne sait pas comment, mais il sait qu'on en est capable, et de toute façon, où est-ce qu'on irait trouver un Détraqueur ?} 

\emph{Peut-être} , suggéra Poufsouffle, \emph{que tout le système de Horcruxe du Seigneur des Ténèbres se ferait court-circuiter par la résonnance si on l'attrapait et qu'on se tenait à lui, qu'on sacrifiait toute notre vie pour le détruire pour toujours.} 

\emph{N'importe quoi,}  répondit Serdaigle. \emph{Mais j'imagine qu'il n'y a pas de mal à fantasmer un peu avant de mourir, aussi stupide que ce soit.} 

\emph{Si Lord Voldemort a assez peur de la mort,}  continua Poufsouffle, \emph{s'il avait suffisamment envie de ne juste plus avoir à penser à la mort, alors le système de Horcruxe}  pourrait \emph{avoir ce genre de faille. Il n'a jamais pensé à tester ses Horcruxes sur quelqu'un d'autre, ça pourrait indiquer qu'il n'est pas capable d'y réfléchir sainement…} 

\emph{Donc son point faible, c'est sa peur de la mort ?}  dit Serdaigle. \emph{Désolé mais non. Je crois que quelqu'un qui a plus de cent Horcruxes n'aurait pas créé pas un système aussi peu sécurisé.} 

Et le cerveau de Harry continua de réfléchir.

Une véritable asymétrie dans la résonnance magique qui les unissait… voilà qui semblait improbable. L'effet n'avait aucune raison de marcher comme ça. Mais le retour de bâton magique touchait peut-être le sorcier proportionnellement à sa force, la magie la plus puissante explosait peut-être plus dangereusement. Cela expliquait l'événement de Godric's Hollow (Voldemort explose et le bébé survit) aussi bien que celui d'Azkaban (Voldemort sévèrement touché par la force de sa propre magie, Survivant en première année moins touché par sa magie plus faible). Ou bien seule la magie du lanceur de sortilège résonnait : cela expliquait aussi les deux observations. Cela pouvait aussi expliquer pourquoi le professeur Quirrell ne s'était pas empressé de dire à Harry de ne lui lancer aucun sortilège. Mais il y avait une autre explication très simple à la réticence du professeur Quirrell à parler de résonnance : c'était un indice colossal pour Godric's Hollow, et Harry aurait pu faire le lien.

La partie de lui qui était anesthésiée par le chagrin et la culpabilité en profita pour remarquer, puisqu'on parlait de choses évidentes, que, depuis le jour où les choses étaient devenues sérieuses à Poudlard, ils auraient vraiment, vraiment, \emph{vraiment} , \emph{VRAIMENT}  dû reconsidérer la décision qu'ils avaient prise le premier jeudi de l'année, sur le conseil du professeur McGonagall : \emph{de ne pas parler à Dumbledore de la sensation funeste que Harry ressentait à proximité du professeur Quirrell} . Harry n'avait effectivement pas su à qui faire confiance, et pendant un moment il avait été plausible que Dumbledore soit le méchant et que le professeur Quirrell soit son héroïque opposant, mais…

Dumbledore aurait compris.

Il aurait compris tout de suite.

Le sage vieux sorcier et son vrai phénix auraient tout de suite su, et Harry ne lui avait pas fait confiance, il ne lui avait pas donné les informations vitales, et la raison n'était autre qu'une pure et simple négligence : celle de ne pas reconsidérer une décision enregistrée quatre jours après le début de l'année. Même après Azkaban, l'étiquette 'ne pas en parler à Dumbledore' avait été là ; même après la mort de Hermione, même après tout le reste. Harry avait simplement oublié de ramener la question dans le domaine conscient, de reconsidérer son choix.

Une autre vague de tristesse et de honte se déversa sur Harry. Pendant un moment, il continua sans la dernière voix. D'autres furent heureuses de prendre sa place.

Après au moins plusieurs kilomètres, et de nombreuses pensées tristes, le tunnel de pierre s'acheva.

Le Seigneur des Ténèbres monta quelques escaliers, et Harry le suivit.

Ils émergèrent dans un bâtiment de pierre noire et humide. De vieilles portes de pierre sales s'ouvrirent sans être touchées.

Devant eux, des plaques de marbres jaillissant du sol ; sur elles, des noms et des dates. Les tombes étaient éparpillées au hasard, et tout le cimetière était laissé à l'abandon.

La lune, au-dessus d'eux, était aux trois-quarts pleine, déjà brillante avant même la tombée de la nuit.

Harry s'était arrêté en voyant le cimetière. Il y avait une alarme stridente dans son cerveau qui lui disait d'être \emph{n'importe où sauf ici} , mais il ne voyait pas comment accomplir cela. L'alarme continua donc, alors que derrière Harry, les portes du mausolée se refermaient et se scellaient de nouveau.

Le Seigneur des Ténèbres arriva au centre du vétuste cimetière. Il s'arrêta de marcher et fit un petit geste circulaire au-dessus de sa tête, baguette en main.

Il y eut un grondement, et un autel émergea doucement du sol, d'au moins deux mètres de large, de pierre noire couverte de runes grises. Autour de l'autel surgirent six obélisques de marbre noir, disposés à intervalles réguliers. Ils scintillaient sombrement sous le ciel crépusculaire.

La vaine alarme dans le cerveau de Harry devint plus forte.

"Ceci," dit le Seigneur des Ténèbres à la façon du professeur Quirrell, "est un plan de travail que je me suis fait, accessible depuis Poudlard et Pré-au-Lard." Le Seigneur des Ténèbres tendit une main vers l'autel. "C'est là que Mlle Granger vivra à nouveau, et aussi là que je renaîtrai dans mon vrai corps. Je vais me reconstruire d'abord, évidemment. \emph{Magie pour resssussciter fille-enfant plus ssimple avec vrai corps.} " Un étrange rire serpentin accompagna ces mots. "\emph{Ssois asssuré que même ssi la résssurection de fille-enfant peut être conssidérée comme magie noire, fille-enfant ne souffrira pas, ne ssera pas rendue laide. Elle resssemblera à elle-même, sson essprit ssera le ssien, et ni moi ni les miens ne lui ferons de mal ensuite.} "

Harry avait la bouche sèche et son esprit avait du mal à fonctionner. "Professeur, s'il-vous-plaît, pourriez-vous me dire en Fourchelangue la véritable raison pour laquelle vous ressuscitez Mlle Granger ?"

"\emph{Pour te rendre les consseils et la prudence d'amie fille-enfant. Pour asssurer que tu te ssoucies d'un monde où elle sse trouve. Ceci, petit, est ma principale raisson d'agir.} " Un nouveau rire de serpent accompagna ses mots, évoquant la sardonique compréhension de quelque vaste ironie.

Une petite étincelle d'espoir s'embrasa en Harry, accompagnée d'une bien plus grande note de confusion, de la peur qu'un Occlumens parfait soit en fait capable de mentir en Fourchelangue. Harry ne comprenait pas pourquoi le Seigneur des Ténèbres faisait cela si son prochain coup serait de tuer le Survivant ou d'en faire son esclave…

Peut-être n'avait-il simplement jamais compris le professeur Quirrell, peut-être que le modèle que Harry avait de Tom Jedusor était à ce point mauvais… peut-être que le Survivant serait Oublietté de cette journée et largué quelque part avec une Hermione Granger désorientée, pendant que Lord Voldemort continuait sa conquête du monde… ?

L'espoir flamba en Harry, mais c'était un espoir embrouillé qui n'avait pas de sens. Cela ne correspondait pas au Seigneur des Ténèbres qui s'était moqué de Dumbledore, avait rit de sa défaite. Harry ne trouvait aucun ensemble de motivations cohérentes qui expliquait les actes du professeur Quirrell.

\emph{Je ne sais pas ce qui doit se passer ensuite.} 

Le Seigneur des Ténèbres s'était avancé vers l'autel. Il s'y agenouilla et sembla plonger sa main dans la pierre de l'autel lui-même pour en extraire une fiole d'un liquide qui, sous la lumière du crépuscule, semblait noir.

Lorsque le Seigneur des Ténèbres parla à nouveau, sa voix fut précise, saccadée : "Du sang, du sang, du sang si sagement caché," dit-il.

Et les obélisques autour de l'autel se mirent à chanter en choeur, des voix émanant de la pierre immobile, des cadences plus anciennes que le Latin.

\emph{Apokatastethi, apokatastethi, apokatastethi to soma mou emoi.} 

\emph{Apokatastethi, apokatastethi, apokatastethi to soma mou emoi.} 

Le chant des obélisques se faisait écho après chaque phrase, comme s'ils n'étaient pas synchrones. Du sang de la fiole fut versé et il sembla couler sans toucher l'autel, s'étendre lentement dans les airs, adopter une forme.

\emph{Apokatastethi, apokatastethi, apokatastethi to soma mou (emoi).} 

\emph{Apokatastethi, apokatastethi, apokatastethi to soma mou (emoi).} 

Une grande silhouette reposait sur l'autel, et même dans le crépuscule, elle semblait trop pâle.

Le professeur de Défense plaça sa main dans ses robes et en sortit un petit fragment irrégulier de verre rouge.

Il le plaça sur le grand corps pâle.

La Pierre demeura là un moment, au moins quelques minutes. Le morceau de verre rouge irrégulier n'émit ni éclair ni lueur, ne donna aucun signe de pouvoir.

Puis la Pierre bougea, juste un peu, elle tourna sur le corps.

Le professeur de Défense remit la Pierre dans ses robes et toucha la grande silhouette qui gisait sur l'autel, il toucha ses yeux de ses doigts, toucha sa poitrine de sa baguette.

Puis il rejeta sa tête en arrière et rit.

"Incroyable," dit le Seigneur des Ténèbres, de la voix du professeur de Défense que Harry avait connu. "Permanente, sa forme est permanente ! Une simple apparence donnée par la magie devient la véritable substance dès que la Pierre la touche ! Et je n'ai rien senti ! Rien ! Je croyais avoir été dupé, avoir obtenu une fausse Pierre, mais la substance résiste à tous mes tests !" Le professeur de Défense glissa de nouveau le morceau de verre dans ses robes. "J'admets que c'est profondément étrange, même pour moi."

Puis le professeur de Défense tourna autour de l'autel, cinq fois, en chantant trop bas pour que Harry entende.

Il plaça sa baguette sur la main de la silhouette allongée sur l'autel.

Il plaça ses deux mains sur le front du corps.

Puis le Seigneur des Ténèbres parla. "\emph{Fal. Tor. Pan.} "

Sans avertissement, une lueur vive comme un éclair inonda tout le cimetière et Harry fit un pas en arrière ; ses mains touchèrent instinctivement son front. Il avait l'impression qu'on lui avait tiré dessus, ou qu'une abeille avait piqué sa cicatrice.

Le professeur de Défense s'effondra.

Et la silhouette trop grande se redressa sur l'autel.

Il pivota élégamment et se mit debout. Il faisait au moins une tête de plus qu'un homme normal. Ses membres étaient fins et pâles, sans muscles, mais ils semblaient receler une terrible force.

Harry fit un autre pas vacillant en arrière, ses mains toujours sur sa cicatrice. Même s'ils étaient loin l'un de l'autre, Harry sentit une terrifiante appréhension emplir l'espace, comme si la sensation funeste avait toujours été \emph{floue}  et venait de devenir nette, de se concentrer en une douleur physique située dans la cicatrice de Harry.

Est-ce que Voldemort était \emph{censé}  ressembler à ça ? Le nez avait l'air, il avait l'air d'avoir subi une \emph{défaillance}  pendant la résurrection…

La silhouette trop grande renversa sa tête en arrière et rit en levant ses mains et sa baguette pour mieux les observer. Sa main gauche s'ouvrit grand, elle était pâle, comme une demi-araignée avec quatre pattes trop grandes ; ses doigts caressaient la baguette tenue dans son autre main. Des feuilles se soulevèrent du sol du cimetière, s'approchèrent pour danser autour de la silhouette trop grande, l'entourèrent, le vêtirent, prirent la forme d'une chemise à col montant et de robes ; et Lord Voldemort riait toujours. Exactement le rire sans joie que Harry s'était souvenu entendre émaner de sa propre gorge dans le cauchemar du Détraqueur, précisément de ce ton, de ce timbre.

Des yeux rouges brillaient dans le crépuscule, leurs pupilles fendues comme celles d'un chat.

Le corps que Voldemort avait abandonnée se releva, tremblant ; et d'une voix que Harry pouvait à peine entendre, Quirinus Quirrell haleta : "Libre… oh, libre…"

"\emph{Stupéfix} ," dit la voix froide et flutée de Voldemort, et Quirinus Quirrell fut projeté au sol. Puis, d'un mouvement de son autre main, il fut soulevé et envoyé loin de l'autel.

Voldemort s'éloigna de l'autel puis se retourna et regarda Harry ; la douleur dans sa cicatrice s'embrasa.

"Effrayé, petit ?" siffla Voldemort, et on aurait dit que le Fourchelangue sous-tendait même la parole humaine du Seigneur des Ténèbres. "Bien. Place la fille sur l'autel, et mets fin à ta métamorphose. \emph{Il est temps de la resssussciter.} "

\emph{C'est vrai ? On va vraiment le faire ?} 

Harry déglutit et maîtrisa sa peur grâce à cet impossible espoir entouré de confusion. Il marcha vers l'autel, puis enleva sa chaussure gauche, sa chaussette gauche, et l'anneau de pied qu'était Hermione Granger. La forme métamorphosée était identique au Portoloin d'urgence qu'on lui avait donné. Il eut un pincement de regret de ne pas avoir le véritable Portoloin, mais seulement un pincement : si Severus avait dit vrai, les Mangemorts hauts placés mettaient souvent en place des barrières contre les Portoloin. Derrière Harry, Voldemort rit à nouveau, un rire appréciateur et surpris.

"J'ai besoin de ma baguette pour lancer \emph{Finite Incatatem} ," dit Harry.

"Non." La voix aigüe était cruelle. "Tu as appris à maintenir un métamorphose par le toucher, sans utiliser de baguette. Tu peux aussi mettre fin à ta propre métamorphose sans baguette, en ordonnant à ta magie de s'écouler. Fais-le maintenant."

Harry déglutit et toucha l'anneau de pied. Il dut s'y reprendre à trois fois et vider son esprit avant de parvenir à pousser la magie hors de l'anneau, comme il l'y avait doucement fait entrer.

Le sortilège prit fin beaucoup plus lentement comme pour un \emph{Finite Incantatem} , comme si une métamorphose accélérée avait lieu à l'envers. L'anneau se tordit, ondula, grandit. Les couleurs et les textures changèrent.

Deux tiers d'une fille morte gisaient sur l'autel, sur le flanc, un bras pendant au bord de l'autel, dans la position où les hasards du processus d'inversion de la métamorphose l'avaient placée. Le sang ne coulait plus des moignons mâchonnés de ses cuisses. La fille morte avait le visage de Hermione Granger, mais tordu, pâle. Elle était comme Harry l'avait vue dans l'hôpital. C'était l'image qui avait été gravée au fer rouge dans son cerveau pendant les trente longues minutes de métamorphoses, l'image qu'il avait reproduite pendant les quatre heures, encore plus longues, nécessaires à la métamorphose du leurre. La fille morte était nue, car ses vêtements ne faisaient pas partie d'elle et n'avaient pas été métamorphosés.

Cette vue fit remonter des souvenirs des heures passées dans l'infirmerie, des cauchemars qui avaient suivis et qu'il avait réprimés.

"Éloignes-toi," dit la voix aigüe de Voldemort. "C'est à mon tour, maintenant."

Harry déglutit et s'écarta de l'autel vers l'entrée du long couloir où il s'était tenu plus tôt. "Son corps est, devrait être à cinq degrés Celsius, je l'ai refroidie pour qu'il n'y ait pas de lésions cérébrales…" la voix de Harry passait d'une octave à l'autre. \emph{Il va vraiment faire ça ? Vraiment ?}  Il devait y avoir un piège, mais lequel ? Voldemort avait dit que ni lui ni les siens ne feraient de mal à Hermione, que son esprit et son corps seraient sien… \emph{pourquoi ?} 

Voldemort s'avança à nouveau vers l'autel et redressa le corps d'un geste de la main. Le Seigneur des Ténèbres parlait d'un ton monotone et très précis. "Chair, chair, chair si sagement cachée."

Les obélisques chantèrent à nouveau.

\emph{Apokatastethi, apokatastethi, apokatastethi to soma mou (emoi).} 

\emph{Apokatastethi, apokatastethi, apokatastethi to soma mou (emoi).} 

De la chair repoussa des moignons de ses cuisses, suinta comme de la vase et se solidifia.

Les obélisques cessèrent leur chant. Un corps nu entier gisait sur l'autel.

Il ne ressemblait pas à Hermione. Une Hermione Granger aurait été debout, aurait parlé, aurait porté son uniforme de Poudlard.

Voldemort leva une main puis siffla, comme agacé. Après un geste violent, la moitié des robes qui entouraient le corps endormi de Quirinus Quirrell furent déchirées, sa cravate violette et verte fut déchiquetée, sa veste attirée vers Voldemort. Une partie de Harry tressailli, comme s'il venait de voir le Seigneur des Ténèbres attaquer le professeur Quirrell.

Voldemort plongea sa main dans la veste, qui eut un soubresaut, comme si quelque chose venait d'être brisé ; puis il jeta vivement la veste au sol et déversa son contenu. La bourse de Harry tomba, ainsi que son Retourneur de Temps, un balais, le pistolet de Voldemort, la Cape et plusieurs amulettes, anneaux et appareils étranges que Harry ne reconnut pas.

Et enfin un morceau de verre rouge, qui fut déposé sur Hermione Granger et laissé là un moment.

Plusieurs minutes s'écoulèrent. Le Seigneur des Ténèbres enfila une amulette tirée du tas d'objets à côté de l'autel. De ce tas, Voldemort prit aussi quatre courts bâtons de bois équipés de sangles qu'il attacha par-dessous ses robes ; ils se fixaient apparemment aux avant-bras et aux cuisses. Le Seigneur des Ténèbres s'éleva dans les airs, alla à gauche, à droite, en haut et en bas, légèrement vacillant au début ; puis son vol devint plus stable.

Le morceau de verre rouge pivota, légèrement.

Le Seigneur des Ténèbres redescendit jusqu'au sol et toucha le corps de Hermione Granger avec sa baguette.

"\emph{Il y a un obsstacle} ," siffla Voldemort.

Harry s'était tellement attendu à une trahison ou à un échec que la confirmation ne fut qu'un choc sourd, sans violence. "\emph{Quel obsstacle ?} "

"\emph{Corps de fille est resstauré. Ssubstance réparée. Mais ni magie, ni vie… c'est le cadavre d'une Moldue.} " Voldemort se détourna de l'autel et se mit à faire les cent pas. "La version complète du rituel résoudrait ça. Mais cela prendrait du temps… du temps et le sang de l'ennemi de Granger, et je ne pense pas que Draco Malfoy fasse toujours l'affaire, pas plus que je ne peux prendre mon propre sang contre ma volonté… idiot." La voix de Voldemort était devenue un sourd sifflement. "Idiot, j'aurais dû le prévoir et me préparer. Je connais assez la médecine Moldue pour savoir que son cerveau se réveillera peut-être grâce à un choc électrique… mais sa magie lui reviendrait-elle ? Je ne sais pas, et je pense que si elle se réveille Moldue, elle le sera pour toujours. Mais je n'ai pas de meilleure idée." Le Seigneur des Ténèbres leva sa baguette…

"Attendez !" lâcha soudain Harry. Il sentit l'espoir remonter. \emph{Elle a besoin d'une étincelle de vie et de magie, juste d'une étincelle pour la relancer…} 

Voldemort se tourna pour le regarder. Le visage de serpent parut légèrement surpris.

"\emph{Je crois que j'ai quelque chosse qui pourrait marcher,} " siffla Harry. "\emph{Bessoin de baguette. Pas l'intention de l'utilisser contre vous.} " Harry ne dit rien quant à la possibilité que ses intentions changent ; il avait simplement dit son idée assez vite pour n'avoir pas encore formé d'intention spécifique.

"Alors là," siffla Voldemort, "j'aimerais voir ça." Le Seigneur des Ténèbres tendit la main vers le tas d'objets à côté de l'autel et en sortit la baguette de Harry enrobée dans du tissu. Elle fut jetée, glissa dans les airs et tomba aux pieds de Harry ; puis le Seigneur des Ténèbres recula, et le tas d'objets le suivit en glissant.

Harry sortit sa baguette et s'avança.

\emph{On a notre baguette, c'est la première étape} , dit la dernière voix, la voix de l'espoir.

Aucune partie de Harry ne savait ce que la deuxième étape pourrait être, mais au moins la première étape avait été accomplie.

Et Harry se tint devant le corps réparé de Hermione Granger, toujours nue et morte, sur un autel à la lumière du crépuscule.

"Lord Voldemort," dit Harry, "je vous en prie, donnez-lui quelques vêtements. Cela pourrait m'aider."

"Accordé," siffla Voldemort. La douleur de la cicatrice de Harry se réveilla quand le corps nu de la fille s'éleva, puis s'intensifia à nouveau quand des feuilles mortes dansèrent autour d'elle et qu'elle fut soudain vêtue d'un faux uniforme de Poudlard aux bordures vertes plutôt que bleues. Ses mains se joignirent sur sa poitrine, ses jambes se tendirent, et son corps redescendit doucement.

Harry la regarda.

Se concentra sur elle, maintenant qu'elle ressemblait à nouveau à une humaine.

\emph{Elle a l'air de dormir, pas d'être morte.}  Il dut faire un effort pour chercher à observer sa respiration, constater son absence et faire la déduction. Au premier abord… elle aurait aussi bien pu être en vie.

Il semblait acquis que Hermione Granger n'aurait pas approuvé cette situation prise dans son ensemble. Cela ne signifiait pas pour autant que, toutes choses étant égales par ailleurs (ce qui n'était pas forcément le cas), elle aurait préféré être morte que vivante.

\emph{Parce que tu veux vivre, parce que je crois sincèrement que tu voudrais vivre…} 

Harry tendit une main gauche tremblante et toucha le front de Hermione. Il avait maintenant une certaine tiédeur, pas la froideur de cinq degrés Celsius. Soit Voldemort avait restauré sa température corporelle habituelle, soit le rituel l'avait fait automatiquement. Ce qui, maintenant qu'il y songeait, signifiait que le cerveau de Hermione était en ce moment à la fois chaud et privé d'oxygène.

Ce nouveau sentiment d'urgence lui suffit.

Les pieds de Harry adoptèrent la posture habituelle, sa baguette se leva, se dirigea vers le cadavre de Hermione Granger. La \emph{seule}  chose qui n'allait pas chez elle, c'était sa mort ; tout le reste allait bien. Il n'y avait qu'une seule chose à changer.

\emph{Tu n'es pas bienvenue ici, mort.} 

"\emph{Expecto,} " hurla Harry en sentant \emph{la magie et la vie}  alimenter le sortilège du Patronus, "\emph{PATRONUM !} "

La fille en uniforme de Poudlard fut entourée d'une étincelante aura de feu argenté lorsque le Patronus naquit en elle.

Harry vacilla et il sentit un \emph{creux} , une morsure. Son intuition ou les souvenirs de Tom Jedusor lui dirent que ni la vie ni la magie qui venaient de passer à Hermione ne lui reviendraient jamais. Ça n'avait été ni toute sa vie ni toute sa magie, loin de là, car il n'avait pas eu le \emph{temps}  d'en user autant, mais ce qu'il venait de perdre était parti pour toujours.

Et Hermione Granger respirait au rythme du sommeil, d'inhalations et exhalations rythmées. Le ciel crépusculaire était devenu plus sombre, si bien que Harry ne vit pas si ses couleurs lui étaient revenues ; mais cela aurait dû être le cas, c'était presque certain. Elle avait l'air de dormir paisiblement, et ce n'était pas parce que la mort ressemblait au sommeil : c'était parce qu'elle dormait, et son corps allait bien, et rien ne la faisait souffrir.

Une partie de Harry, la partie qui était parvenue à se taire un peu plus tôt, fit doucement remarquer qu'ils étaient toujours dans un cimetière, que le récemment victorieux Lord Voldemort tenait toujours les rênes, et qu'il n'était pas certain que Hermione préfère être en vie.

Harry souriait toujours lorsqu'il abaissa lentement sa baguette. Les feux d'artifices festifs qui détonaient dans son esprit fusaient avec retenue ; il ne criait pas, ne courait pas en cercle comme l'aurait fait le professeur Flitwick, mais ça…

Ça…

\emph{ÇA} , dit Harry à voix haute dans sa tête, \emph{ÇA c'est ce que j'appelle une deuxième étape.} 

"Intéressant," dit la voix aigüe et froide. "Ton Patronus utilise ta vie autant que ta magie… je l'avais deviné, car c'est un sortilège beaucoup trop puissant pour qu'un élève en première année puisse l'alimenter avec sa seule magie. Et pourtant le puzzle est encore incomplet, car un autre sortilège alimenté par de la vie n'aurait pas fonctionné… ta pensée joyeuse, était-ce l'image de son retour à la vie ? Est-ce que c'était ça, l'astuce ?" Lord Voldemort jouait à nouveau avec sa baguette, et son regard rouge et fendu exprimait un sombre intérêt. "Je crois que je me sentirai très stupide quand je comprendrai enfin ce sortilège, un jour de mon éternité. Maintenant, éloigne-toi de la fille. \emph{J'ai l'intention de pourssuivre le travail, pour qu'elle ait le plus de chances posssibles de continuer à vivre.} "

Harry fit un pas en arrière avec réticence, la tension de nouveau présente. Il faillit trébucher sur une stèle anonyme, et le Seigneur des Ténèbres continua d'avancer.

Debout devant l'autel, le Seigneur des Ténèbres posa un doigt sur le front de Hermione Granger.

Puis il frappa son front du même doigt et dit d'une voix si basse que Harry faillit ne pas l'entendre : "\emph{Requiescus.} "

Voldemort agita une main en direction d'un obélisque qui se mit à pivoter et finit allongé au sol, pointé vers l'extérieur. "Tout à fait fascinant," siffla Voldemort. "Elle est en vie et magique, mais contrairement à ce que je craignais, ce n'est pas un autre Tom Jedusor."

La tension montait de nouveau en Harry. Il aurait bien remis sa baguette à sa ceinture noire, mais il ne \emph{voulait pas}  rappeler à Voldemort qu'il l'avait toujours sur lui. "Que lui faites-vous ?"

Un autre obélisque pivota, s'allongea au sol. "\emph{Il exisste un ancien rituel qui transsfère la nature magique d'une créature au ssujet en la ssacrifiant. Contraintes très fortes. Transsfert est temporaire, sseulement quelques heures. Ssujet meurt parfois quand transsfert prend fin. Mais Pierre le rendra permanent.} "

Quatre obélisques étaient étendus au sol, à intervalles réguliers ; les deux autres s'étaient éloignés en lévitant.

Voldemort plongea la main dans sa propre bouche, remarqua quelque chose, et eut un autre sifflement agacé. Il fit un geste vers la bouche d'un Quirinus Quirrell endormi, et de celle-ci s'envolèrent deux dents, presque invisibles en ce début de nuit. L'une d'elles alla sur le tas d'objets, l'autre flotta jusqu'à l'autel.

Quelques instants plus tard, Harry poussa un cri et fit un bon en arrière.

Immense et difforme, à la peau grumeleuse, aux jambes larges comme des troncs d'arbres, une petite tête qui rappelait une noix de coco perchée sur un rocher.

Un troll des montagnes se tenait entre les obélisques, immobile, comme s'il dormait debout.

"\emph{Que faites-vous ?} "

La bouche de Voldemort dessinait un large sourire qui le rendait \emph{horrible} . On aurait cru qu'il avait trop de dents. "\emph{Je vais ssacrifier mon arme de ssecours et la fille-enfant obtiendra les pouvoirs de régénération du troll. Ss'il n'a pas déjà été réglé par le rituel précédent, le mal de métamorphosse ne résisstera pas à cela. Et aucune lame, aucun ssortilège de coupure ne tranchera la fille-enfant, aucune maladie ne l'atteindra.} "

Harry déglutit. "Je suis très confus." Est-ce que Voldemort \emph{s'entraînait à être gentil}  ? Cette hypothèse ne semblait pas suffire à expliquer la situation.

"Reste bien en arrière," dit froidement Voldemort. "Ce rituel est plus noir que le précédent." Le Seigneur des Ténèbres entonna un nouveau chant, des syllabes plus douces qui semblaient être vivantes, tourbillonner dans les airs, et Harry, saisit d'une nouvelle poussée d'appréhension, fit un pas en arrière.

Puis il cria quand la douleur rugit dans sa cicatrice. Le troll des montagnes s'effondra sur lui-même, se transforma en cendres saisies suspendues puis en poussière, et celle-ci sembla être emportée sans aller nulle part ; elle ne fut plus là.

Quel qu'ait été le sortilège de repos que Voldemort avait lancé à Hermione Granger, il marchait : elle dormait paisiblement.

"Euh," dit Harry d'une petite voix, "ça a marché ?"

"\emph{Diffindo.} "

Harry s'avança en poussant un cri étranglé puis s'arrêta, à la fois parce que la stupidité de son geste lui était apparue, et parce que la coupure ouverte par le sortilège de Découpe sur la jambe de Hermione s'était refermée presque aussi vite qu'elle était apparue. Au bout de quelques secondes, il ne restait plus qu'une légère tache de sang.

La Pierre fut à nouveau déposée sur Hermione et tourna après quelques instants. Voldemort rit une fois de plus en passant sa main au-dessus d'elle. "Extraordinaire."

Puis une autre petite dent flotta jusqu'au cercle d'obélisques, et un instant plus tard, une licorne se tenait là où le troll avait été, les yeux ternes, la tête baissée.

"Quoi ?" dit Harry. "Pourquoi une \emph{licorne}  ?"

"\emph{Capacité du ssang de licorne à présserver vie sse combine très bien avec le ssoin du troll. À compter de ce jour, fille-enfant n'aura plus à craindre que le Feudeymon et le ssortilège de la Mort.} " Un éclat de rire serpentin. "\emph{Et puis j'avais une licorne en trop, autant en faire ussage.} "

"Le sang de licorne a des effets secondaires…"

"\emph{Sseulement quand le pouvoir de sson ssang est volé par un autre. Avec ssortilège, le pouvoir de la licorne viendra de l'intérieur de fille-enfant, comme ssi elle était née avec.} "

Le sinistre chant et ses mots tourbillonnants reprirent.

Harry regarda sans rien y comprendre.

\emph{Sans même vouloir comprendre, qu'est-ce que je vois ?} 

\emph{Je vois le Seigneur des Ténèbres faire tout ce qu'il peut pour ressusciter Hermione et la garder en vie. On dirait qu'il croit que sa vie dépend de celle de Hermione Granger.} "

Les parties perplexes de Harry cherchèrent une procédure à suivre. Sa première pensée fut : 'Fais une prédiction à partir de ta meilleure hypothèse', mais cela ne le mena nulle part. Le méchant avait gagné, mais le fil narratif ne continuait pas comme il aurait dû.

Une nouvelle poussée de douleur dans sa cicatrice, comme un coup sur le front. La licorne oscilla puis se désintégra comme le troll avant elle.

Le Seigneur des Ténèbres posa une fois de plus la Pierre sur Hermione et l'entoura de ses mains.

Il observa l'invisible processus un moment, puis se retourna, laissant la Pierre sur elle, et émit un fredonnement aigu. "Ah oui," siffla-t-il. "Ce serait parfaitement adapté. As-tu toujours le journal que je t'ai donné, petit ? Celui du célèbre scientifique ?"

Le cerveau de Harry ne comprit pas tout de suite de quoi Voldemort parlait. Dans la chambre de Marie, chez Marie, en octobre, ce précieux cadeau venu d'un ami. L'idée aurait dû faire monter une vague de chagrin terrible pour ce professeur Quirrell perdu ou factice, mais cette émotion avait été trop présente et son cerveau l'avait temporairement mise de côté.

"Oui," dit Harry. "Je crois que c'est dans ma bourse, je peux aller voir ?" Il \emph{savait}  que c'était dans sa bourse. Il l'avait remplie de tout ce dont il pourrait jamais avoir besoin, ses possessions et d'autres achats ; et de tout ce qui aurait pu être un objet de quête.

La bourse en peau de Moke de Harry fut extraite du tas d'objets près de l'autel et jetée aux pieds de Harry.

"Le journal de Roger Bacon," dit Harry en y plongeant la main, et le journal apparut. Le professeur Quirrell avait dit que le journal ressortirait intact d'un incendie, alors il le jeta vers l'autel de Voldemort ; il y avait plus important à penser que le traitement correct des livres, même de celui-ci.

Voldemort ramassa le journal et l'examina d'un air captivé.

Harry, aussi doucement et discrètement que possible, attacha la bourse à l'arrière de sa ceinture, là où on ne la verrait pas, là où Harry avait mis sa baguette.

\emph{Troisième étape, la bourse.} 

"Oui," siffla Voldemort en feuilletant le journal, "cela conviendra très bien." La Pierre se déplaça légèrement et l'autre main du Seigneur des Ténèbres rangea la Pierre dans ses robes.

"Quel était le but caché derrière ce journal ?" dit Harry après avoir attaché la bourse à sa ceinture et placé ses mains vides là où Voldemort pouvait les voir. "J'ai commencé par essayer de le traduire, mais ça n'allait pas très vite…" Le travail avait en fait été épouvantablement lent et Harry avait trouvé des choses plus importantes à faire.

"\emph{Le journal était exactement ce qu'il avait l'air d'être : un cadeau pour te sséduire, t'amener de mon côté.} " Voldemort fit des gestes complexes sans jamais regarder sa main, tout en tenant le journal de l'autre. L'espace d'un instant, Harry cru voir une trainée de ténèbres flotter en l'air, mais la lune était trop peu lumineuse pour qu'il puisse en être certain. "Et maintenant, mon cher petit," dit la voix aigüe de Voldemort, teintée d'une joie sinistre, au moment où il touchait brièvement le front de Hermione Granger d'un geste nonchalant, "je fais de ce journal un cadeau bien plus précieux, un témoin de la sagesse que tu m'as transmise. Car je ne voudrais jamais te voir privé des conseils et de la prudence de Hermione Granger, pas tant que les étoiles demeurent. \emph{Avadakedavra.} "

Le rayon vert du sortilège de la Mort atteint sa cible avant que Harry ne puisse lancer le Patronus, avant qu'il ait la moindre chance de bouger. Quand il cria et tendit la main vers sa baguette, c'était déjà fini.

Au moment de mourir, le corps inconscient de Quirinus Quirrell ne bougea même pas. La lumière verte le frappa et ce fut tout.

Des ténèbres émettaient une anti-lumière semblable aux trainées que Voldemort avait créées plus tôt, et le journal de Roger Bacon s'assombrit comme si une force corruptrice s'était emparée de lui au moment même où l'air autour de Hermione Granger se mit à trembler.

La douleur dans la cicatrice de Harry s'embrasa, l'écrasa, comme un fer rouge enfoncé sur son front. Les réflexes de Tom Jedusor prirent le dessus et le firent esquiver, se jeter de côté.

Et Voldemort criait aussi, glapissait, le journal était tombé à terre, il se tenait la tête, il hurlait.

\emph{Chance…} 

La dernière voix d'espoir dit cela, et Harry tenta désespérément de réfléchir, de comprendre. Tuer Voldemort n'avait pas de \emph{sens} , ça n'aurait fait que l'\emph{agacer} , les armes ne pouvait rien contre lui tant que ses centaines de Horcruxes existaient encore…

Mais il semblait toujours utile de lui enlever son corps, temporairement, de prendre la Pierre et Hermione et de fuir.

La main droite de Harry avait déjà saisi sa baguette. Sa main gauche passa dans son dos, atteint difficilement sa bourse et traça silencieusement un symbole, celui d'un mot de huit lettres.

"Non !" s'écria Voldemort. Il laissa tomber sa baguette, regarda le corps de Hermione, comme ahuri. "Non, non !"

L'objet passa de la bourse à la main de Harry et il commença à s'avancer aussi doucement de possible, cherchant à se rapprocher de Voldemort, à atteindre la distance viable indiquée par ses brèves expériences.

"Ma grande création," haleta Voldemort. Sa voix était flutée, comme paniquée. "Deux esprits différents ne peuvent exister dans le même monde - il est parti, il est rompu ! Un Horcruxe, je dois faire un Horcruxe immédiatement…" son regard tomba sur le corps endormi de Hermione Granger et il commença à lever sa baguette, à reproduire ses gestes précédents.

Harry leva son pistolet et appuya trois fois sur la détente.
\par\noindent\rule{\textwidth}{0.4pt}
\textbf{NdT : Le prochain chapitre (112) paraîtra demain vendredi 10 avril à 18h, heure française.} 

