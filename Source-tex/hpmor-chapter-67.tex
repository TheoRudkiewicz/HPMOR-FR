
\chapter{Accomplissement de soi, pt 2}

Dans les hauteurs de Poudlard, là où les couloirs et les pièces se transformaient quotidiennement, là le territoire devenait aussi incertain que la carte, là où la stabilité du château commençait à s'effilocher en rêves et en chaos sans changer de style architectural ni perdre de son apparente solidité - dans les hauteurs de Poudlard, une bataille allait bientôt avoir lieu.

La présence de tant d'élèves stabiliserait les couloirs pendant un moment par le biais de leur observation permanente. Les pièces et les couloirs de Poudlard \emph{bougeaient}  parfois même quand on était en train de les regarder mais il ne se \emph{changeaient } pas. Même après huit siècles, Poudlard était toujours un peu timide à l'idée de se changer devant les autres.

Mais en dépit de l'éphémère permanence (avait dit le professeur de Défense), les hauteurs de Poudlard possédaient pourtant un réalisme militaire : il fallait réapprendre le terrain à chaque fois et toujours revérifier les placards à la recherche de corridors secrets.

On était dimanche, dimanche premier mars. Le professeur Quirrell avait suffisamment récupéré pour recommencer à superviser les batailles et ils devaient tous rattraper le retard.

Draco Malfoy, le général Dragon, regarda les deux compas qu'il tenait, un dans chaque main. L'un avait la couleur du Soleil et l'autre avait un lustre iridescent et multicolore qui signifiait le Chaos. Draco savait que les deux autres généraux avaient chacun reçu leurs compas ; sauf qu'une main de Hermione Granger et une main de Harry Potter tenaient un compas rouge-orange dont les reflets vacillaient tels une flamme et pointait toujours en direction du contingent le plus imposant de troupes Dragon.

Sans ces deux compas, ils auraient pu se chercher pendant des jours et des jours sans jamais se trouver, ce qui constituait un des risques du terrain une fois aux plus hauts étages de Poudlard.

Draco avait un mauvais pressentiment quant à ce qu'il se passerait lorsque l'armée Dragon trouverait la légion du Chaos. Harry Potter avait changé depuis que Bellatrix Black s'était échappée ; l'héritier de Serpentard commençait vraiment à avoir l'air d'un vrai seigneur (et comment le professeur Quirrell avait-il su que cela se produirait ?). Draco se serait sentit beaucoup mieux avec Hermione Granger à ses côtés, accompagnée de ses vingt-trois soldats Soleil en rangs, mais non, le général Soleil et sa stupide fierté refusaient d'accepter une alliance contre le général Potter. Elle lui avait dit qu'elle voulait abattre Potter elle-même.

La Noble et Très Ancienne maison Malfoy avait maintenu son influence sur l'Angleterre pendant des siècles en comprenant qu'on ne pouvait pas toujours être \emph{le}  plus fort. Parfois un autre seigneur était tout simplement plus puissant et il fallait se contenter de n'être \emph{que}  son premier lieutenant. Il était possible d'accumuler \emph{pas mal}  d'influence, de richesses et de pouvoir au fil de douze générations passées à être commandants en second. Il fallait juste prendre à chaque fois garde de ne pas laisser sa maison se faire entraîner dans la chute du seigneur qu'elle servait. C'était là la tradition Malfoy, perfectionnée par des siècles de pratique.

Et Père avait donc minutieusement expliqué à Draco que s'il rencontrait quelqu'un d'évidemment plus fort que lui, Draco ne lui en voudrait \emph{pas} , il ne le nierait \emph{pas} , il ne piquerait \emph{pas}  une crise capable de saboter sa potentielle future position mais il \emph{s'assurerait}  que sa place dans la prochaine structure de pouvoir ne descendrait pas en-dessous de second.

Apparemment, Granger n'avait jamais reçu une telle leçon de ses parents et niait encore l'évidence : Harry Potter devenait plus fort qu'elle.

Il avait donc rencontré en secret les capitaines Goldstein, Bones et Macmillan et ils s'étaient tous mis d'accord pour faire de leur mieux afin que Dragon et Soleil ne s'affrontent pas avant d'avoir fait face à la plus importante menace qu'était Chaos.

Cela ne rompait pas \emph{vraiment}  l'accord contre les traîtres, ce n'était pas appeler à la traîtrise si on avait l'intention de vraiment \emph{aider}  l'autre armée.

Un son de cloche aigu retentit à travers les couloirs pour marquer le signal du début de bataille ; Draco cria "\emph{Allez !} " un instant plus tard et les Dragons commencèrent à courir. Cela fatiguerait ses soldats, cela aurait un coût même après qu'ils se soient arrêtés et qu'ils aient repris leur respiration, mais il \emph{fallait}  qu'ils mettent Chaos directement entre eux et le régiment Soleil.
\par\noindent\rule{\textwidth}{0.4pt}
Harry et Neville marchaient d'un pas tranquille dans les couloirs ; Harry regardait le compas jaune-or qui pointait en direction du régiment Soleil et Neville restait aux aguets juste au cas où ils rencontreraient quelqu'un.

Si vous aviez écouté attentivement, vous auriez remarqué que leurs bruits de pas étaient assez forts.

"Donc," dit le lieutenant chaotique après un moment. "C'est pour ça que tu nous a fait pratiquer le duel avec tous ces poids attachés ?"

Harry hocha la tête et garda ses yeux sur le compas qui pointait vers Soleil ; si la direction se mettait à changer trop vite cela voudrait dire qu'ils étaient trop près.

"Je ne voulais rien dire devant les autres mais deux semaines ne laissent pas beaucoup de temps aux muscles pour se développer," dit Neville. "Et l'équilibre est différent ; et je pense que ça pèse \emph{plus}  lourd ; et est-ce que ça n'est pas comme de métamorphoser un objet moldu ?"

"Nan," dit Harry. "J'ai vérifié à l'avance. On peut en voir sur des statues de Poudlard, certains sorciers en \emph{portaient} , même si c'était juste la mode du Moyen-Âge." Et puisque personne n'essaierait ça \emph{sans } se battre contre des élèves de première année armés de sortilèges faibles comme \emph{Somnium} , ça ne révélait aucune bonne idée non plus.

Ils arrivèrent à une intersection en Y, du genre énervante : aucun des couloirs n'était au bon angle pour les placer sur un chemin qui croiserait la destination de Soleil, alors que ceux-ci suivaient la légion du Chaos qui suivait elle-même l'armée Dragon. Harry choisit donc la direction qui semblait la meilleure et Neville le suivit.

"On ferait mieux d'essayer un rapide sortilège de silence sur ce truc quand on s'approchera," dit Neville. "Ça fait pas mal de bruit, ils pourraient comprendre."

Harry hocha la tête puis dit "bonne idée" au cas où Neville ne l'aurait pas regardé.

Ils continuèrent dans le couloir au sol de pierre situé dans les hauteurs de Poudlard, éclairé par des fenêtres au verre parfois transparent et parfois fumé, dépassant parfois des statues de sorcières, de dragons, et même d'occasionnels chevaliers mages en armures de plaques et cottes de mailles.
\par\noindent\rule{\textwidth}{0.4pt}
Les soldats Soleil avançaient rapidement dans un couloir long et large, baguettes brandies. Ils ne pouvaient pas utiliser le bouclier prismatique pendant leurs manœuvres mais Parvati Patil et Jenny Rustad maintenaient pour le moment un \emph{Contego } autour du groupe des officiers, qui seraient les premières cibles de toute embuscade.

Elle et ses officiers avaient décidé que la tactique de la prochaine bataille serait de se mêler directement aux les soldats ennemis aussi vite que possible après avoir pratiqué \emph{entre eux}  comment se soutenir, comment éviter de se tirer les uns sur les autres et comment se placer de façon à ce que l'ennemi hésite à faire feu. Ils n'avaient pu pratiquer que quatre heures mais elle considérait que ses troupes seraient ainsi déjà meilleures à ce genre de combat rapproché que d'autres soldats ne l'ayant jamais pratiqué. Ça ressemblait au genre de tactiques de Chaos mails ils n'avaient pas encore utilisé celle-ci.

Elle pensait que c'était une bonne stratégie. Et pourtant, peu importe le nombre de fois où elle leur faisait la leçon, ses soldats persistaient à chuchoter d'effrayantes rumeurs sur ce que Harry et Neville étaient en train d'apprendre à faire. Elle avait fini par partir en parler au capitaine Goldstein, qui comprenait le Moral des Troupes et autres choses de ce genre, et Anthony avait suggéré -

"C'est bizarre," dit soudain le capitaine Macmillan, fronçant les sourcils devant les compas flamboyants et iridescents qu'il tenait dans chacune de ses mains (Ernie était, comme Harry l'aurait formulé, 'bon en visualisation spatiale', et avait donc été désigné pour porter les compas et essayer comprendre ce que leurs ennemis fabriquaient). "Je pense... que Dragon ne bouge plus très vite... je pense qu'ils sont passés de l'autre côté de Chaos avant nous... et on dirait que Chaos avance pour les attaquer au lieu de manœuvrer pour ne plus être entre nous deux ?"

Hermione fronça les sourcils, essaya de comprendre, et vit des froncements de sourcils similaires chez Anthony et Ron. Si Chaos et Dragon s'attaquaient directement et épuisaient leurs forces l'un contre l'autre, ce serait quasiment comme de concéder la bataille à Soleil...

"Potter pense qu'on est alliés donc il attaque Malfoy maintenant avant que Dragon ne puisse nous rejoindre," dit Blaise Zabini depuis les rangs des soldats ordinaires. "Ou il pense qu'il peut battre les deux armées à la suite s'il les attaque séparément." Le Serpentard eut un soupir condescendant. "Allez-vous me re-promouvoir officier, maintenant ? Vous êtes des incapables sans moi, vous savez."

Ils ignorèrent tous les bruits qui émanaient de la bouche de Zabini.

"On va toujours dans la bonne direction ?" demande Anthony.

"Ouais," dit Ernie.

"On s'approche d'eux ?" dit Ron.

"Pas encore -"

C'est alors que les immenses portes de bois noir à l'extrémité du couloir s'ouvrirent grand et s'écrasèrent contre le mur, révélant deux silhouettes presque entièrement enveloppées dans des capes grises, du tissu gris plaqué contre leur visages, eux-mêmes placés sous des capuches grises, et l'une de ces silhouettes élevait déjà sa baguette, la pointait directement vers elle.

Et la physionomie du jeu fut radicalement transformée lorsque la voix de Harry, aiguë et tendue sous l'effort, hurla le mot :

"\emph{Stupéfix !} "

Le sortilège d'étourdissement de duel fonça vers elle, et elle fut si stupéfaite qu'elle ne commença pas à bouger avant qu'il ne soit presque trop tard, le jet de lumière rouge \emph{s'écrasa}  contre le bouclier de \emph{Contego}  situé face à eux, elle évita à peine, ressentit un léger picotement sur le bras lorsque la lumière rouge la dépassa, vit du coin de l'œil Susan recevoir le coup, se faire propulser vers Ron -

"\emph{Somnium !} " mugit la voix d'Anthony, suivit un instant plus tard par une douzaine de vois criant : "\emph{Somnium !} "

Hermione se remit sur pied d'une poussée frénétique et, alors qu'elle s'élevait, elle vit les deux silhouettes en capes grises qui étaient là, debout.

On ne pouvait pas \emph{voir}  les sortilèges de sommeil, ils étaient trop faibles -

Mais il était impossible qu'ils aient tous \emph{raté} .

"\emph{Stupéfix !} " glapit la voix de Neville Londubat, et un autre jet rouge jaillit vers elle, si bien qu'elle s'effondra en un tas peu digne en se contorsionnant désespérément pour se mettre hors de son chemin, puis elle se releva du mieux qu'elle put, haletante, et vit que cette fois-ci le coup avait eu Ron pendant qu'il se relevait.

"Coucou, Soleil," dit la voix de Harry, depuis sa capuche.

"Nous sommes les Chevaliers Gris du Chaos," dit la voix de Neville.

"Nous serons vos ennemis lors de cette bataille," dit la voix de Harry, "pendant que l'\emph{autre } armée du Chaos massacrera les Dragons."

"Et au fait," dit la voix de Neville, "nous sommes invincibles."
\par\noindent\rule{\textwidth}{0.4pt}
Les deux garçons, sous leur robes et leur cape grise, du tissu gris sur leur visage, faisaient face à toute l'armée Soleil, visiblement indifférents à une douzaine de sortilèges de sommeil.

Daphné entendit un doux soupir venir de derrière elle, et lorsqu'elle tourna la tête elle vit que les lèvres de Hannah s'étaient ouvertes, que les yeux de la Poufsouffle étaient immense et qu'elle regardait -

Il aurait été difficile de décrire le fatras de pensées qui traversa l'esprit de Daphné en un éclair lorsqu'elle se rendit compte que Hannah regardait Neville plutôt que Harry, ce qui poussa à son tour une partie d'elle à remarquer que de fait, en matière de garçons, Neville \emph{était}  devenu plutôt intéressant dernièrement, et à vrai dire, là, tout de suite, le dernier héritier Londubat avait l'air carrément \emph{cool} , ce sur quoi quelque chose s'éveilla en elle, ses propres lèvres s'ouvrirent, et tout ce que Dame sa Mère lui avait jamais instruit en matière de modestie, de flatterie et de shampoing parfumé fut propulsé hors de son esprit si fort que ça aurait dû lui en ébouriffer les cheveux autour des oreilles ; parce qu'elle avait observé Hermione et Harry et qu'elle savait comment elle voulait que \emph{sa}  cour se déroule -

Dame sa Mère lui avait aussi récemment enseigné certains sortilèges qu'il aurait été embarrassant de ne pas connaître quand on appartenait à la Noble et Très Ancienne maison Greengrass.

Sa baguette s'éleva, pointa vers sa gauche, et Daphné hurla : "\emph{Tonare !} "

Puis sa baguette passa au-dessus de sa tête et elle prononça l'incantation : "\emph{Ravum Calvaria !} "

Elle prit enfin sa baguette dans deux main et s'écria : "\emph{Lucis Gladius !} "

L'hémorragie de magique la fit presque tomber à genoux mais elle parvint à le supporter, et le drain diminua lorsque la forme étincelante se fut entièrement matérialisée et stabilisée.

Elle avait quand même l'impression qu'elle ferait mieux de ne pas essayer de se battre avec ça pendant trop longtemps.

Le fait que tout le monde l'observe allait \emph{sans dire}  et elle aurait \emph{dû}  bondir, cheveux flottants au vent, pour faire face à Neville, mais elle put seulement s'avancer d'un pas régulier et mettre sa Très Ancienne Lame devant lui. Il va aussi sans dire que tout le monde s'écarta et lui laissa la voie libre.

"\emph{Je suis Daphné, de la Noble et Très Ancienne maison Greengrass !} " s'écria-t-elle. "\emph{Greengrass du Soleil !} ". Les bonnes manières de duel lui étaient complètement sorties de l'esprit, et même si elle avait vu assez de pièces de théâtre pour se souvenir des défis à mort et des défis du Sang, elle n'arrivait pas à se rappeler ce qui convenait à la situation présente, si bien qu'elle pointa l'épée incandescente vers l'objet de ses émois et rugit : "\emph{On va voir ce que t'as dans le ventre, Nevy !} "

La voix de Harry glapit "\emph{Stupéfix}  \emph{!} " une fois de plus. Plus tard, en se rappelant de ce moment, elle n'arriva jamais tout à fait à croire qu'elle avait réussi : elle fit tournoyer sa lame de lumière comme si c'était une batte de base-ball et \emph{renvoya le rayon vers Harry} , qui parvint à peine à s'écarter de la trajectoire.

"\emph{Tonare !} " hurla Neville, de la Noble et Très Ancienne maison Londubat. "\emph{Ravum Calvaria, Lucis Gladius !} "
\par\noindent\rule{\textwidth}{0.4pt}
Pendant quelques secondes personne ne fit autre chose que regarder Neville et Daphné qui commençaient à se mettre une raclée l'un à l'autre. Ils se déplaçaient lentement et Hermione conjectura que le sortilège requérait beaucoup de force. Si on était une Moldue et qu'on avait vu certains films, ça n'était pas très impressionnant.

Mais il fallait quand même reconnaître qu'ils avaient déjà du mérite d'utiliser des sabres lasers.

"Interruption au sujet des règles," dit la voix de Harry. "Je sais que le professeur de Défense regarde mais je me dois quand même de poser la question : quelqu'un sait-il s'ils se découperont en deux si jamais ils se touchent vraiment -"

"Non," dit Hermione d'un ton absent. Ça avait été dans un de ses livres d'Histoire, même si elle ne savait pas que les épées de duel magiques ressemblaient à \emph{ça} . "Ils les ont invoquées de façon à ce qu'elles ne fassent qu'étourdir quand elles touchent."

"Tu \emph{connais}  ce sortilège ?"

"Oh, non, c'est le sort de la Lame Très Ancienne, c'est seulement légal pour les membres des maisons Nobles et Très Anciennes -"

Hermione se tut et regarda alors Harry, ou plutôt sa capuche grise.

"Eh bien," dit la voix de Harry, "dans ce cas je suppose que je pourrais descendre le reste du régiment Soleil à moi tout seul." Elle ne pouvait pas voir son visage mais il avait la voix de quelqu'un qui souriait.

"Tu as esquivé quand Daphné t'a renvoyé ton propre sort," dit Hermione. "Donc quoi que tu aies fait, tu n'es \emph{pas}  invincible. Un \emph{Stupéfix}  peut toujours t'avoir."

"Théorie intéressante," dit la voix de Harry sous sa capuche. "Tu as quelqu'un dans ton armée qui peut tester ça ?"

"J'ai lu quelque chose sur le sortilège d'étourdissement, un jour," dit Hermione. "Il y a quelques mois. Je me demande si je pourrais me souvenir des instructions ?" Sa baguette s'éleva vers Harry.

Il y eut une courte pause tandis que non loin, un garçon et fille aux halètements sonores se mettaient lentement la pâtée à coups de sabre laser.

"Bien sûr," dit Harry, pointant sa propre baguette vers elle, "\emph{je}  peux juste utiliser Somnium sur toi. Ça me demandera beaucoup moins d'efforts."

De nouveaux \emph{Contego}  apparurent devant elle, lancés par Jenny et Parvati, avant même que Harry n'aie fini sa phrase.

Le bout de la baguette de Hermione faisait de petits mouvements, un diamant dans un cercle, un diamant dans un cercle ; elle répétait le geste pour correspondre exactement à ce qu'elle se souvenait avoir vu dans le livre. Ce serait une prouesse de haut vol, même pour elle, mais il \emph{fallait}  qu'elle réussisse du premier coup car elle ne pouvait se permettre de perdre de l'énergie dans des incantations échouées.

"Tu sais," dit Hermione Granger, "je comprends que ce n'est pas vraiment ta faute, mais je commence à en avoir assez d'entendre les gens parler du Survivant comme si tu étais - comme si tu étais une sorte de \emph{dieu}  ou quelque chose dans le genre."

"Franchement, pareil pour moi," dit Harry Potter. "C'est triste de voir à quel point les gens continuent de me sous-estimer."

Sa baguette continua de répéter le diamant dans le cercle, encore et encore. Harry allait récupérer des forces, elle le savait, pendant qu'elle pratiquait son sortilège au maximum avant d'attaquer. "Je commence à penser que vous avez besoin qu'on vous remette à votre place, général Chaos."

"Tu as peut-être raison," dit calmement Harry. Sa pieds commencèrent à se mouvoir selon ce qu'elle savait être une danse de duettiste. "Malheureusement, il n'y a plus rien qui puisse me vaincre à part un autre Harry Potter."

"Permettez-moi d'être précise, M. Potter. \emph{Je}  vais vous remettre à votre place."

"Toi et quelle autre armée ?"

"Tu crois que t'as plutôt la classe, hein ?" dit Hermione.

"Mais oui," dit Harry. "Oui, en effet. Certains dirons peut-être que c'est arrogant, mais suis-je censé être la dernière personne à Poudlard à remarquer à quel point je suis génial ?"

Hermione leva sa main gauche et ferma le poing.

C'était un signal. Huit soldats choisis de son armée pointeraient leur baguette vers elle et lanceraient discrètement \emph{Wingardium Leviosa} ;

Ils avait aussi pratiqué \emph{ça}  après que Hermione, ayant abandonné l'idée de faire la leçon à ses soldats, avait essayé sur une suggestion d'Anthony de leur donner un général Soleil qui aurait \emph{l'air}  de pouvoir vaincre des ennemis invincibles.

"Tu te prends pour Superman," dit Hermione. Elle leva son gauche plus haut et les huit soldats qui la soutenaient la firent léviter au-dessus du sol. "\emph{Eh bien voilà Super Hermione !} " Elle brandit sa main en avant, et, tout en se faisant propulser vers Harry, regrettant seulement de ne pas pouvoir voir l'expression de son visage, sa baguette fit un diamant dans un cercle, elle fit appel à toute la magie dont elle disposait, ce qui lui donna la sensation de toucher un fil conducteur lorsque le sortilège bien trop puissant se déversa à travers elle au son de son cri : \emph{"Stupéfix !"} 

Le jet rouge jaillit de sa baguette, parfait.

Harry l'évita.

Et alors, parce qu'ils n'avaient pas pratiqué ça dans les couloirs, elle s'écrasa contre un mur.
\par\noindent\rule{\textwidth}{0.4pt}
"\emph{Somnium}  !" glapit Draco, puis au bout de quelques secondes pour se recharger : "\emph{SOMNIUM, BON SANG !} "

Il \emph{savait}  qu'il touchait Theodore car l'autre garçon n'essayait même pas d'éviter, mais l'héritier de Nott se contenta d'avoir un sourire aussi maléfique que celui de son père, de lever sa baguette -

Draco parvint à bondir de côté lorsque Theodore dit "\emph{Somnium !} " mais l'essoufflement le gagnait et il ne pouvait pas maintenir ce rythme ; Theodore ne se fatiguait pas à éviter, Draco bougeait sans cesse, c'était de la \emph{folie} .

Il avait regagné assez de force pour faire feu à nouveau mais -

\emph{La stupidité consiste à refaire la même chose en s'attendant à un résultat différent} , avait dit Harry, et c'était \emph{là}  l'œuvre de \emph{Harry} , ça ne pouvait plus être un objet moldu, mais Draco n'avait n'arrivait pas à imaginer ce que ça \emph{pouvait}  être, et il aurait dû penser à des hypothèses et à des façons de les tester mais il était trop occupé à désespérément éviter les autre sortilèges de sommeil que Theodore lui lançait en riant ; Draco sentit un léger engourdissement au flanc cette-fois ci, ça l'avait raté de très très peu, et il finit par en avoir assez, il ne se fatigua même pas à détailler la théorie qu'il testait ni pourquoi et il se contenta de -

"\emph{Luminos !} " s'écria Draco, et Theodore fut entouré d'une lumière rouge, "\emph{Dulak !} " et il s'éteignit de nouveau (donc Theodore \emph{était}  toujours affecté par la magie), "\emph{Expelliarmus !} " et la baguette de Theodore s'envola (et Draco se rendit alors compte que de toute façon ça aurait été un bon sortilège à lancer), mais Theodore fonçait vers Draco bras tendus en avant pour le saisir si bien que Draco hurla "\emph{Flipendo !} ", les pied de l'autre garçon se renversèrent abruptement -

- et son dos heurta le sol dans un fracas \emph{métallique}  étonnamment fort.

Draco voyait trouble après avoir lancé quatre sortilèges si rapidement, Theodore se relevait déjà du mieux qu'il pouvait, il n'y n'avait même pas le temps de penser en mots mais il parvint à dire "\emph{Somnium !} " et \emph{cette fois}  il visa le visage de Theodore plutôt que sa poitrine.

Theodore évita (il \emph{évita !} ) et hurla :"\emph{Code sept sur Malfoy !} "

"\emph{Prismatis !} " cria la voix de Padma et un arc-en-ciel chatoyant apparut soudain devant Draco au moment même où quatre voix chaotiques s'écriaient \emph{Somnium !} 

Puis il y eut un silence tandis que tout le monde observait l'immense sphère prismatique qui protégeait les restes de l'armée Dragon.

Lancer ce cinquième sort avait mis Draco à genoux mais il releva les yeux et parvint à dire aussi clairement que possible : "Si le sortilège de sommeil - ne fonctionne pas - visez leur visage - je pense que les Lieutenants portent des hauts de métal."

"Tu as déjà perdu trop de soldats," dit Finnigan d'une voix forte depuis l'autre côté de la barrière, "on vous battra quand même," puis le Gryffondor eut un rire démoniaque. Il le faisait maintenant presque aussi bien que Harry Potter, et les autres légionnaires du Chaos se mirent rapidement à rire avec lui.

Draco pouvait voir du coin de l'œil Gregory et Vincent, inconscients. Padma maintenait encore la sphère prismatique, la plus grande qu'il l'ait jamais vue lancer ; mais sa respiration était hachée, elle était encore recouverte de sueur, épuisée depuis le moment où ils avaient tous couru pour se mettre en position ; la sorcière Serdaigle était forte mais pas \emph{athlétique} .

Il espérait vraiment que le général Granger arriverait vite et attaquerait Chaos par derrière. Le général Potter et Neville du Chaos manquaient à l'appel et Draco pouvait deviner quelle avait été leur destination, mais deux soldats ne pouvaient pas faire prendre trop de retard au régiment Soleil à eux seuls, si ?
\par\noindent\rule{\textwidth}{0.4pt}
Elle savait que ce n'était pas juste, que l'autre fille avait tout donné, mais Hermione souhaita quand même que Daphné aie tenu plus longtemps.

"\emph{Lagann !} " dit la voix de Neville derrière elle alors qu'elle s'enfuyait, et il y eut le bruit d'un mur prismatique fracassé, la voix de Hannah qui hurlait désespérément "\emph{Somnium !" } et quelques instants plus tard celle de Neville qui disait calmement "Somnium" et le bruit mat d'un autre de ses soldats qui tombait au sol.

Et la force qui la maintenait en l'air diminua à nouveau, elle put sentir la poigne des sortilèges de Lévitation qui se distendait sous son poids, qui cessait d'être suffisante.

Son vol s'arrêta et elle commença à chuter vers le sol au ralenti, et elle aurait dû dire à ses soldats de juste la laisser \emph{tomber}  mais elle était trop en colère, trop troublée, elle ne pensait pas assez vite, elle essayait encore de trouver la force nécessaire pour un dernière sortilège d'étourdissement, et il n'y eut donc nulle part où aller lorsque Harry pointa sa baguette vers elle et dit "\emph{Somnium} " ; ce fut le dernier mot que Hermione Granger entendit de toute cette bataille.

