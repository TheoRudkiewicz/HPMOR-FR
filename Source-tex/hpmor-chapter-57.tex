
\chapter{EPS, Cognition sous contraintes, pt 7}

Harry avait \emph{espéré}  avoir accompli une fusion avec son mystérieux côté obscur, être devenu capable de faire usage de tous les avantages de celui-ci sans avoir à subir aucun de ses inconvénients et de pouvoir mobiliser sur commande la clarté cristalline et l'invincible volonté sans pour autant avoir à devenir froid ou colérique.

Il avait une fois de plus surestimé ses progrès. \emph{Quelque chose}  s'était produit, mais il avait toujours un mystérieux côté obscur, toujours séparé de lui, et son lui ordinaire était toujours vincible. Et malgré toutes les réparations qu'il avait effectuées sur la peur de la mort qu'avait son côté obscur, il n'osait pas devenir sombre sans protection à l'intérieur d'Azkaban ; cela aurait été tenter le destin de façon exagérée.

C'était dommage, car un peu de nonvincibilité aurait \emph{certainement été bien pratique là, tout de suite.} 

Ce qui rendait les choses plus difficiles, c'était qu'il ne pouvait ni s'appuyer contre un mur ni éclater en sanglots ni même se permettre un soupir. Sa chère Bella le regardait et ce n'était pas le genre de chose qu'un Seigneur des Ténèbres aurait fait.

"Seigneur -", dit Bellatrix. Sa voix, basse, était tendue. "Les Détraqueurs - ils approchent - je peux les sentir, seigneur -"

"Merci, Bella," dit une voix sèche, "je suis au courant."

Harry ne pouvait pas ressentir les trous dans le monde aussi bien que lorsqu'il avait revêtu la Relique de la Mort mais il pouvait sentir l'attraction du vide monter en intensité. Il avait d'abord cru que c'était parce que Bellatrix et lui descendaient les escaliers, jusqu'à ce qu'ils s'arrêtent et que l'attraction continue d'augmenter... puis de diminuer lorsque les Détraqueurs s'éloignèrent de la spirale, puis d'augmenter de nouveau lorsqu'ils entamèrent une nouvelle volée de marches... Les Détraqueurs étaient maintenant à l'intérieur d'Azkaban, et ils venaient le chercher. Bien sûr qu'ils venaient le chercher. Harry leur résistait peut-être, mais il n'était pas \emph{caché} .

\emph{Nouveau besoin} , dit Harry à son cerveau. \emph{Trouve un moyen de vaincre les Détraqueurs qui ne nécessite pas mon Patronus. Autrement, trouve encore un autre moyen de cacher quelqu'un à la vue des Détraqueurs, à part la Cape d'Invisibilité -} 

\emph{Je démissionne,}  dit son cerveau. \emph{Trouve-toi un autre morceau substrat opératoire pour résoudre des problèmes dotés de contraintes ridicules et exagérées.} 

\emph{Je suis sérieux} , pensa Harry.

\emph{Moi aussi, } dit son cerveau. \emph{Lance ton Patronus et attends que les Aurors viennent te chercher. Sois raisonnable. C'est fini.} 

\emph{Abandonne...} 

Lorsque cette pensée lui vint, la succion du vide sembla devenir plus forte ; et Harry comprit ce qui se passait et se concentra d'autant plus sur les étoiles, détourna son esprit du désespoir -

\emph{Tu sais} , fit remarquer la partie logique de son être, \emph{si tu n'as pas le droit d'avoir des pensées négatives parce que cela ouvrira ton esprit aux Détraqueurs, } c'estaussi\emph{ un biais cognitif : comment sauras-tu qu'il est vraiment temps d'abandonner ?} 

Un cri sanglotant et désespéré vint d'en-dessous, porteur de mots tels que "non", et "partez". Les prisonniers savaient, ils pouvaient les sentir.

Les Détraqueurs arrivaient.

"Seigneur, vous - vous ne devriez pas prendre de risque pour moi - reprenez votre Cape -"

"Tais-toi, imbécile," siffla une voix en colère. "Lorsque je déciderai de te sacrifier, je t'en ferai part."

\emph{Elle marque un point} , dit Serpentard. \emph{Tu}  \emph{ne } devrais pas \emph{prendre de risque pour elle, il est impossible que sa vie ait autant de valeur que la tienne.} 

L'espace d'un instant, Harry considéra la possibilité de sacrifier Bellatrix pour se sauver lui -

Et à cet instant, une partie de la lumière orangée des lampes à gaz sembla fuir le couloir, un soupçon de froid s'introduisit au bout des doigts de Harry. Et il sut alors qu'envisager la possibilité de laisser Bellatrix aux ombres de la Mort le rendrait de nouveau vulnérable. Prendre cette décision pourrait même le rendre incapable de lancer le Patronus car il aurait alors abandonné la pensée même qui l'avait sauvé plus tôt.

L'idée lui vint que même s'il n'était plus capable de lancer le Patronus, il pourrait toujours prendre la cape de Bellatrix ; il lui fallut alors arracher ses pensées à cette possibilité et se concentrer fermement sur la décision qu'il avait prise de ne \emph{pas}  le faire, sans quoi il serait peut-être tombé au sol, car le tourbillon de vide qui s'enroulait autour de lui avait atteint une puissance mortelle ; des cris venaient d'en-haut, et ceux d'en-dessous s'étaient tus.

\emph{C'est ridicule} , dit son côté logique. \emph{Les agents rationnels ne devraient pas avoir à gérer ce genre de raisonnement censuré, tous les théorèmes postulent que la façon dont tu penses n'affecte pas la réalité, tes actes mis à part, c'est pour ça que tu es libre de choisir un algorithme optimal sans t'inquiéter de la façon dont tes pensées interagissent avec celles de Détraqueurs...} 

...

\emph{C'est une idée vraiment stupide} , dit Gryffondor. \emph{Même moi je pense que c'est une idée stupide, et je suis ton côté Gryffondor. Tu ne vas pas vraiment rester là et -} 
\par\noindent\rule{\textwidth}{0.4pt}
"On a sa position !" s'écria Ora, brandissant son miroir magique en signe de triomphe. "Le Détraqueur à l'extérieur du mur interne indique le niveau 7, spirale C, c'est là qu'ils sont !"

Les Aurors la regardaient, attendaient un réponse.

"Non," dit Amélia d'une voix égale. "C'est là que \emph{l'un}  d'eux se trouve. Les Détraqueurs ne peuvent toujours pas trouver Bellatrix Black. Nous n'allons pas nous précipiter là-bas et la laisser partir dans la mêlée, et nous n'allons pas diviser nos forces pour mieux nous faire prendre en embuscade. Tant que nous nous déplaçons précautionneusement, nous ne pouvons pas perdre. Dites à Scrimgeour et à Shacklebolt de continuer de descendre un étage à la fois comme avant -"

Le vieux sorcier s'était déjà élancé. Cette fois Amélia ne se fatigua même pas à l'insulter car leurs boucliers savamment construits s'écartèrent à nouveau comme s'ils avaient été liquides et ondulèrent doucement dans son sillage.
\par\noindent\rule{\textwidth}{0.4pt}
Harry attendait à l'entrée du couloir, juste devant les escaliers qui menaient à l'étage supérieur. Bellatrix et le serpent étaient derrière lui, masqués par la Relique de la Mort dont Harry s'était rendu maître ; il savait, même s'ils ne pouvait pas le voir, que la sorcière émaciée était assise sur les marches le dos courbé car il avait rétracté son sortilège de lévitation afin de libérer sa magie et son esprit.

Ses yeux étaient braqués sur l'extrémité opposée du couloir, tout près des escaliers qui menaient à l'étage inférieur. Cette fois ci ce n'était pas dans son esprit mais dans la réalité que les lumières s'étaient assombries et que la température avait chuté. La peur tonnait autour de lui comme une mer fouettée par les vents d'un ouragan, et la succion du vide était devenue la force d'attraction hurlante d'un trou noir en approche.

En haut des escaliers situés de l'autre côté ils s'approchèrent, flottants entre les airs mourants, les vides, les absences, les blessures du monde.

Et Harry s'attendait à ce qu'ils s'arrêtent.

Avec toute la volonté et la concentration qu'il pouvait assembler, il \emph{s'attendait à ce qu'ils s'arrêtent.} 

Anticipait leur arrêt.

Croyait qu'ils s'arrêteraient.

...en tout cas, c'était ça l'idée...

Harry éteint la dangereuse pensée errante et \emph{s'attendit à ce que les Détraqueurs s'arrêtent.}  Ils n'avaient aucune intelligence propre, ils n'étaient que des blessures infligées au monde, leur forme et leur structure étaient empruntées aux attentes des autres. Si des gens avaient pu négocier avec eux, s'ils avaient pu leur offrir des victimes en échange de leur coopération, ce n'était que parce qu'ils \emph{croyaient que les Détraqueurs marchanderaient. } Donc si Harry croyait assez fort que les vides se détourneraient et partiraient, ils se détourneraient et ils partiraient.

Mais les blessures continuaient d'approcher, la peur tourbillonnante semblait maintenant être devenue solide, le vide arrachait la matière comme la pensée, la substance comme l'esprit, et l'on pouvait voir le métal se ternir à mesure que les trous dans le monde défilaient.

Un petit bruit vint de derrière lui, de Bellatrix, mais elle ne dit rien car elle avait reçu l'ordre de rester silencieuse.

\emph{Ne les vois pas comme des créatures mais comme des objets psychosensibles, si je peux me contrôler alors je peux les contrôler -} 

Le problème était qu'il ne \emph{pouvait pas}  se contrôler si facilement, qu'il ne pouvait pas se faire croire que bleu était vert par un acte de volonté. Il ne pouvait pas réprimer toutes ces pensées sur l'irrationalité qu'il y avait à se \emph{faire}  croire quelque chose. Sur \emph{l'impossibilité}  qu'il y avait à se tromper soi-même afin de croire quelque chose lorsqu'on \emph{savait}  ce que l'on était en train de faire. Tout l'entraînement contre le mensonge à soi-même que Harry s'était prodigué refusait de s'éteindre, \emph{peu importe les dommages que cela allait causer dans ce cas précis -} 

Les ombres de la Mort franchirent le milieu du couloir et Harry leva sa main, doigts écartés, et dit de la voix ferme et confiante d'un commandant : "Arrêtez-vous."

Les ombres de la Mort s'arrêtèrent.

Derrière lui Bellatrix eut un hoquet étranglé, comme s'il avait été arraché de sa gorge.

Harry fit un geste vers elle, le signal qu'il avait préparé à l'avance et qui signifiait : \emph{répète ce que tu les as entendus dire} .

"Ils disent," répondit Bellatrix, sa voix tremblait, "ils disent : 'Bellatrix Black nous a été promise. Dis-nous où elle se cache et tu seras épargné.'"

"Bellatrix ?" dit Harry, donnant à sa voix un ton amusé. "Elle s'est échappée il y a un moment."

Un instant plus tard, Harry se rendit compte qu'il aurait dû dire que Bellatrix était parmi les Aurors de l'étage supérieur, cela aurait semé une plus grande confusion -

Non, c'était une erreur de penser qu'on pouvait tromper les Détraqueurs, ils n'étaient que des \emph{choses} , ils n'étaient contrôlés que par les \emph{attentes}  -

"Ils disent," dit Bellatrix d'une voix brisée, "ils disent qu'ils savent que vous mentez."

Les vides recommencèrent à avancer.

\emph{Ses anticipations sont plus fermes que la mienne ; elle les contrôle involontairement -} 

"Ne résiste pas," dit Harry, pointant sa baguette derrière lui.

"Je, je vous aime, adieu, seigneur -"

"\emph{Somnium.} "

Entendre ces horribles mots et comprendre l'erreur de Bellatrix l'aida étrangement ; cela lui rappela pourquoi il se battait.

"Arrêtez-vous," répéta Harry. Bellatrix dormait ; maintenant, seule sa volonté, ou plutôt seules ses attentes devraient contrôler ces sphères d'annihilation -

Mais les sphères continuèrent de glisser vers l'avant, et Harry ne pu s'empêcher de s'inquiéter à l'idée que l'expérience précédente avait endommagé sa confiance, ce qui voulait dire qu'il ne \emph{pourrait pas}  les arrêter, et alors même qu'il s'observait penser cela il se mit à douter encore plus - il avait besoin de plus de temps pour se préparer, il fallait vraiment qu'il commence par pratiquer le contrôle d'un seul Détraqueur dans une cage -

Il n'y avait plus qu'un quart de couloir entre lui et les ombres de la Mort, les vents du vide étaient si forts que Harry pouvait sentir l'érosion de ses propres failles qui commençait.

Et la pensée lui vint qu'il avait peut-être tort, peut-être que les Détraqueurs \emph{avaient}  leur propres désirs et leurs propres capacités à prévoir. Ou peut-être qu'ils étaient contrôlés par la façon dont \emph{tout le monde } pensait qu'ils fonctionnaient, pas seulement par les pensées de la personne la plus proche d'eux. Et dans un cas comme dans l'autre -

Harry leva sa baguette, en position de départ pour le Patronus, et il parla :

"L'un de vous s'est rendu à Poudlard et n'est pas revenu. Il n'existe plus ; cette Mort est morte."

Les Détraqueurs firent halte, une douzaine de blessures du monde se tinrent immobiles alors que le néant hurlait autour d'eux tel un vent fatal sans destination.

"Retournez-vous et partez et ne parlez de cela à personne, petites ombres, ou je vous détruirai aussi."

Les doigts de Harry glissèrent vers la position de départ du Patronus et il se prépara à le lancer ; dans son esprit, la Terre brillait entre les étoiles, le côté jour resplendissait de bleu sous l'éclat du soleil, le côté nuit étincelait de villes humaines. Harry ne bluffait pas, il n'essayait pas de jouer avec ses pensées. Les ombres de la Mort avanceraient et seraient annihilées, ou elles repartiraient, il était prêt à une éventualité comme à l'autre...

Et les vides battirent en retraite aussi doucement qu'ils étaient venus, les vents de néant s'amenuisant à chaque mètre qu'ils traversaient, à mesure qu'ils glissaient le long des escaliers et partaient.

Qu'ils aient vraiment leur propre pseudo-intelligence ou que Harry ait enfin réussi à \emph{s'attendre}  à ce qu'ils partent... cela, il l'ignorait.

Mais ils étaient partis.

Harry prit un moment pour s'asseoir sur les escaliers à côté d'une Bellatrix inconsciente et il s'avachit autant qu'elle, ferma ses yeux un moment, juste un moment, il ne comptait certainement pas dormir à Azkaban mais il avait besoin de ce moment. Harry espérait que les Aurors allaient toujours lentement descendre les escaliers et que ça ne pourrait donc pas faire de mal de se reposer cinq minutes. Il prenait garde à avoir des pensées positives et enjouées, \emph{oh, je vais juste prendre un bon repos regénératif de cinq minutes} , plus que, disons, \emph{oh, je vais juste m'effondrer sous le coup de l'épuisement émotionnel et physique}  parce que les Détraqueurs n'étaient pas partis bien loin.

\emph{Ah, et au fait,}  dit Harry à son cerveau, \emph{tu es viré} .
\par\noindent\rule{\textwidth}{0.4pt}
"Je l'ai trouvé !", s'écria la voix d'un vieux sorcier.

\emph{Qui ?}  pensa Amélia tout en pivotant afin d'observer le retour de Dumbledore, qui portait dans ses bras -

- la seule vue, la seule personne à laquelle elle ne se serait jamais attendue -

- un homme vêtu de robes rouges déchirées, écorché comme s'il avait vécu une petite guerre, du sang séché le long de nombreuses coupures. Ses yeux étaient ouverts et il mâchait une barre de chocolat tenue dans son unique main viable.

Bahry Une-Main était \emph{en vie} .

Un cri de joie s'éleva, les Aurors abaissèrent leur baguettes, certains d'entre eux commençaient déjà à s'élancer.

"\emph{Restez sur vos gardes !} " mugit Amélia. "Vérifiez qu'ils ne sont pas sous Polynectar - inspectez Bahry, cherchez de petits Animagus ou des pièges -"
\par\noindent\rule{\textwidth}{0.4pt}
"\emph{Innerver. Wingardium Leviosa.} "

Il y eut une pause. Harry sentit, même s'il ne pouvait pas tout à fait le voir, que la femme invisible se relevait, qu'elle s'appuyait sur ses pieds et qu'elle tournait la tête pour regarder autour d'elle. "Je suis... en vie... ?"

Harry était douloureusement tenté de répondre non, juste pour voir ce qu'elle en ferait. Au lieu de cela il siffla : "Ne pose pas de questions stupides."

"Que s'est-il passé ?" murmura Bellatrix.

Et le Seigneur des Ténèbres laissa échapper un rire fou et suraigu, et il dit : "J'ai effrayé les Détraqueurs, ma chère Bella."

Il y eut une pause. Harry aurait souhaité pouvoir observer le visage de Bellatrix ; s'était-il trompé dans son choix de mots ?

Après un moment, d'une voix chevrotante : "Se pourrait-il, seigneur, que sous cette forme nouvelle vous ayez commencé à vous soucier de moi -"

"Non," répondit Harry d'une voix froide, et il se détourna d'elle (mais il garda sa baguette pointée dans sa direction) puis commença à marcher. "Et assures-toi de ne plus m'offenser ou je t'abandonnerai ici que tu me sois utile ou pas. Maintenant suis, ou reste en arrière ; j'ai à faire."

Harry avança à grand pas sans écouter les halètements qui venaient de derrière lui ; il savait que Bellatrix suivait.

...parce que la dernière chose dont cette femme avait besoin, la dernière chose qu'il lui fallait penser avant que le guérisseur psychiatrique ne commence à essayer de la déprogrammer, c'était de croire que le Seigneur des Ténèbres pourrait un jour lui rendre son amour.
\par\noindent\rule{\textwidth}{0.4pt}
Le vieux sorcier lissa sa barbe d'un air contemplatif en regardant l'Auror Bahry se faire transporter hors de la pièce par deux Aurors costauds.

"Amélia, comprends-tu cela ?"

"Non," répondit-elle simplement. Elle soupçonnait un piège qui lui aurait pour l'instant échappé et c'était pourquoi l'Auror Bahry serait maintenu sous bonne garde, à l'écart du groupe principal.

"Peut-être," répondit-il lentement, "que celui qui parmi eux est capable de lancer le Patronus est plus qu'un simple otage. P'têt ben quelqu'un qu'on a dupé pour qu'il vienne ? Quelle que soit la raison, ils ont gardé notre Auror en vie ; ne soyons pas les premiers à exercer des sortilèges mortels lorsque nous les trouverons -"

"Je vois," dit la vieille sorcière, comprenant soudain, "c'était \emph{ça}  leur plan. L'Oublietter et le garder en vie ne leur coûte rien et \emph{nous}  fait hésiter -" Amélia hocha la tête, résolue, et elle dit à son équipe : "Nous continuons comme avant."

Le vieux sorcier soupira. "Des nouvelles des Détraqueurs ?"

"Si je te le dis," lâcha-t-elle, "est-ce que tu vas encore déguerpir ?"

"Cela ne te coûte rien, Amélia," dit le vieux sorcier avec douceur, "et pourrait éviter le combat à l'un des tiens."

\emph{Me coûte rien à ma part ma chance de me venger -} 

Mais ce n'était rien comparé à l'autre poids dans la balance ; le vieux sorcier agaçant finissait souvent par avoir raison et c'était en partie ce qui le rendait si agaçant.

"Les Détraqueurs ont cessé de répondre aux questions concernant l'autre personne qu'ils ont dit avoir vue," lui dit-elle, "et ils refusent de dire pourquoi, ni où il l'ont vue."

Dumbledore se tourna vers le flamboyant phénix d'argent perché sur son épaule et dont l'éclat illuminait le couloir entier, et il reçut un signe de dénégation silencieux en réponse. "Je ne peux pas non plus les détecter," dit Dumbledore. Puis il haussa les épaules. "Je suppose que je vais juste parcourir toute la spirale de haut en bas et voir si quelque chose se passe, non ?"

Amélia lui aurait ordonné de ne pas le faire si elle avait pensé que cela aurait eut le moindre effet.

"Albus," dit Amélia alors que le vieux sorcier se retournait pour partir, "même toi tu peux être pris en embuscade."

"Absurde, ma chère," dit le vieux sorcier d'un ton guilleret alors qu'il s'en allait de nouveau, secouant en signe d'admonestation sa baguette de près d'un demi-mètre faite d'un bois noir-gris impossible à identifier, "je suis invincible."

Il y eut une pause.

("Il ne vient pas de dire ça -" murmura le nouvel Auror, une jeune demoiselle encore pincée du nom de Noëlle Curry, à l'intention Brooks, de l'Auror senior de son trio. "Si ?")

("Il peut se le permettre," lui murmura Isabel en retour, "c'est \emph{Dumbledore} , même le Destin ne le prend plus au sérieux.")

"Et c'est pour ça," dit Amélia d'un ton grave au bénéfice des Aurors plus jeunes, "qu'on ne l'appelle jamais pour rien à moins d'y être absolument obligé."
\par\noindent\rule{\textwidth}{0.4pt}
Harry était allongé, totalement immobile, sur le banc dur qui faisait office de lit dans cette cellule avec une couverture tirée par-dessus lui, et il essayait de rester le plus statique possible en attendant que la peur revienne. Un Patronus approchait, et il était puissant. Bellatrix était masquée par une Relique de la Mort et aucune sortilège simple ne pouvait pénétrer cela ; mais Harry ne savait pas quels autres arts les Aurors pourraient employer afin de le détecter lui, et il n'osait pas révéler son ignorance en l'interrogeant. Alors Harry restait allongé sur le dur matelas, dans une cellule à la porte fermée à clé, derrière une immense porte de métal elle aussi fermée, dans une noirceur absolue, une fine couverture tirée par-dessus lui, et il espérait que la personne qui approchait ne regarderait pas à l'intérieur, ou au moins qu'elle n'y regarderait pas de trop près -

Ce n'était vraiment pas un paramètre sur lequel il pouvait avoir le moindre effet, cette partie de son destin reposait entièrement dans les mains des Variables Cachées. La majeure partie de son esprit se concentrait sur la métamorphose qu'il était en train d'opérer.

À l'écoute du silence, Harry entendit les pas rapide s'approcher ; ils s'interrompirent devant la porte, puis -

- continuèrent.

La peur revint bientôt.

Il ne s'autorisa pas à remarquer son soulagement pas plus qu'il ne s'était autorisé à remarquer sa peur. Il maintenait dans son esprit la forme d'un appareil Moldu bien plus gros qu'une batterie de voiture et appliquait lentement cette Forme à la substance d'un cube de glace (que Harry avait congelé en utilisant \emph{Frigideiro}  sur de l'eau venue d'une bouteille de sa bourse). On n'était pas censé métamorphoser des choses destinées à être brûlées, mais entre le fait que la substance originale était aqueuse et le sortilège de bulle qui protégeait son arrivée d'air, Harry espérait que cela ne rendrait personne malade.

Maintenant, la seule question était de savoir si, avant que les Aurors ne fassent une inspection poussée de ce groupe de cellules, il y aurait assez de temps pour que Harry puisse finir sa métamorphose et la métamorphose partielle qu'il comptait faire ensuite -
\par\noindent\rule{\textwidth}{0.4pt}
Lorsque le vieux sorcier revint les mains vides, même Amélia commença à ressentir un soupçon d'inquiétude. Elle et les deux autres équipes d'Aurors avaient balayé un tiers des trois spirales de façon synchronisée afin de ne créer aucune ouverture qui pourrait être franchie en passant à travers un plafond et ils attendait encore de trouver le moindre indice.

"Pourrais-je te demander un rapport ?" dit Amélia, gardant le mordant hors de sa voix.

"D'abord un simple passage de bas en haut," dit le vieux sorcier. Il fronçait les sourcils, ce qui rendait son visage encore plus ridé qu'à l'habitude. "J'ai examiné la cellule de Bellatrix et j'ai trouvé une poupée de mort à sa place. Je pense que l'évasion n'était pas censée être remarquée. Il y a quelque chose de caché dans un coin, sous un morceau de tissu ; je n'y ai pas touché afin que les Aurors l'examinent. Sur le chemin du retour, j'ai ouvert chaque porte et j'ai regardé dans les cellules. Je n'ai rien vu qui soit Désillusionné, seulement les prisonniers -"

Ils furent interrompus par un cri venant du phénix rouge-or et tous les Aurors tressaillirent en l'entendant. Il exprimait une condamnation et une demande urgente qui poussa presque Amélia à se mettre à courir.

"- qui sont dans une condition assez alarmante," dit Dumbledore d'un ton doux. L'espace d'un instant, les yeux bleus derrière les verres en croissant de lune devinrent très froids. "L'un de vous me parlera-t-il des conséquences de leurs actions ?"

"\emph{Je}  n'ai pas - " commença Amélia.

"Je sais," dit le vieux sorcier. "Mes excuses, Amélia." Il soupira. "Parmi ceux que j'ai vu, certains des prisonniers les plus récents avaient encore quelques restes de leur magie, mais je n'ai ressenti aucune puissance intacte ; le plus fort avait autant de magie qu'un enfant de première année. De nombreuses fois j'ai entendu Fumseck crier de détresse, mais jamais de défi. Il semble que vous allez devoir continuer votre recherche ; ils se cachent assez bien pour échapper à mon simple regard."
\par\noindent\rule{\textwidth}{0.4pt}
Lorsque Harry eut fini sa première métamorphose, il s'assit, ôta la couverture qui l'avait recouvert, lança un rapide \emph{Lumos} , jeta un coup d'œil à sa montre et fut surpris de découvrir que près d'une heure et demie s'étaient écoulée. Quelle fraction de ce temps datait d'après le passage de la personne qui avait ouvert puis refermé la porte - Harry n'avait bien sûr pas regardé dans cette direction - cela, il ne pouvait pas le deviner.

"Seigneur... ?" chuchota la voix de Bellatrix, douce et très hésitante.

"Tu peux parler à présent," dit Harry. Il lui avait dit de rester silencieuse pendant qu'il travaillait.

"C'est Dumbledore qui a posé son regard sur nous."

Pause.

"Intéressant." dit Harry d'un ton neutre. Il était heureux de ne pas s'en être rendu compte lorsque cela s'était produit. Il semblait qu'il avait eu \emph{plutôt chaud} .

Harry parla à sa bourse et commença à extraire l'appareil magique qu'il associerait au produit de son heure de travail. Puis, lorsque cela fut fait, un autre mot fit apparaître un tube de colle industrielle ; avant de l'utiliser, Harry lança le sortilège de bulle sur lui-même et sur Bellatrix, et il demanda à cette dernière de lancer le même sortilège sur le serpent afin que les émanations de la colle ne les atteignent pas dans l'espace confiné de la cellule.

Lorsque la colle eut commencé à prendre, liant la technologie à la magie, Harry déposa le tout sur le lit et s'assit au sol, reposant sa magie et sa volonté pendant un moment avant de s'essayer à la métamorphoser suivante.

"Seigneur..." dit Bellatrix d'une voix hésitante.

"Oui ?" dit la voix sèche.

"Quel est cet appareil que vous fabriquez ?"

Harry réfléchit rapidement. Cela semblait être une bonne opportunité de vérifier ses plans auprès d'elle sous couvert de questions orientées.

"Considère ceci, ma chère Bella," dit Harry d'un voix soyeuse. "À quel point est-il difficile, pour un sorcier puissant, de découper les murs d'Azkaban ?"

Il y eut une pause, puis la voix de Bellatrix lui parvint, lente et perplexe : "Pas difficile du tout, seigneur... ?"

"En effet," dit la voix sèche et haute perchée du maître de Bellatrix. "Suppose que quelqu'un le fasse et s'envole par le trou ainsi formé au moyen d'un balais, vers les cieux puis vers le large. Faire évader un prisonnier d'Azkaban semble alors facile, n'est-ce pas ?"

"Mais seigneur..." dit Bella. "Les Aurors - ils ont leur propre balais, seigneur, des balais rapides -"

Harry écouta : les choses étaient telles qu'il les avait imaginées. Le Seigneur des Ténèbres répondit, de nouveau du ton soyeux de l'interrogatoire Socratique, et Bellatrix posa une autre question, à laquelle Harry ne s'était pas attendu, mais la question que Harry posa en guise de réponse révéla que cela n'aurait en fin de compte pas d'importance. Et en réponse à la dernière question de Bellatrix, le Seigneur des Ténèbres se contenta de sourire et dit qu'il était temps qu'il reprenne son travail.

Puis Harry se releva, alla jusqu'au fond de la cellule et toucha la dure surface du mur du bout de sa baguette - le mur d'Azkaban, le métal solide qui les protégeait d'une exposition directe à la fosse des Détraqueurs.

Et Harry entama une métamorphose partielle.

Il espérait que le sortilège serait plus rapide. Il avait passé des heures et des heures à pratiquer cette magie unique, la rendant ainsi routinière, pas beaucoup plus difficile que la métamorphose ordinaire. Le volume de la forme qu'il changeait n'était pas si grand que ça et la forme métamorphosée serait peut-être haute, large et longue, mais elle serait très fine. Un demi-millimètre, avait-il songé, un demi-millimètre suffirait étant donné la régularité parfaite de la surface...

Sur le long banc qui servait de lit de prison, là où Harry avait déposé l'appareil technologie métamorphosé et l'objet magique afin que la colle qui les reliait puisse sécher, de petites lettres d'or scintillaient à la surface de l'objet Moldu. Harry n'avait pas réellement \emph{prévu}  leur présence mais elles avaient été perpétuellement présentes dans ses pensées et semblaient donc être devenues partie intégrante de la forme métamorphosée.

Il y avait de nombreuses choses que Harry aurait pu dire avant de faire usage de ce triomphe d'ingéniosité technologique. Toute une collection de choses qui auraient été appropriées, d'une façon ou d'une autre. Ou du moins des choses que Harry \emph{aurait}  pu dire, \emph{aurait dites}  si Bellatrix n'avait pas été là.

Mais il n'y avait qu'une seule chose à dire, une chose qu'il n'aurait l'opportunité de dire que cette fois-ci, pour laquelle une meilleure opportunité d'être dite (ou du moins d'être \emph{pensée} , s'il ne pouvait la dire) ne se présenterait jamais plus. Il n'avait pas vu le film mais il avait vu une bande-annonce, et pour une raison ou une autre la phrase était restée scotchée dans son esprit.

Les petites lettres d'or sur l'appareil Moldu disaient :

\emph{Très bien, bande de nazes primitifs ! Écoutez-moi !} 

