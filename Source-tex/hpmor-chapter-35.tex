
\chapter{Problèmes de coordination, partie 3}

Ils s'étaient rendus dans le bureau du professeur de Défense, et celui-ci avait scellé la porte avant de se renverser dans sa chaise et de prendre la parole.

La voix du professeur de Défense était très calme, et cela perturbait Harry bien plus que si le professeur Quirrell n'avait crié.

"J'essaie", dit doucement le professeur Quirrell, "de me montrer indulgent en raison de votre jeunesse. Du fait que j'étais moi-même, à votre âge, un extraordinaire idiot. Vous parlez comme un adulte, vous vous mêlez de jeux adultes, et j'oublie parfois que vous n'êtes qu'une mouche du coche. J'espère, M. Potter, que vos interventions puériles ne viennent pas de vous tuer, de ruiner ce pays, et de perdre la prochaine guerre."

Harry avait beaucoup de mal à contrôler sa respiration. "Professeur Quirrell, j'en ai dit bien moins que je ne l'aurais souhaité, mais il fallait que je dise quelque chose. Vos propositions sont extrêmement alarmantes pour quiconque doté de la moindre notion d'Histoire Moldue du dernier siècle. Les fascistes Italiens, des gens extrêmement désagréables, ont prit leur nom de \emph{fasces} , un ensemble de morceaux de bois liés ensembles, pour symboliser l'idée que l'unité est la force -"

"Alors les vilains fascistes Italiens croyaient que l'unité est plus forte que la division," dit le professeur Quirrell. La dureté commençait à poindre dans sa voix. "Peut-être qu'ils croyaient aussi que le ciel est bleu et qu'ils acceptaient l'idée selon laquelle il est néfaste de se faire tomber des rochers sur la tête."

\emph{La stupidité inversée n'est pas l'intelligence ; la personne la plus stupide du monde peut dire que le soleil brille, il n'en fait pas nuit pour autant...}  "Très bien, vous avez raison, c'était un argument ad hominem, ce n'est pas faux \emph{parce que}  les fascistes l'ont dit. Mais professeur Quirrell, vous ne pouvez pas demander à tout le monde de porter la Marque d'un dictateur ! C'est un point unique de défaillance ! Écoutez, je vais le formuler ainsi : imaginez que l'ennemi lance Imperius à celui qui contrôle la Marque -"

"Les puissants sorciers ne sont pas si simples à soumettre à l'Imperius," dit sèchement le professeur Quirrell. "Et si vous ne pouvez pas trouver un chef de valeur, alors vous êtes de toute façon foutus. Mais les chefs de valeur existent, la question est de savoir si les gens les suivront."

Harry se passa les mains dans les cheveux sous l'effet de la frustration. Il voulait demander une pause et faire lire \emph{La montée et la chute du troisième Reich}  au professeur Quirrell puis recommencer la conversation. "J'imagine que si je suggérais que la démocratie est une meilleure forme de gouvernement que la dictature -"

"Je vois," dit le professeur Quirrell. Il ferma brièvement les yeux, puis les rouvrit. "M. Potter, la stupidité du Quidditch vous est limpide parce que vous n'avez pas grandi en vénérant ce jeu. Si vous n'aviez jamais entendu parler d'élections, M. Potter, et que vous pouviez simplement voir \emph{ce qui est} , ce que vous verriez ne vous plairait pas. Regardez notre ministre, élu, de la magie. Est-il le plus sage, le plus fort, le plus grand de notre pays ? Non ; c'est un bouffon possédé par l'argent de Lucius Malfoy. Les sorciers ont été aux urnes et ont choisi entre Cornelius Fudge et Tania Leach, qui s'étaient battus dans un grand et divertissant concours après que la Gazette du Sorcier, que Lucius Malfoy contrôle aussi, ait décidé qu'ils étaient les deux seuls candidats sérieux. Personne ne pourrait suggérer avec sérieux que Cornelius Fudge a vraiment été choisi comme le meilleur chef que notre pays avait à offrir. Ce n'est pas différent dans le monde Moldu, à ce que j'en ai compris ; le dernier journal Moldu que j'ai lu mentionnait que le dernier président des États-Unis avait été un acteur de cinéma à retraite. Si vous n'aviez pas grandi avec les élections, M. Potter, leur idiotie vous serait aussi limpide que celle du Quidditch."

Harry se tint là, la bouche ouverte, cherchant ses mots. "Le but des élections n'est pas de produire le meilleur chef, c'est de maintenir les politiciens dans une peur suffisamment grande des électeurs pour qu'ils ne deviennent pas complètement maléfiques comme les dictateurs qui -"

"La dernière guerre, M. Potter, se jouait entre le Seigneur des Ténèbres et Dumbledore. Et même si Dumbledore était un chef doté de nombreux défauts et qui perdait la guerre, il serait \emph{ridicule}  de suggérer que \emph{n'importe lequel}  des ministres de la magie élus pendant cette période auraient pu prendre sa place ! La force vient de la puissance des sorciers et de ceux qui les suivent, pas des élections et des idiots qu'elles élisent. C'est la leçon de l'histoire récente de l'Angleterre magique ; et je doute que la prochaine guerre vous en enseigne une autre. \emph{Si}  vous survivez, M. Potter, ce que vous ne ferez \emph{pas}  à moins d'abandonner les enthousiastes illusions de l'enfance !"

"Si vous pensez qu'il n'y a aucun danger dans la ligne de conduite que vous soutenez," dit Harry, sa voix devenait tranchante en dépit de ses efforts, "alors cela est aussi de l'enthousiasme puéril."

Harry regarda sombrement dans les yeux du professeur Quirrell, qui le fixa en retour sans ciller.

"De tels dangers," dit froidement le professeur Quirrell, "doivent être étudiés dans des bureaux tels que celui-ci, pas dans des discours. Les idiots qui ont élu Cornelius Fudge ne s'intéressent pas aux complexités et à la prudence. Présentez leur quelque chose de plus nuancé qu'une acclamation rugissante et vous ferez face à votre guerre tout seul. \emph{Cela} , M. Potter, a constitué votre erreur puérile, que Draco Malfoy n'aurait pas faite à huit ans. Cela aurait dû être évident, même pour \emph{vous} , que vous auriez dû rester silencieux et \emph{me consulter avant } au lieu de mentionner vos inquiétudes devant la foule !"

"Je ne suis certainement pas un ami d'Albus Dumbledore," dit Harry, un froid dans sa voix en réponse à celui du professeur Quirrell. "Mais il n'est pas un enfant, et il ne semblait pas trouver mes inquiétudes puériles, ni que j'aurais dû attendre avant d'en parler."

"Oh," dit le professeur Quirrell, "alors vous vous faites souffler la réplique par le directeur, maintenant ?" et il se leva.
\par\noindent\rule{\textwidth}{0.4pt}
Lorsque Blaise passa l'angle, en chemin vers le bureau, il vit que le professeur Quirrell était déjà là, appuyé contre le mur.

"Blaise Zabini," dit le professeur de Défense en se redressant ; ses yeux étaient incrustés dans son visage tels deux pierres noires, et sa voix envoya un frisson de peur le long de la colonne vertébrale de Blaise.

\emph{Il ne peut rien contre moi, je dois juste me souvenir que -} 

"Je crois," dit le professeur Quirrell, d'une voix claire et forte, "avoir déjà deviné le nom de votre employeur. Mais j'aimerais l'entendre de vos lèvres, et aussi entendre le prix qui vous a acheté."

Blaise savait qu'il suait sous ses robes et que l'humidité était probablement visible sur son front. "J'ai eu une chance de montrer que j'étais meilleur que les trois généraux et je l'ai saisie. Beaucoup de gens me détestent maintenant, mais il y a aussi plein de Serpentard qui m'adoreront parce que j'ai fait ça. Qu'est-ce qui vous fait penser que je -"

"Vous n'avez pas inventé le plan de la bataille d'aujourd'hui, M. Zabini. Dites-moi qui l'a fait."

Blaise déglutit bruyamment. "Eh bien... je veux dire, dans ce cas... alors vous savez déjà qui l'a fait, c'est ça ? Le seul assez fou, c'est Dumbledore. Et il me protégera si vous essayez de faire quelque chose."

"En effet. Dites-moi le prix." Le regard du professeur de Défense était toujours dur.

"C'est ma cousine Kimberly," dit Blaise, déglutissant de nouveau et essayant de contrôler sa voix. "Elle existe, et elle se fait vraiment martyriser, Potter a vérifié ça, il n'était pas stupide. Seulement Dumbledore a dit qu'il avait un peu poussé les brutes à le faire, juste pour aider pour son plan, et que si je travaillais pour \emph{lui}  elle irait bien, mais que si \emph{j'allais}  avec Potter, elle pourrait avoir encore plus d'ennuis !"

Le professeur Quirrell resta silencieux un long moment.

"Je vois," dit le professeur Quirrell, sa voix maintenant plus douce. "M. Zabini, si un tel événement devait de nouveau survenir, je vous invite à me contacter directement. J'ai mes propres moyens de protéger mes amis. Maintenant, une dernière question : même avec tout le pouvoir que vous aviez, forcer une égalité aurait été difficile. Qui Dumbledore vous a-t-il ordonné de faire gagner le cas échéant ?"

"Soleil", dit Blaise.

Le professeur Quirrell hocha la tête. "Comme je le pensais." Le professeur de Défense soupira. "Dans votre carrière future, M. Zabini, je vous suggère de ne pas vous essayer à des intrigues aussi complexes. Elles ont tendance à échouer."

"Euh, en fait, j'ai dit ça au directeur," dit Blaise, "et il a dit que c'était pour ça qu'il était important d'avoir plus d'une intrigue en cours à la fois."

Le professeur Quirrell passa une main lasse sur son front. "On se demande comment le Seigneur des Ténèbres n'est pas devenu fou en \emph{le}  combattant. Vous pouvez vous rendre à votre rendez-vous avec le directeur, M. Zabini. Je ne dirai rien de cela, mais si le directeur découvrait d'une façon ou d'une autre que nous avons parlé, souvenez-vous de mon offre de vous offrir la protection dont je dispose. Allez-y."

Blaise n'attendit pas un mot de plus. Il se contenta de pivoter et de s'enfuir.
\par\noindent\rule{\textwidth}{0.4pt}
Le professeur Quirrell attendit un moment, puis dit "C'est bon, M. Potter."

Harry arracha la Cape d'Invisibilité de sa tête et la fourra dans sa bourse. Il tremblait d'une rage telle qu'il pouvait à peine parler. "Il a \emph{quoi}  ? Il a fait \emph{quoi}  ?"

"Vous auriez dû le déduire vous-même, M. Potter," dit doucement le professeur Quirrell. "Vous devez apprendre à brouiller votre vision jusqu'à réussir à voir la forêt masquée par les arbres. Toute personne entendant des histoires à votre sujet et ne sachant pas que vous étiez le mystérieux Survivant pourrait aisément déduire que vous possédez une cape d'invisibilité. Prenons du recul, brouillons les détails, et qu'observons-nous ? Il y avait une grande rivalité entre élèves, et leur compétition s'est achevée sur un nul parfait. Ce genre de chose n'a lieu que dans les histoires, M. Potter, et il y a une seule personne dans cette école qui pense en termes d'histoires. C'était une intrigue étrange et complexe, et vous auriez dû vous rendre compte qu'elle n'était pas caractéristique du jeune Serpentard auquel vous faisiez face. Et je vous ai prévenu qu'il y avait un agent quadruple ; vous saviez que Zabini était au moins un agent triple, vous auriez donc dû deviner que c'était très probablement lui. Non, je ne déclarerai pas la bataille invalide. Vous avez tous trois échoué au test et perdu contre votre ennemi commun."

À ce stade, les tests importaient peu à Harry. "Dumbledore a \emph{fait chanter}  Zabini en \emph{menaçant sa cousine}  ? Juste pour que notre bataille s'achève par une égalité ? \emph{Pourquoi}  ?"

Le professeur Quirrell eut un rire sans joie. "Peut-être le directeur pensait-il que la rivalité était bonne pour son héros de compagnie et souhaitait la voir continuer. Pour le plus grand bien, vous comprenez. Ou peut-être qu'il est simplement devenu fou. Vous voyez, M. Potter, tout le monde sait que la folie de Dumbledore est un masque, qu'il est sain et prétend être fou. Tout le monde est fier de sa fine perspicacité, et, connaissant l'explication secrète, arrête de chercher. Il ne leur vient pas à l'esprit qu'il est \emph{aussi}  possible d'avoir un masque sous le masque, d'être un fou faisant semblant d'être normal faisant semblant d'être fou. Et j'ai peur, M. Potter, d'avoir des affaires urgentes à régler ailleurs, et de devoir partir ; mais je vous conseillerais avec force de ne pas vous faire souffler la réplique par Albus Dumbledore lorsque vous êtes en guerre. À plus tard, M. Potter."

Et le professeur de Défense inclina sa tête avec une certaine ironie puis il s'éloigna dans la même direction que celle vers laquelle Zabini avait fuit, et Harry se tint là, la bouche ouverte sous l'effet du choc.
\par\noindent\rule{\textwidth}{0.4pt}
\emph{Après-coup}  \emph{: Harry Potter} 

Harry se traîna lentement vers le dortoir Serdaigle, ses yeux aveugles aux murs, aux peintures, aux autres élèves ; il monta des escaliers et descendit des rampes sans ralentir, ni accélérer, ni se rendre compte d'où il allait.

Après le départ du professeur Quirrell, il avait mis plus d'une minute à se rendre compte que ses seules sources d'information au sujet de la participation de Dumbledore étaient (a) Blaise Zabini, et il aurait dû être un abruti complet pour lui faire de nouveau confiance, et (b) le professeur Quirrell, qui aurait facilement pu contrefaire une intrigue dans le style de Dumbledore, et qui pourrait aussi penser qu'un peu de rivalité estudiantine était pour le mieux ; et qui avait, si vous preniez du recul et brouilliez les détails, proposé à l'instant de transformer le pays en une dictature de sorciers.

Et il était aussi possible que Dumbledore \emph{soit}  derrière Zabini, et que le professeur Quirrell ait sincèrement essayé de combattre le Marque des Ténèbres d'égal à égal et d'empêcher la répétition d'un numéro qu'il avait trouvé pathétique. Qu'il ait tenté de s'assurer que Harry ne se retrouve pas à combattre le Seigneur des Ténèbres seul pendant que tout le monde se serait caché, effrayé, essayant de rester à l'écart des lignes de tir, attendant que Harry les sauve.

Mais la vérité, c'était que...

Eh bien...

L'idée ne dérangeait pas Harry plus que ça.

Il savait que c'était le genre de chose censée rendre les héros amers et pleins de ressentiment.

Bah. Harry était tout à fait en faveur d'une situation ou tout le monde \emph{restait hors de danger}  tandis que le Survivant s'occupait du Seigneur des Ténèbres tout seul, plus ou moins un petit nombre de compagnons. Si le prochain conflit contre le Seigneur des Ténèbres en arrivait à une seconde guerre des sorciers où pleins de gens mouraient et que tout un pays se retrouvait impliqué, ça voudrait dire que Harry aurait \emph{déjà échoué} .

Et si après cela une guerre se déclarait entre les sorciers et les Moldus, il importerait peu de savoir qui gagnerait, Harry aurait déjà échoué en laissant les choses aller jusque là. Et puis, qui avait dit que les sociétés ne pouvaient pas s'intégrer pacifiquement quand le secret volait inévitablement en éclats ? (Même si Harry pouvait entendre la voix sèche du professeur Quirrell dans son esprit qui lui demandait s'il était un idiot, qui lui disait toutes les choses évidentes...) Et si les mages et les Moldus ne pouvaient vivre en paix, alors, plutôt que de laisser une guerre éclater, Harry combinerait la magie et la science et trouverait comment évacuer tous les sorciers sur Mars ou ailleurs.

Parce que si on en arrivait à une guerre pour l'extermination...

C'était ce dont le professeur Quirrell ne s'était pas rendu compte, la plus importante question qu'il avait oublié de poser à son jeune général.

La véritable raison pour laquelle Harry n'avait aucune intention de se laisser convaincre de soutenir l'idée d'une Marque de la Lumière, peu importe \emph{à quel point}  cela l'aiderait dans son combat contre le Seigneur des Ténèbres.

Un Seigneur des Ténèbres et cinquante partisans marqués avaient été un péril contre toute l'Angleterre magique.

Si toute l'Angleterre prenait la Marque d'un chef fort, elle serait un péril pour tout le monde magique.

Et si tous les sorciers prenaient une Marque, ils seraient un danger pour le reste de l'humanité.

Personne ne savait exactement combien il y avait de sorciers dans le monde. Il avait fait quelques estimations avec Hermione et avait trouvé des nombres de l'ordre d'un million.

Mais il y avait six milliards de Moldus.

Si on en arrivait à une guerre totale...

Le professeur Quirrell avait oublié de demander quel camp Harry protégerait.

Une civilisation scientifique, allant vers l'avant, les yeux levés vers le ciel, sachant que son destin était d'atteindre les étoiles.

Et une civilisation magique, dont la magie s'effaçait tandis que son savoir se perdait, toujours gouvernée par une noblesse qui voyait les Moldus comme des êtres pas tout à fait humains.

C'était une sensation horriblement triste, mais pas une qui comportait la moindre trace de doute.
\par\noindent\rule{\textwidth}{0.4pt}
\emph{Après-coup : Blaise Zabini} 

Blaise avançait dans les couloirs avec une lenteur composée et attentive tandis que son cœur battait follement et qu'il essayait de le calmer -

"Ahem," dit une voix sèche et chuchotante depuis une alcôve ombragée devant laquelle il passait.

Blaise fit un bond mais ne cria pas.

Lentement, il se retourna.

Dans ce sombre recoin se trouvait une cape noire si large et tourbillonnante qu'il était impossible de déterminer si la silhouette qu'elle masquait était mâle ou femelle, et au-dessus de la cape se trouvait un chapeau noir aux larges rebords, et un nuage noir semblait s'amonceler en-dessous, cachant le visage de celui ou de ce qui se trouvait derrière.

"Votre rapport," chuchota M. Chapeau et Cape.

"J'ai dit exactement ce que vous m'aviez dit de dire," dit Blaise. Sa voix était un peu plus calme maintenant qu'il n'était plus en train de mentir à quelqu'un. "Et le professeur Quirrell a réagit exactement de la façon dont vous vous y attendiez."

Le large chapeau noir s'était incliné et redressé, comme si la tête en-dessous avait opiné. "Excellent," dit le chuchotement impossible à attribuer à qui que ce soit de connu. "La récompense que je t'ai promise est déjà en chemin vers ta mère, par chouette."

Blaise hésita, mais sa curiosité le dévorait. "Est-ce que je peux vous demander, maintenant, pourquoi vous voulez créer des problèmes entre le professeur Quirrell et Dumbledore ?" Le directeur n'avait rien eu à faire avec les brutes de Gryffondor dont Blaise avait entendu parler, et en plus d'aider Kimberly, le directeur avait proposé que le professeur Binns lui donne d'excellentes notes en Histoire de la Magie même s'il rendait un parchemin vide en lieu et place de ses devoirs, auquel cas il aurait quand même dû se rendre aux cours et faire semblant de rendre son travail. À vrai dire, Blaise aurait trahi les trois généraux gratuitement, cousine ou pas cousine, mais il n'avait pas considéré qu'il fut nécessaire de mentionner ce fait.

Le large chapeau noir s'inclina d'un côté, comme pour exprimer un regard interloqué. "Dis moi, ami Blaise, t'est-il venu à l'esprit que les traîtres qui trahissent autant rencontrent souvent de funestes fins ?"

"Nan," dit Blaise, regardant droit vers le nuage noir sous le chapeau. "Tout le monde sait qu'il n'arrive jamais rien de \emph{vraiment}  mauvais aux élèves dans l'enceinte de Poudlard."

M. Chapeau et Cape chuchota un gloussement. "En effet," dit le chuchotement. "Avec le meurtre d'un élève voilà cinq décennies comme exception pour prouver la règle, puisque Salazar Serpentard aurait mit son monstre sous clé dans les anciens murs à un niveau plus élevé que celui du directeur lui-même."

Blaise regarda le nuage noir, se sentant maintenant un peu mal à l'aise. Mais il aurait fallu qu'il s'agisse d'un professeur de Poudlard pour qu'il puisse lui faire quoi que ce soit de grave sans déclencher d'alarme. Quirrell et Rogue étaient les seuls professeurs qui pourraient faire une chose pareille, et Quirrell n'aurait aucun intérêt à se duper \emph{lui-même} , et Rogue ne ferait pas de mal à un de ses Serpentard... non ?

"Non, ami Blaise," chuchota le nuage noir, "je souhaitais seulement te conseiller de ne jamais rien tenter de tel dans ta vie adulte. Tant de trahisons mèneraient certainement à au moins une vengeance."

"Ma \emph{mère}  n'a jamais subit de vengeance," dit fièrement Blaise. "Même si elle a épousé \emph{sept}  maris et que chacun d'eux est mort mystérieusement en lui laissant beaucoup d'argent;"

"Vraiment ?" dit le chuchotement. "Comment a-t-elle persuadé le septième de l'épouser après qu'il ait entendu ce qui était arrivé aux six autres ?"

"J'ai demandé à maman," dit Blaise," et elle a dit que je ne pourrais pas savoir avant d'être assez vieux, et je lui ai demandé à quel âge je serais assez vieux, et elle a dit, plus vieux qu'elle."

Encore le gloussement chuchoté. "Eh bien dans ce cas, ami Blaise, félicitations pour avoir suivi les traces de ta mère. Vas, et si tu ne dis rien de cela, nous ne nous rencontrerons plus."

Blaise recula maladroitement, sentant une étrange réticence à l'idée de lui tourner le dos.

Le chapeau s'inclina. "Oh, allons, petit Serpentard. Si tu étais vraiment l'égal de Harry Potter ou Draco Malfoy, tu te serais déjà rendu compte que mes menaces voilées n'avaient pour but que d'assurer ton silence face à Albus. Aurais-je eu l'intention de te faire du mal, je n'aurais pas fait de sous-entendu ; si je n'avais rien dit, \emph{alors}  tu aurais dû t'inquiéter."

Blaise se redressa, se sentant un peu insulté, et opina en direction de M. Chapeau et Cape, puis il se tourna d'un mouvement volontaire et partit à grand pas vers son rendez-vous avec le directeur.

Il avait espéré jusqu'au bout que quelqu'un d'\emph{autre}  surgisse et lui donne une opportunité de vendre M. Chapeau et Cape.

Mais après tout, maman n'avait pas trahi sept maris différents \emph{en même temps} . Si on regardait les choses sous cet angle, il se débrouillait encore mieux qu'elle.

Et Blaise Zabini continua de marcher vers le bureau du directeur, souriant, heureux d'être un agent quintuple -

Le garçon trébucha l'espace d'un instant, puis il se redressa, chassant l'étrange sensation de désorientation.

Et Blaise Zabini continua de marcher vers le bureau du directeur, souriant, heureux d'être un agent quadruple.
\par\noindent\rule{\textwidth}{0.4pt}
\emph{Après-coup : Hermione Granger} 

Le messager ne l'approcha que lorsqu'elle fut seule.

Hermione quittait juste les toilettes des filles, où elle se cachait parfois pour penser, et un grand chat brillant bondit de nulle part et dit : "Miss Granger ?"

Elle laissa échapper un petit glapissement avant de se rendre compte que le chat avait parlé de la voix du professeur McGonagall.

Même si elle n'avait pas été effrayée, seulement surprise : le chat était clair, brillant et magnifique, irradiant d'une lumière blanche-argentée, comme des rayons de soleil couleur lune, et elle ne pouvait pas s'imaginer avoir peur.

"Qu'êtes-vous ?" dit Hermione.

"C'est un message du professeur McGonagall," dit le chat, toujours de la voix du professeur. "Peux-tu te rendre dans mon bureau et ne parler de ceci à personne ?"

"J'arrive tout de suite," dit Hermione, encore surprise, et le chat bondit et disparu ; seulement il ne disparut pas, il s'en fut ; du moins c'est ce que son esprit lui dit, même si ses yeux venaient de le voir disparaître.

Lorsque Hermione atteint le bureau de son professeur préféré, son esprit était plein de spéculations tourbillonnantes. Y avait-il quelque chose qui n'allait pas avec ses notes de métamorphose ? Mais alors pourquoi le professeur McGonagall dirait-elle de n'en parler à personne ? C'était probablement au sujet des exercices de métamorphose partielle de Harry...

Le visage du professeur McGonagall semblait inquiet, pas sévère, et Hermione s'assit face à son bureau - essayant d'empêcher ses yeux d'aller jusqu'au mur d'alcôves qui contenait les devoirs du professeur McGonagall, elle s'était toujours demandé quel genre de travail les adultes devaient accomplir pour faire fonctionner l'école et s'ils pourraient avoir besoin de son aide...

"Mlle. Granger," dit le professeur McGonagall, "laissez-moi commencer par vous dire que je sais déjà que le directeur vous a demandé de faire ce vœu -"

"Il vous l'a \emph{dit}  ?" lâcha Hermione sous l'effet de la surprise. Le directeur avait dit que personne d'autre n'était censé savoir !

Le professeur McGonagall s'interrompit, regarda Hermione, et gloussa avec tristesse. "Il est bon de voir que M. Potter ne vous a pas trop corrompue. Mlle. Granger, vous n'êtes pas censée \emph{admettre}  quelque chose simplement parce que je le dis. Il se trouve que le directeur ne me l'a \emph{pas}  dit, c'est juste que je le connais trop bien."

Hermione rougissait à présent furieusement.

"Tout va bien, Mlle. Granger !" dit le professeur McGonagall avec hâte. "Vous êtes une Serdaigle en première année, personne ne vous demande d'être Serpentard."

Cela fit \emph{vraiment}  mal.

"Très bien," dit Hermione d'un ton quelque peu acerbe, "j'irai demander des leçons de Serpentard à Harry Potter, alors."

"Ce n'\emph{est pas}  ce que je voulais..." dit le professeur McGonagall, et elle laissa sa voix en suspens. "Mlle. Granger, cela m'inquiète \emph{parce que}  les jeunes filles Serdaigle ne devraient pas avoir à être Serpentard ! Si le directeur vous demande de vous impliquer dans quelque chose qui ne vous met pas à l'aise, il est tout à fait normal de refuser. Et si vous vous sentez mise sous pression, dites au directeur que vous voudriez que je sois là, ou que vous voudriez m'en parler avant."

Les yeux de Hermione étaient très écarquillés. "Le directeur fait-il des choses mauvaises ?"

Le professeur McGonagall eut l'air un peu triste en entendant cela. "Pas exprès, Mlle. Granger, mais je pense... eh bien, il \emph{est}  probablement vrai que le directeur a du mal à se souvenir de ce que c'est que d'être un enfant. Même lorsqu'il en était un, je suis sûre qu'il devait être brillant, un esprit et un cœur fort, avec assez de courage pour trois Gryffondors. Parfois le directeur en demande trop de la part de ses jeunes élèves, Mlle. Granger, ou il ne prend pas assez de précautions pour qu'ils ne souffrent pas. C'est un homme bon, mais parfois ses intrigues peuvent aller trop loin."

"Mais il est \emph{bon}  pour un élève d'avoir de la force et du courage," dit Hermione. "C'est pour cela que vous m'avez suggéré Gryffondor, n'est-ce pas ?"

Le professeur McGonagall eut un sourire sarcastique. "Peut-être que j'étais simplement égoïste, que je vous voulais pour ma Maison. Le Choixpeau vous a-t-il offert - non, je n'aurais pas dû vous demander."

"Il m'a dit que je pouvais aller partout sauf à Serpentard," dit Hermione. Elle avait \emph{failli}  demander pourquoi elle n'était pas assez bonne pour Serpentard avant de réussir à s'en empêcher... "Alors \emph{j'ai}  du courage, professeur !"

Le professeur McGonagall se pencha au-dessus de son bureau. L'inquiétude était à présent clairement visible sur son visage. "Mlle. Granger, il ne s'agit pas de courage, il s'agit de ce qui est bon pour des jeunes filles ! Le directeur vous attire dans ses intrigues, Harry Potter vous donne des secrets à garder, et maintenant vous faites des alliances avec Draco Malfoy ! Et j'ai promis à votre mère que vous seriez en sécurité à Poudlard !"

Hermione ne savait tout simplement pas quoi répondre à cela. Mais l'idée lui vint que le professeur McGonagall ne l'aurait peut-être pas mise en garde si elle avait été un garçon à Gryffondor et pas une fille à Serdaigle, et \emph{ça} , c'était, eh bien... "J'essaierai d'être quelqu'un de bien," dit-elle, "et je ne laisserai personne me dire de ne pas l'être."

Le professeur McGonagall se comprima les mains sur les yeux. Lorsqu'elle les enleva, son visage ridé semblait très vieux. "Oui," dit-elle dans un souffle, "vous auriez bien réussi dans ma Maison. Soyez prudente, Mlle. Granger. Et si vous êtes jamais inquiète ou mal à l'aise au sujet de quoi que ce soit, venez me voir immédiatement. Je ne vous retiendrai pas plus longtemps."
\par\noindent\rule{\textwidth}{0.4pt}
\emph{Après-coup, Draco Malfoy}  \emph{:} 

Presque aucun d'eux ne voulait faire quoi que ce soit de compliqué ce samedi, pas après avoir combattu plus tôt. Alors Draco était juste assis dans une salle inoccupée et essayait de lire un livre intitulé \emph{Penser la physique} . C'était une des choses les plus fascinantes que Draco avait lues de sa vie, du moins les parties qu'il pouvait comprendre, du moins quand le \emph{maudit imbécile}  qui refusait de laisser ses livres sortir de son champ de vision parvenait à se la \emph{fermer}  et à laisser Draco se \emph{concentrer}  -

"Hermione Granger est une \emph{Sang-de-Bouuurbe} ," chanta Harry Potter depuis un bureau non loin de celui de Draco, lisant lui-même un livre beaucoup plus pointu.

"Je sais ce que tu essaies de faire," dit calmement Draco sans lever les yeux de son livre. "Ça ne va pas marcher. On va quand même s'allier et t'écraser."

"Un \emph{Maaaalfoy}  travaille avec une \emph{Sang-de-Bouuurbe} , qu'est-ce-que tous les \emph{amiiiis } de ton père vont penser -"

"Ils penseront que les Malfoys ne sont pas aussi simple à manipuler que \emph{tu}  sembles le croire, \emph{Potter}  !"

Le professeur de Défense était encore plus fou que Dumbledore, aucun futur sauveur du monde n'aurait pu être si \emph{puérile}  et \emph{indigne}  à quelque âge que ce soit.

"Eh Draco, tu sais ce qui va être vraiment chiant ? \emph{Tu } sais que Hermione Granger a deux copies de l'allèle magique, comme toi et moi, mais tous tes camarades de Serpentard ne le savent pas et \emph{tuuuuuu } n'as pas le droit de \emph{l'expliqueeeer}  -"

Les doigts de Draco blanchissaient là où ils serraient le livre. Il était impossible que se faire battre puis cracher dessus exige un tel contrôle de soi, et s'il ne se vengeait pas rapidement, il allait faire quelque chose d'incriminant -

"Et qu'est-ce que \emph{tu}  as souhaité la première fois ?" dit Draco.

Harry ne répondit rien, alors Draco leva les yeux de son livre et sentit un tiraillement de satisfaction malsaine à la vue de l'air triste sur le visage de Harry.

"Hm," dit Harry. "Beaucoup de gens m'ont demandé ça, mais je ne pense pas que le professeur Quirrell aurait voulu que j'en parle."

Draco se donna un air sérieux. "Tu peux m'en parler à \emph{moi} . Ce n'est probablement pas important comparé aux autres secrets que tu m'as révélés, et à quoi d'autres les amis servent-ils ?" \emph{C'est ça, je suis ton ami ! Sens-toi coupable !} 

"Ce n'était vraiment pas si intéressant que ça," dit Harry avec une légèreté évidemment artificielle. "Juste, \emph{je souhaite que le professeur Quirrell enseigne la magie de bataille l'année prochaine} ."

Harry soupira et se replongea dans son livre.

Puis il dit, quelques secondes plus tard : "Ton père va probablement être assez énervé contre toi à Noël, mais si tu promets que tu trahiras la fille Sang-de-Bourbe et que tu balaieras son armée, tout reviendra dans l'ordre et tu auras quand même tes cadeaux de Noël."

Peut-être que si Draco et Granger le demandaient super poliment au professeur Quirrell et qu'ils utilisaient une partie de leurs points Quirrell, ils recevraient le droit de faire quelque chose d'un peu plus intéressant au général Chaos que de simplement l'endormir.

