
\chapter{Le péché capital}

Brillant le soleil, brillants les airs, brillants les élèves et brillants leurs parents, propre le sol pavé de la plate-forme 9,75, le soleil d'hiver suspendu bas dans le ciel à 9h45 le matin du cinq janvier 1992. Certains des élèves les plus jeunes portaient des écharpes et des mitaines, mais la plupart étaient simplement vêtus de leurs robes ; après tout, ils étaient des sorciers.

Une fois loin de la plate-forme d'arrivée, Harry retira son écharpe et son manteau, ouvrit un compartiment de sa malle et y fourra ses affaires d'hiver.

Il se tint là pendant un long moment, laissant l'air de janvier le mordre, juste pour voir comment c'était.

Il finit par sortit sa baguette ; et il ne put s'empêcher de penser aux parents qu'il venait d'embrasser et auxquels il venait de dire au revoir, au monde dont il laissait les problèmes derrière lui...

Se sentant étrangement coupable de cette inéluctabilité, Harry dit : "\emph{Thermos} "

La chaleur se répandit en lui.

Et le Survivant fut de retour.

Harry bailla et s'étira, se sentant plus léthargique qu'autre chose au terme de ces vacances. Ce matin, il n'était pas d'humeur à lire ses manuels, ni même de la science-fiction sérieuse ; ce dont il avait besoin, c'était de quelque chose d'entièrement frivole et qui puisse occuper son attention.

Eh bien, s'il était prêt à se défaire de quatre Noises, ça ne serait pas difficile à trouver.

De plus, si la \emph{Gazette du sorcier}  était corrompue et que le \emph{Chicaneur}  était le seul journal concurrent, il pourrait y avoir de véritables informations étouffées là-dedans.

Harry marcha d'un pas lourd jusqu'au même kiosque que la dernière fois, se demandant si le \emph{Chicaneur } surpasserait le gros titre qu'il avait vu la fois précédente.

Le vendeur commença à sourire à l'approche de Harry, puis son visage changea brutalement lorsqu'il aperçut la cicatrice.

"\emph{Harry Potter}  ?" s'étrangla le vendeur.

"Non, M. Durian," dit Harry, ses yeux ayant brièvement déclinés jusqu'au badge de l'homme, "juste une incroyable imitation -"

Puis la voix de Harry se bloqua dans sa gorge alors qu'il apercevait la partie supérieure du \emph{Chicaneur} .


\begin{center}\emph{VOYANTE BOURRÉE CRACHE LE MORCEAU :} \\\emph{} \emph{LE SEIGNEUR DES TÉNÈBRES VA REVENIR,} \end{center}


Pendant juste un instant, Harry tenta de réprimer les mouvements des muscles de son visage avant de se rendre que de ne \emph{pas}  être choqué pourrait être, en un sens, tout autant révélateur -

"Excusez-moi," dit Harry. Il avait un ton quelque peu alarmé, et il ne savait même pas si cela en révélait trop, ni même ce que sa réaction normale \emph{aurait été}  s'il n'avait rien su. Il avait passé trop de temps entouré d'élèves Serpentard, il oubliait comment cacher des secrets aux gens normaux. Quatre Noises heurtèrent le comptoir. "Un exemplaire du \emph{Chicaneur} , s'il vous plaît."

"Oh, pas de problème, M. Potter !" dit le vendeur avec hâte en battant des mains. "C'est - non, rien, juste que -"

Un journal traversa les airs, tomba dans les doigts de Harry, et il le déplia.


\begin{center}\emph{VOYANTE BOURRÉE CRACHE LE MORCEAU :} \\\emph{} \emph{LE SEIGNEUR DES TÉNÈBRES VA REVENIR,} \\\emph{} \emph{MARIÉ À DRACO MALFOY} \end{center}


"C'est gratuit," dit le vendeur, "pour \emph{vous} , je veux dire -"

"Non," dit Harry, "j'allais en acheter un de toute façon."

Le vendeur prit les pièces et Harry continua de lire.

"Mince alors," dit Harry une demi-minute plus tard, "Vous rendez une voyante complètement ivre avec six lampées de scotch et voilà qu'elle vend la mèche sur toutes \emph{sortes}  de choses secrètes. Je veux dire, qui aurait pensé que Sirius Black et Peter Pettigrew étaient secrètement la même personne ?"

"Pas moi," dit le vendeur.

"Ils ont même une image où ils sont ensembles, pour qu'on sache qui c'est qui est secrètement la même personne."

"Ouaip," dit le vendeur. "Plutôt malin comme déguisement, eh ?"

"Et j'ai secrètement soixante-cinq ans."

"Vous n'en faites pas la moitié," répondit aimablement le vendeur.

"Et je suis fiancé à Hermione Granger, \emph{et}  à Bellatrix Black, \emph{et}  à Luna Lovegood, et oh oui, à Draco Malfoy aussi..."

"En v'la un mariage qui va être intéressant," dit le vendeur.

Harry releva les yeux du journal et dit d'un ton plaisant, "Vous savez, j'ai commencé par entendre que Luna Lovegood était folle, et je me suis demandé si elle l'était vraiment ou si elle inventait tout ça et qu'elle en riait en son fort intérieur. Et puis quand j'ai lu mon deuxième gros titre du \emph{Chicaneur} , j'ai décidé qu'elle ne \emph{pouvait pas}  être folle, je veux dire, ça ne peut pas être \emph{facile}  d'inventer ces trucs, on ne pourrait pas le faire par \emph{accident} . Et \emph{maintenant} , vous savez ce que je pense ? Je pense qu'elle doit être folle après tout. Quand les gens normaux essaient d'inventer des choses, ils n'inventent pas \emph{ça} . Quelque chose doit vraiment \emph{clocher}  à l'intérieur de votre tête avant que ce soit \emph{ça}  qui en sorte lorsque vous vous mettez à inventer des choses !"

Le vendeur fixa Harry.

"Sérieusement," dit Harry. "Qui \emph{lit}  ça ?"

"Vous," dit le vendeur.

Harry s'éloigna pour lire son journal.

Il ne s'assit pas à la table, située non loin, où il s'était assis avec Draco la \emph{première}  fois qu'il s'était préparé à embarquer dans le train. Il lui semblait que ça aurait été comme de donner à l'Histoire la tentation de se répéter.

Ce n'était pas \emph{seulement}  que, à en croire le \emph{Chicaneur} , sa première semaine à Poudlard avait duré cinquante-quatre ans. C'était surtout que, à l'humble avis de Harry, sa vie n'avait pas \emph{besoin}  d'une once de complexité supplémentaire.

Alors Harry alla ailleurs et trouva une petite chaise en fer, loin du gros de la foule et des craquements étouffés occasionnels qui survenaient lorsque les parents transplanaient avec leurs enfants, et il s'assit et lut le \emph{Chicaneur}  pour voir s'il contenait de véritables informations qui auraient été étouffées.

Et mis à part la folie évidente (que le ciel les aide si une seule de \emph{ces}  nouvelles était vraie), il y avait pas mal ragots romantiques sournois ; mais rien qui aurait été particulièrement \emph{important}  si ça avait été vrai.

Harry était en train de s'informer sur la loi sur le mariage proposée par le ministère, destinée à bannir tous les mariages, quand -

"Harry Potter," dit un voix soyeuse qui propulsa un jet d'adrénaline dans le sang de Harry.

Il releva les yeux.

"Lucius Malfoy," dit Harry, sa voix usée. La prochaine fois il serait malin et il attendrait dans la partie Moldue de King's Cross jusqu'à 10h55.

Lucius inclina sa tête courtoisement, envoyant ses longs cheveux blancs glisser le long de ses épaules. L'homme portait toujours la même cane, laquée de noir, avec pour poignée la tête d'un serpent d'argent ; et quelque chose dans la façon dont il la tenait disait : \emph{ceci est une arme puissante et mortelle} , et non pas : \emph{je suis faible et je m'appuie dessus} . Son visage était vide d'expression.

Deux hommes l'encadraient, leurs yeux scannant continuellement, leur baguette déjà serrée dans leur main abaissée. Ils bougeaient comme un seul organisme doté de quatre jambes et quatre bras, les Crabbe-et-Goyle senior, et Harry pensa qu'il pouvait deviner lequel était lequel, mais ça n'avait pour le moment pas vraiment d'importance. Ils n'étaient que les appendices de Lucius, aussi certainement que s'ils avaient été deux orteils de son pied gauche.

"Je vous demande pardon de vous avoir dérangé, M. Potter," dit la voix douce et soyeuse. "Mais vous n'avez répondu à aucune de mes chouettes ; et ceci, ai-je pensé, pourrait bien être ma seule opportunité de vous rencontrer."

"Je n'ai reçu aucune de vos chouettes," dit Harry calmement. "Je suppose que Dumbledore les a interceptées. Mais je n'y aurais pas répondu si je les avais reçues, sauf par l'entremise de Draco. Car avoir directement affaire avec vous, sans que Draco le sache, constituerait un abus de notre amitié."

\emph{Va-t-en s'il te plaît, va-t-en s'il te plaît...} 

Les yeux scintillèrent. "Si cela est votre position, alors..." dit Malfoy senior. "Bien. Je jouerai le jeu un moment. Quel était votre but en manipulant votre bon ami, mon fils, vers une alliance publique avec cette fille ?"

"Oh," dit Harry d'un ton léger, "c'est évident, non ? Travailler avec Granger lui fera se rendre compte que les Moldus sont humains, en fin de compte. Bah. Ha. Ha."

La trace d'un fin sourire déplaça les lèvres de Lucius. "Oui, cela ressemble à l'un des plans de Dumbledore. Et ce n'en est \emph{pas}  un."

"En effet,'" dit Harry. "Cela fait partie de mon jeu avec Draco, et ne résulte d'aucun acte de Dumbledore, et c'est tout ce que j'en dirai."

"Passons-nous de jeux," dit Malfoy senior, les yeux gris se durcissant soudain. "Si mes soupçons sont fondés, vous n'êtes de toute façon pas du genre à obéir à Dumbledore, \emph{M. Potter} ."

Il y eut une courte pause.

"Alors vous savez," dit Harry d'une voix froide. "Dites moi. À quel moment exactement vous en êtes-vous rendu compte ?"

"Lorsque j'ai lu votre réponse au petit discours du professeur Quirrell," dit l'homme aux cheveux blancs, et il gloussa d'un ton grave. "J'étais d'abord perplexe, car cela ne semblait pas servir vos intérêts ; il m'a fallu des jours pour comprendre quels intérêts étaient servis, et tout est enfin devenu clair. Et il est de même évident que vous êtes faible à certains égards, sinon à d'autres."

"Très intelligent de votre part," dit Harry, toujours froid. "Mais peut-être vous méprenez-vous sur mes intérêts."

"Peut-être est-ce le cas." Un soupçon d'acier se glissa dans la voix soyeuse. "En effet, c'est précisément ce dont j'ai peur. Vous jouez à d'étranges jeux avec mon fils, pour un but que je ne puis deviner. Ce n'est pas un acte amical, et vous ne pouvez vous attendre à ce que je ne sois pas préoccupé !"

Lucius s'appuyait maintenant sur sa cane de ses deux mains, et les deux mains étaient blanches, et ses gardes du corps s'étaient soudainement tendus.

Un instinct en Harry déclara que ce serait une très mauvaise idée que de montrer sa peur et de laisser Lucius voir qu'il pouvait être intimidé. De toute façon, ils étaient dans une gare, en public -

"Je trouve intéressant," dit Harry, mettant de l'acier dans sa voix, "que vous pensez que je pourrais tirer bénéfice d'un tort que je causerais à Draco. Mais c'est sans rapport avec le sujet, Lucius. \emph{Il}  est mon ami, et je ne trahis pas mes amis."

"\emph{Quoi ?} " murmura Lucius. Son visage montrait un choc profond.

Alors -

"Compagnie," dit l'un des laquais, et en entendant la voix, Harry pensa que ce devait être Crabbe senior.

Lucius se raidit et se tourna, puis il laissa échapper un sifflement désapprobateur.

Neville approchait, l'air effrayé mais déterminé, dans le sillage d'une grande femme qui n'avait pas l'air effrayée du tout.

"Mme. Londubat," dit Lucius d'un ton de glace.

"M. Malfoy," répondit la femme du même ton glacial. "Êtes-vous un désagrément pour notre Harry Potter ?"

L'aboiement de rire qui jaillit de Lucius sembla étrangement amer. "Oh, je ne pense pas. Vous êtes venu le protéger de moi, c'est cela ?" L'homme aux cheveux blanc se déplaça vers Neville. "Et ce serait là le loyal lieutenant de M. Potter, le dernier descendant des Londubat, Neville, auto-intitulé du Chaos. Comme le monde devient étrange. Parfois je pense qu'il doit être totalement fou."

Harry ne savait pas du tout quoi répondre à cela, et Neville semblait confus et apeuré.

"Je doute que ce soit le monde qui soit fou," dit Mme. Londubat. Sa voix prit un ton pavoisant. "Vous semblez d'une triste humeur, M. Malfoy. Le discours de notre cher professeur Quirrell vous a-t-il coûté quelques alliés ?"

"C'était une calomnie assez intelligente de mes capacités," dit froidement Lucius, "mais seulement efficace pour les idiots qui croient que j'étais réellement un Mangemort."

"\emph{Quoi ?} " lâcha Neville.

"J'étais victime de l'\emph{Imperius} , jeune homme," dit Lucius, l'air maintenant fatigué. "Le Seigneur des Ténèbres n'aurait certainement pas pu commencer à recruter parmi les familles de Sang-Pur sans le soutien de la Maison Malfoy. J'ai soulevé des objections, et il s'est simplement assuré de mon soutien. Ses propres Mangemorts ne le surent qu'ensuite, d'où la fausse Marque que je porte ; même si, puisque je n'y avais pas vraiment consenti, elle ne me lie pas. Certains des Mangemorts croient encore que j'étais le plus grand d'entre eux, et pour la paix de notre nation je les laisse le croire, pour les garder sous contrôle. Mais je n'étais pas assez idiot pour soutenir de mon plein gré cet aventurier au destin funeste -"

"Ignore-le," dit Mme. Londubat, l'ordre était adressé à Harry ainsi qu'à Neville. "Il doit passer le reste de sa vie à prétendre, par peur de ton témoignage sous Veritaserum," dit-elle avec une méchante satisfaction.

Lucius lui tourna dédaigneusement le dos et fit de nouveau face à Harry. "Demanderez-vous à cette harpie de partir, \emph{M. Potter}  ?"

"Je ne pense pas," dit Harry d'une voix sèche. "Je préfère avoir affaire à la partie de la Maison Malfoy qui a mon âge."

Il y eut alors une longue pause. Les yeux gris le scrutèrent.

"Bien sûr..." dit lentement Lucius. "Je me \emph{sens}  à présent être un idiot. Pendant tout ce temps vous faisiez juste semblant de n'avoir aucune idée de ce dont je parlais."

Harry croisa le regard et ne dit rien.

Lucius leva sa cane de quelques centimètres et la frappa durement contre le sol.

Le monde disparut dans une brume pâle, tous les sons s'effacèrent, il n'y avait rien dans l'univers hormis Harry, Lucius Malfoy, et la cane à tête de serpent.

"Mon fils est mon cœur," dit Malfoy senior, "la dernière chose de valeur qui me reste en ce monde, et je vous dit cela dans un esprit d'amitié : s'il devait lui arriver malheur, je consacrerais ma vie à sa vengeance. Mais tant que qu'il ne lui arrive \emph{pas}  malheur, je vous souhaite de réussir dans vos entreprises. Et comme vous ne m'en avez pas demandé plus, je ne vous en demanderai pas plus moi-même."

Puis la brume pâle disparut, révélant une Mme. Londubat outragée, bloquée dans son avancée par Crabbe senior ; sa baguette était à présent dans sa main.

"Comment \emph{osez} -vous !" siffla-t-elle.

Les sombres robes de Lucius tourbillonnèrent autour de lui de même que ses cheveux blanc, et il se tourna vers Goyle senior. "Nous retournons au manoir Malfoy."

Il y eut trois pouf de Transplanage, et ils étaient partis.

Un silence s'ensuivit.

"Par les \emph{cieux} ," dit Mme. Londubat. "De quoi s'agissait-t-il ?"

Harry haussa les épaules d'un air impuissant. Puis il regarda Neville.

Il y avait de la sueur sur son front.

"Merci beaucoup, Neville," dit Harry. "Ton aide a été grandement appréciée, Neville. Et maintenant, Neville, je pense que tu devrais t'asseoir."

"Oui, général," dit Neville, et au lieu d'aller jusqu'à l'une des chaises proches de Harry, il s'écroula à moitié sur le pavage en position assise.

"Vous avez provoqué de grands changements chez mon petit-fils," dit Mme. Londubat. "J'approuve certains, mais pas tous."

"Envoyez moi une liste de chaque," dit Harry. "Je verrai ce que je peux faire."

Neville grogna mais ne dit rien.

Mme. Londubat gloussa. "Je le ferai, jeune homme, merci." Sa voix baissa d'un ton. "M. Potter... le discours donné par le professeur Quirrell est quelque chose que notre nation avait besoin d'entendre depuis longtemps. Je ne peux pas en dire autant de votre commentaire à son sujet."

"Je prendrai votre opinion en considération," dit Harry d'un ton neutre.

"J'espère fort que vous le ferez," dit Mme. Londubat, et elle se retourna vers son petit-fils. "Dois-je encore -"

"Tu peux y aller, grand-mère," dit Neville. "Je me débrouillerai tout seul cette fois."

"Et \emph{cela} , je l'approuve," dit-elle, et il y eut un pouf et elle disparut comme une bulle de savon.

Les deux garçons restèrent assis en silence pendant un moment.

Neville parla le premier, sa voix usée. "Tu vas essayer de réparer tous les changements qu'elle \emph{approuve} , c'est ça ?"

"Pas \emph{tous} ," dit Harry d'un ton innocent. "Je veux juste m'assurer que je ne suis pas en train de te corrompre."
\par\noindent\rule{\textwidth}{0.4pt}
Draco avait l'air \emph{très}  inquiet. Il n'arrêtait pas de jeter des coups d'œil autour de lui, en dépit du fait qu'il avait insisté pour qu'ils se rendent dans la malle de Harry et qu'ils utilisent un Vrai sort de Silence et pas seulement la barrière d'étouffement sonore.

"\emph{Qu'est-ce}  que tu as dit à Père ?" lâcha Draco au moment où le sort de Silence fut jeté et que les sons de la plate-forme 9 3/4 eurent disparu.

"Je... écoute, peux-tu me dire ce qu'il \emph{t'a}  dit avant de te déposer ?" dit Harry.

"Que je devrais le lui dire tout de suite si tu semblais me menacer," dit Draco. "Que je devrais le lui dire tout de suite s'il y avait quoi que ce soit que \emph{je}  faisais qui pourrait \emph{te}  menacer ! Père pense que tu es \emph{dangereux} , Harry, quoi que tu lui aies dit, ça lui a fait \emph{peur}  ! \emph{Ce n'est pas une bonne idée de faire peur à Père !} "

\emph{Oh, bon sang...} 

"De \emph{quoi}  avez-vous parlé ?" exigea Draco.

Harry s'enfonça péniblement dans la petite chaise pliante posée au fond de la caverne de sa malle. "Tu sais Draco, tout comme la question fondamentale de la rationalité est 'Pourquoi est-ce que je pense ce que je pense et comment est-ce que je sais que je le sais ?', il y a aussi un péché mortel, un façon de penser qui est l'opposé de celle-ci. Comme les anciens philosophes Grecs. Ils n'avaient aucune idée de ce qui se passait, alors ils se promenaient en disant des choses comme 'Tout est de l'eau.' ou 'Tout est du feu', et ils ne se demandèrent jamais : 'Attends une minute, si tout \emph{est}  de l'eau, comment est-ce que je peux le \emph{savoir}  ?' Ils ne se demandèrent pas s'ils avaient les preuves leur permettant de discriminer \emph{cette}  possibilité parmi toutes les \emph{autres}  possibilités imaginables, des éléments de preuve qu'ils auraient eu bien peu de chances de rencontrer si leur théorie n'avait \emph{pas}  été vraie -"

"\emph{Harry,} " dit Draco, la voix tendue, "\emph{De quoi avez-vous parlé avec Père ?} "

"À vrai dire, je ne sais pas," dit Harry, "et il est donc très important que me contente \emph{pas } d'inventer quelque chose -"

Jamais auparavant Harry n'avait jamais entendu Draco glapir d'horreur à une octave si élevée.

