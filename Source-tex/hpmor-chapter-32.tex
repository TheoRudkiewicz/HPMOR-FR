
\chapter{Interlude   Gestion des finances pers}

[NdT : Titre complet :\emph{ Interlude : Gestion des finances personnelle} s]
\par\noindent\rule{\textwidth}{0.4pt}
"Mais, professeur," argua Harry, une partie de son désespoir à présent audible, "laisser tous mes biens dans un coffre fort non diversifié rempli de pièces d'or... c'est de la folie, professeur ! C'est comme, je ne sais pas, comme de faire des expériences de métamorphose sans avoir consulté un professeur ! On ne fait tout simplement pas ça avec l'argent !"

Depuis le visage parcheminé du vieux sorcier - en-dessous du chapeau de vacances festif qui ressemblait à une collision entre une automobile en tissu vert et une autre en tissu rouge - un regard triste et grave perça vers Harry.

"Je suis désolé, Harry," dit Dumbledore, "et je te demande pardon, mais te laisser contrôler tes finances te donnerait bien trop d'autonomie."

La bouche de Harry s'ouvrit et aucun son ne s'échappa. Il était littéralement sans voix.

"Je t'autoriserai à retirer cinq Gallions pour les cadeaux de Noël," dit Dumbledore, "ce qui est plus que ce qu'un garçon de ton âge devrait dépenser mais ne présente, je pense, aucune menace -"

"\emph{Je n'arrive pas à croire que vous venez de dire ça !} " les mots avaient jaillit hors de la bouche de Harry. "Vous \emph{admettez}  être à ce point manipulateur ?"

"Manipulateur ?" dit le vieux sorcier, un léger sourire aux lèvres. "Non, ce serait manipulateur si je ne l'admettais \emph{pas}  ou si j'avais un autre motif caché derrière l'évidence. C'est assez simple, Harry. Tu n'es pas encore prêt à jouer au jeu et te laisser des milliers de Gallions avec lesquels chambouler le plateau serait insensé."
\par\noindent\rule{\textwidth}{0.4pt}
Les joyeux tiraillements du Chemin de Traverse avaient augmentés au centuple puis encore redoublés à l'approche de Noël, avec tous les magasins ensevelis sous les brillants objets magiques qui étincelaient tellement qu'on aurait dit que l'esprit de Noël allait perdre les pédales et transformer tout le quartier en un joyeux cratère de vacances. Les rues étaient si bondées de sorciers et de sorcières vêtus d'habits festifs et \emph{criards}  qu'on en avait les yeux presque aussi sévèrement agressés que les oreilles ; et, à en voir la déconcertante variété d'acheteurs, il était clair que le Chemin de Traverse était une attraction internationale. Il y avait des sorcières emmitouflées dans d'énormes pans de tissu, telles des momies en serviette, et des sorciers portant des chapeaux hauts-de-forme et des peignoirs très formels, et de jeunes enfants sachant à peine marcher qui étaient décorés de lumières presques aussi brillantes que celles des magasins eux-mêmes, et leurs parents les tenaient en main, les menant à travers ce pays des merveilles magiques et les laissant piailler tout leur saoûl. C'était le moment d'être joyeux.

Et au milieu de toute cette légèreté et de ces réjouissances, une note venue de la plus sombre des nuits ; une atmosphère froide et noire qui maintenait quelques coudées d'espace vide même au milieu de ce chahut.

"Non", dit le professeur Quirrell, arborant un air de révulsion sinistre, comme s'il venait de mordre dans un morceau de nourriture qui n'aurait pas seulement eu un goût horrible mais aurait en plus été moralement répugnante. C'était l'expression qu'une personne ordinaire aurait eu après avoir mordu dans une tourte à la viande puis découvert que non seulement elle était pourrie mais qu'en plus elle avait été faite à partir de bébés chats.

"Oh, \emph{allons} ," dit Harry. "Vous devez bien avoir \emph{quelques}  idées."

"M. Potter," dit le professeur Quirrell, les lèvres jointes en une fine ligne, "j'ai accepté de jouer le rôle de gardien adulte pour cette expédition. Je n'ai pas accepté de vous conseiller en matière de cadeaux. Je ne fête pas Noël, M. Potter."

"Et Newtonoël ?" dit Harry d'un ton enjoué. "Isaac Newton est \emph{vraiment}  né un 25 décembre, contrairement à certains autres personnages historiques que je pourrais nommer."

Cela n'impressionna en rien le professeur Quirrell.

"Écoutez," dit Harry, "je suis désolé, mais je dois faire \emph{quelque chose}  de spécial pour Fred et George et je ne sais absolument pas quelles sont mes options."

Le professeur Quirrell émit un bourdonnement pensif. "Vous pourriez leur demander quels sont les membres de leur famille qu'ils aiment le moins, puis engager un assassin. Je connais un membre d'un certain gouvernement-en-exil qui est assez compétent, et il pourrait même vous donner une réduction pour plusieurs Weasley."

"\emph{Cette année} ," dit Harry, faisant descendre sa voix d'une octave, "offrez à vos amis le cadeau... de la \emph{mort} ."

Cela fit naître un sourire sur le visage du professeur Quirrell. Il remonta jusqu'à ses yeux.

"Eh bien," dit Harry, "au moins vous n'avez pas suggéré que je leur offre un rat de compagnie -" la bouche de Harry se referma aussi sec et il regretta les mots à peine sortis de sa bouche.

"Pardon ?" dit le professeur Quirrell.

"Rien," répondit immédiatement Harry, "une longue histoire stupide." Et la raconter lui semblait étrangement répréhensible, peut-être parce que Harry était inquiet à l'idée que le professeur Quirrell aurait rit même s'il s'était révélé que Bill Weasley \emph{n'avait}  \emph{pas}  été soigné et que tout n'était pas revenu à la normale...

Et où avait \emph{été } le professeur Quirrell pour n'avoir jamais entendu cette histoire ? Harry avait eu l'impression que toute l'Angleterre magique était au courant.

"Écoutez," dit Harry, "j'essaie de \emph{solidifier leur loyauté envers moi} , d'accord ? Faire des jumeaux Weasley mes laquais ? Comme dit le vieux proverbe : un ami, ce n'est pas quelqu'un qu'on utilise une fois et qu'on jette ensuite, c'est quelqu'un qu'on utilise encore et encore. Professeur Quirrell, Fred et George sont deux des amis les plus utiles que j'ai à Poudlard, et je compte les utiliser encore et encore. Alors si vous pouviez m'aider à être un peu Serpentard et que vous me suggériez une chose pour laquelle ils me seraient \emph{très}  reconnaissants..." Harry laissa sa voix en suspens, invitant un réponse.

Il fallait juste présenter ce genre chose sous le bon angle.

Ils marchèrent un bon moment avant que le professeur Quirrell ne parle de nouveaux, sa voix suintant presque de dégoût. "Les jumeaux Weasley utilisent des baguettes de seconde main, M. Potter. Ils se souviendraient de votre générosité à chaque fois qu'ils jetteraient un sort."

Harry applaudit involontairement sous le coup de l'excitation. Il suffirait de mettre l'argent sur un compte chez Ollivander et de dire à M. Ollivander de ne jamais le rendre - ou mieux encore, qu'il l'envoie à Lucius Malfoy si les jumeaux Weasley ne venaient pas acheter une baguette avant le début de l'année prochaine. "C'est \emph{génial} , professeur !"

Le professeur Quirrell ne sembla pas apprécier le compliment. "Je suppose que je peux tolérer Noël quand c'est dans \emph{cet}  esprit, M. Potter, mais à peine." Puis il sourit légèrement. "Bien sûr, cela vous coûterait quatorze Gallions, et vous n'en avez que cinq."

"\emph{Cinq}  Gallions," dit Harry avec un reniflement outragé. "Mais à qui le directeur pense-t-il qu'il a affaire ?"

"Je pense," dit le professeur Quirrell, "qu'il ne lui est simplement pas venu à l'idée de craindre les conséquences qui se produiraient si vous concentriez votre ingéniosité sur l'obtention de fonds. Mais vous avez bien fait de perdre plutôt que de formuler une menace explicite. Par simple curiosité, M. Potter, qu'\emph{auriez}  vous fait si je ne m'étais pas détourné sous le coup de l'ennui tandis que vous, dans un accès de puérilité, récupériez vos cinq Gallions une Noise à la fois ?"

"Eh bien, la méthode la plus simple aurait été d'emprunter de l'argent à Draco Malfoy," dit Harry.

Le professeur Quirrell gloussa brièvement. "Sérieusement, M. Potter."

\emph{Bien noté} . "J'aurais probablement fait acte de présence en certains lieux en tant que célébrité. Je n'aurais recours à rien d'économiquement perturbateur dans le seul but de dépenser de l'argent." Harry avait vérifié, il \emph{aurait}  le droit de conserver son Retourneur de Temps lorsqu'il rentrerait chez lui pour les vacances afin que son cycle de sommeil ne commence pas à se décaler. Mais il était \emph{aussi}  possible que quelqu'un garde un oeil sur les boursicoteurs magiques. L'astuce avec l'or et l'argent demanderait du travail côté Moldu, et des financements, et les gobelins deviendraient peut-être soupçonneux après le premier cycle. Et ouvrir une vraie banque demanderait \emph{beaucoup}  de travail... Harry n'avait pas \emph{tout à fait}  trouvé une façon de faire de l'argent qui soit à la fois rapide \emph{et}  certaine \emph{et } sûre, et il avait donc été très heureux lorsque le professeur Quirrell s'était révélé si facile à duper.

"J'espère que ces cinq Gallions vous dureront assez longtemps, vu l'attention avec laquelle vous les avez comptés," dit le professeur Quirrell. "Je doute que le directeur sois si empressé de m'accorder la garde des clés de votre coffre-fort une seconde fois lorsqu'il aura découvert que vous m'avez trompé."

"Je suis sûr que vous avez fait de votre mieux," dit Harry avec une expression de gratitude profonde.

"Avez-vous besoin d'une aide quelconque pour entreposer toutes ces Noises, M. Potter ?"

"Eh bien, en quelque sorte," dit Harry. "Connaîtriez-vous quelques bonnes opportunités d'investissement, professeur Quirrell ?"

Et ils poursuivirent leur promenade dans leur petite sphère de silence et d'isolation, au travers de la foule éclatante et chaotique ; et si vous prêtiez attention, vous verriez que, là où ils allaient, les branches feuillues se flétrissaient, les fleurs se fanaient, et les jouets d'enfants qui jouaient de joyeux sons de cloche passaient à des notes plus graves et plus effrayantes.

Harry \emph{le remarqua} , mais il ne dit rien, il se sourit juste à lui-même.

Chacun avait sa façon de célébrer Noël, et le Grinch en faisait autant partie que le Père-Noël lui-même.

