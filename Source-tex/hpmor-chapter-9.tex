
\chapter{Titre effacé, Partie 1}

All your base are belong to J.K. Rowling.
\par\noindent\rule{\textwidth}{0.4pt}
1.000 CRITIQUES EN 26 JOURS WOOHOO MEGA PUISSANCE ! 30 JOURS 1.189 CRITIQUES LE COMBO CONTINUE ! YEAH ! VOUS ETES LES MEILLEURS ! CECI EST SPARTAAAAA !

Hum.

Une remarque \emph{préliminaire}  inhabituelle pour ce chapitre : Lee Jordan est le compère farceur de Fred et George (dans le canon). Je trouve que "Lee Jordan" ressemble à un nom de né-Moldu ; il serait donc capable d'enseigner certaines choses à Fred et George. Ce n'était pas autant évident pour certains lecteurs que ça ne l'était pour l'auteur.

Les quarks de troisième génération étaient aussi nommés "vérité" et "beauté" avant que "dessus" et "dessous" ne remportent la bataille ; je suis né à peu près en même temps que Hermione et lorsque j'avais onze ans j'utilisais "vérité" et "beauté".

Lorsque la Partie 1 de ce chapitre a été publiée pour la première fois, j'ai dit que si qui quelqu'un devinait le sujet de la dernière phrase avant la mise à jour suivante, celle du 3 avril, je lui dirais le reste de l'intrigue.
\par\noindent\rule{\textwidth}{0.4pt}
"\emph{On ne sait jamais quel petit événement va bouleverser le cours de votre grand projet."} 
\par\noindent\rule{\textwidth}{0.4pt}
"Abbott, Hannah !"

Pause.

"POUFSOUFFLE !"

"Bones, Susan !"

Pause.

"POUFSOUFFLE !"

"Boot, Terry !"

Pause.

"SERDAIGLE !"

Harry jeta un bref regard à sa nouvelle camarade de Maison, plus pour avoir un aperçu rapide de son visage qu'autre chose. Il essayait encore se contrôler suite à sa rencontre avec les fantômes. Ce qui était triste, vraiment triste, vraiment réellement triste, c'était qu'il \emph{réussissait}  à retrouver le contrôle de lui-même. Ça ne semblait pas convenable. Ça aurait dû lui prendre, quoi, au moins une journée. Peut-être une vie entière. Peut-être toujours, en fait.

"Corner, Michael !"

Longue pause.

"Serdaigle !"

Au pupitre devant la table d'honneur se tenait le professeur McGonagall, avec son air sec, scrutant sévèrement l'assistance tandis qu'il appelait les élèves les uns après les autres ; même si elle avait sourit pour Hermione et quelques autres. Derrière elle, sur la plus grande chaise de la table - en fait, c'était une sorte de trône doré - se trouvait un vieillard à lunettes à la barbe d'argent dont on aurait pu imaginre qu'elle allait jusqu'au sol ; il regardait le Choixpeau d'un air bienveillant et il était aussi proche du stéréotype du Vieil Homme Sage qu'on pouvait l'être sans devenir Oriental (même si harry avait appris à faire attention aux stéréotypes depuis que, lors de sa première rencontre avec le professeur McGonagall, il s'était attendu à la voir ricaner). Le vieux sorcier avait applaudi tous les élèves après leur répartition avec un sourire immuable qui semblait pourtant exprimer un ravissement toujours renouvelé.

A gauche du trône doré, un homme au regard sévère et au visage austère n'avait applaudi personne et parvenait mystérieusement à sembler regarder Harry droit dans les yeux à chaque fois que Harry l'observait. Plus à gauche encore, l'homme au visage pâle que Harry avait vu au Chaudron Baveur, dont les yeux parcouraient la foule comme s'il était en proie à une crise de panique et dont le corps tressautait occasionallement ; sans savoir pourquoi, Harry se retrouvait souvent le regard braqué sur lui. A gauche de cet homme, un série de trois sorcières plus âgées qui ne semblaient pas accorder beaucoup d'intérêt aux élèves. Puis, à droite de la grande chaise dorée, un sorcière d'âge mur au visage rond et coiffée d'un chapeau jaune, qui avait applaudi tous les élèves non-Serpentard. Un petit homme debout sur sa chaise, avec une barbe blanche bouffante, qui avait applaudi tous les élèves, mais n'avait souri que pour les Serdaigle. Et, à l'extrême droite, et occupant le même espace que les 3 êtres inférieurs, l'entité montagneuse qui les avait accueillis à leur sortie du train et s'était présenté sous le nom de Hagrid, gardien des Clés et des Lieux.

"Est-ce que l'homme debout sur sa chaise est à la tête de Serdaigle ?" chuchota Harry à Hermione.

Pour une fois, Hermione ne répondit pas immédiatement ; elle n'arrêtait pas de se dandiner en fixant le Choixpeau avec tant d'énergie que Harry cru que ses pieds allaient se décoller du sol.

"Oui," répondit l'un des préfets qui les accompagnait, une jeune femme aux couleurs de Serdaigle. Mademoiselle Deauclaire, si Harry se souvenait bien. Sa voix était douce, mais avait une note de fierté. "Cest le professeur d'enchantements de Poudlard, Filius Flitwick, l'enchanteur le plus savant à être encore en vie, et un ancien champion de duel..."

"Pourquoi est-il si \emph{court sur pattes}  ?" siffla un élève dont Harry avait oublié le nom. "Est-ce que c'est un \emph{hybdride}  ?"

Un regard glacé de la préfète. "Le professeur a effectivement des ancêtres gobelins."

"Quoi ?" dit Harry par réflexe, et Hermione et d'autres élèves lui intimèrent de se taire.

Maintenant le regard étonnamment intimidant de la préfète Serdaigle était posé sur Harry.

"Je veux dire..." chuchota-t-il... "Ce n'est pas que ça me \emph{dérange...}  c'est juste... enfin... comment est-ce \emph{possible } ? On ne peut pas simplement mélanger deux espèces et obtenir une progéniture viable ! Ca devrait brouiller les instructions génétiques pour tous les organes non communs aux deux espèces, ça serait comme de vouloir construire..." ils n'avaient pas de voitures, impossible donc de faire une analogie avec des moteurs aux plans entremêls, "un mi-chariot mi-bateau, ou quelque chose comme ça..."

La préfète Serdaigle regardait toujours Harry avec un air sévère. "Et \emph{pourquoi}  est-ce qu'on ne pourrait pas avoir un mi-chario dmi-bateau ?"

"\emph{Chuuuut}  !" siffla un autre préfèt, alors que celle de Serdaigle avait parlé doucement.

"C'est que..." continua Harry encore plus doucement tout en essayant de savoir comment demander si les gobelins descendaient de shumains ou d'un ancêtre commun comme \emph{Homo erectus} , ou si les gobelins avaient été \emph{faits}  par les humains... par exemple s'ils étaient toujours génétiquement humains mais sous l'effet un enchantement transmissible dont l'effet magique n'était dilué que si un seul des parents était 'gobelin', ce qui expliquerait que les métissages soient possibles, auquel cas les gobelins ne seraient \emph{pas}  un inestimable indice quant à la possibilité que l'intelligence apparaisse chez des espèces autres que \emph{Homo sapiens} ... en y réfléchissant, les gobelins le Gringotts ne lui avaient \emph{pas}  semblé posséder des intelligences particulièrement étranges et non-humaines, rien qui ressemble aux Marionnetistes ou au Dirdir... "je me demande d'où les gobelins \emph{viennent} ."

"Lituanie", murmura distraitement Hermione, ses yeux toujours fixés sur le Choixpeau.

Maintenant, la préfète souriait à Hermione.

"Laisse tomber," murmura Harry.

Au pupitre, le professeur McGonagall s'exclama : "Goldstein, Anthony !"

"SERDAIGLE !"

A côté de Harry, Hermione trépignait si fort que ses pieds se décollaient du sol à chaque saut.

"Goyle, Gregory !"

Il y eut un long moment de silence tendu sous le Choixpeau. Presque une minute.

"SERPENTARD !"

"Granger, Hermione !"

Hermione se libéra de son immobilité et courut à pleine vitesse vers le Choixpeau Magique, le ramassa et fourra avec force le vieux morceau de tissu rapiécé sur sa tête. Harry grimaça. Hermione avec été celle qui \emph{lui}  avait expliqué ce qu'était le Choixpeau Magique, mais elle ne le \emph{traitait}  certainement pas comme un artefact irremplaçable d'une importance vitale vieux de 800 ans et fait de magie oubliée sur le point de réaliser une opération télépathique complexe sur son esprit et n'ayant pas l'air d'être en très bonne condition physique.

"SERDAIGLE !"

Parlons-en, des prédictions courues d'avance. Harry ne voyait vraiment pas pourquoi Hermione avait été aussi tendue. Dans quel univers alternatif bizarre cette fille n'aurait-elle \emph{pas}  été triée à Serdaigle ? Si Hermione Granger n'allait pas à Serdaigle, alors il n'y avait aucune bonne raison pour que la Maison Serdaigle existe.

Hermione arriva à la table de Serdaigle et reçut une acclamation respectueuse ; Harry se demanda si l'acclamation aurait été plus forte ou plus douce si ils avaient eu la moindre idée du niveau de compétition qu'ils venaient d'accueillir à leur table. Harry connaissait pi jusqu'à 3,141592 parce qu'une précision de un pour un million suffisait dans la plupart des situations pratiques. Hermione connaissait les cent premiers chiffres de pi parce que c'était le nombre de chiffres qui avait été imprimé à l'arrière de son manuel de maths.

Harry fut heureux de voir Neville Londubat aller à Poufsouffle. Si cette Maison possédait vraiment la loyauté et la camaraderie dont elle était censée faire exemple, alors une Maison pleine d'amis fiables ferait un bien fou à Neville. Les enfants intelligents à Serdaigle, les méchants à Serpentard, ceux qui rêvaient d'être des héros à Gryffondor, et tous ceux qui accomplissaient le vrai travail à Poufsouffle.

(Cela dit, Harry \emph{avait } eu raison de commencer par consulter un préfet de Serdaigle. La jeune femme n'avait même pas sorti sa tête de sa lecture, elle avait juste pointé une baguette en direction de Neville et marmonné quelque chose. Après quoi Neville avait acquis un expression hébétée et était allé s'égarer dans la cinquième voiture en partant de l'avant et dans le quatrième compartiment à gauche, qui contenait en effet sa tortue.)

"Malfoy, Draco !", alla à Serpentard, et Harry lâcha un soupir de soulagement. Ça avait \emph{semblé}  certain, mais vous ne saviez jamais quel petit événement pouvait bouleverser le cours de votre grand projet.

Le professeur McGonagall appela "Perks, Sally-Anne", et des enfants réunis s'échappa une fille pâle et maigrelette, étrangement éthérée - comme si elle risquait de disparaître dès l'instant où vous arrêteriez de la regarder, pour être immédiatement oubliée à jamais.

Puis (avec une note d'inquiétude si fermement masquée dans sa voix et écartée de son visage qu'il aurait fallu très bien la connaître pour la remarquer), Minerva McGonagall inspira profondément et appela : "Potter, Harry !"

Il y eut un silence soudain.

Toutes les conversations se turent.

Tous les yeux se braquèrent sur lui.

Pour la première fois de sa vie, Harry sentit qu'il avait peut-être la chance de découvrir ce qu'était le trac.

Il bloqua immédiatement cette émotion. S'il comptait vivre en Angleterre magique, il allait devoir s'habituder à des foules concentrées sur lui ; en fait, ce serait le cas s'il comptait faire quoi que ce soit d'intéressant dans sa vie. Accrochant un sourire faussement sûr de lui à son visage, il leva un pied...

"Harry Potter !" cria la voix de Fred ou George Weasley, puis l'autre jumeaux reprit : "Harry Potter !", et un moment plus tard, toute la table Gryffondor les imita, avant d'être rejointe par une bonne partie de Serdaigle et Poufsouffle.

"\emph{Harry Potter ! Harry Potter ! Harry Potter !} "

Et Harry Potter s'avança. Il se rendit compte trop tard qu'il allait beaucoup trop lentement, mais à ce stade il aurait été très gênant de changer d'allure.
\par\noindent\rule{\textwidth}{0.4pt}
Ne sachant que trop bien ce qu'elle allait voir, Minerva se tourna pour observer le reste de la table d'honneur.

Trelawney s'éventait frénétiquement, Flitwick observait avec curiosité, Hagrid applaudissait au rythme de la musique, Sprout avait l'air sévère, Vector et Sinistra s'amusaient, et Quirrell regardait dans la vide. Albus avait un sourire bienveillant. Et Rogue serrait sa coupe de vin vide, ses jointures blanches, avec tant de force que l'argent de la coupe se déformait lentement.

Avec un grand sourire, Harry Potter s'inclinait d'un côté puis de l'autre en avançant entre les quatre tables des Maisons à un pas fort mesuré, tel un prince reçevant son château en héritage.

\emph{"Sauve-nous d'autres Seigneurs !} " s'écria l'un des jumeaux Weasley, et l'autre enchaîna alors : "\emph{Surtout si ce sont des professeurs !} ", ce qui fit éclater de rire toutes les tables, à l'exception de Serpentard.

Les lèvres de Minerva formèrent une ligne blanche. Elle aurait un mot avec les Horreurs Weasley au sujet de cette remarque ; s'ils pensaient qu'elle était sans pouvoir parce que c'était le premier jour d'école et que Gryffondor n'avait pas de points à perdre... Si les détentions ne leur faisaient pas peur, elle trouverait autre chose.

Puis, avec un hoquet d'horreur, elle regarda en direction de Rogue ; il devait \emph{forcément}  s'être rendu compte que l'enfant Potter ignorait totalement la cible de cette remarque...

Le visage de Rogue avait dépassé le stade de la rage et avait atteint celui de l'indifférence aimable. Un faible sourire se promenait sur ses lèvres. Il regardait en direction de Harry Potter, pas de la table de Gryffondor, et il serrait dans ses mains les restes froissés de ce qui avait jadis été une coupe de vin...
\par\noindent\rule{\textwidth}{0.4pt}
Et Harry s'avança, sourire figé, se sentant à la fois très bien et particulièrement mal.

Ils l'acclamaient pour un travail qu'il avait fait quand il avait un an. Un travail qu'il n'avait pas vraiment terminé. Quelque part, sous une forme ou une autre, le Seigneur des Ténèbres était toujours en vie. L'auraient-ils acclamé avec autant de force si ils l'avaient su ?

Mais le pouvoir du Seigneur des Ténèbres \emph{avait}  été brisé une fois.

Et Harry les protègerait à nouveau. Il y avait même une prophétie à ce sujet, et c'était exactement ce qu'elle annonçait. Enfin, il le ferait quoi que puisse en dire une quelconque satanée prophétie.

Tous ces gens qui croyaient en lui et l'acclamaient - Harry ne supportait pas l'idée que rien de cela ne soit vrai. De briller puis de disparaître comme tant d'autres enfants prodiges. D'être une déception. De ne pas vivre à la hauteur de sa réputation de symbole de la Lumière, quelle que soit la \emph{façon}  dont il avait obtenu ce statut. Il allait absolument, sans faute, peu importe le temps que ça lui prendrait et même si ça le tuait, répondre à leurs attentes. Puis il allait \emph{excéder}  ces attentes, pour que les gens s'étonnent, en regardant en arrière, de lui en avoir demandé si peu.

\emph{"HARRY POTTER ! HARRY POTTER ! HARRY POTTER !"} 

Harry fit ses derniers pas vers le Choixpeau Magique alors que la musique se terminait. Il fit une courbette à l'Ordre du Chaos à la table de Gryffondor, puis se tourna et fit une autre courbette à l'autre coté du Hall, et attendit que les applaudissements et les gloussements s'estompent.

Au fond de lui-même, il se demandait si le Choixpeau Magique était réellement \emph{conscient} , au sens d'être conscient de sa propre conscience, et si c'était le cas, si ça le satisfaisait de ne parler qu'à des enfants de onze ans, une fois par an. Sa chanson le sous-entendait : \emph{Oh, je suis le Choixpeau et je vais bien, je dors toute l'année, je travaille une journée...} 

Après que la salle soit devenue un peu plus silencieuse, Harry s'assit sur le tabouret et plaça \emph{précautionneusement}  l'artefact télépathique vieux de 800 ans et fait de magie oubliée...

...pensant, aussi fort qu'il le pouvait : \emph{Ne me trie pas tout de suite ! J'ai des questions qu'il faut que je te pose ! Ai-je jamais été Oublietté ? As-tu trié le Seigneur des Ténèbres quand il était enfant et peux-tu me parler de ses faiblesses ? Peux-tu me dire pourquoi j'ai eu la baguette sœur de celle du Seigneur des Ténèbres ? Le fantôme du Seigneur des Ténèbres est-il relié à ma cicatrice et est-ce pour ça que je me mets parfois en colère ? Ce sont les questions les plus importantes, mais si tu as un moment de plus peux tu me dire quelque chose sur la façon dont je pourrais redécouvrir les magies perdues qui t'ont créés ?} 

Dans le silence de l'esprit de Harry, où il n'y avait eu auparavant aucune voix hormis la sienne, s'éleva une seconde voix inconnue, à l'air nettement soucieuse :

"\emph{Oh là. Ça n'était jamais arrivé avant..."} 

