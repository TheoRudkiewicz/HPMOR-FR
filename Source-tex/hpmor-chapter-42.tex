
\chapter{Courage}

"\emph{Romantiques ?} " dit Hermione. "Ce sont tous les deux des \emph{garçons !} "

"Waoh," dit Daphné d'un ton légèrement choqué. "Tu veux dire que les Moldus détestent \emph{vraiment}  ça ? Je pensais que c'était juste quelque chose que les Mangemorts avaient inventé."

"Non," dit une Serpentard plus âgée que Hermione ne reconnu pas, "c'est vrai, ils doivent se marier en secret et si jamais ils sont découverts ils passent au bûcher tous les deux. Et si tu es une fille et que tu trouves ça romantique, ils te brûlent aussi."

"Ça n'est pas possible !" protesta une Gryffondor, alors que Hermione essayait encore de découvrir ce qu'elle pouvait bien répondre à ça. "Il ne resterait plus \emph{aucune}  fille Moldue !"

Elle avait continué de lire en silence, et Harry Potter avait continué d'essayer de lui présenter ses excuses, et elle avait rapidement comprit que Harry, peut-être pour la première fois de sa vie, avait prit conscience du fait qu'il avait commis quelque chose d'agaçant ; et que Harry, certainement pour la première fois de sa vie, était \emph{terrifié}  à l'idée de perdre son amitié ; et elle avait commencé à se sentir (a) coupable et (b) inquiète quant à la direction que prenaient les efforts de plus en plus désespérés de Harry. Mais elle n'avait toujours pas la moindre idée du genre d'excuse qui pourrait convenir, alors elle avait dit que les filles de Serdaigle devraient voter - et cette fois elle ne prédéterminerait pas le résultat, mais elle ne l'avait pas mentionné - ce que Harry avait immédiatement accepté.

Le lendemain, quasiment toutes les filles de Serdaigle âgées de plus de treize ans avaient voté pour que Draco fasse tomber Harry.

Hermione avait été légèrement déçue que ce soit si simple, même si c'était évidemment juste.

Mais là, maintenant, se tenant juste à l'extérieur des grandes portes du château au milieu de la moitié de la population féminine de Poudlard, Hermione commençait à soupçonner qu'il y avait certaines \emph{choses}  qui se passaient ici, et qu'elle ne comprenait pas, et dont elle souhaitait désespérément qu'elles n'arrivent jamais aux oreilles de ses collègues généraux.
\par\noindent\rule{\textwidth}{0.4pt}
On ne pouvait pas distinguer les détails depuis le sommet, seulement un flot indistinct de visages féminins attentifs.

"Tu n'as pas la moindre idée de ce dont il est question, n'est-ce pas ?" dit Draco d'un ton amusé.

Harry avait lu un certain nombre de livres qu'il n'était pas censé lire, sans parler de quelques gros titres du \emph{Chicaneur} .

"Le Survivant fait tomber Draco Malfoy enceinte ?" dit Harry.

"D'accord, tu \emph{sais}  de quoi il est question," dit Draco. "Je croyais que les Moldus détestaient ça ?"

"Seulement les idiots," dit Harry. "Mais, euh, ne sommes-nous pas, euh, un peu \emph{jeunes ?} "

"Pas trop jeunes pour \emph{elles} ," dit Draco. Il renifla. "Ah, les\emph{ filles !} "

Ils marchèrent en silence jusqu'au bord du toit.

"Donc \emph{je}  fais ça par vengeance contre toi," dit Draco, "mais pourquoi est-ce que \emph{tu}  le fais ?"

L'esprit de Harry fit un calcul éclair, pesant les facteurs, décidant s'il était trop tôt...

"Honnêtement ?" dit Harry. "Parce que je comptais lui faire grimper des murs gelés mais que je ne comptais \emph{pas}  la faire tomber du toit. Et, euh, je me suis en quelque sorte \emph{vraiment}  senti très mal à ce sujet. Je veux dire, j'imagine qu'au bout d'un moment j'ai vraiment fini par voir ma relation avec elle comme une rivalité amicale. Alors ce sont de vraies excuses, pas une ruse ni quoi que ce soit."

Il y eut une pause.

Puis -

"Ouais," dit Draco. "Je comprends."

Harry ne sourit pas. Ça avait peut-être été le non-sourire le plus difficile de toute sa vie.

Draco observa le rebord du toit et fit une grimace. "Ça va être beaucoup plus difficile à faire exprès que par accident."
\par\noindent\rule{\textwidth}{0.4pt}
L'autre main de Harry tenait le toit d'une poigne réflexe terrifiée, ses doigts blancs sur la pierre de glace.

Vous pouviez savoir que vous aviez bu la potion de chute plumée avec votre esprit conscient. C'était une toute autre paire de manches que de le dire à votre esprit inconscient.

C'était tout aussi effrayant que ce que Harry avait imaginé que ça avait dû être pour Hermione, ce qui était fort juste, en somme.

"Draco," dit Harry, il avait du mal à contrôler sa voix mais les filles de Serdaigle lui avaient donné un script, "Tu dois me laisser tomber !"

"D'accord !" dit Draco, et il lâcha le bras de Harry.

L'autre main de Harry s'accrocha désespérément au rebord, puis, sans qu'aucune décision n'ait été prise, ses doigts lui firent défaut et il tomba.

Il y eut un bref moment où l'estomac de Harry essaya de bondir dans sa gorge et où son corps essaya désespérément de se réorienter sans avoir aucun moyen de le faire.

Il y eut un bref moment où Harry put sentir la potion de chute plumée s'activer, commencer à le ralentir, une sensation d'embardée, d'amortissement.

Puis quelque chose le \emph{tira}  et il accéléra de nouveau vers le bas \emph{plus vite que la gravité}  -

La bouche de Harry s'était à peine ouverte et avait commencé à crier qu'une partie de son cerveau essayait de trouver une façon créative de s'en sortir, qu'une autre essayait de calculer le temps qu'il lui restait pour être créatif, et qu'une autre petite partie vestigiale remarquait qu'il n'allait même pas finir le calcul du temps qui lui restait avant de heurter le sol -
\par\noindent\rule{\textwidth}{0.4pt}
Harry essayait désespérément de contrôler son hyperventilation, et entendre les hurlements de toutes les filles qui gisaient maintenant en tas au sol et les unes sur les autres ne l'aidait pas beaucoup.

"Par les cieux," dit l'homme peu familier, l'homme aux vêtement étranges et au visage légèrement balafré qui tenait Harry dans ses bras. "De toutes les façons dont j'avais imaginé que nous pourrions un jour nous croiser de nouveau, je ne m'attendais pas à vous voir tomber du ciel."

Harry se souvint de la dernière chose qu'il avait vue, le corps s'écroulant, et il parvint à hoqueter : "Professeur... Quirrell..."

"Il ira bien dans quelques heures," dit l'homme peu familier qui tenait Harry. "Il est juste épuisé. Je n'aurais pas cru cela possible... il doit avoir assommé \emph{deux cents élèves}  juste pour s'assurer d'atteindre celui qui vous jetait un sort..."

L'homme remit gentiment Harry sur ses pieds, le soutenant un moment.

Harry reprit précautionneusement son équilibre puis hocha la tête en direction de l'homme.

Celui-ci le lâcha, et Harry s'effondra promptement.

L'homme l'aida à se relever, s'assurant à chaque instant qu'il était entre Harry et les filles qui se relevaient elles aussi, son visage jetant des coups d'œils constants dans leur direction.

"Harry," dit doucement l'homme, d'une voix extrêmement sérieuse, "as-tu la moindre idée de qui parmi ces filles aurait pu vouloir te tuer ?"

"Pas un meurtre," dit une voix épuisée, "juste de la stupidité."

Cette fois c'est l'homme peu familier qui sembla être sur le point de s'effondrer, alors qu'un choc profond se dessinait sur son visage.

Le professeur Quirrell était déjà assis, là où il avait été effondré dans l'herbe.

"Par les cieux !" hoqueta l'homme. "Vous ne devriez pas être -"

"M. Lupin, vos inquiétudes sont injustifiées. Aucun sorcier, peu importe son pouvoir, ne peut jeter un tel sortilège par sa seule force. Il faut le faire en étant \emph{efficace} ."

Le professeur Quirrell ne se leva quand même pas.

"Merci," chuchota Harry. Puis "merci", à l'homme qui se tenait à côté de lui.

"Que s'est-il passé ?" dit l'homme.

"J'aurais dû le prévoir," dit le professeur Quirrell, sa voix rêche et chargée de désapprobation. "Un certain nombre de filles ont essayé de faire venir M. Potter dans leurs bras à elles. J'imagine que chacune pensait le faire avec douceur."

Oh.

"Considérez ceci comme une leçon sur les vertus de la préparation, M. Potter," dit le professeur Quirrell. "Si je n'avais pas \emph{insisté}  pour qu'il y ait plus d'un témoin adulte présent à ce petit événement et que nous ayons \emph{tous les deux}  nos baguettes sorties, M. Lupin n'aurait pas été disponible pour ralentir votre chute après que je me sois effondré et vous auriez été gravement blessé."

"\emph{Monsieur !} " dit l'homme - M. Lupin, apparemment. "Vous ne devriez pas dire des choses pareilles au garçon !"

"Qui est -" commença à dire Harry.

"La seule autre personne qui était disponible pour regarder à part moi," dit le professeur Quirrell. "Je vous présente Remus Lupin, qui est temporairement ici pour enseigner aux élèves le sortilège du Patronus. Bien que j'ai cru comprendre que vous vous êtes déjà rencontrés."

Harry observa l'homme, perplexe. Il aurait dû se souvenir de ce visage légèrement balafré, de cet étrange et doux sourire.

"Où nous sommes-nous rencontrés ?" dit Harry.

"À Godric's Hollow," dit l'homme. "J'ai changé un certain nombre de vos couches."
\par\noindent\rule{\textwidth}{0.4pt}
Le bureau temporaire de M. Lupin était une petite pièce de pierre dotée d'un petit bureau de bois, et Harry ne pouvait pas voir ce sur quoi M. Lupin était assis, ce qui suggérait qu'il s'agissait d'un petit tabouret tout à fait semblable à celui qui faisait face à son bureau. Harry en déduit qu'il ne serait pas à Poudlard très longtemps, et n'utilisant par conséquent pas beaucoup ce bureau il avait dit aux elfes de maison de ne pas se fatiguer. Cela en disant long sur quelqu'un de voir qu'il essayait de ne pas embêter les elfes de maison. Plus précisément, cela voulait dire qu'il avait été Trié à Poufsouffle, puisque, pour ce que Harry en savait, Hermione était la seule non-Poufsouffle qui se souciait de ne pas embêter les elfes de maison (Harry avait quant à lui trouvé ces scrupules assez bêtes. Quiconque avait créé les elfes de maison en premier lieu avait été atrocement maléfique, bien sûr ; mais cela ne voulait pas dire que Hermione agissait bien \emph{maintenant}  en privant des êtres sentients de la corvée qu'ils avaient été créés pour aimer).

"Assieds-toi s'il te plaît, Harry," dit doucement l'homme. Ses robes d'enseignant étaient de mauvaise qualité, pas tout à fait en loques mais visiblement usées par le passage du temps, d'une façon qu'un simple sortilège de réparation n'aurait pu arranger ; \emph{miteux}  venait à l'esprit. Et malgré cela, d'une façon incompréhensible, il était entouré d'une aura de dignité, qui n'aurait pu être obtenue par des robes raffinées et chères, qui ne serait pas \emph{allée}  avec des robes raffinées, qui était la propriété exclusive de ce qui était miteux. Harry avait \emph{entendu parler}  de l'humilité, mais il n'y avait encore jamais fait face - seulement la modestie satisfaite des gens qui pensaient que cela faisait partie de leur style et qui voulaient que vous le remarquiez.

Harry s'assit sur le petit tabouret de bois, face à l'étroit bureau de M. Lupin.

"Merci d'être venu," dit l'homme.

"Non, merci à \emph{vous}  de m'avoir sauvé," dit Harry. "Si vous avez jamais besoin que quelque chose d'impossible se produise, faites-le moi savoir."

L'homme sembla hésiter. "Harry, puis-je... te poser une question personnelle ?"

"Vous pouvez la poser, certainement," dit Harry, "j'ai moi aussi beaucoup de questions pour vous."

Lupin hocha la tête. "Harry, tes parents adoptifs prennent-ils bien soin de toi ?"

"Mes \emph{parents} ," dit Harry. "J'en ai quatre. Michael, James, Pétunia et Lily."

"Ah," dit M. Lupin. Puis "Ah" de nouveau. Il semblait cligner très fort des yeux. "Je... c'est bon à entendre, Harry, Dumbledore ne voulait pas nous dire où tu étais... j'avais peur qu'il ne pense qu'il te faudrait d'horribles beaux-parents, ou quelque chose comme ça..."

Étant donné sa première rencontre avec Dumbledore, Harry n'était pas sûr que les inquiétudes de M. Lupin aient été mal placées ; mais tout s'était assez bien passé, et il s'abstint donc de répondre. "Qu'en est-il de mes..." Harry chercha un mot qui ne les rabaisserait pas plus qu'il ne les mettrait sur un piédestal... "\emph{autres}  parent ? Je veux savoir... eh bien, je veux tout savoir."

"C'est une gageure," dit M. Lupin. Il s'essuya le front d'une main. "Eh bien, commençons au commencement. Lorsque tu es né, James était si heureux qu'il ne pouvait toucher sa baguette sans qu'une lueur dorée ne s'en échappe, et ce pendant une semaine. Et même après cela, à chaque fois qu'il te tenait dans ses bras, ou qu'il voyait Lily te tenir, ou qu'il pensait à toi, cela se produisait de nouveau -"
\par\noindent\rule{\textwidth}{0.4pt}
De temps à autres, Harry regardait sa montre et découvrait qu'une autre demi-heure s'était écoulée. Il n'était pas très à l'aise à l'idée de faire manquer son dîner à Remus, en particulier puisque Harry reviendrait simplement à 19h plus tard, mais cela ne suffit pas à les empêcher de continuer.

Enfin, Harry amassa assez de courage pour poser la question capitale, alors que Remus était au milieu d'un long discours sur les merveilles dont James était capable au Quidditch, et Harry n'avait pas pu trouver le cœur de l'interrompre plus directement.

"Et c'est là que," dit Remus, les yeux brillants, "James réussit un \emph{triple Plongeon renversé de Mulhanney } avec \emph{effet coupé !}  Toute la foule est partie en délire, même certains des Poufsouffles acclamaient -"

\emph{J'imagine qu'il fallait être là} , pensa Harry - non que cela aurait aidé de quelque façon que ce soit - et il dit : "M. Lupin ?"

Quelque chose dans la voix de Harry dut atteindre l'homme car il s'arrêta au milieu de sa phrase.

"Mon père était-il une brute ?" dit Harry.

Remus regarda longuement Harry. "Pendant un petit temps," dit Remus. "Il mûrit et s'en éloigna bien assez tôt. Où as-tu entendu cela ?"

Harry ne répondit pas, il essayait de trouver une phrase vraie qui dévierait les soupçons, mais il ne réfléchit pas assez vite.

"Oublie cela," dit Remus, et il soupira. "Je peux deviner de qui il s'agit." Le visage légèrement balafré était pincé par une expression de désapprobation. "Quelle chose horrible à dire -"

"Mon père avait-il la moindre circonstance atténuante ?" dit Harry. "Vie familiale difficile, ou quelque chose comme ça ? Ou était-il juste... naturellement méchant ?" \emph{Froid ?} 

Les mains de Remus passèrent dans ses cheveux, le premier geste nerveux que Harry ait pu voir chez lui. "Harry," dit Remus, "tu ne peux pas juger ton père par ce qu'il faisait lorsqu'il était un jeune garçon !"

"Je\emph{ suis}  un jeune garçon," dit Harry, "et je me juge \emph{moi} ."

En entendant cela, Remus cligna deux fois des yeux.

"Je veux savoir \emph{pourquoi} ," dit Harry. "Je veux \emph{comprendre} , parce que pour moi, il me semble qu'il n'existe pas la moindre excuse !" - voix un peu tremblante. "Dites-moi tout ce que vous savez sur ses raisons, même si ça ne semble pas gentil." \emph{Pour que je ne tombe pas dans le même piège moi-même, quel qu'il soit} .

"C'était la chose à faire quand on était à Gryffondor," dit Remus lentement, avec réticence. "Et... je ne le pensais pas à l'époque, je pensais que ça avait été l'inverse, mais... c'est peut-être \emph{Black}  qui a entraîné \emph{James} , en fait... Black voulait tellement montrer à tout le monde qu'il était contre Serpentard, tu vois, nous voulions tous croire que le sang de quelqu'un n'était pas sa destinée -"
\par\noindent\rule{\textwidth}{0.4pt}
"Non, Harry," dit Remus. "Je ne sais pas pourquoi Black a poursuivi Peter au lieu de s'enfuir. C'est comme si ce jour là, Black a provoqué des tragédies pour le plaisir de le faire." La voix de l'homme était instable. "Il n'y a pas eu d'indice, pas d'avertissement, nous pensions tous - de penser qu'il serait -" La voix de Remus s'interrompit.

Harry pleurait, il ne pouvait pas s'en empêcher, cela faisait plus mal de l'entendre dit par Remus, plus mal que tout ce qu'il avait ressentit en y pensant. Harry avait perdu deux parents dont il ne se souvenait pas, qu'il ne connaissait que par le biais d'histoires. Remus Lupin avait perdu ses quatre meilleurs amis en moins de vingt-quatre heures ; et il n'y avait pas eu la moindre justification pour la perte de Peter Pettigrew, le dernier qui lui restait.

"Parfois, cela me fait encore mal de penser qu'il est à Azkaban," finit Remus, sa voix presque un murmure. "Harry, je suis heureux que les Mangemorts n'aient pas le droit aux visiteurs. Cela veut dire que je n'ai pas à me sentir honteux de ne pas lui rendre visite."

Harry dut déglutir plusieurs fois avant de pouvoir parler de nouveau. "Pouvez-vous me parler de Peter Pettigrew ? Il semble qu'il était l'ami de mon père, et il me semble - que je devrais savoir, que je devrais me souvenir -"

Remus hocha la tête, de l'eau scintillait à présent dans ses yeux. "Harry, je pense que si Peter avait su que cela se finirait ainsi -" la voix de l'homme s'étrangla. "Peter avait plus peur du Seigneur des Ténèbres que n'importe lequel d'entre nous, et s'il avait su que cela se finirait ainsi, je ne pense pas qu'il aurait pu continuer. Mais Peter connaissait le \emph{risque} , Harry, il savait que le risque était réel, que ça \emph{pouvait}  arriver, et pourtant il est resté aux côtés de James et Lily. Pendant toute ma scolarité je me suis demandé pourquoi Peter n'avait pas été Trié à Serpentard, ou peut-être à Serdaigle, parce qu'il adorait tellement les secrets, il ne pouvait pas leur résister, il découvrait des choses au sujets des gens, des choses qu'ils souhaitaient maintenir cachées. Et alors l'ombre du Seigneur des Ténèbres a tout recouvert, et Peter est resté aux côtés de James et Lily et il a mit ses talents à bon usage, et j'ai compris pourquoi le Choixpeau l'avait envoyé à Gryffondor." La voix de Remus était féroce à présent, et fière. "C'est \emph{facile}  de rester avec ses amis quand on est un héros comme Godric, intrépide et fort comme les gens pensent qu'un Gryffondor devrait être. Mais si Peter avait plus peur que n'importe lequel d'entre nous, cela ne fait-il pas de lui le plus courageux ?"

"Si," dit Harry. Sa propre gorge était si serrée qu'il ne pouvait presque pas parler. "Si vous le pouvez, M. Lupin, si vous avez le temps, je pense qu'il y a quelqu'un d'autre qui devrait entendre l'histoire de Peter Pettigrew, un élève de Poufsouffle en première année, Neville Londubat."

"Le garçon de Frank et Alice," dit Remus d'une voix devenue triste. "Je vois. Ce n'est pas une histoire joyeuse, Harry, mais je peux la raconter à nouveau si tu penses que cela l'aidera."

Harry hocha la tête.

Un bref silence s'abattit.

"Y avait-il \emph{quoi que ce soit}  d'inachevé entre Peter Pettigrew et Black ?" dit Harry. "\emph{N'importe quoi}  qui aurait pu le faire partir à la recherche de Peter Pettigrew, même sans mériter un meurtre ? Comme un secret que M. Pettigrew aurait connu et que Black aurait voulu connaître, ou qu'il aurait voulu protéger en le tuant ?"

Quelque chose scintilla dans les yeux de Remus, mais l'homme plus âgé secoua la tête et dit : "Pas vraiment."

"Cela veut dire qu'il y \emph{a}  quelque chose," dit Harry.

Le sourire narquois apparut de nouveau sous la moustache poivre et sel. "Tu as un peu de Peter en toi, tu vois. Mais ce n'est pas important, Harry."

"Je suis Serdaigle, je ne suis pas \emph{censé}  résister à la tentation des secrets. Et," dit Harry d'un ton plus sérieux, "si Black trouvait que cela méritait qu'il se fasse prendre, je ne peux pas m'empêcher de penser que c'est peut-être important."

Remus semblait assez mal à l'aise. "J'imagine que je pourrai te le dire quand tu seras plus vieux, mais vraiment, Harry, ce n'est \emph{pas}  important ! Juste quelque chose qui date de notre scolarité."

Harry n'aurait pas pu mettre le doigt sur ce qui lui mit la puce à l'oreille ; ça aurait pu être quelque chose dans le ton de nervosité particulier de la voix de Remus, ou la façon dont l'homme avait dit \emph{quand tu seras plus vieux} , ce détail qui fit jaillir une étincelle dans l'intuition de Harry...

"En fait," dit Harry, "je pense que j'ai plus ou moins déjà deviné, désolé."

Remus leva les sourcils. "Vraiment ?". Il semblait un peu sceptique.

"Ils étaient amants, n'est-ce pas ?"

Il y eut une longue pause gênée.

Remus hocha lentement la tête d'un air grave.

"Une fois," dit Remus. "Il y a longtemps. Une triste affaire, qui s'acheva par une vaste tragédie, c'est du moins ainsi que nous l'avons perçue lorsque nous étions jeunes." La perplexité et le mécontentement étaient clairement présents sur son visage. "Mais j'avais cru que tout cela était terminé, enterré sous une amitié adulte ; jusqu'au jour où Black a tué Peter."

