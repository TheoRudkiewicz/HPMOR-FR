
\chapter{Empathie}

J.K. Rowling est sûre à 87\% que vous allez prendre feu.

Roger Bacon vivait au 13ème siècle et est reconnu comme l'un des premiers partisans de la méthode scientifique. Donner son journal de bord à un scientifique est un peu comme donner à un écrivain non pas la plume de Shakespeare mais celle de quelqu'un qui aurait participé à l'invention de l'écriture.
\par\noindent\rule{\textwidth}{0.4pt}
Ce n'était pas tous les jours que vous pouviez voir Harry Potter supplier.

"\emph{S'il vous plaaaaaaaît," } gémit Harry Potter.

Fred et George secouèrent la tête en souriant.

Harry Potter avait l'air d'agoniser. "Mais je vous ai \emph{dit}  comment j'ai fait pour le chat de Kevin Soufflebranche, et pour Hermione et la disparition du soda, et je ne \emph{peux pas}  vous dire pour le Choixpeau ni pour le Rapeltout ni pour le professeur Rogue..."

Fred et George haussèrent les épaules et se détournèrent, se préparant à partir.

"Si jamais tu trouves la réponse," dirent les jumeaux Weasley, "assure-toi de nous le faire savoir."

"\emph{Vous êtes méchants ! Vous êtes tous les deux méchants !} "

Fred et George rabattirent fermement la porte de la salle vide et s'assurèrent de maintenir leur sourire pendant un moment, juste au cas où Harry Potter aurait pu voir à travers les portes.

Puis ils passèrent l'angle d'un couloir et leurs visages s'affaissèrent.

"J'imagine que les suppositions de Harry -"

"- ne t'ont pas donné d'idées ?" se dirent-ils en même temps l'un à l'autre, et leurs épaules s'abattirent encore plus.

Leur dernier souvenir en rapport avec l'affaire était Flume refusant de les aider même s'ils n'arrivaient pas à se souvenir de \emph{ce}  qu'ils lui avaient demandé de faire...

...mais ils devaient avoir été chercher ailleurs, avoir trouvé \emph{quelqu'un}  prêt à les aider à faire \emph{quelque chose}  d'illégal sans quoi ils n'auraient pas accepté de subir ensuite un sortilège d'Amnésie.

Comment était-il \emph{possible}  qu'ils soient parvenus à accomplir tout ça avec seulement quarante Gallions ?

Au début, ils s'étaient inquiétés d'avoir fabriqué des preuves tellement bonnes que Harry devrait \emph{vraiment}  épouser Ginny... mais il semblait qu'ils avaient aussi prévu cela. Le compte-rendu du Magenmagot avait \emph{de nouveau } été trafiqué et remis dans l'état où il était à l'origine, le faux contrat de fiançailles avait disparu de son coffre-fort de Gringotts gardé un dragon, et ainsi de suite. C'était à vrai dire plutôt effrayant. La plupart des gens pensaient maintenant que la \emph{Gazette du sorcier}  avait tout inventé pour des raisons qui échappaient à tous, et le \emph{Chicaneur}  avait obligeamment retourné le couteau dans la plaie avec son gros titre du lendemain, HARRY POTTER SECRÈTEMENT FIANCÉ À LUNA LOVEGOOD.

Ils souhaitaient désespérément que, quelle que soit la personne qu'ils avaient engagée, elle leur dirait tout après que le délai de prescription ait expiré. Mais en attendant, c'était horrible, ils avaient accompli la plus grande farce de leur vie, peut-être la plus grande farce de l'histoire de la farce, et ils ne se \emph{savaient pas comment} . C'était fou, ils avaient réussi à trouver un moyen la \emph{première}  fois, alors pourquoi ne pouvaient-ils pas en trouver un rétrospectivement, \emph{sachant}  tout ce qu'ils avaient fait ?

Leur seule consolation, c'était que Harry ne savait pas qu'ils ne savaient pas.

Même Maman ne les avait pas interrogé à ce sujet en dépit de l'évidente connexion avec la famille Weasley. Quoi qu'ils aient accompli, c'était loin hors de la portée de tout étudiant de Poudlard... à part peut-être d'\emph{un} , qui, si certaines rumeurs étaient vraies, aurait pu le faire en claquant des doigts. \emph{Harry}  avait été interrogé sous Veritaserum, leur avait-il dit, et Dumbledore était là, jetant des regards inquiétants aux Aurors. Les Aurors l'avaient interrogé juste assez pour déterminer que Harry n'avait pas commis la farce lui-même et qu'il n'avait fait disparaître personne puis ils s'étaient précipités hors de Poudlard.

Fred et George s'étaient demandé s'ils devaient se sentir insultés par le fait que Harry Potter se soit fait interroger par les Aurors pour \emph{leur}  farce mais l'expression de \emph{Harry}  probablement causée par ce même fait les avait convaincus que ça en valait la peine.

Il n'avait pas été surprenant d'apprendre que Rita Skeeter et le rédacteur en chef de la \emph{Gazette du sorcier}  avaient tous deux disparus et qu'ils étaient probablement tous deux dans un autre pays. Ils \emph{auraient}  aimé raconter ça à leur famille. Ils pensaient que Papa les aurait probablement félicités après que Maman ait fini de les tuer et que Ginny ait brûlé leur dépouille.

Mais tout allait bien, ils le diraient à Papa un jour, et en attendant...

...en attendant, Dumbledore avait éternué par hasard en les croisant dans le couloir et un petit paquet était accidentellement tombé de ses poches, et à l'intérieur s'étaient trouvés deux monocles de cambrioleurs assortis d'une \emph{incroyable}  qualité. Les jumeaux Weasley avaient testé leurs monocles sur le couloir "interdit" du troisième étage en faisant un rapide aller-retour jusqu'au miroir magique, et s'ils n'avaient pas pu clairement voir \emph{toutes}  les toiles de détection, les monocles leur en avaient toutefois montré \emph{beaucoup}  plus que ce qu'ils avaient vu lors de leur passage précédent.

Bien sûr, ils faudrait qu'ils fassent très attention de ne jamais se faire prendre avec les monocles en leur possession ou ils finiraient dans le bureau du directeur à se faire donner la leçon avec beaucoup de sévérité, et peut-être même à recevoir des menaces de renvoi.

Il était bon de savoir que tous ceux qui étais répartis à Gryffondor ne finissaient pas comme le professeur McGonagall.
\par\noindent\rule{\textwidth}{0.4pt}
Harry était assis à un bureau dans une pièce blanche sans fenêtre et sans décoration face à un homme sans expression habillé de robes d'un noir uni et formel.

La pièce était protégée contre toute détection et l'homme avait jeté exactement vingt-sept sorts avant de dire ne serait-ce que "Bonjour, M. Potter."

Il était étrangement de circonstance que ce même homme en noir s'apprête à lire l'esprit de Harry.

"Préparez-vous," dit l'homme d'une voix sans timbre.

Le livre d'Occlumancie de Harry avait dit qu'un esprit humain n'était exposé à un Legilimens que par certaines \emph{surfaces} . Si vous échouiez à défendre vos surfaces, le Legilimens allait \emph{au travers}  d'elles et il était alors capable d'accéder à toutes les parties de votre esprit qu'il était capable de comprendre...

...ce qui n'était généralement pas grand-chose. Il semblait que pour les humains, les esprits d'autres humains étaient difficiles à comprendre passé le plus superficiel des niveaux. Harry s'était demandé si le fait d'avoir beaucoup de connaissances en sciences cognitives ferait de lui un Legilimens incroyablement puissant mais ses expériences passées avaient \emph{fini}  par lui inculquer qu'il ferait mieux de s'exalter un peu moins vite à ce genre de sujet. Ce n'était pas comme si les scientifiques cognitifs comprenaient les humains assez bien pour pouvoir en fabriquer un.

Pour apprendre le contre - l'Occlumancie - la première étape consistait à s'imaginer être quelqu'un d'autre, à le prétendre aussi minutieusement que possible, à s'immerger entièrement dans cette personnalité alternative. Vous n'auriez pas toujours à faire ça, mais au début, c'était comme ça que vous appreniez où vos surfaces se trouvaient. Le Legilimens essaierait de vous lire, et, si vous prêtiez assez attention, vous le sentiriez essayer d'entrer. Et votre travail était de vous assurer qu'il touchait toujours votre personnalité imaginaire et jamais la vraie.

Une fois que vous étiez assez bon, vous pouviez imaginer être une personne très \emph{simple} , faire semblant d'être un caillou et prendre l'habitude de laisser le déguisement en place aux endroits où se trouvaient vos surfaces. C'était une barrière Occlumantique standard. Faire semblant d'être un rocher était difficile à apprendre au début mais simple à faire ensuite, et la surface exposée d'un esprit étant bien moins profonde que son intérieur, vous pouviez avec assez de pratique garder le déguisement activé simplement par habitude.

Ou alors, si vous étiez un \emph{Occlumens parfait} , vous pouviez aller \emph{au-devant}  de n'importe quelle sonde et répondre aux questions aussi vites qu'elles étaient posées pour que le Legilimens entre par vos surfaces et ne voie qu'un esprit indistinguable de la personne que vous prétendiez être.

Le fait que l'on puisse tromper les meilleurs des télépathes humains en faisant semblait d'être quelqu'un d'autre rappelait tristement à quel point les humains se comprenaient mal et le peu qu'un sorcier pouvait espérer appréhender des profondeurs qui se cachaient sous la surface de l'esprit.

Mais les humains ne se comprennent en premier lieu que parce qu'ils font semblant. Vous ne faites pas de prédictions au sujet des autres en modélisant les cent billions de synapses de leur esprit sous la forme d'objets distincts. Demandez au meilleur manipulateur de la Terre de vous fabriquer un Intelligence Artificielle à partir de rien et ils vous regardera d'un air bête. Vous prédisez les actes des autres en disant à \emph{votre}  cerveau d'agir comme le leur. Vous vous \emph{mettez à leur place} . Si vous voulez savoir ce qu'une personne en colère ferait, vous activez le circuit de colère de votre propre cerveau et ce que ce circuit produira sera votre prédiction. À quoi ressemble la circuiterie neuronale de la colère ? Qui sait ? Le meilleur manipulateur de la Terre ne sait peut-être même pas ce que \emph{sont } les neurones, pas plus que ne le sait le meilleur des Legilimens.

Tout ce qu'un Legilimens pouvait \emph{comprendre} , un Occlumens pouvait le \emph{prétendre} . C'était la même technique qui était utilisée dans les deux cas - probablement implémentée par la même circuiterie neuronale, un seul ensemble de circuits de contrôle destinés à reconfigurer votre cerveau pour qu'il agisse comme le modèle de celui d'un autre.

Et ainsi la course entre l'offense et la défense télépathique s'était achevée par une victoire décisive pour la défense. Autrement le monde magique et peut-être même la Terre entière auraient été des endroits bien différents...

Harry prit une profonde inspiration et se concentra. Il avait un léger sourire sur le visage.

Pour \emph{une fois} , juste \emph{une fois} , Harry ne s'était pas fait rouler rayon pouvoirs mystérieux.

Après presque un mois de travail, et plus sur un coup de tête qu'à cause d'une vraie intuition, Harry avait décidé de se mettre froidement en colère et d'essayer à nouveau les exercices du livre d'Occlumancie. Il en était alors presque au point d'abandonner tout espoir dans le domaine mais ça avait semblé mériter un rapide essai -

Il avait fait les exercices les plus difficiles du livre à toute vitesse, en moins de deux heures, et le lendemain il était allé voir le professeur Quirrell et il lui avait dit qu'il était prêt.

Il s'était avéré que son côté obscur était très, \emph{très}  bon pour faire semblant d'être quelqu'un d'autre.

Harry pensa à son déclencheur standard, qui datait de la première fois qu'il était entièrement passé à son côté obscur...

\emph{Severus marqua une pause, l'air plutôt content de lui. "Et ce sera... cinq points ? Non, disons dix points retirés à Serdaigle pour impertinence."} 

Le sourire de Harry devint plus froid, et il regarda l'homme en robes noires qui pensait qu'il allait lire l'esprit de Harry.

Puis il devint une toute autre personne, quelqu'un qui lui avait semblé approprié pour l'occasion.
\par\noindent\rule{\textwidth}{0.4pt}
...assis à un bureau dans une pièce blanche sans fenêtre et sans décoration face à un homme sans expression habillé de robes d'un noir uni et formel.

Kimball Kinnison regarda l'homme en robes noires qui pensait qu'il allait lire l'esprit d'un Surfulgur de la Patrouille Galactique.

De dire que Kimball Kinnisen était confiant quant au résultat de la tentative aurait été un euphémisme. Il avait été entraîné par Mentor d'Arisia, l'esprit le plus puissant connu de cet univers et de tout autre, et le simple sorcier assis en face de lui verrait précisément ce que le Fulgur Gris voudrait qu'il voit...

...l'esprit du garçon sous les traits duquel il était pour l'instant déguisé, un garçon innocent nommé Harry Potter.

"Je suis prêt" dit Kimball Kinnison d'un ton nerveux qui était exactement celui qu'aurait eu un garçon de onze ans.

"\emph{Legilimens,} " dit le sorcier en robes noires.

Il y eut un flottement.

Le sorcier en robes noires cligna des yeux comme s'il avait vu quelque chose de si choquant que ça avait mérité qu'il en bouge \emph{ses } paupières. Sa voix n'était pas aussi vide de timbre qu'elle l'avait été : "Le Survivant a un \emph{mystérieux}  \emph{côté obscur}  ?"

Une chaleur grimpa lentement le long des joues de Harry.

"Eh bien," dit l'homme. Son visage était redevenu parfaitement calme. "Excusez-moi, M. Potter, c'est une chose que de connaître ses avantages, mais c'en est une autre que d'être follement trop sûr d'eux. Peut-être pourrez-vous en effet apprendre l'Occlumancie à l'âge de onze ans. Cela me subjugue. J'avais cru que M. Dumbledore faisait encore semblant d'être fou. Votre talent dissociatif est si fort que je suis surpris de ne voir aucun autre signe d'abus infantile, et vous pourriez peut-être finir par devenir un Occlumens parfait. Mais il existe une différence considérable entre ça et s'attendre à ériger une barrière Occlumantique fonctionnelle du premier coup. C'est simplement ridicule. Avez-vous senti quoi que ce soit pendant que je lisais votre esprit ?"

Harry secoua la tête, rougissant furieusement.

"Alors faites plus attention la prochaine fois. Le but n'est pas de créer une image parfaite lors votre premier jour de cours. Le but est d'apprendre où se trouvent vos surfaces. Préparez-vous."

Harry essaya de faire à nouveau semblant d'être Kimball Kinnisen, il essaya de faire plus attention, mais ses pensées étaient un peu éparses, et il avait soudain conscience de toutes les choses auxquelles il ne devrait pas penser...

Oh, ça allait être nul.

Harry grinça des dents. Au moins l'instructeur subirait un sort d'Amnésie.

"\emph{Legilimens} ."

Il y eut une pause -

...assis à un bureau dans une pièce blanche sans fenêtre et sans décoration face à un homme sans expression habillé de robes d'un noir uni et formel.

C'était leur quatrième jour, un dimanche soir. Quand vous payiez autant, vous aviez vos cours quand ça vous chantait et vous n'aviez pas à vous préoccuper du concept de week-ends.

"Bonjour, M. Potter," dit le télépathe d'une voix sans timbre après avoir jeté l'ensemble des sorts de protection de l'intimité.

"Bonjour, M. Bester," dit Harry d'une voix lasse. "Débarrassons-nous déjà du choc initial, voulez-vous ?"

"Vous êtes parvenu à me surprendre ?" dit l'homme, maintenant légèrement intéressé. "Eh bien alors," il pointa sa baguette et se plongea dans les yeux de Harry. "\emph{Legilimens} ."

Il y eut une pause, puis le sorcier en robes noires tressaillit, comme si quelqu'un l'avait touché avec un aiguillon à bétail.

"Le Seigneur des Ténèbres est \emph{en vie}  ?" s'étrangla-t-il. Ses yeux étaient soudain devenus fous. "\emph{Dumbledore se rend invisible et se faufile dans les dortoirs des filles ?} "

Harry soupira et regarda sa main. Dans à peu près trois secondes...

"Donc," dit l'homme. Il n'avait pas entièrement récupéré son absence de timbre. "Vous croyez honnêtement que vous allez découvrir les règles secrètes de la magie et devenir tout-puissant."

"C'est ça," dit Harry d'une voix égale en regardant toujours sa montre. "Je suis \emph{aussi}  présomptueux \emph{que ça} ."

"Je me le demande. Il semble que le Choixpeau pense que vous allez devenir le prochain Seigneur des Ténèbres."

"Et \emph{vous}  savez que je fais de mon mieux pour ne \emph{pas}  le devenir, et vous avez vu que nous avons déjà eu une longue discussion pour déterminer si vous acceptiez de m'enseigner l'Occlumancie, et vous avez fini par décider que vous le ferez, alors pourrions-nous juste passer à la suite ?"

"Très bien," dit l'homme exactement six secondes plus tard, pareil que la dernière fois. "Préparez-vous." Il marqua une paus puis dit, sa voix assez mélancolique : "Bien que \emph{j'aimerais}  pouvoir me souvenir de cette astuce avec l'or et l'argent."

Harry trouvait assez dérangeant de se rendre compte à quel point les pensées humaines étaient reproductibles lorsque vous réinitialisiez les gens aux mêmes conditions initiales et que vous les exposiez aux mêmes stimulus. Cela dissipait des illusions qu'un bon réductionniste n'aurait pas dû avoir en premier lieu.
\par\noindent\rule{\textwidth}{0.4pt}
Le lundi matin suivant, Harry sortait de son cours de Botanique d'un pas lourd, d'assez mauvaise humeur.

Hermione bouillonnait à côté de lui.

Les autres enfants étaient toujours à l'intérieur, rassemblant leurs affaires avec lenteur car ils piaillaient avec excitation au sujet de la victoire de Serdaigle au deuxième match de Quidditch de l'année.

Il semblait que la nuit dernière, après dîner, une fille avait volé sur un balais pendant une demi heure et avait ensuite attrapé une sorte de moustique géant. Il existait d'autres faits relatant ce qui s'était passé pendant le match mais ils n'étaient pas pertinents à l'issue de celui-ci.

Harry avait raté ce palpitant événement sportif à cause de son cours d'Occlumancie et aussi parce qu'il avait une vie, lui.

Il avait donc évité toute conversation du dortoir Serdaigle, les sorts de Sourdinam et les malles magiques n'étaient-ils pas merveilleux. Il avait pris son petit déjeuner à la table Gryffondor.

Mais Harry n'avait pas pu éviter le cours de Botanique, et les Serdaigles en avaient parlé avant le cours, et après le cours, et \emph{pendant}  le cours jusqu'à ce que Harry ait relevé les yeux du bébé furcot dont il changeait la couche et qu'il ait annoncé d'une voix forte que certains ici essayaient d'apprendre quelque chose au sujet des \emph{plantes}  et que les Vifs ne poussaient ni dans les arbres ni ailleurs alors pourraient-ils arrêter de parler du Quidditch \emph{par pitié} . Toutes les personnes présentes lui avaient jeté des regards choqués mis à part Hermione qui avait eu l'air de vouloir applaudir et le professeur Chourave qui lui avait donné un point pour Serdaigle.

Un point pour Serdaigle.

\emph{Un}  point.

Les sept idiots sur leur balais idiots avaient gagné \emph{cent quatre-vingt-dix points}  pour Serdaigle en jouant à leur jeu idiot.

Il semblait que les scores de Quidditch \emph{s'ajoutaient directement au total d'une Maison} .

En d'autres mots, attraper un moustique doré valait 150 points.

Harry n'arrivait même pas à \emph{imaginer}  ce qu'il devrait faire pour mériter cent cinquante points.

À part, vous savez, sauver \emph{cent cinquante Poufsouffles} , ou trouver \emph{quinze idées aussi bonnes que celle consistant à ajouter une coque protectrice autour des machines à remonter le temps} , ou inventer \emph{mille cinq cent façons créatives de tuer des gens} , ou être Hermione Granger pendant \emph{toute l'année} .

"On devrait les tuer," dit Harry à Hermione, qui marchait à côté de lui et qui avait un air tout aussi offensé que le sien.

"Qui ?" dit Hermione. "L'équipe de Quidditch ?"

"Je pensais à toute personne sur Terre ayant quoi que ce soit à voir avec le Quidditch, mais oui, l'équipe de Serdaigle serait un bon début."

Les lèvres de Hermione était pincées d'un air désapprobateur. "Tu \emph{sais}  que c'est mal de tuer des gens, Harry ?"

"Oui," dit Harry.

"D'accord, je voulais juste vérifier," dit Hermione. "Occupons-nous d'abord de l'Attrapeuse. J'ai lu quelques mystères d'Agatha Christie, sais-tu comment on pourrait la faire monter dans un train ?"

"Deux étudiants qui fomentent un meurtre," dit une voix sèche. "Que c'est choquant."

Un homme vêtu de robes légèrement tachées passa un angle proche, sans se presser. Ses cheveux longs, gras et négligés tombaient le long de ses épaules. Un danger mortel semblait irradier de lui, remplir le couloir de potions mal mélangées, de chutes accidentelles et de gens mourant dans leur lit de causes que les Aurors déclareraient plus tard comme étant naturelles.

Sans y penser, Harry se plaça devant Hermione.

Il y eut une inspiration venant de derrière lui, puis un instant plus tard, Hermione le poussa et se plaça devant \emph{lui} . "Cours, Harry !" dit-elle. "Les garçons ne devraient pas être mis en danger."

Severus Rogue eut un faux sourire. "Amusant. Je demande un peu de votre temps, Potter, si vous pouvez vous arracher à votre flirt avec mademoiselle Granger."

Il y eut soudain un air très soucieux sur le visage de Hermione. Elle se retourna vers Harry et ouvrit la bouche, puis s'interrompit, l'air perturbée.

"Oh, ne vous en faites pas, mademoiselle Granger," dit la voix soyeuse de Severus. "Je promets de vous rendre votre beau [NdT: en français dans le texte] non estropié." Son sourire disparut. "Maintenant, Potter et moi allons partir et avoir une conversation en privé, juste entre nous. J'espère qu'il est clair que vous n'êtes pas invitée, mais juste au cas où, considérez cela comme un ordre venant d'un professeur de Poudlard. Je suis sûr qu'une bonne petite fille comme vous ne désobéira pas."

Et Severus se détourna et repartit vers l'angle d'où il était venu. "Vous venez, Potter ?" dit sa voix.

"Euh," dit Harry à Hermione. "Est-ce que je peux, euh, partir et le suivre et \emph{te}  laisser trouver ce que j'aurais dû dire pour m'assurer que tu ne sois pas toute inquiète et offensée ?"

"Non," dit Hermione d'une voix tremblante.

Le rire de Severus leur parvint par échos depuis derrière l'angle du couloir.

Harry inclina la tête. "Désolé," dit-il bassement, "vraiment," et il partit à la poursuite du maître de Potions.
\par\noindent\rule{\textwidth}{0.4pt}
"Donc," dit Harry. Il n'y avait pas d'autre son que celui de leurs deux paires de jambes, les longues et les courtes, qui avançaient à pas feutrés sur un couloir de pierre quelconque ; le maître de Potions déambulait rapidement, mais pas trop, pour que Harry puisse le suivre, et dans la mesure où Harry pouvait appliquer le concept de direction à Poudlard, ils s'éloignaient des zones fréquentées. "De quoi s'agit-il ?"

"J'imagine que vous ne pourriez pas expliquer," dit sèchement Severus, "pourquoi vous fomentiez tous deux le meurtre de Cho Chang ?"

"J'imagine que \emph{vous}  ne pourriez pas expliquer," dit sèchement Harry, "en tant qu'officiel du système scolaire de Poudlard, pourquoi attraper un moustique en or est considéré comme une réussite scolaire qui mérite cinquante points ?"

Un sourire passa sur les lèvres de Severus. "Eh bien, moi qui croyais que vous étiez censé être perspicace. Êtes-vous réellement incapable de comprendre vos camarades de classe, Potter, ou votre aversion pour eux est-elle si grande que vous ne voulez même pas essayer de le faire ? Si les points de Quidditch n'étaient pas comptés pour la coupe des Maisons, alors aucun élève ne s'intéresseraient aux points. Il ne s'agirait plus que d'un obscur concours pour les élèves tels que vous et mademoiselle Granger."

C'était une réponse étonnamment bonne.

Et cet étonnement réveilla entièrement l'esprit de Harry.

Rétrospectivement, il n'aurait pas dû être surprenant que Severus comprenne ses étudiants, et qu'il les comprenne même très bien.

Il avait lu leurs esprits.

Et... le livre disait qu'un Legilimens efficace était extrêmement rare, bien plus rare qu'un Occlumens parfait, parce que presque personne n'avait assez de discipline mentale.

\emph{Discipline mentale}  ?

Harry avait récolté des histoires au sujet d'un homme qui perdait régulièrement son tempérament en cours et qui criait sur de jeunes enfants.

... mais le même homme, quand Harry avait mentionné que le Seigneur des Ténèbres était toujours en vie, avait répondu instantanément et parfaitement, en réagissant exactement comme quelqu'un qui aurait totalement ignoré ce fait.

L'homme déambulait dans Poudlard avec un air d'assassin, irradiant le danger...

...ce qui était exactement ce qu'un vrai assassin n'aurait \emph{pas } fait. Les vrais assassins devaient ressembler à d'humble petits comptables jusqu'à ce qu'ils vous tuent.

Il était le directeur de la fière et aristocratique maison Serpentard, et il portait une robe tachée de potions et de bouts d'ingrédients que deux minutes de magie auraient enlevés.

Harry remarqua qu'il était confus.

Et son estimation de la menace que représentait \emph{le directeur de la maison Serpentard}  grimpa en flèche jusqu'à atteindre des niveaux astronomiques.

Dumbledore avait eu l'air de penser que Severus était sien, et il n'y avait rien eu pour venir contredire cela ; le maître de Potions avait été 'effrayant mais pas abusif', comme promis. Harry s'était donc dit un peu plus tôt qu'il s'agissait là d'une affaire liée à la Communauté. Si Severus avait prévu de lui faire du mal, il ne serait certainement pas venu chercher Harry devant Hermione, un témoin, alors qu'il aurait simplement pu attendre un moment où Harry serait seul...

Harry se mordit discrètement la lèvre.

"J'ai un jour connu un garçon qui adorait le Quidditch," dit Severus Rogue. "C'était un abruti fini. Comme nous nous y serions attendus tous deux."

"De \emph{quoi}  s'agit-il ?" dit lentement Harry.

"Patience, Potter."

Severus tourna la tête, puis il glissa à sa manière d'assassin vers une ouverture proche entre les murs du couloir, un corridor plus petit et plus étroit, qui dérivait.

Harry le suivit en se demandant s'il aurait été plus intelligent de simplement s'enfuir.

Il tourna, puis il tourna encore, et il arriva à un cul-de-sac, un simple mur vide. Si Poudlard avait vraiment été construite et pas invoquée ou conjurée ou née ou quelque chose du genre, Harry aurait eu quelques mots acerbes à dire à l'architecte qui avait payé des gens pour construire des corridors qui ne menaient nulle part.

"\emph{Silencio} ," dit Severus, ainsi que quelques autres choses.

Harry se pencha en arrière, croisa ses bras sur sa poitrine, et regarda le visage de Severus.

"On me regarde dans les yeux, Potter ?" dit Severus Rogue. "Vos leçons d'Occlumancie ne peuvent avoir progressé suffisamment pour que vous puissiez bloquer la Légilimancie. Mais peut-être avez-vous assez progressé pour être capable de la détecter. Puisque je ne peux le savoir, je m'abstiendrai d'essayer. L'homme sourit légèrement. "Et il en va de même pour Dumbledore, je pense. Qui est la raison pour laquelle nous sommes \emph{en train}  d'avoir cette conversation."

Les yeux de Harry s'écarquillèrent involontairement.

"Pour commencer," dit Severus, les yeux brillants, "je voudrais votre promesse que vous ne parlerez pas de notre conversation à \emph{quiconque} . En ce qui concerne l'école, nous discutons de vos devoirs de Potions. Qu'ils le croient ou non n'a aucune importance. En ce qui concerne Dumbledore et McGonagall, je trahis les confidences que Draco Malfoy m'a faites, et aucun de nous deux ne pense qu'il serait de bon ton d'en dire plus long sur les détails."

Le cerveau de Harry essaya de calculer les ramifications et les implications de ce que Rogue venait de dire et il manqua de mémoire virtuelle.

"Eh bien ?" dit le maître de Potions.

"Très bien," dit lentement Harry. Il était difficile de voir comment le fait d'avoir une conversation et de ne pouvoir dire à personne que vous l'aviez eue était plus contraignant que de ne \emph{pas}  l'avoir eue, auquel cas vous ne pouviez pas \emph{non plus}  en dire le contenu à quelqu'un. "Je promets."

Severus regardait Harry avec attention. "Vous avez un jour dit, dans le bureau du directeur, que vous ne toléreriez pas qu'on malmène ou qu'on abuse des enfants. Je me demande donc, Harry Potter. À quel point ressemblez-vous à votre père ?"

"À moins que nous ne parlions de Michael Verres-Evans," dit Harry, "la réponse est que j'en sais très peu au sujet de James Potter."

Severus hocha la tête, comme s'il s'adressait à lui-même. "Il y a un Serpentard en cinquième année. Un garçon nommé Lesath Lestrange. Il est malmené par des Gryffondors. Je suis... restreint dans mes capacités à gérer une telle situation. \emph{Vous}  pourriez peut-être l'aider. Si vous le souhaitiez. Je ne vous demande pas une faveur et je ne vous en devrai aucune. Il s'agit simplement d'une opportunité de faire ce que vous voulez."

Harry regarda Severus en réfléchissant.

"Vous vous demandez si c'est un piège ?" dit Severus, un léger sourire sur les lèvres. "Ce n'est pas un piège. C'\emph{est}  un test. Considérez cela comme une marque de ma curiosité. Mais les problèmes de Lesath sont réels, tout comme le sont mes propres difficultés à intervenir."

C'était le problème quand les gens savaient que vous étiez un des gentils. Même si vous saviez qu'ils le savaient, vous ne pouviez toujours pas ignorer l'appât.

Et si son père avait lui aussi protégé les étudiants contre les petits durs... ça n'avait pas d'importance que Harry sache pourquoi Severus lui avait donné cette information. Ça lui réchauffait quand même le cœur, le rendait fier, et l'empêchait de refuser l'opportunité.

"Très bien," dit Harry. "Parlez-moi de Lesath. Pourquoi est-il malmené ?"

Le visage de Severus perdit son léger sourire. "Vous pensez qu'il y a des \emph{raisons} , Potter ?"

"Peut-être pas," dit doucement Harry. "Mais l'idée m'est venue qu'il pourrait avoir poussé une Sang-de-Bourbe sans importance du haut de quelque escalier."

"Lesath Lestrange," dit Severus, sa voix à présente froide, "est le fils de Bellatrix Black, la servante la plus fanatique et la plus maléfique du Seigneur des Ténèbres. Lesath est le bâtard reconnu de Rastaban Lestrange. Peu après la mort du Seigneur des Ténèbres, Bellatrix, Rastaban et le frère de Rastaban, Rodolphus, furent capturés alors qu'ils torturaient Alice et Frank Londubat. Ils sont tous trois emprisonnés à vie à Azkaban. Les Londubat ont été rendus fous par des Cruciatus répétés et demeurent à l'aile des Incurables de Sainte Mangouste. Est-ce là une bonne raison de le maltraiter, Potter ?"

"Absolument pas," dit Harry, toujours doucement. "Et Lesath lui-même n'a commis aucun tort dont vous soyez au courant ?"

Le léger sourire traversa de nouveau les lèvres de Severus. "Il n'est pas plus saint qu'un autre. Mais il n'a poussé aucune Sang-de-Bourbe dans les escaliers, pas que j'ai entendu."

"Ou lu dans son esprit," dit Harry.

L'expression de Severus était glacée. "Je n'ai pas envahi son intimité, Potter. J'ai plutôt regardé dans les Gryffondors. Il constitue une cible commode pour la satisfaction de leurs petites envies."

Un douche froide de colère courut le long de l'épine dorsale de Harry, et il dut se rappeler que Severus n'était peut-être pas une bonne source d'information.

"Et vous pensez," dit Harry, "qu'une intervention par Harry Potter, le Survivant, pourrait s'avérer efficace."

"En effet," dit Severus Rogue, et il dit à Harry où et quand les Gryffondors avaient prévu leur prochain petit jeu.
\par\noindent\rule{\textwidth}{0.4pt}
Il existe un grand couloir qui traverse le centre du deuxième étage de Poudlard le long de l'axe Nord-Ouest, et près du centre de ce couloir se trouve une ouverture menant à un petit corridor qui se prolonge sur douze mètres avant de tourner à angle droit, formant ainsi un L, puis il continue sur douze mètres de plus avant de s'arrêter au pied d'une fenêtre large et éclairée qui offre une vue venant de trois étage plus haut et révélant la légère bruine qui tombe sur les terrains à l'est de Poudlard. En se tenant à cette fenêtre, on ne peut rien entendre de ce qui a lieu dans le couloir principal, et personne dans le couloir ne pourrait entendre ce qui se passe près de cette fenêtre. Si vous pensez qu'il y a là quoi que ce soit d'étrange, c'est que vous n'êtes pas à Poudlard depuis très longtemps.

Quatre garçons en robes bordées de rouge rient, et un garçon vêtu de robes bordées de vert crie et s'agrippe frénétique au rebord de la fenêtre ouverte tandis que les quatre garçons font mine de le pousser à l'extérieur. C'est juste une blague, bien sûr, et puis une chute de cette hauteur ne tuerait pas un sorcier. Une bonne plaisanterie. Si vous pensez qu'il y a là quoi que ce soit d'étrange -

"\emph{Que faites-vous ?} " dit la voix d'un sixième garçon.

Les quatre garçons en robes bordées de rouge pivotent en tressaillant et le garçon en robes bordées de vert s'écarte frénétiquement de la fenêtre et tombe au sol, le visage strié de larmes.

"Oh," dit le plus beau des garçons en robes à bordures rouges, l'air soulagé, "c'est \emph{toi } ? Hé, Lessy, tu sais qui c'est ?"

Il n'y a aucune réponse venant du garçon au sol, qui essaie de contrôler ses reniflements, et le garçon en robes bordées de rouge replie sa jambe, se préparant à frapper -

"\emph{Arrête !} " crie le sixième garçon.

Le garçon en robes bordées de rouge vacille en interrompant son coup de pied. "Euh," dit-il, "sais-\emph{tu}  qui il est ?"

La respiration du sixième garçon semble étrange. "Lesath Lestrange," dit-il, le souffle très court, "et \emph{il}  n'a rien fait à mes parents, il avait cinq ans."
\par\noindent\rule{\textwidth}{0.4pt}
Neville Londubat fixa les quatre brutes immenses qui lui faisaient face, essayant de contrôler ses tremblements du mieux qu'il le pouvait.

Il aurait juste dû dire non à Harry Potter.

"Pourquoi est-ce que \emph{tu}  le défends ?" dit celui qui était beau, lentement, sur un ton perplexe qui recelait déjà les premières notes de l'offense. "C'est un \emph{Serpentard} . Et un \emph{Lestrange} ."

"C'est un garçon qui a perdu ses parents," dit Neville Londubat. "Je sais ce que c'est." Il ne savait pas d'où les mots étaient venus. Ça avait eu l'air trop cool, comme quelque chose que Harry Potter aurait dit.

Cela dit, les tremblements continuèrent.

"Tu te prends pour \emph{qui}  ?" dit celui qui était beau, un début de colère dans la voix.

\emph{Je suis Neville, le dernier descendant de la Noble et Ancienne Maison Londubat -} 

Neville n'arrivait pas à le dire.

"Je pense que c'est un \emph{traître} ," dit l'un des autres Gryffondors, et l'estomac de Neville se noua soudain.

Il l'avait su, il l'avait juste su. Enfin de compte, Harry Potter avait eu tort. Les brutes ne s'arrêtaient pas seulement parce que Neville Londubat leur disait d'arrêter.

Celui qui était beau fit un pas en avant et les trois autres suivirent.

"Alors tu vois ça comme ça," dit soudain Neville, impressionné par la fermeté de sa voix. "Tu te fiches que ce soit Lesath Lestrange ou Neville Londubat."

Lesath Lestrange émit soudain un glapissement depuis l'emplacement où il gisait.

"Le mal est le mal," gronda celui qui avait déjà parlé, "et si tu es ami avec le mal, alors tu es mauvais toi aussi."

Les quatre avancèrent d'un pas.

Lesath se leva, vacillant. Son visage était gris, il fit quelques pas vers l'avant, se pencha contre le mur, et ne dit rien. Ses yeux étaient rivés sur l'angle du couloir, sur la sortie.

"Amis," dit Neville. Sa voix était maintenant montée d'une octave. "Oui, j'ai des amis. L'un d'eux est le Survivant."

Deux des Gryffondors eurent soudain l'air inquiet. Le plus beau ne broncha pas. "Harry Potter n'est pas là", dit-il, la voix dure, "et s'il l'était, je ne pense pas qu'il aimerait voir un Londubat défendre un Lestrange."

Et le Gryffondor fit un autre pas en avant, et derrière eux, Lesath se glissa le long du mur, attendant sa chance.

Neville déglutit et leva la main droite avec son pouce et son index pressés l'un contre l'autre.

Il ferma les yeux parce que Harry lui avait fait promettre de ne pas regarder.

Si ça ne marchait pas, il ne ferait plus jamais confiance à personne.

Sa voix s'éleva, étonnamment claire au vu des circonstances.

"Harry James Potter-Evans-Verres. Harry James Potter-Evans-Verres. Harry James Potter-Evans-Verres. Par la dette que tu as auprès de moi et par le pouvoir de ton vrai nom je t'invoque, je t'ouvre le chemin, je t'appelle à te manifester devant moi."

Neville claque des doigts.

Puis il ouvrit les yeux.

Lesath Lestrange le fixait.

Les quatre Gryffondors le fixaient.

Le plus beau commença à ricaner, ce qui fit partir les trois autres.

"Harry Potter était censé apparaître à l'angle de ce couloir ?" dit celui qui était beau. "Ouille. On dirait que tu t'es fait poser un lapin."

Le plus beau fit un pas menaçant en direction de Neville.

Les trois autres suivirent automatiquement.

"Ahem," dit Harry Potter depuis quelque part derrière eux, appuyé contre le mur près de la fenêtre, dans le cul-de-sac du corridor, là où personne n'aurait pu se rendre sans être vu.

Si regarder les gens crier était toujours autant agréable, alors Neville comprenait plus ou moins pourquoi certains décidaient de devenir des brutes.

Harry Potter glissa vers l'avant, se plaçant entre Lesath Lestrange et les autres. Il fit passer son regard de glace le long des quatre garçons vêtus de robes à bordures rouges, puis ses yeux vinrent s'arrêter sur le plus beau, le chef de bande. "M. Carl Sloper," dit Harry Potter. "Je crois avoir pleinement compris la situation. Si Lesath Lestrange a jamais commis le moindre mal, à part celui d'être né des mauvais parents, ce fait n'est pas connu de \emph{vous} . Si je me trompe à ce sujet, M. Sloper, je suggère que vous m'en informiez immédiatement."

Neville vit la peur et l'admiration sur le visage des autres garçons. Il la ressentait lui-même. Harry avait \emph{dit}  que ce ne serait qu'un tour, mais comment était-ce possible ?

"Mais c'est un \emph{Lestrange} ," dit le chef de bande.

"C'est un garçon qui a\emph{ perdu ses parents} ," dit Harry Potter, sa voix devenant encore plus froide.

Cette fois, les trois autres Gryffondor tressaillirent.

"Donc," dit Harry Potter. "Vous avez vu que Neville ne voulait pas que vous tourmentiez un garçon innocent au nom des Londubat. Cela ne vous a pas convaincu. Si je vous dit que le Survivant pense \emph{lui aussi}  que vous avez tort, que ce que vous avez fait aujourd'hui était une terrible erreur, cela change-t-il quelque chose ?"

Le chef de bande fit un pas vers Harry.

Les autres ne le suivirent \emph{pas} .

"Carl," dit l'un d'eux en déglutissant. "On devrait peut-être y aller."

"Ils disent que tu vas devenir le prochain Seigneur des Ténèbres," dit le chef de bande, fixant Harry.

Un sourire apparut sur le visage de Harry Potter. "Ils disent aussi que je suis secrètement fiancé à Ginevra Weasley et qu'il y a une prophétie annonçant que nous allons conquérir la France." Le sourire s'effaça. "Puisque vous êtes fermement décidé à insister, M. Carl Sloper, laissez-moi rendre les choses claires. \emph{Laissez Lesath tranquille} . Si vous ne le faites pas, je le saurai."

"Alors comme ça Lessy est allé rapporter," dit froidement le chef de bande.

"Bien sûr," dit sèchement Harry Potter, "et il m'a aussi dit ce que vous avez fait aujourd'hui après le court de Sortilèges, dans un endroit isolé et intime où personne ne pouvait vous voir, avec une certaine fille de Poufsouffle qui porte un ruban blanc dans les cheveux -"

La mâchoire du chef s'affaissa sous l'effet du choc.

"Aaaah !" dit l'un des autres Gryffondors d'une voix aiguë, et il pivota sur ses talons puis il s'élança et dépassa l'angle du corridor. On entendit le battement de ses pas rapide s'estomper peu à peu.

Et ils ne furent plus que six.

"Ah," dit Harry Potter, "voilà que part un jeune homme légèrement intelligent. Vous autres pourriez apprendre de l'exemple de Bertram Kirke avant d'avoir à faire face à des, disons à des problèmes."

"Tu menaces de nous dénoncer ?" dit le beau Gryffondor, essayant de mettre de la colère dans sa voix plutôt vacillante. "Les rapporteurs, il leur arrive des bricoles !"

Les deux autres Gryffondors commencèrent à reculer lentement.

Harry Potter se mit à rire. "Oh, tu ne viens pas de dire ça. Essaies-tu \emph{vraiment}  de m'intimider ? \emph{Moi } ? Non sincèrement, tu te crois plus effrayant que Peregrine Derrick, que Severus Rogue ou tant qu'on y est que Tu-Sais-Qui ?"

Même le chef de bande vacilla en entendant cela.

Harry Potter leva sa main, prêt à claquer des doigts, et les trois Gryffondors se jetèrent en arrière, et l'un d'eux laissa échapper : "Non, ne... !"

"Vous voyez," dit Harry Potter, "c'est là que je claque des doigts et que vous devenez l'ingrédient dune histoire hilarante qui sera contée au milieu beaucoup de rires nerveux ce soir au dîner. Mais le truc, c'est que des gens en qui j'ai confiance n'arrêtent pas de me dire de ne pas faire ça. Le professeur McGonagall m'a dit que je choisissais la solution de facilité, et le professeur Quirrell a dit que je devais apprendre à perdre. Vous vous rappelez de cette histoire où je me suis laissé battre par des Serpentards plus âgés ? On pourrait faire ça. Vous pourriez me brutaliser un moment et je pourrais vous laisser faire. Seulement, vous vous rappelez le moment à la fin où je dis à mes très nombreux amis de ne rien faire pour me venger ? Cette fois-ci on sautera cette partie. Alors allez-y. Brutalisez-moi."

Harry Potter s'avança, les bras grands ouverts en signe d'invitation.

Les trois Gryffondor cédèrent sous la pression et se mirent à courir, et Neville dut faire un pas de côté pour ne pas se faire courir dessus.

Il y eut un silence tandis que les bruits de pas s'estompaient, et plus de silence ensuite.

Et ils ne furent plus que trois.

Harry Potter prit une profonde inspiration puis exhala. "Waoh," dit-il. "Comment ça va, Neville ?"

La voix de Neville sortit sous la forme d'un couinement aigu. "Alors \emph{ça} , c'était vraiment cool."

Un sourire apparut brièvement sur le visage de Harry. "\emph{Tu}  étais très cool toi aussi, tu sais."

Neville savait que Harry Potter disait ça comme ça, pour essayer de le faire se sentir bien, mais ça alluma quand même une chaude lueur à l'intérieur de sa poitrine.

Harry se tourna vers Lesath Lestrange -

"Ça va, Lestrange ?" dit Neville avant que Harry ne puisse ouvrir la bouche.

Alors ça, c'était quelque chose qu'il ne serait jamais imaginé dire un jour.

Lesath Lestrange pivota lentement et regarda Neville, le visage serré, ne pleurant plus, ses larmes étincelant tandis qu'elles séchaient.

"Tu crois que tu sais ce que c'est ?" dit Lesath, sa voix perchée et tremblante. "\emph{Tu crois que tu sais}  ? Mes parents sont à \emph{Azkaban} , j'essaie de ne pas y penser, mais ils me le rappellent toujours, ils pensent que c'est \emph{génial}  que Mère soit là dans le froid et dans le noir pendant que les Détraqueurs absorbent sa vie, j'aimerais être comme Harry Potter, au moins ses parents ne souffrent pas, mes parents souffrent toujours, chaque seconde, chaque jour, j'aimerais être comme toi, au moins tu peux voir tes parents parfois, au moins tu sais qu'ils t'aimaient, si Mère m'a jamais aimé les Détraqueurs ont probablement mangé cette pensée maintenant -"

Les yeux de Neville étaient écarquillés sous l'effet de la surprise. Il ne s'était pas attendu à ça.

Lesath se tourna vers Harry, dont les yeux étaient emplis d'horreur.

Lesath se jeta au sol, face à Harry, puis toucha la pierre de son front et murmura : "Aide-moi, Seigneur."

Il y eut un horrible silence. Neville n'arrivait pas à trouver ce qu'il pouvait bien dire, et vu le choc intense qui se lisait sur le visage de Harry, lui non plus ne savait pas quoi répondre.

"Ils disent que tu peux tout faire, s'il te plaît, s'il te plaît Seigneur, fais sortir mes parents d'Azkaban, je serai à jamais ton loyal serviteur, ma vie sera tienne et ma mort aussi, seulement, s'il te plaît -"

"Lesath," dit Harry, et sa voix se brisa, "Lesath, je ne peux pas, je ne peux pas vraiment faire des choses pareilles, ce ne sont que des tours idiots."

"\emph{Non !} " dit Lesath, sa voix perchée et désespérée. "Je l'ai \emph{vu} , les histoires sont vraies, tu \emph{peux le faire}  !"

Harry déglutit. "Lesath, j'ai tout organisé avec Neville, nous avons tout planifié à l'avance, demande-lui !"

C'était le cas ; même si Harry n'avait pas dit \emph{comment}  ils allaient s'y prendre...

Quand Lesath releva la tête, son visage était livide et sa voix leur parvint dans un crissement qui déchira les oreilles de Neville. "\emph{Espèce de fils de Sang-de-Bourbe ! Tu pourrais la faire sortir, mais tu ne le feras pas ! Je me suis mis à genoux et je t'ai supplié et tu ne m'aideras pas ! J'aurais dû le savoir, tu es le Survivant, tu crois que c'est sa place, là-bas !"} 

"Je ne \emph{peux pas}  !" dit Harry, sa voix aussi désespérée que celle de Lesath. "La question n'est pas ce que je veux, je n'en ai pas le \emph{pouvoir}  !"

Lesath se releva et cracha aux pieds de Harry puis il se détourna et s'éloigna. Lorsqu'il eut tourné l'angle, le bruit de ses pas s'accéléra, et alors qu'ils s'estompaient Neville crut entendre le bruit d'un unique sanglot.

Et ils ne furent plus que deux.

Neville regarda Harry.

Harry regarda Neville.

"Waoh," dit doucement Neville. "Il n'avait pas l'air très reconnaissant qu'on l'aie sauvé."

"Il pensait que je pouvais l'aider," dit Harry d'une voix rauque. "Il espérait pour la première depuis des années."

Neville avala sa salive avec difficulté, et le dit enfin : "Je suis désolé."

"Hein ?" dit Harry, l'air totalement dérouté.

"Je n'étais pas reconnaissant quand tu m'as aidé -"

"Tout ce que tu as dit était vrai," dit le Survivant.

"Non," dit Neville, "pas tout."

Ils se donnèrent simultanément un bref sourire triste, chacun regardant l'autre avec condescendance.

"Je sais que ce n'était pas réel", dit Neville, "je sais que je n'aurais rien pu faire si tu n'avais pas été là, mais merci de m'avoir laissé faire comme si."

"N'importe quoi," dit Harry.

Harry s'était détourné de Neville et regardait les nuages lugubres visibles par la fenêtre.

Une idée complètement idiote vint à Neville. "Te sens-tu coupable parce que tu ne peux pas faire sortir les parents de Lesath d'Azkaban ?"

"Non," dit Harry.

Quelques secondes s'écoulèrent.

"Oui," dit Harry.

"Tu es bête," dit Neville.

"J'en suis conscient," dit Harry.

"Dois-tu littéralement faire \emph{tout}  ce qu'on te demande ?"

Le Survivant pivota et regarda de nouveau Neville. "\emph{Le faire}  ? Non. Me sentir coupable de ne pas le faire ? Oui."

Neville avait du mal à choisir ses mots. "Quand le Seigneur des Ténèbres est mort, Bellatrix Black est alors devenue la pire personne du monde, littéralement, et c'était \emph{avant}  qu'elle aille à Azkaban. Elle a torturé ma mère et mon père jusqu'à ce qu'ils deviennent fous parce qu'elle voulait savoir ce qui était arrivé au Seigneur des Ténèbres -"

"Je sais," dit doucement Harry. "Je comprends ça, mais -"

"Non ! Tu ne comprends \emph{pas}  ! Elle avait une \emph{raison}  de faire ça, et mes parents étaient tous les deux des Aurors ! C'est \emph{loin}  d'être la pire chose qu'elle ait jamais faite !" la voix de Neville vacillait.

"Quand même," dit le Survivant, les yeux distants, regardant quelque part, vers l'ailleurs, vers un endroit que Neville ne pouvait pas imaginer. "Il existe peut-être une solution incroyablement intelligente qui permettrait de sauver tout le monde et qu'ils vivent heureux pour toujours, et si j'étais assez intelligent, j'y aurais déjà pensé -"

"Tu as un problème," dit Neville. "Tu penses que tu te dois d'être la personne que Lesath Lestrange pense que tu es."

"Ouais," dit le Survivant, "tu as carrément mis le doigt dessus. Chaque fois que quelqu'un crie une prière à laquelle je ne peux pas répondre, je me sens coupable de ne pas être Dieu."

Neville ne comprenait pas vraiment mais... "Ça n'a pas l'air d'aller."

Harry soupira. "Je comprends que j'ai un problème, et je sais ce que je dois faire pour le résoudre, d'accord ? J'y travaille."
\par\noindent\rule{\textwidth}{0.4pt}
Harry regarda Neville s'en aller.

Bien sûr, Harry n'avait pas dit quelle était la solution.

La solution était évidemment de se dépêcher de devenir Dieu.

Les bruits de pas de Neville se déplacèrent et il fut bientôt inaudible.

Et il ne fut plus qu'un.

"Ahem," dit la voix de Severus Rogue, directement depuis son dos.

Harry laissa échapper un petit cri et se haït instantanément.

Il se retourna lentement.

Le grand homme graisseux vêtu de robes tachées se tenait appuyé contre un mur dans la même position que celle que Harry avait occupée.

"Une excellente cape d'invisibilité," dit le maître de Potions d'une voix traînante. "Voilà qui explique beaucoup."

Ah, bordel de fiente.

"Et peut-être que j'ai passé trop de temps aux côtés de Dumbledore," dit Severus, "mais je ne peux m'empêcher de me demander si ce n'est pas \emph{La}  Cape d'Invisibilité."

Harry se transforma immédiatement en quelqu'un qui n'avait jamais entendu parler de la Cape d'Invisibilité et qui était \emph{exactement}  aussi intelligent que Harry croyait que Severus croyait que Harry était.

"Oh, peut-être," dit Harry. "Je vous fais confiance pour comprendre ce que cela impliquerait si c'était le cas ?"

La voix de Severus était condescendante. "Vous n'avez pas la moindre idée de ce dont je parle, n'est-ce pas, Potter ? Une tentative bien maladroite d'essayer d'aller à la pêche aux informations."

(Le professeur Quirrell avait remarqué pendant leur déjeuner que Harry devait vraiment dissimuler son état d'esprit mieux qu'en adoptant un visage neutre à chaque fois que quelqu'un discutait d'un sujet dangereux, et il avait parlé de la duplicité de premier niveau, de deuxième niveau, et ainsi de suite. Donc, soit Severus \emph{voyait}  en effet Harry comme un joueur du premier niveau, ce qui faisait de Severus un niveau deux, auquel cas le coup de troisième niveau que Harry venait de porter avait réussi ; ou Severus était un joueur de quatrième niveau et voulait que Harry \emph{pense}  que la supercherie avait réussi. Harry, en souriant, avait demandé au professeur Quirrell à quel niveau \emph{lui } jouait, et le professeur Quirrell, en souriant, avait répondu : \emph{un niveau au-dessus du vôtre} .)

"Alors vous regardiez depuis le début," dit Harry. "Je crois qu'on appelle ça le sortilège de Désillusion."

Un léger sourire. "Il aurait été stupide de ma part de prendre le risque qu'il vous arrive malheur."

"Et vous vouliez voir le résultat de votre test vous-même," dit Harry. "Alors. Est-ce que je ressemble à mon père ?"

Un étrange expression se dessina sur le visage de l'homme, une expression qui ne lui ressemblait pas. "Je dirais plutôt, Harry Potter, que vous ressemblez à -"

Severus s'interrompit abruptement.

Il fixa Harry.

"Lestrange vous a traité de fils de Sang-de-Bourbe," dit lentement Severus. "Ça n'a pas eu l'air de beaucoup vous embêter."

Harry fronça les sourcils. "Non, pas dans ces circonstances."

"Vous veniez de l'aider," dit Severus. Ses yeux étaient braqués sur Harry. "Et il vous l'a renvoyé au visage. Ce n'est sûrement pas quelque chose que vous allez juste pardonner ?"

"Il venait de vivre une expérience plutôt traumatisante," dit Harry. "Et je ne pense pas qu'être sauvé par des première année l'aie non plus beaucoup aidé."

"J'imagine que c'était simple à pardonner," dit Severus, et sa voix était étrange, "puisque Lestrange n'a aucune importance à vos yeux. Juste un étrange Serpentard. S'il avait été votre ami, peut-être vous seriez-vous senti bien plus blessé par ce qu'il a dit."

"S'il avait été un ami," dit Harry, "raison de plus de le pardonner."

Il y eut un long silence. Harry sentit, et il n'aurait pu dire ni pourquoi ni d'où, que l'air se remplissait d'une horrible tension, comme de l'eau qui monterait, monterait et monterait.

Puis Severus sourit, semblant soudain à nouveau détendu, et toute la tension disparut.

"Vous êtes une personne très clémente," dit Severus, souriant toujours. "J'imagine que c'est votre père, Michael Verres-Evans, qui vous a enseigné cela."

"Plutôt la collection de science-fiction et de fantasy de papa," dit Harry. "En fait, c'était une sorte de cinquième parent. J'ai vécu les vies de tous les personnages de tous mes livres, et tout leur puissant savoir résonne dans mon esprit. Quelque part dans ces livres, je suis sûr qu'il y avait quelqu'un comme Lesath, même si je ne pourrais pas dire qui. Ce n'était pas difficile de me mettre à sa place. Et ce sont mes livres qui m'ont aussi dit quoi faire. Les gentils pardonnent."

Severus rit d'un rire léger. "J'ai peur de ne pas bien savoir ce que les gentils font."

Harry le regarda. C'était à vrai dire assez triste. "Je vous prêterai des livres avec des gentils si vous voulez."

"Je voudrais vous demander un conseil," dit Severus d'un ton badin. "Je connais un autre Serpentard de cinquième année qui se faisait brutaliser par des Gryffondor. Il faisait la cour à une magnifique née-Moldue qui le croisa tandis qu'il se faisait brutaliser, et elle essaya de le sauver. Et il la traita de Sang-de-Bourbe, et tout fut fini entre eux. Il s'excusa de nombreuses fois, mais elle ne le pardonna jamais. Avez-vous la moindre idée de ce qu'il aurait pu dire ou faire pour obtenir d'elle le pardon que vous avez octroyé à Lestrange ?"

"Euh", dit Harry, "en me basant uniquement sur cette information, je ne suis pas certain que ce soit \emph{lui}  qui ait eu un problème. Je lui aurais dit de ne pas sortir avec quelqu'un incapable de pardonner. Imaginez qu'ils se soient mariés, pouvez-vous imaginer la vie dans leur ménage ?"

Il y eut une pause.

"Oh, mais elle \emph{pouvait}  pardonner," dit Severus d'un ton amusé, "puisqu'elle devint ensuite la petite amie de celui qui brutalisait le Serpentard. Dites-moi, pourquoi aurait-elle pardonné la brute et non la victime ?"

Harry haussa les épaules. "C'est une folle conjecture, mais parce que la brute avait fait beaucoup de mal à quelqu'un \emph{d'autre} , et que la victime lui avait fait un peu de mal à \emph{elle} , ce qui, pour une raison ou une autre, lui a semblé bien plus impardonnable. Ou, sans vouloir être trop direct, la brute était-elle belle ? Ou tant qu'on y est, était-elle riche ?"

Il y eut une autre pause.

"Oui aux deux," dit Severus.

"Et voilà," dit Harry. "Non pas que j'ai moi-même jamais été au lycée, mais mes livres m'ont laissé entendre qu'il existe un certain genre d'adolescente qui sera outragée par une seule insulte venant d'un garçon pauvre à l'apparence normale, et pourtant parviendra à trouver dans son cœur la force de pardonner le riche et magnifique garçon qui brutalise les autres. En d'autre mots, elle était superficielle. Dites à la personne dont vous parlez qu'elle ne le méritait pas, qu'il doit l'oublier et passer à autre chose, et que la prochaine fois, il devrait sortir avec des filles qui ont de l'esprit, au lieu de celles qui sont simplement mignonnes."

Severus regarda Harry en silence, ses yeux étincelants. Le sourire s'était dissipé, et bien que le visage de Severus soit agité de soubresauts, le sourire ne revint pas.

Harry commençait à se sentir un peu nerveux. "Euh, non pas que j'ai la moindre expérience dans le domaine, bien sûr, mais je pense que c'est ce qu'un sage conseiller de mes livres aurait dit."

Il y eut plus de silence et plus d'étincelles.

C'était probablement le bon moment pour changer de sujet.

"Donc," dit Harry. "Ai-je passé votre test, quel qu'il ait été ?"

"Je pense," dit Severus, "qu'il ne devrait plus y avoir la moindre conversation entre nous, Potter, et vous seriez extrêmement sage de ne jamais mentionner celle-ci."

Harry cligna des yeux. "Accepteriez-vous de me dire ce que j'ai fait de mal ?"

"Vous m'avez offensé," dit Severus. "Et je ne fais plus confiance à votre capacité à la fourberie."

Harry regarda Severus, plutôt pris par surprise.

"Mais vous m'avez donné des conseils bien intentionnés," dit Severus Rogue, "aussi je vous donnerai un vrai conseil en retour." Sa voix était presque parfaitement stable. Comme une corde tendue presque parfaitement à l'horizontale malgré l'énorme poids accroché en son milieu, maintenue grâce aux millions de tonnes de tension tirant de chaque côté. "Vous avez failli mourir aujourd'hui, Potter. À l'avenir, ne partagez jamais votre sagesse avec quiconque à moins de savoir exactement ce dont vous parlez tous les deux."

Le cerveau de Harry fit enfin la connexion.

"\emph{Vous}  étiez ce -"

La bouche de Harry se referma net lorsque \emph{failli mourir}  le percuta enfin, deux secondes trop tard.

"Oui," dit Severus, "c'était moi."

Et la terrible tension se déversa à nouveau dans la pièce, comme de l'eau sous pression venue du fond de l'océan.

Harry ne pouvait pas respirer.

\emph{Perds. Maintenant} .

"Je ne savais pas", murmura Harry. "Je suis dé-"

"Non," dit Severus. Juste cet unique mot.

Harry se tint là, silencieux, son esprit frénétiquement à la recherche d'alternatives. Severus se tenait entre lui et la fenêtre, ce qui était vraiment dommage parce qu'une chute de cette hauteur n'aurait pas tué un sorcier.

"Vos livres vous ont trahi," dit Severus, toujours de cette voix étirée par des millions de tonnes de traction. "Ils ne vous ont pas dit la seule chose que vous deviez savoir. Vous ne pouvez apprendre ce que c'est que de perdre la personne que vous aimez en lisant une histoire. C'est quelque chose que vous ne pourrez jamais comprendre sans l'avoir ressenti vous-même."

"Mon père," murmura Harry. C'était sa meilleure conjecture, la seule chose qui pourrait le sauver. "Mon père a essayé de vous protéger contre les brutes."

Un sourire funèbre s'étira sur le visage de Severus, et l'homme se déplaça vers Harry.

Et le dépassa.

"Au revoir, Potter," dit Severus sans se retourner. "Nous aurons bien peu à nous dire à partir d'aujourd'hui."

Et arrivé à l'angle, l'homme s'arrêta, et sans se retourner, parla une dernière fois.

"Votre père était la brute," dit Severus Rogue, "et ce que votre mère a vu en lui est quelque chose que je n'avais jamais compris avant aujourd'hui."

Il partit.

Harry se retourna et marcha vers la fenêtre. Ses mains tremblantes se posèrent sur le rebord.

\emph{Ne partagez jamais votre sagesse avec quiconque à moins de savoir exactement ce dont vous parlez tous les deux. Compris.} 

Harry regarda les nuages et la légère bruine pendant un moment. La fenêtre donnait sur la partie Est du domaine et on était l'après-midi ; si le soleil avait été visible à travers les nuages, Harry n'aurait donc pas pu le voir.

Ses mains cessèrent de trembler, mais il avait une sensation d'écrasement dans la poitrine, comme si elle avait été compressée par des courroies de métal.

Alors son père avait été une brute.

Et sa mère avait été superficielle.

Peut-être qu'ils avaient ensuite grandis. Des gens bien tels que le professeur McGonagall semblaient les porter aux nues, et ça pouvait ne pas être \emph{seulement}  dû au fait qu'ils avaient été d'héroïques martyrs.

Bien sûr, c'était une maigre consolation quand vous aviez onze ans et que vous étiez sur le point de devenir un adolescent et que vous vous demandiez quel genre d'adolescent vous pourriez devenir.

Si horrible.

Si triste.

Quelle terrible vie avait Harry.

Apprendre que ses parents génétiques n'avaient pas été parfaits, allons, il devrait passer un moment à se morfondre sur le sujet, à s'apitoyer sur son sort.

Peut-être qu'il devrait aller se plaindre à Lesath Lestrange.

Harry avait lu au sujet des Détraqueurs. Le froid et la ténèbre les entourait, et la peur, et ils aspiraient toutes vos pensées joyeuses, et en leur absence vos pires souvenirs remontaient à la surface.

Il pouvait s'imaginer à la place de Lesath, sachant que ses parents étaient à Azkaban à vie, cet endroit dont personne ne s'était jamais échappé.

Et Lesath s'imaginait à la place de sa mère, dans le froid et les ténèbres et la peur, seule avec ses pires souvenirs, même dans ses rêves, chaque seconde de chaque jour.

Pendant un moment, Harry imagina papa et maman à Azkaban avec les Détraqueurs aspirant leur vie, drainant les souvenirs heureux et leur amour pour lui. Juste un moment, avant que son imagination ne fasse brûler un fusible, n'appelle une extinction d'urgence et ne lui dise de ne plus jamais imaginer ça.

Était-il juste de faire cela à qui que ce soit, même à la deuxième personne la plus maléfique de la Terre ?

\emph{Non} , dit la sagesse des livres de Harry, \emph{pas s'il existe un autre moyen, n'importe quel autre moyen} .

Et à moins que le système judiciaire sorcier ne soit aussi parfait que ses prisons - et tout bien considéré, ça avait l'air assez improbable - quelque part à Azkaban se trouvait une personne qui était entièrement innocente, et probablement plus d'une personne.

Il y avait une sensation de brûlure dans la gorge de Harry et de l'humidité qui s'amoncelait dans ses yeux, et il voulut téléporter tous les prisonniers d'Azkaban en sécurité et invoquer le feu du ciel et faire exploser cet horrible endroit jusqu'à ses fondations. Mais il ne pouvait pas, parce qu'il n'était pas Dieu.

Et Harry se souvint de ce que le professeur Quirrell avait dit sous la lumière stellaire : \emph{Parfois, quand ce monde vicié me semble inhabituellement empli de haine, je me demande s'il y aurait un autre endroit, loin, où j'aurais dû vivre... Mais les étoiles sont tellement, tellement lointaines... Et je me demande ce que seraient mes rêves si je rêvais pendant longtemps, très longtemps.} 

Pour l'instant, ce monde vicié semblait inhabituellement empli de haine.

Et Harry ne pouvait pas comprendre les paroles du professeur Quirrell, elles auraient aussi bien pu être celles d'un alien ou d'une Intelligence Artificielle, celles de quelque chose qui aurait été construit selon une structure si différente de celle du cerveau de Harry que celui-ci n'aurait jamais pu être amené à fonctionner dans ce mode.

Vous ne pouviez pas quitter votre planète maternelle tant qu'elle contenait un lieu comme Azkaban.

Il fallait rester et se battre.

