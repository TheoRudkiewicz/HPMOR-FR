
\chapter{EPS, Secret et franchise, pt 11}

Ils tourbillonnèrent au travers les flammes vertes et virevoltèrent au travers du réseau de cheminées tandis que le cœur de Minerva battait au rythme d'une horreur qu'elle n'avait pas ressenti depuis dix ans et trois mois jusqu'à ce que les passages situés entre les plis de l'espace aient un soubresaut et les expulsent dans le hall de Gringotts (c'était l'arrivé de cheminée la plus sûre du Chemin de Traverse, la connexion la plus difficile à intercepter et la façon la plus rapide de quitter Poudlard sans phénix). Un employé gobelin se tourna vers eux, ses yeux s'écarquillèrent et il entama un salut respectueux -

\emph{Détermination, Destination, Délibération !} 

Et ils furent tous deux dans l'allée située derrière Chez Marie, baguettes déjà sorties et levées, déjà dos à dos, les paroles d'un sortilège anti-Désillusion déjà aux lèvres de Severus.

L'allée était vide.

Lorsqu'elle se retourna pour regarder Severus, la baguette de ce dernier s'abattait déjà sur sa propre tête dans un bruit semblable à celui un œuf qui se brisait et ses lèvres prononçaient déjà les mots de l'invisible ; il adopta les couleurs de son environnement et devint un amalgame flou de leur image, puis l'amalgame, correspondit parfaitement à ce qui se trouvait derrière lui et il n'y eut alors plus rien.

Elle abaissa sa baguette et fit un pas en avant afin de recevoir sa propre Désillusion -

Derrière elle, le son immanquable d'un feu qui naissait.

Elle pivota et vit Albus, sa longue baguette déjà dans sa main droite, prête. Ses yeux étaient sombres derrière les demi-cercles de ses lunettes et Fumseck, sur son épaule, avait ouvert ses ailes de feu, prêt à l'envol vol comme au combat.

"Albus !" dit-elle. "Je pensais -" Elle venait de le voir quitter Azkaban et elle pensait que même les phénix ne pouvaient en revenir si facilement.

Puis elle comprit.

"Elle s'est échappée," dit Albus. "Ton Patronus l'a-t-il trouvé ?"

Les battements de son cœur s'intensifièrent, l'horreur qui coulait dans ses veines se solidifia. "Il a dit qu'il était là, aux lavabos -"

"Espérons qu'il a dit vrai," dit Albus, puis la baguette frappa sa tête et lui procura la sensation d'une eau s'écoulant sur elle ; un instant plus tard ils fonçaient tous les quatre (même Fumseck avait été rendu invisible même si l'on pouvait parfois voir une lueur flamboyante là où il venait de passer) vers l'entrée du restaurant. Il s'interrompirent devant la porte pendant que Albus murmurait quelque chose et soudain l'un des clients visibles à travers la vitre se leva, un air absent sur le visage, puis ouvrit la porte comme pour jeter un rapide coup d'œil à l'extérieur à la recherche d'une connaissance ; et ils étaient entrés, dépassant les clients insouciants (Minerva savait que Severus gravait leur visage dans sa mémoire et qu'Albus saurait repérer toute Désillusion) en direction du panneau qui indiquait l'emplacement des lavabos -

Une vieille porte de bois marquée d'une pancarte s'ouvrit grand et quatre sauveurs invisibles prirent les toilettes d'assaut.

Le petite pièce était vide et l'évier témoignait d'une utilisation récentes mais aucun signe de Harry, seulement une feuille de papier laissée sur le couvercle refermé des toilettes.

Elle n'arrivait plus à respirer.

La feuille de papier s'éleva lorsqu'Albus s'en saisit, et un moment plus tard elle fut projetée vers Minerva.

\emph{M: Que le Choixpeau m'a-t-il dit de vous dire ?} 

\emph{- H} 

"Ah," dit Minerva sous le coup de la surprise, et son esprit mit un moment à remettre la question dans son contexte car même si ce n'était pas le genre de chose qu'on oubliait elle n'avait pas vraiment eu ce \emph{genre}  de choses en tête - "que suis une jeune impudente et que je devrais déguerpir de sa pelouse."

"\emph{Hein ?} " dit le vide de la voix d'Albus comme si même lui pouvait être surpris.

La tête de Harry Potter apparut alors, suspendue à côté des toilettes, le visage froid et alerte, le Harry trop adulte qu'elle avait parfois vu, ses yeux dardant d'un coin à l'autre de la pièce.

"Que se passe-" commença le garçon.

Albus, de nouveau visible, tout comme Fumseck et McGonagall, s'avança immédiatement, tendit la main gauche, arracha un cheveu de la tête de Harry (produisant un cri surpris chez le garçon) que Minerva accepta dans sa paume ; un instant plus tard Albus souleva le garçon en majeure partie invisible dans ses bras, il y eut un flash de feu rouge et or...

Et Harry Potter fut en sécurité.

Minerva fit quelques pas en avant, s'appuya contre le mur proche de là où Albus et Harry s'étaient trouvé et essaya de retrouver son calme.

Elle avait... perdu quelques habitudes pendant les dix ans qui s'étaient écoulés depuis le démantèlement de l'Ordre du Phénix.

À côté, Severus redevint visible dans un chatoiement. Sa main droite extrayait déjà la flasque de ses robes et la gauche se tendait d'un geste quémandeur. Elle lui donna le cheveu de Harry et il le lâcha un instant après dans la flasque de Polynectar inachevé qui commença immédiatement à crépiter et à buller à mesure qu'il acquérait le pouvoir qui permettrait à Severus de jouer son rôle d'appât.

"C'était inattendu," dit lentement le maître des potions. "Je me demande pourquoi notre directeur n'a pas récupéré M. Potter \emph{plus tôt}  s'il était prêt à aller jusqu'à tordre le temps ? Il n'y aurait rien dû y avoir pour l'empêcher de le faire... de fait, votre Patronus aurait dû le rencontrer alors qu'il était déjà en sécurité..."

Elle n'y avait pas pensé car autre chose venait de prendre le pas sur toutes ses autres pensées. C'était loin d'être aussi horrible que l'évasion de Bellatrix Black mais tout de même -

"Harry a une \emph{cape d'invisibilité ?} " dit-elle.

Le maître de Potion ne répondit pas : il rétrécissait.
\par\noindent\rule{\textwidth}{0.4pt}
Tic-clic, plic-blip, ding-zing-ting-

bien que cela finisse toujours par passer en-dessous de son seuil de perception, l'énervement n'en demeurait pas moins : elle comptait tous les faire taire le jour où elle deviendrait directrice ; si elle devenait directrice. Elle se demandait quel directeur de Poudlard avait le premier été suffisamment sans gêne pour créer un appareil qui faisait du \emph{bruit}  afin de le transmettre ses successeurs.

Elle était assise dans le bureau du directeur munie de son propre bureau, rapidement métamorphosé, et travaillait sur les mille petites tâches de paperasserie nécessaires à ce que Poudlard ne s'arrête pas brusquement de fonctionner ; elle pouvait s'y perdre facilement et cela l'empêchait de penser à autre chose. Comme Albus l'avait un jour fait remarqué d'un ton plutôt narquois, Poudlard semblait bien mieux fonctionner lorsqu'il y avait une crise extérieure à laquelle elle devait éviter penser...

...il y a dix ans, c'était la dernière fois qu'Albus avait dit cela.

Le carillon indiqua l'approche d'un visiteur.

Minerva continua de lire son parchemin.

La porte s'ouvrit avec fracas, révélant Severus Rogue, qui s'avança de trois pas et demanda son interrompre son mouvement : "Des nouvelles de Fol-Oeil ?"

Albus se levait déjà de sa chaise alors qu'elle mettait ses parchemins de côté et dissipait le bureau. "Le Patronus de Maugrey m'apporte ses rapports d'Azkaban," dit Albus. "Son Œil n'a rien vu ; et si l'Œil de Vance ne voit pas quelque chose, alors cette chose n'existe pas. Et toi ?"

"Personne n'a essayé de me prendre mon sang de force," dit Severus. Il eut une rapide grimace en guise de sourire. "À part le professeur de Défense."

"\emph{Quoi ?} " dit Minerva.

"Il a repéré mon imposture avant même que je puisse ouvrir les lèvres et m'a immédiatement attaqué, ce qui était assez raisonnable, tout en exigeant de savoir où se trouvait M. Potter." Une autre grimace-sourire. "Pour une raison ou une autre, hurler que j'étais Severus Rogue ne sembla pas le rassurer. Je suis persuadé que cet homme me tuerait pour une Mornille et rendrait cinq Noises de monnaie. J'ai du étourdir notre bon professeur Quirrell, ce qui ne fut pas simple, et il a mal réagit au maléfice. 'Harry Potter', naturellement effrayé, a couru prévenir le patron, et le professeur de Défense a été transporté à Ste Mangouste -"

"\emph{Ste Mangouste ?} " dit Minerva.

"- où on l'a trouvé dans un état d'épuisement tel que l'on m'a dit qu'il s'était probablement surmené pendant plusieurs semaines avant de s'effondrer. Votre précieux professeur de Défense va bien, Minerva, le sortilège d'étourdissement l'a peut-être aidé en le forçant à prendre quelques jours de repos. Après avoir décliné l'offre d'une cheminée vers Poudlard, je suis retourné au Chemin de Traverse, où j'ai erré ; mais personne n'a semblé désireux d'obtenir du sang de M. Potter."

"Notre professeur de Défense est entre de bonnes mains, j'en suis certain," dit Albus. "Minerva, des affaires plus importantes requièrent ton attention."

Elle lui fallut faire un effort considérable pour arracher son attention à ce problème mais elle se rassit, Severus fit venir une chaise à lui d'un geste, et ils commencèrent leur conseil.

Entre ces deux-là, elle avait l'impression d'être un imposteur sous Polynectar. La guerre et l'intrigue n'étaient certainement pas de son ressort. Il lui fallait faire des efforts pour garder un temps d'avance sur les jumeaux Weasley, et lui arrivait parfois même d'échouer à cela. En fin de compte, elle n'était assise là que parce qu'elle avait entendu la prophétie...

"Nous faisons face," commença le directeur, "à un mystère plutôt alarmant. Je ne peux songer qu'à deux sorcier capables de manigancer cette évasion."

Minerva inspira brusquement. "Il y a une chance que ce ne soit \emph{pas } Vous-Savez-Qui ?"

"J'en ai peur," dit le directeur.

Elle jeta un regard de côté et vit que Severus était aussi perplexe qu'elle. \emph{Peur}  qu'il ne s'agisse pas du retour du Seigneur des Ténèbres ? Elle aurait presque tout donné pour que ce soit le cas.

"Donc," dit Albus d'une voix grave. "Notre premier suspect est Voldemort, de nouveau parmi nous, cherchant à se faire revivre. J'ai étudié de nombreux livre que j'aimerais ne pas avoir lus à la recherche de tous les moyens par lesquels il pourrait revenir et je n'en ai trouvé que trois. Sa voie la plus sûre est la Pierre Philosophale, dont Flamel m'assure que même Voldemort ne pourrait la créer seul ; en suivant cette voie il reviendrait plus fort et plus terrible que jamais auparavant. Je n'aurais pas cru Voldemort capable de résister à la tentation de la Pierre, d'autant plus qu'elle est un piège évident, un défi à son intelligence. Mais sa deuxième option est presque aussi prometteuse : la chair de son serviteur, délibérément offerte ; le sang de son ennemi, pris de force ; et l'os de son ancêtre, involontairement légué. Voldemort est un perfectionniste -" Albus jeta un coup d'œil à Severus qui acquiesça d'un hochement de tête, "- et il cherchera certainement la combinaison la plus puissante : la chair de Bellatrix, le sang de Harry Potter et les ossements de son père. Sa dernière possibilité est de séduire une victime et de drainer sa vie sur une longue période, auquel cas il serait faible comparé à son pouvoir d'antan. La raison pour laquelle il a fait disparaître Bellatrix est claire. Et s'il la garde en réserve uniquement pour le cas où il ne pourrait atteindre la Pierre, cela expliquerait pourquoi personne n'a essayé d'enlever Harry aujourd'hui."

Minerva jeta un nouveau regard vers Severus et vit qu'il écoutait attentivement mais qu'il n'était pas surpris.

"Ce qui n'est \emph{pas}  clair," continua le directeur, "c'est la \emph{façon}  dont Voldemort aurait pu manigancer cette évasion. Une poupée de mort a été laissée à la place de Bellatrix, son évasion était donc censée rester inaperçue ; et même si cela a mal tourné, les Détraqueurs ne sont plus parvenus à la trouver après nous avoir prévenus. Azkaban est restée impénétrable pendant des siècles, et je ne peux imaginer par quel moyen Voldemort a accompli cette évasion."

"Cela n'a peut-être que peu d'importance," dit Severus, son visage parfaitement neutre. "Pour accomplir ce que nous sommes incapables d'imaginer, il suffit au Seigneur des Ténèbres d'avoir une meilleure imagination que nous."

Albus acquiesça d'un air lugubre. "Il existe malheureusement un autre sorcier qui se rit de l'impossible. Un sorcier qui, il y a peu, a développé un nouveau et puissant sortilège qui aurait pu rendre les Détraqueurs aveugles à l'évasion de Bellatrix Black. Et il est impliqué lui aussi, pour d'autres raisons."

Le cœur de Minerva manqua quelques battements. Elle ne savait pas \emph{comment}  ni \emph{pourquoi}  mais une appréhension terrible était en train de naître en elle quant à l'\emph{identité}  de cette personne -

"Et de \emph{qui}  s'agirait-il ?" dit Severus, visiblement perplexe.

Albus se pencha et dit les mots funestes qu'elle avait redoutés : "Harry James Potter-Evans-Verres."

"\emph{Potter ?} " demanda le maître des potions, et sa voix d'habitude soyeuse était plus choquée que Minerva ne l'avait jamais entendue être. "Monsieur le directeur, s'agit-il de l'une de vos plaisanteries ? Il est en première année à Poudlard ! Un accès de colère et quelques farces puériles aidées d'une cape d'invisibilité ne font pas de lui -"

"Ce n'est pas une blague," dit Minerva, d'une voix à peine au-dessus du chuchotement. "Harry fait déjà des découvertes novatrices en Métamorphose, Severus. Mais je ne savais pas qu'il faisait aussi des recherches en sortilèges."

"Harry n'est pas un élève de première année ordinaire," dit le directeur d'un ton solennel. "Il est marqué comme l'égal du Seigneur des Ténèbres et possède un pouvoir que le Seigneur des Ténèbres ignore."

Severus la regardait, et il aurait fallu bien le connaître pour voir que c'était un regard suppliant. "Dois-je prendre ceci au sérieux ?"

Minerva se contenta de hocher la tête.

"Quelqu'un d'\emph{autre}  est-il au courant de l'existence de ce... sortilège nouveau et puissant ?" demanda Severus.

Le directeur regarda Minerva d'un air contrit -

Elle le sut, elle le sut avant même qu'il ne le dise et voulut crier à s'en faire brûler les poumons.

- et dit : "Quirinus Quirrell."

"\emph{Pourquoi} ," dit-elle d'une voix qui aurait dû faire fondre la moitié des appareils du bureau, "\emph{M. Potter a-t-il ne serait-ce que PARLÉ de son brillant sortilège d'évasion de prison à notre professeur de Défense -} "

Le directeur passa une main fatiguée et ridée sur son front qui était tout aussi ridé. "Quirinus s'est juste trouvé être présent, Minerva. Même moi je n'y ai vu aucun mal sur le moment." Le directeur hésita. "Et Harry a dit que son sortilège était trop dangereux pour être expliqué à lui ou à moi ; et lorsque je l'ai de nouveau interrogé aujourd'hui, il a insisté sur le fait qu'il ne l'avait toujours pas expliqué à Quirinus et qu'il n'avait pas non plus laissé tomber ses barrières Occlumantiques en présence du professeur de Défense -"

"M. Potter est un \emph{Occlumens}  ? Tu lui as donné une cape d'invisibilité \emph{et}  il est immunisé contre le Veritaserum \emph{et}  il est\emph{ ami avec les jumeaux Weasley ?}  Albus, as-tu la moindre idée de ce que tu as déchaîné sur cette école ?" sa voix était presque devenue un cri perçant. "Lorsqu'il sera arrivé en septième année il ne restera rien de Poudlard hormis un trou fumant !"

Albus se pencha dans son grand fauteuil rembourré et dit en souriant : "N'oublie pas le Retourneur de Temps."

Elle cria alors, mais doucement.

Severus enchaîna d'une voix traînant : "Devrais-je lui apprendre à mijoter du Polynectar ? Je pose la question par pur souci de complétude, au cas où vous ne seriez pas satisfait par l'ampleur de votre désastre de compagnie."

"Peut-être l'année prochaine," dit Albus. "Mes très chers amis, la question est de savoir si Harry Potter a fait disparaître Bellatrix Black d'Azkaban, ce qui, même à l'aune de mon jugement tolérant, dépasse les bornes de l'entrain de la jeunesse."

"Excusez-moi, monsieur le directeur," dit Severus de l'un des sourires les plus secs qu'elle l'avait jamais vu donner à Albus, "mais je ferai savoir mon opinion : la réponse est non. Cela est purement et simplement l'œuvre du Seigneur des Ténèbres."

"Alors comment se fait-il," dit Albus, et tout humour avait soudain quitté sa voix, "que lorsque j'ai essayé de récupérer Harry immédiatement après son arrivée au Chemin de Traverse, j'ai découvert que cela produirait un paradoxe ?"

Minerva s'enfonça encore plus dans sa chaise, laissa son coude gauche tomber sur le dur accoudoir, fit reposer sa tête sur sa main et ferma les yeux de désespoir.

Il y avait un proverbe qui circulait dans des cercles fermés selon lequel seul un Auror sur trente était qualifié pour enquêter sur des cas impliquant des Retourneurs de Temps ; et parmi cette poignée, la moitié qui n'était pas \emph{déjà}  cinglée le serait bientôt.

"Vous soupçonnez donc," disait la voix de Severus, "que Potter est allé du Chemin de Traverse à Azkaban, puis qu'il a ensuite fait une boucle retour jusqu'au Chemin de Traverse où nous l'avons récupéré -"

"Précisément," dit la voix d'Albus. "Bien qu'il soit aussi possible que Voldemort ou ses serviteurs aient été présents afin de s'assurer que Harry était bien arrivé au Chemin de Traverse avant de commencer leur tentative d'évasion à Azkaban. Et qu'ils avaient quelqu'un disposant d'un Retourneur de Temps qui transmettrait le message de leur réussite en arrière dans le temps pour déclencher l'enlèvement. C'était de fait mon soupçon quant à cette éventualité qui m'a poussé à vous envoyer avec Minerva, seuls, avant que je ne me rende à Azkaban. Je pensais alors que leur évasion échouerait, mais si récupérer Harry Potter impliquait l'observation de leur échec final je n'aurais alors pas pu me rendre à Azkaban après avoir interagi avec lui puisque le futur d'Azkaban ne peut pas entrer en contact avec son passé. Lorsque, à Azkaban, je n'ai reçu aucun message ni de toi ni de Minerva ni de Flitwick à qui j'avais dit d'essayer de vous contacter, j'ai su que votre interaction avec Harry Potter avait constitué une interaction avec le futur d'Azkaban, ce qui voulait dire que quelqu'un envoyait des messages dans le Temps -"

La voix d'Albus s'interrompit.

"Mais, monsieur le directeur," dit Severus, "\emph{vous}  êtes revenu du futur d'Azkaban et avez interagi avec nous..."

La voix du maître des potions resta en suspens.

"Mais Severus, si j'avais \emph{reçu}  des message de toi ou de Minerva disant que Harry était en sécurité, je n'aurais pas en premier lieu remonté le temps pour -"

"Monsieur le directeur, je pense que nous devrions dessiner des diagrammes."

"Je suis d'accord, Severus."

Il y eut le bruit d'un parchemin étendu sur une table, puis de plumes qui le grattaient, puis d'autres débats.

Minerva était assise dans sa chaise, la tête dans sa main, les yeux clos.

Elle avait un jour entendu cette histoire au sujet d'un criminel qui était entré en possession d'un Retourneur de Temps parce que le Département des mystères avait pris l'exécrable décision de lui en confier un ; un Auror avait reçu pour tâche de suivre la piste de ce criminel du temps anonyme et avait lui-même reçu un Retourneur de Temps ; et l'histoire se terminait alors qu'ils étaient tous deux dans l'aile de Ste Mangouste réservée au Cinglés Irrécupérables.

Minerva resta assise là les yeux clos et s'efforça de ne pas écouter ce qu'ils disaient, de ne pas y penser, de ne pas devenir folle.

Au bout d'un moment, les débats semblèrent s'être taris et elle dit : "Le Retourneur de Temps de M. Potter ne peut opérer qu'entre neuf heures du soir et minuit. Albus, a-t-on touché à la coque protectrice ?"

"Pas selon mes sortilèges les plus raffinés," répondit-il. "Mais les coques protectrices sont récentes ; se jouer des précautions des Langues-de-Plomb et n'en laisser aucune trace n'est... peut-être \emph{pas}  impossible."

Elle ouvrit les yeux et vit que Severus et le directeur regardaient fixement un parchemin couvert de pattes de mouches entortillées qui l'auraient certainement rendue folle si elle avait essayé de les comprendre.

"En êtes-vous arrivés à la moindre \emph{conclusion}  ?" dit-elle. "Et s'il vous plaît, ne me dites pas comment vous y êtes arrivés."

Severus et le directeur se regardèrent l'un l'autre puis se tournèrent vers elle.

"Nous avons conclu," dit le directeur avec le plus grand sérieux, "que soit Harry est impliqué, soit il ne l'est pas ; que soit Voldemort a accès à un Retourneur de Temps, soit il n'y a pas accès ; et qu'indépendamment de ce qui a eu lieu dans l'enceinte d'Azkaban, personne ne pourrait s'être rendu au cimetière de Little Hangleton pendant la période où Maugrey l'a déjà regardé depuis mon propre passé."

"Plus brièvement," dit Severus d'une voix traînante, "nous ne savons rien, chère Minerva ; même s'il semble au minimum \emph{probable}  qu'un autre Retourneur de Temps est impliqué dans l'affaire d'une façon ou d'une autre. En ce qui me concerne, je soupçonne Potter d'avoir été soudoyé, trompé ou menacé dans le but de lui faire transporter des messages dans le passé, peut-être même au sujet de cette évasion. Je n'énoncerai pas l'évidence quant à qui pourrait tirer ses ficelles. Mais je suggère qu'à neuf heures ce soir nous voyons si Potter est capable de revenir de six heures en arrière, jusqu'à trois heures de l'après-midi, afin de déterminer s'il n'a pas encore utilisé son Retourneur de Temps.

"Cela semble être sage de toute façon," dit Dumbledore. "Assure-toi que ce soit fait, Minerva, et dis à Harry de passer à mon bureau quand cela lui conviendra."

"Mais soupçonne-tu toujours Harry d'avoir été directement impliqué dans l'évasion elle-même ?" répondit-elle.

"Possible mais peu probable," dit Severus à l'instant même où Albus disait : "Oui."

Minerva pinça l'arête de son nez, inspira profondément et laissa l'air sortir. "Albus, Severus, quelle \emph{raison}  M. Potter pourrait-il avoir de faire une chose pareille !"

"Je ne peux en concevoir aucune," dit Albus, "mais il demeure que seule la magie de Harry, de tous les moyens que je connaisse, aurait pu -"

"Attendez," dit Severus. Son visage s'était vidé de toute expression. "Une idée me vient, je dois vérifier -" Le maître des potions s'empara d'une pincée de cheminette et traversa la pièce d'un pas vif en direction du feu - vers lequel Albus agita hâtivement sa baguette afin de l'allumer - et Severus disparut dans un grand éclat de feu vert au son des mots "Serpentard, bureau du directeur".

Elle et Albus se regardèrent et haussèrent les épaules ; puis Albus retourna à son étude du parchemin.

Ce n'est que quelques minutes plus tard que Severus surgit hors de la cheminée en époussetant des traces de cendres de ses vêtements.

"Bien," dit le maître des potions. De nouveau le visage vide de tout expression. "J'ai peur que M. Potter ait un motif."

"Parle !" dit Albus.

"J'ai trouvé Lesath Lestrange en train d'étudier dans la salle commune de Serpentard," dit Severus. "Il n'a pas hésité à croiser mon regard. Et il semble que M. Lestrange n'apprécie pas l'idée que ses parents soient à Azkaban dans le froid et dans les ténèbres tandis que les Détraqueurs drainent leur vie et les font souffrir chaque seconde de chaque jour, et il l'a dit à M. Potter en des termes aussi éloquents avant de le supplier de les faire sortir. Car voyez-vous, M. Lestrange a entendu dire que le Survivant est capable de tout."

Albus et elle échangèrent un regard.

"Severus," dit Minerva, "même Harry... a \emph{certainement} ... plus de bons sens que \emph{ça} ..."

Sa voix resta en suspens.

"M. Potter pense qu'il est Dieu," dit Severus, le visage parfaitement neutre, "et Lesath Lestrange est tombé à genoux devant lui en le priant de tout son cœur."

Minerva regarda Severus. Elle se sentit malade. Elle avait étudié la religion Moldue - c'était la raison la plus courante pour laquelle on devait effacer la mémoire des parents de nés-Moldus - et elle en savait assez pour comprendre ce que Severus venait de dire.

"Quoi qu'il en soit," dit le maître des potions. "J'ai regardé en M. Lestrange pour savoir s'il était au courant de l'évasion de sa mère. Il n'en a pas entendu parler. Mais à l'instant où il l'apprendra, il en conclura que le responsable est Harry Potter."

"Je vois..." dit lentement Albus. "Merci, Severus. C'est une bonne nouvelle."

"\emph{Une bonne nouvelle}  ?" éclata Minerva.

Albus la regarda d'une expression à présent aussi neutre que celle de Severus ; et elle se souvint brusquement qu'Albus aussi avait - "C'est la meilleure raison possible pour laquelle on pourrait vouloir faire sortir Bellatrix d'Azkaban," dit ce dernier d'une voix basse. "Et si ce n'est pas \emph{Harry} , rappelons-nous que c'est alors certainement Voldemort lui-même en train de jouer son premier coup. Mais ne jugeons pas hâtivement alors que nous ignorons tant et sommes sur le point d'en apprendre beaucoup."

Il se leva une fois de plus, alla jusqu'à la cheminée qui était toujours allumée et jeta une autre pincée de poudre verte avant de passer sa tête dans les flammes. "Département de la justice magique," dit-il, "bureau du directeur."

Au bout d'un moment, on put distinctement entendre la voix abrupte de madame Bones : "De quoi s'agit-il, Albus ? Je suis plutôt occupée."

"Amélia," dit Albus, "je te prie de me faire part de tout ce que tu as découvert sur cette affaire."

Il y eut un moment de silence. "Oh," dit la voix froide de madame Bones à travers le feu brûlant, "et j'imagine que ce sera réciproque ?"

"Cela se pourrait," dit Albus d'une voix calme.

"Si un Auror meurt à cause de tes réticences, vieux fouineur, je t'en tiendrai pour entièrement responsable."

"Je comprends, Amélia," répondit-il, "mais je ne souhaite certainement pas être une source d'incrédulité ou de panique inutile -"

"\emph{Bellatrix Black}  s'est échappée d'\emph{Azkaban}  ! Quel genre de panique ou d'incrédulité penses-tu que je vais qualifier \emph{d'inutile}  dans une situation pareille ?"

"Je te demanderai peut-être de te souvenir de ces mots," dit le vieux sorcier à travers les flammes vertes. "Car si j'apprends que mes peurs ne sont pas sans fondement, alors tu les \emph{sauras} . Et maintenant Amélia, je t'en prie, si tu as appris quoi que ce soit, dis-le moi."

Il y eut un autre silence, puis la voix de Madame Bones dit : "j'ai une information qui me vient de quatre heures dans le futur, Albus. En veux-tu toujours ?"

Albus s'interrompit -

(soupesant, Minerva le savait, l'éventualité qu'il puisse vouloir revenir plus de deux heures dans le passé ; car on ne pouvait envoyer des informations plus de six heures dans le passé, pas même en utilisant plusieurs Retourneurs de Temps).

- et dit enfin : "Oui, s'il te plaît."

"Nous avons eu de la chance," dit la voix de madame Bones, "l'un des Aurors témoins de l'évasion est un Moldu et il nous a dit que le sortilège de feu volant, comme nous l'avons appelé, n'est peut-être pas un sort du tout mais un produit de l'artisanat Moldu."

Ce fut comme un coup de poing dans l'estomac et la nausée de Minerva redoubla. Quiconque avait observé la Légion du Chaos savait de qui cela révélait la marque...

La voix de madame Bones continua : "Nous avons fait venir Arthur Weasley du Service des détournements de l'artisanat Moldu - il en sait plus sur les fabrications Moldues que tout autre sorcier - nous lui avons donné les descriptions que les Aurors avaient fait de la scène et il percé ce mystère. C'est un produit de l'artisanat moldu appelé une fouliée, et ils appellent ça comme ça parce qu'il faudrait être fou à lier pour monter à bord d'une chose pareille. Il n'y a pas six ans, l'une de leurs fouliées a explosé, ce qui a tué des centaines de Moldus en l'espace d'un instant et a failli mettre le feu à la Lune. Weasley dit que les fouliées utilisent un type de science particulier appelé réaction opposée, donc le plan est de développer un sort qui empêchera ce type de science de fonctionner près d'Azkaban."

"Merci Amélia," dit gravement Albus. "Est-ce tout ?"

"Je vais vérifier si nous n'avons rien de six heures plus tard," dit la voix de madame Bones, "on ne me l'aurait pas dit même si c'était le cas, mais je vais ordonner qu'on te le dise. Y-a-t-il quelque chose que \emph{tu}  voudrais me dire, Albus ? Laquelle des deux possibilités est la plus probable ?"

"Pas encore, Amélia," dit Albus, "mais j'aurais peut-être quelque chose à te dire très bientôt."

Il s'écarta alors du feu, qui redevint des flammes jaunes ordinaires. Chaque minute de la vie du vieux sorcier, chaque seconde depuis sa naissance, chaque seconde ajoutée par le Retourneur de Temps, tout cela et quelques décennies de plus causées par le stress étaient visibles sur son visage ridé.

"Severus ?" dit le vieux sorcier. "De quoi s'agissait-il vraiment ?"

"Une fusée," dit le maître des potions au sang mêlé qui avait grandi dans la ville Moldue appelée l'Impasse du Tisseur. "Une technologie Moldue des plus impressionnantes."

"Est-il probable que \emph{Harry}  connaisse un tel art ?"

Severus répondit d'une voix traînante : "Oh, un garçon tel que M. Potter saura \emph{tout}  sur les fusées ; cela est certain, chère Minerva. Il faut vous souvenir que les choses sont différentes dans le monde Moldu." Il fronça les sourcils. "Mais les fusées \emph{sont}  dangereuses et chères..."

"Harry a volé et caché une fraction inconnue de l'argent placé dans sa chambre forte à Gringotts, peut-être des milliers de Gallions," dit le directeur, puis, à ses comparses : "tel n'était \emph{pas}  mon plan, mais j'ai commis l'erreur d'envoyer le professeur de Défense superviser Harry lorsqu'il eut à retirer cinq Gallions afin d'acheter des cadeaux de Noël..." il haussa les épaules. "Oui, j'admets que rétrospectivement, c'était de la folie. Continuons."

Minerva se cogna doucement la tête plusieurs fois contre l'appuie-tête de sa chaise.

"Cela dit, monsieur le directeur," dit Severus. "Ce n'est pas parce que les Mangemorts n'ont pas utilisé d'engin Moldu pendant la première guerre qu'\emph{il}  ignore leur existence. Des fusées transformées en armes sont tombées sur l'Angleterre côté Moldu pendant la guerre de Grindelwald. Si, comme vous nous l'avez dit, il a passé les étés de ces années dans un orphelinat Moldu... alors il connaît lui aussi les fusées Moldues. Et s'il a écouté les comptes-rendus de l'utilisation que M. Potter fait des engins Moldus pendant les simulations de combat, il a certainement appris à utiliser la force de son ennemi et à les améliorer lui-même. Il pense \emph{exactement}  comme ça ; tout pouvoir sur lequel il pose les yeux devient l'objet de son désir."

Le vieux sorcier était immobile, parfaitement figé, même les cheveux de sa barbes s'étaient solidifiés, tels des fils de pierre ; et l'idée vint à Minerva, la plus effrayante qu'elle ait jamais eu, que Albus Dumbledore avait été pétrifié par l'horreur.

"Severus," dit-il, et sa voix failli se briser, "te rends-tu compte de ce que tu dis ? Si Harry Potter et Voldemort combattent au moyen d'armes Moldues, le monde ne sera plus qu'un champ de flammes !"

"\emph{Quoi ?} " dit Minerva. Bien sûr, elle avait entendu parler des armes à feu, mais elles n'étaient pas \emph{si}  dangereuses que ça pour une sorcière expérimentée -

Severus parla comme si elle n'avait pas été là. "Alors peut-être envoie-t-il délibérément cet avertissement à Harry Potter, disant ainsi que toute attaque par une arme Moldue provoquera des représailles du même type. Ordonnez à M. Potter de cesser d'utiliser la technologie Moldue en bataille ; cela montrera que le message est reçu... et le privera d'une source d'inspiration." Severus fronça les sourcils. "Quoique, maintenant que j'y pense, M. Malfoy - et bien sûr Mlle Granger - eh bien, à la réflexion un interdit général sur la technologie semblerait plus sage -"

Le vieux sorcier se pressa les mains sur le front et une voix chancelante s'échappa de ses lèvres : "Je commence à \emph{espérer}  que Harry est derrière cette évasion... que Merlin nous protège, qu'ais-je fait, qu'ais-je fait, que va-t-il advenir du monde ?"

Severus haussa les épaules. "À en croire les rumeurs, les armes Moldues ne sont que légèrement pires que les aspects les plus... \emph{ésotériques}  de la magie -"

"\emph{Pires ?} " s'étrangla Minerva, puis elle ferma la bouche comme si on l'y avait forcée.

"Pire que tout autre danger, en ces temps incertains," dit Albus. "Mais pas pire que ce qui a fait disparaître Atlantis du Temps."

Minerva le regardait fixement et sentait que de la sueur commençait à perler le long de sa colonne vertébrale.

Severus continua de s'adresser à Albus : "Tous les Mangemorts à part Bellatrix l'auraient trahi, tous ses sympathisants se seraient retournés contre lui et tous les pouvoirs du monde auraient convergé afin de le détruire s'il s'était montré imprudent dans son maniement d'une source de puissance dangereuse. La situation actuelle est-elle si différente ?"

Un peu de vie et un peu de couleur étaient revenus sur le visage du vieux sorcier. "Peut-être pas..."

"Et quoi qu'il en soit," dit Severus avec un sourire légèrement condescendant, "les armes Moldues ne sont pas si faciles que ça à obtenir, ni pour mille Gallions ni pour un millier de millier de Gallions."

\emph{Harry ne métamorphose-t-il pas simplement les appareils qu'il utilise en bataille ?}  pensa Minerva, mais avant qu'elle puisse ouvrir la bouche pour poser la question -

La cheminée cracha des flammes vertes et entre elles apparut le visage de l'assistant de madame Bones, Pius Thicknesse. "Monsieur le président sorcier ?" dit-il. "J'ai un rapport à vous rendre, transmis -" les yeux de Thicknesse passèrent sur Minerva et Severus, "il y a six minutes."

"Vous voulez dire dans six heures," dit Albus. "Ces deux-là peuvent l'entendre ; donnez-moi votre rapport."

"Nous savons comment ils ont fait," dit Thicknesse. "Dans la cellule de Bellatrix Black, cachée dans un coin, se trouvait une fiole vide ; et des tests sur le fluide encore restant ont révélé qu'il s'agissait d'une potion d'Animagus."

Il y eut un long silence.

"Je vois..." dit Albus avec difficulté.

"Pardon ?" dit Minerva. Elle ne comprenait pas.

La tête de Thicknesse se tourna vers elle. "Madame McGonagall, les Animagus sous forme animale intéressent moins les Détraqueurs. Tous les prisonniers sont testés avant leur arrivée à Azkaban et s'il sont Animagus alors leur forme animale est détruite. Mais nous n'avions pas envisagé la possibilité que l'un d'eux puisse devenir un Animagus \emph{après}  être entré à Azkaban en buvant la potion et en méditant sous la protection d'un Patronus -"

"J'ai cru comprendre," dit Severus, ayant de nouveau adopté son ton hautain habituel, "que les méditations d'Animagus requéraient un temps considérable."

"Eh bien, M. Rogue," aboya Thicknesse, "nos archives montrent que Bellatrix Black était un Animagus \emph{avant}  d'être envoyée à Azkaban et que sa forme animale soit détruite ; alors peut-être que sa \emph{seconde}  méditation n'a pas pris autant de temps que la première !"

"Je n'aurais pas cru qu'un prisonnier d'Azkaban soit capable d'une telle chose..." dit Albus."Mais Bellatrix Black était une sorcière des plus puissantes avant son incarcération, et si certaines personnes le peuvent, Bellatrix est l'une d'elles. Azkaban peut-elle être protégée contre cette méthode ?"

"Oui," dit la tête très sûre d'elle de Pius Thicknesse. "Nos experts disent qu'il est quasiment inimaginable qu'une méditation d'Animagus puisse être terminée en moins de trois heures, avec ou sans expérience préalable. Toutes les visites des prisonniers qui y ont droit seront dorénavant limitées à deux heures et les Détraqueurs nous informeront si un Patronus est maintenu dans l'enceinte de la prison pendant plus longtemps."

Albus sembla attristé en entendant cela mais il hocha la tête. "Je vois. Il n'y aura pas d'autres tentatives de la sorte, bien sûr, mais ne relâchez pas votre vigilance. Et lorsqu'Amélia aura entendu tout ceci, dites-lui que j'ai des informations pour elle."

La tête de Pius Thicknesse disparut sans ajouter un mot.

"Pas d'autres tentatives...?" dit Minerva.

"Parce que, chère Minerva," dit Severus de sa voix traînante, pas tout à fait débarrassé de son ton hautain usuel, "si le Seigneur des Ténèbres avait prévu de libérer un autre des ses serviteurs d'Azkaban, il n'aurait pas laissé derrière lui une fiole révélant comment il a procédé." Severus fronça les sourcils. "J'avoue... que je ne comprends quand même pas pourquoi la fiole a été laissée là."

"C'est une sorte de message..." dit lentement Albus. "Et je vois pas ce qu'il signifie, pas du tout...". Il battit des doigts sur son bureau.

Pendant une longue minute, ou peut-être trois, le vieux sorcier regarda dans le vide, les sourcils froncés, tandis que Severus restait lui aussi assis en silence.

Puis Albus secoua la tête en signe de consternation et dit : "Severus, comprends-\emph{tu } ceci ?"

"Non," dit le maître des potions, puis avec un sourire sardonique : "ce qui vaut probablement mieux pour nous tous ; quoi que nous ayons été \emph{censés}  en déduire, cette partie du plan a échouée."

"Vous êtes maintenant certains que \emph{c'est}  Vous-Savez... que c'est Voldemort ?" dit Minerva. "Un autre Mangemort ne pourrait pas avoir trouvé cette idée ingénieuse ?"

"Et qui saurait aussi ce qu'est une fusée ?" dit sèchement Severus. "Je ne crois pas que les autres Mangemorts étaient aussi férus d'études des Moldus. C'est lui."

"Je suis d'accord, c'est lui," dit Albus. "Azkaban est demeurée invaincue pendant des siècles pour tomber face à une potion Animagus ordinaire. C'est trop malin et trop impossible, ce qui a toujours été la marque de Voldemort depuis l'époque où il était connu sous le nom de Tom Riddle. Celui qui voudrait falsifier cette signature devrait être aussi rusé que Voldemort lui-même. Et il n'y a personne d'autres dans le monde qui surestimerait mon intelligence par erreur et me laisserait un message que je ne peux absolument pas comprendre."

"À moins qu'il ne vous ait parfaitement estimé," dit Severus d'une voix sans timbre, "auquel cas c'est exactement ce qu'il voulait que vous pensiez."

Albus soupira. "En effet. Mais même s'il m'a parfaitement trompé, nous pouvons au moins en conclure avec certitude que ce n'était pas Harry Potter."

Cela aurait dû être un soulagement et pourtant Minerva sentit le froid se répandre à travers sa colonne vertébrale et ses veines, à travers ses poumons et ses os.

Elle se souvenait de conversations semblables.

Elle se souvenait de conversations semblables il y a dix ans, à une époque où le sang coulait tant à travers l'Angleterre qu'il formait des fleuves, où les sorciers et les sorcières qu'elle avait un jour eu pour élèves se faisaient massacrer par centaines ; elle se souvenait de maisons qui brûlaient, d'enfants qui hurlaient, d'éclats de lumière verte -

"Que direz-vous à madame Bones ?" chuchota-t-elle.

Albus se leva et marcha jusqu'au centre de la pièce, sa main effleurant les appareils, ici un instrument lumineux, là un autre sonore ; il réajusta ses lunettes d'une main et utilisa l'autre pour recentrer sa longue barbe d'argent sur ses robes, puis le vieux sorcier se retourna et leur fit face.

"Je lui dirai le peu que je sais de l'Art Noir nommé horcrux par lequel une âme est privée de la mort," dit Albus Dumbledore d'une douce voix qui semblait emplir toute la pièce, "et je leur dirai ce qui peut être fait avec la chair du serviteur."

"Je lui dirai que je reconstitue l'Ordre du Phénix."

"Je lui dirai que Voldemort est de retour."

"Et que la seconde guerre des sorciers a commencée."
\par\noindent\rule{\textwidth}{0.4pt}
\emph{Quelques heures plus tard...} 

L'horloge antique placée contre le mur du bureau de la directrice adjointe avait des aiguilles d'or et des nombres d'argents ; elle tictaquait et s'ébranlait sans bruit à mesure qu'elle avançait car un sortilège de silence avait été lancé sur elle.

L'aiguille d'or des heures s'approcha du neuf d'argent, l'aiguille d'or des minutes fit de même, les deux composants du temps proches l'un de l'autre, sur le point de se trouver au même endroit sans jamais être entrés en collision.

Il était 8h43 et le moment approchait où le Retourneur de Temps de Harry s'ouvrirait afin d'être testé par une méthode qu'aucun sortilège imaginable ne pourrait tromper à moins de savoir passer outre les lois du Temps lui-même. Ni corps ni âme, ni savoir ni substance ne pouvait insérer sept heures supplémentaires dans une journée. Elle inventerait un message sur le champ et dirait à Harry d'apporter ce message au professeur Flitwick six heures plus tôt, à trois heures de l'après-midi, et elle demanderait au professeur s'il l'avait reçu à cette heure.

Et le professeur Flitwick lui dirait qu'il l'avait bien reçu à trois heures.

Et elle dirait à Severus et à Albus de faire \emph{un peu}  plus confiance à Harry la prochaine fois.

Le professeur McGonagall invoqua son Patronus et dit à son chat étincelant : "Vas voir M. Potter et dis-lui ceci : 'M. Potter, veuillez vous rendre dans mon bureau dès que vous entendez ceci sans rien faire d'autre en chemin.'"

