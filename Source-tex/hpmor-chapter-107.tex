
\chapter{La Vérité, partie 4}

Les feuilles en spirale du dieffenbachia donnaient à Harry l'impression de marcher sur un sol forestier. Elles soutenaient son poids sans avoir la dureté du béton. Il garda un œil inquiet sur les petites tiges, mais elles demeuraient immobiles.

Lorsqu'il atteint le pied de l'escalier de feuilles, elles se tendirent soudain, saisirent ses bras et ses pieds.

Après avoir résisté un instant, il choisit de se détendre.

"Intéressant," dit le professeur Quirrell en lévitant depuis la trappe, sans toucher ni les feuilles ni les tiges. "Je constate que ça ne t'a pas dérangé de perdre contre une plante."

Maintenant que sa vision n'était plus obscurcie par la peur, Harry observa le professeur de Défense avec plus d'attention. Il se tenait droit et lévitait sans difficulté apparente. La sensation funeste était très forte, mais ses yeux étaient enfoncés dans leurs orbites, ses bras minces et dévastés. La maladie n'avait \emph{pas}  été un bluff ; l'hypothèse évidente était que le professeur de Défense avait récemment mangé une licorne pour recouvrer sa santé.

Et il parlait toujours comme aurait parlé le professeur Quirrell, pas comme Lord Voldemort. Ce qui n'était peut-être pas au détriment de Harry. Il en ignorait la raison - peut-être que le professeur de Défense avait encore besoin de lui -, mais participer à cette mascarade semblait servir les intérêts de Harry.

"Vous m'avez sciemment laissé tomber dans ce piège, professeur," répondit Harry exactement comme il l'aurait fait face au professeur Quirrell. \emph{Rôles, masques, rappelle-lui notre ancienne relation…}  "Seul, j'aurais utilisé un balai."

"Peut-être. Comment un élève de première année ordinaire résoudrait-il ce puzzle ? Baguette en main, évidemment." La plante tendait maintenant des tiges vers le professeur Quirrell, mais il lévitait juste hors de leur portée.

Harry venait de se rappeler que le professeur Chourave leur avait parlé du Filet du Diable. Selon leur livre de Botanique, la plante aimait les endroits frais et sombres, comme des grottes. C'était d'ailleurs très surprenant venant d'une plante à larges feuilles. "Si je devais deviner, je dirais que c'est un Filet du Diable et qu'il fuirait une source de lumière ou de chaleur. Peut-être qu'un élève de première année pourrait utiliser Lumos ? Aujourd'hui, j'utiliserais \emph{Inflammare} , mais je ne l'ai appris qu'en mai."

D'un mouvement du poignet, le professeur de Défense fit jaillir de sa baguette un liquide qui atteint la plante près de la base de ses tiges dans un bruit d'eau suivi d'un sifflement. Toutes les tiges en contact avec Harry repartirent et se mirent à frapper les plaies qui s'ouvraient à la surface de la plante, comme pour mettre fin à une douleur. Quelque chose chez la plante laissait penser qu'elle poussait un hurlement muet.

Le professeur Quirrell continua de dériver jusqu'au sol. "Maintenant, elle a peur de la lumière, de la chaleur, de l'acide et de moi."

Harry quitta les dernières feuilles et posa les pieds au sol après avoir précautionneusement inspecté ses robes et la pierre, afin d'être sûr que l'acide n'avait pas rebondi quelque part. Il avait commencé se dire que le professeur Quirrell essayait de lui envoyer un message, mais il n'arrivait pas à le comprendre. "Je pensais que nous avions une mission, professeur. Je ne peux rien contre vous, mais est-ce que c'est \emph{malin}  de perdre autant de temps à vous amuser à mes dépens ?"

"Oh, nous avons le temps," répondit-il, comme amusé. "Il y aurait un tumulte général si on découvrait que nous étions ici, protégés par un Inferius. Lorsque tu es venu et as parlé à Rogue, tu n'avais pas l'air d'avoir vu un tumulte lors du match de Quidditch."

Lorsqu'il comprit ça, un léger frisson parcourut Harry. Quoi qu'il fasse contre le professeur Quirrell, cela ne devrait \emph{pas}  s'entendre dans l'école, ou au moins au match de Quidditch, parce qu'il n'\emph{avait pas}  été interrompu. Ce ne serait pas facile d'assembler assez de monde pour faire face à Lord Voldemort si ni le professeur McGonagall, ni le professeur Flitwick, ni aucun spectateur du match ne pouvaient se rendre compte de quoi que ce soit…

Se battre contre un ennemi intelligent : c'était difficile.

Mais quand bien même… Harry avait la nette impression que, à la place du professeur Quirrell, il n'aurait pas fait la conversation, il n'aurait pas joué aux devinettes. Le professeur Quirrell gagnait \emph{quelque chose}  en prenant son temps. Mais quoi ? Un autre processus devait-il encore aboutir ?

"Mais au fait, m'as tu déjà trahi ?" demanda le professeur Quirrell.

"\emph{Pas encore} " siffla Harry.

Le professeur de Défense indiqua de sa main armée la grande porte de bois située au fond de la pièce. Harry se dirigea vers elle et l'ouvrit.
\par\noindent\rule{\textwidth}{0.4pt}
La salle suivante était plus étroite et son plafond était plus haut. La lumière qui émanait des alcôves était plus blanche que bleue.

Autour d'eux sifflaient des centaines de clés ailées. Après les avoir observées quelques secondes, il devint clair qu'une seule avait la couleur dorée du Vif, mais qu'elle se déplaçait plus lentement qu'un véritable Vif d'Or.

De l'autre côté de la pièce, une porte avec une grande serrure impossible à manquer.

Contre le mur de gauche, un balai, le modèle standard Brossdur 7.

"Professeur," dit Harry en regardant les nuages de clés, "vous avez dit que vous répondriez à mes questions. Qu'est-ce qui se passe ici ? Quand on a une porte blindée que seule une clé peut ouvrir, on garde la clé à l'abri et on n'en donne des copies qu'aux personnes de confiance. On ne \emph{donne pas des ailes à la clé}  et on ne \emph{laisse pas un balai volant appuyé contre le mur} . Alors qu'est-ce qu'on fait là et qu'est-ce qui se passe ? Il est très probable que seul le miroir magique garde vraiment la pierre, alors pourquoi tout ça… pourquoi encourager les élèves de première année à venir ?"

"Je n'en suis pas certain," dit le professeur de Défense. Il était entré après Harry et s'était placé à sa droite, à une distance respectable. "Mais je répondrai, comme je l'ai promis. La méthode de Dumbledore consiste à faire dix choses folles en apparence, et à donner un but caché à sept, peut-être huit d'entre elles. Je crois qu'il veut donner l'impression que je suis invité à envoyer un élève à ma place, précisément pour que Lord Voldemort, tel que Dumbledore se l'imagine, ait moins de chances de se croire malin en le faisant. Imagine Dumbledore se demandant pour la première fois comment protéger la pierre. Hésitant à placer de véritables dangers sur le chemin vers le miroir. Imagine-le s'imaginer quelque jeune élève butant contre ces dangers pour mon compte. Je pense que c'est ça qu'il cherche à éviter en donnant l'impression que cette approche est bienvenue - et donc peu rusée. À moins bien sûr que j'ai mal compris ce que Dumbledore s'imagine que Lord Voldemort pensera." Le professeur Quirrell eut le sourire le plus naturel que Harry ait jamais vu sur son visage. "Dumbledore n'a aucun instinct pour l'intrigue, mais il s'y essaie parce qu'il le doit. Il s'y attelle avec de l'intelligence, du dévouement, une capacité à apprendre de ses erreurs et une absence totale de talent. Ce qu'il fait est profondément difficile à prévoir pour cette unique raison."

Harry se détourna et regarda la porte de l'autre côté de la salle. \emph{Ce n'est pas un jeu pour lui, professeur Quirrell.}  "Je pense que la solution prévue est d'ignorer le balai et d'utiliser \emph{Wingardium Leviosa}  pour prendre la clé puisque ce n'est pas un match de Quidditch et qu'aucune règle ne l'interdit. Alors, quel sortilège d'une puissance effarante allez-vous utiliser cette fois ?"

On n'entendit que le sifflement des clés pendant un bref instant.

Harry s'éloigna du professeur Quirrell de quelques pas. "Je n'aurais probablement pas dû dire ça."

"Oh, si", répondit-il. "Je pense que c'est tout à fait raisonnable de dire ça au plus puissant mage noir du monde quand il se tient à moins de dix pas de vous."

Le professeur Quirrell remit sa baguette dans la manche de sa main droite, celle où le pistolet apparaissait parfois.

Puis il mit une main dans sa bouche et en sortit ce qui ressemblait à une dent. Il jeta la fausse dent en l'air, mais un fois retombée, elle s'était transformée en une baguette que Harry eut l'étrange sentiment de reconnaître, comme si une partie de lui trouvait que cette baguette faisait… partie de lui…

\emph{Treize pouces et demi, en if, et le cœur en plume de phénix} . Harry avait mémorisé l'information fournie par le fabricant de baguettes magiques, Olli-quelque chose, car elle lui avait semblé avoir un rapport potentiel avec l'intrigue. Ce moment, et la façon de penser qu'il avait eue alors, tout cela semblait être un souvenir d'une autre vie.

Le professeur de Défense leva sa baguette et dessina d'un geste une rune de flammes dentelée et d'apparence maléfique ; d'instinct, Harry fit un pas en arrière. Puis le professeur Quirrell parla : "Az-reth. Az-reth. Az-reth."

La rune de flammes se mit à déverser un feu… \emph{tordu} , comme si les bords dentelés de la rune avaient donné au feu sa nature profonde. Il était d'un cramoisi étincelant, d'un rouge plus profond que celui du sang, aussi violemment lumineux que l'arc d'un soudeur. Cet éclat, combiné à cette teinte, semblait \emph{anormal}  ; un rouge si profond n'aurait pas dû être capable de briller autant ; et ce violent cramoisi était veiné d'un noir qui paraissait absorber la lumière du feu. Entre les flammes noirâtres, dessinées par les frontières entre le cramoisi et les ténèbres, des formes de prédateurs défilaient vivement, du cobra à la hyène et au scorpion.

"Az-reth. Az-reth. Az-reth." Au bout de six répétitions, le volume du feu sorti de la rune aurait pu occuper l'espace d'un petit buisson.

Le feu maudit ralentit lorsque le professeur Quirrell le fixa du regard, et il adopta une dernière forme, celle d'un phénix noirâtre au sang de feu.

Et quelque chose donna à Harry la terrible certitude que si ce phénix de feu croisait Fumseck, le véritable phénix mourrait pour ne jamais renaître.

Le professeur Quirrell fit un seul geste de sa baguette et le feu noirâtre s'envola à travers la pièce. Il toucha la porte et sa serrure, et d'un mouvement de ses ailes cramoisies consuma la majeure partie de la porte et de l'arche qui l'entourait. Puis l'éclat cramoisi poursuivit son chemin.

Harry n'eut que le temps de jeter un coup d'oeil et d'apercevoir d'immenses statues sur le point de lever leurs épées et leurs massues avant que le feu noirâtre ne les atteigne et qu'elles se décomposent en brûlant.

Après en avoir fini, le phénix de feu noirâtre revint en passant par le trou et se posa juste au-dessus de l'épaule gauche du professeur Quirrell. Les serres cramoisies à l'intensité solaire demeurèrent quelques centimètres au-dessus des robes.

"Continue," dit le professeur Quirrell. "La voie est libre, maintenant."

Harry dut faire appel aux motifs mentaux de son côté obscur pour garder son calme et s'avancer. Il enjamba les restes encore lumineux de la porte et observa ceux fracassés des immenses pièces sur l'échiquier. Les dalles de marbre noires et blanches s'arrêtaient à cinq mètres de la porte détruite, allaient d'un mur latéral à l'autre, mais s'arrêtaient aussi à cinq mètres de la porte suivante située de l'autre côté de la pièce. Le plafond était largement hors de portée des statues.

"Je suppose," dit Harry d'une voix maintenue calme par les motifs mentaux de son côté obscur, "que, comme le balai était inutile à l'obtention de la clé, la solution prévue consiste à voler au-dessus des statues en utilisant le balai de la chambre précédente ?"

Derrière lui, le professeur Quirrell rit, et c'était le rire de Lord Voldemort. "Continue," dit une voix devenue plus froide, plus flûtée. "Va dans la pièce suivante. J'aimerais voir ce que tu penseras de ce qui s'y trouve."

\emph{Mis en place par Dumbledore pour des élèves en première année} , se remémora Harry, \emph{ce SERA sans danger} , et il traversa l'échiquier détruit, posa la main sur la poignée de la porte, et s'avança.
\par\noindent\rule{\textwidth}{0.4pt}
Un demi-seconde plus tard, Harry claqua la porte et bondit en arrière.

Il lui fallut plusieurs secondes pour reprendre son souffle et se maîtriser. De l'autre côté, de forts mugissements continuaient de se faire entendre, accompagnés de grands coups de massue contre la porte.

"Je suppose," dit Harry d'une voix elle aussi devenue froide, "que, puisque Dumbledore ne mettrait certainement pas un véritable troll des montagnes dans cette pièce, la prochaine épreuve est une illusion faite de mes pires souvenirs. Comme un Détraqueur, avec le souvenir projeté dans la réalité. Très amusant, professeur."

Le professeur Quirrell s'avança vers la porte et Harry se plaça bien à l'écart. Mis à part la forte sensation funeste qui émanait toujours du professeur, le côté obscur de Harry ou son instinct lui conseillaient de ne surtout pas s'approcher du feu noir-cramoisi qui voletait au-dessus de l'épaule du professeur Quirrell.

Le professeur Quirrell ouvrit grand la porte et regarda à l'intérieur. "Hmm", dit-il. "Seulement le troll, comme tu l'as dit. Bon. J'espérais apprendre quelque chose de plus intéressant à ton sujet. C'est un Kokohekkus, aussi connu sous le nom d'Épouvantard commun."

"Un Épouvantard ? Qu'est-ce qu'il… non, je crois savoir ce que ça fait."

"Un Épouvantard," continua le professeur Quirrell, et sa voix était à présent celle d'un professeur de Poudlard en pleine leçon, "gravite à proximité d'endroits sombres, fermés, rarement ouverts, tel un placard oublié dans un grenier. Il veut qu'on le laisse tranquille et adoptera l'apparence qu'il jugera la plus effrayante."

"M'effrayer ?" répondit Harry. "J'ai \emph{tué}  ce troll."

"Tu as bondi en arrière sans réfléchir. Un Épouvantard veut provoquer un mouvement de recul instinctif plus que d'être considéré comme une vraie menace. Autrement, il aurait choisi quelque chose de plus crédible. Quoi qu'il en soit, le sortilège à utiliser contre un Épouvantard est, évidemment, le Feudeymon." Le professeur Quirrell fit un geste et le feu noirâtre bondit de son épaule avant de se répandre dans l'embrasure de la porte.

De derrière la porte vint un glapissement, puis plus rien.

Ils avancèrent dans la salle où s'était trouvé l'Épouvantard avec, cette fois, le professeur Quirrell en tête. Maintenant que le faux troll des montagnes n'était plus là, ce n'était plus qu'une immense salle comme les autres, éclairée de bougeoirs d'un bleu froid.

Le regard du professeur Quirrell semblait lointain, pensif. Il traversa la pièce sans attendre Harry et ouvrit grand la porte située de l'autre côté de la pièce.

Harry le suivit, de loin.
\par\noindent\rule{\textwidth}{0.4pt}
La pièce suivante contenait un chaudron, un râtelier d'ingrédients embouteillés, une planche à découper, des tiges à remuer et tous les autres ustensiles de fabrication de potions. Cette fois, la lumière venue des alcôves était blanche, probablement parce que la préparation de potions nécessitait que l'on distingue bien les couleurs. Le professeur Quirrell se tenait déjà face aux ustensiles et lisait un long parchemin qu'il venait de ramasser. La pièce qui menait à la salle suivante était gardée par un rideau de feu violet qui aurait eu l'air bien plus menaçant s'il n'avait pas semblé pâle et faible comparé aux flammes noirâtres qui volaient au-dessus de l'épaule du professeur Quirrell.

À ce stade, la suspension de l'incrédulité de Harry était partie en congé sans solde depuis bien longtemps, si bien qu'il ne commenta pas sur le fait qu'un véritable système de sécurité servait à \emph{distinguer}  les visiteurs autorisés de ceux qui ne l'étaient pas, ce qui impliquait de proposer des épreuves dont l'issue serait \emph{différente}  selon que l'on soit ou non censé se trouver là. Par exemple, une \emph{bonne}  épreuve aurait été de vérifier que l'arrivant connaissait la combinaison d'un cadenas, connue seulement des personnes autorisées à se rendre ici, et une \emph{mauvais}  épreuve aurait été de vérifier que l'arrivant était capable de préparer une potion en suivant des instructions aimablement fournies.

Le professeur Quirrell jeta le parchemin vers Harry, et il roula par terre. "Qu'en penses-tu?" dit le professeur Quirrell avant de reculer pour que Harry puisse s'approcher et le ramasser.

"Non," dit Harry après avoir parcouru le parchemin. "Vérifier que l'arrivant sait résoudre une énigme ridiculement simpliste pour connaitre l'ordre d'ajout des ingrédients ne constitue toujours pas une épreuve dont l'issue diffère selon le niveau d'accréditation de l'arrivant. Proposer une énigme plus intéressante qui concerne trois idoles ou une rangée de personnes coiffées de chapeaux de couleur n'y change rien, c'est toujours à côté de la plaque."

"Regarde l'autre côté," dit le professeur Quirrell.

Harry retourna le parchemin de deux pieds de long.

De l'autre côté, écrit en patte de mouche, se trouvait la plus longue recette de potions que Harry ait jamais lue. "Mais qu'est-ce que…"

"Une \emph{potion de splendeur} , pour tarir le feu violet," dit le professeur Quirrell. "Il faut ajouter les mêmes ingrédients encore et encore, d'une façon à chaque fois un peu différente. Imagine un groupe de première année enthousiaste qui vient de passer de salle en salle, persuadé d'être sur le point d'atteindre le miroir magique, tomber soudain sur cette épreuve. C'est bien là le l'œuvre du maître des potions."

Harry regarda la forme sombre sur l'épaule du professeur Quirrell. "Du feu peut-il vaincre du feu ?"

"C'est possible," répondit ce dernier. "Je ne suis pas sûr que ce soit souhaitable. Et si la pièce était piégée ?"

Harry n'avait certainement pas l'intention de se retrouver coincé ici à préparer une potion pour le plaisir, ou quelle que soit la raison qui poussait le professeur Quirrell à leur faire traverser ces salles aussi lentement. La recette indiquait d'ajouter des campanules à \emph{trente-cinq}  moments différents et une 'mèche de cheveux clairs' à quatorze moments… "Peut-être que la potion relâche un gaz mortel fatal pour les sorciers adultes, mais pas pour les enfants. Ou qu'elle recèle l'un de cent autres pièges mortels, si on veut parler sérieusement. Est-ce qu'on parle sérieusement ?"

"Cette salle est l'œuvre de Severus Rogue," répondit le professeur Quirrell, de nouveau pensif. "Rogue n'est pas tout à fait un témoin de cette partie. Il n'a pas l'intelligence de Dumbledore, mais il possède l'intention de tuer que Dumbledore n'a jamais eue."

"Eh bien quoi qu'il ait fait, ça n'empêche pas les enfants de passer," fit remarquer Harry. "Beaucoup de première année sont passés. Et si cette pièce permet de ne laisser passer \emph{que}  des enfants, alors du point de vue de Dumbledore, elle force aussi Lord Voldemort à venir avec un enfant. Étant donné les buts de Dumbledore et Rogue, je ne vois pas l'intérêt."

"En effet," dit le professeur en se frottant l'arête du nez. "Mais vois-tu, petit, cette pièce ne contient pas les alarmes qui se trouvaient dans les autres. Il n'y a aucune protection subtile à contourner. C'est comme si j'étais \emph{invité}  à passer outre la potion et à simplement entrer… mais Rogue sait que Lord Voldemort se rendra compte de ça. S'il y avait vraiment un piège destiné à quiconque ne préparerait pas la potion, il serait plus sage de mettre en place des protections et de ne montrer en rien que cette pièce est différente des autres."

Harry écouta, concentré, les sourcils froncés. "Donc… le seul intérêt qu'i enlever les alarmes de cette pièce est d'\emph{éviter}  que vous traversiez cette pièce en trombe."

"J'attends de Rogue qu'il s'attende à ce que je comprenne aussi cela," continua le professeur de défense. "Passé ce cap, je ne peux pas prédire le niveau de jeu qu'il attend de moi. Je suis patient et j'ai consacré beaucoup de temps à ce projet. Mais Rogue ne me connait pas ; il ne connait que Lord Voldemort. Il a parfois vu Lord Voldemort hurler de frustration et suivre des pulsions apparemment contre-productives. Considère les choses du point de vue de Rogue : le maître des potions de Poudlard est en train de dire à Lord Voldemort que s'il veut entrer dans la salle, il doit être patient et obéir aux instructions. Comme si Lord Voldemort était un simple écolier. Il me serait facile d'obtempérer sourire aux lèvres et de me venger plus tard. Mais Rogue ne sait pas que Lord Voldemort peut simplement raisonner ainsi." Le professeur Quirrell regarda Harry. "Tu m'as vu léviter au-dessus du Filet du Diable, n'est-ce pas, petit ?"

Harry hocha la tête. Puis il remarqua qu'il était confus. "Mon livre d'enchantements dit que les sorciers ne peuvent pas se faire léviter eux-mêmes."

"Oui," dit le professeur Quirrell, "c'est bien ce qui y est écrit. Aucun sorcier ne peut se faire voler, ou aucun objet qui supporte son propre poids. Ce serait comme d'essayer de se propulser en orbite en tirant sur ses cheveux. Pourtant, Lord Voldemort peut voler. Comment ? Réponds aussi vite que tu peux."

Si un élève de première année pouvait répondre à la question… "Vous avez demandé à quelqu'un d'autre d'ensorceler vos sous-vêtements, puis vous leur avez effacé la mémoire."

"Pas du tout", dit le professeur Quirrell. "Les sortilèges pour balais nécessitent une forme longue, étroite et solide. Des vêtements ne conviendraient pas."

Harry fronça les sourcils. "Quelle longueur ? Peut-on attacher de petits bouts de bois pour construire un harnais et voler grâce à lui ?"

"J'ai en effet commencé par m'attacher des bâtons aux bras et aux jambes, mais ce n'était que pour m'apprendre une nouvelle façon de voler." Le professeur Quirrell releva une manche et révéla un bras nu. "Comme tu peux le voir, je n'ai rien dans les manches."

Harry absorba cette contrainte supplémentaire. "Vous avez demandé à quelqu'un d'enchanter vos \emph{os}  ?"

Le professeur Quirrell soupira. "Et c'était l'un des exploits de Lord Voldemort les plus craints, ou du moins c'est ce que j'ai entendu dire. Après toute ces années et un peu de Légilimancie faite avec réticente, je ne comprends toujours pas ce qui \emph{cloche}  chez les gens ordinaires… mais tu n'en fais pas partie. Il est temps que tu commences à contribuer à cette expédition. Tu connais Severus Rogue depuis moins longtemps que moi. Donne-moi ton analyse de cette pièce."

Harry hésita en essayant de sembler pensif.

"J'ajouterai," dit le professeur Quirrell, et le phénix de feu noirâtre perché sur son épaule sembla tourner la tête et regarder Harry avec méchanceté, "que si tu me laisses sciemment échouer, je considérerais cela comme une trahison. Je te rappelle que la pierre est la clé de la résurrection de Mlle Granger et que je tiens en otage les vies de centaines d'élèves."

"Je m'en souviens," dit Harry, et juste après, le cerveau merveilleusement inventif de Harry eut une idée.

Il n'était pas certain qu'il fut sage de le dire.

Le silence dura.

"As-tu pensé à quelque chose ?" dit le professeur Quirrell. "Réponds en Fourchelangue."

Non, ce ne serait \emph{pas}  facile, pas contre un adversaire intelligent capable de vous forcer à dire la vérité n'importe quand. "Severus, ou du moins le Severus d'aujourd'hui, respecte beaucoup votre intelligence," préféra répondre Harry. "Je pense… je pense qu'il \emph{s'attend}  à ce que Voldemort croie que Severus penserait Voldemort incapable de réussir son épreuve de patience, et qu'il s'attend à ce que Voldemort en soit en fait capable."

Le professeur Quirrell hocha la tête. "C'est plausible. Est-ce que tu y crois ? Réponds en Fourchelangue."

"\emph{Oui} ," siffla Harry. Peut-être qu'il était dangereux de cacher des informations, ou même des pensées et des idées… "Le but de cette salle est donc de retarder Lord Voldemort d'une heure. Et si je voulais vous tuer, et que je croyais ce que Dumbledore croit, il serait évident de vouloir essayer le baiser du Détraqueur. Je veux dire, comme ils pensent que vous êtes une âme désincarnée… d'ailleurs, est-ce que vous en êtes une ?"

Le professeur Quirrell ne bougea pas. "Dumbledore ne penserait pas à ça," dit le professeur de Défense après un instant. "Mais Severus, peut-être." Il commença à se tapoter la joue d'un doigt et regarda dans le lointain. "Tu es plus fort que les Détraqueurs, petit. Peux-tu me dire s'il y en a non loin ?"

Harry ferma les yeux. S'il y avait des trous dans le monde aux alentours, il ne pouvait pas les sentir. "Aucun que je puisse détecter."

"Réponds en Fourchelangue."

"\emph{Je ne resssens pas de Détraqueur.} "

"Mais tu étais honnête lorsque tu as évoqué cette possibilité ? Tu n'essayais pas de me jouer un tour ?"

"\emph{J'étais honnête. Pas d'arnaque.} "

"Peut-être qu'un Détraqueur peut être masqué, ordonné de bondir et de dévorer la première âme venue…" il se tapait toujours la joue. "Je ne dirais pas que c'est impossible. Pas plus que de lui dire de manger tous ceux qui traverseraient cette pièce trop vite, ou le premier adulte venu. En gardant à l'esprit que je détiens la vie de Hermione et de centaines d'étudiants en otage : utiliserais-tu ton pouvoir contre les Détraqueurs pour me défendre, si l'un d'eux apparaissait ? Réponds en Fourchelangue."

"\emph{Je ne sais pas} ," siffla Harry.

"\emph{Les mange-vie ne peuvent pas me détruire, je ne pensse pas,} " siffla le professeur Quirrell. "\emph{Et j'abandonnerais ssimplement ce corps ss'ils ss'approchent de trop près. Cette fois, je reviendrais très vite, et rien ne m'arrêterait. Je torturerais tes parents pendant des années pour te punir de ne pas m'avoir aidé. Des centaines d'élèves pris en otages mourraient, y compris ceux que tu dis être tes amis. Maintenant, je te posse à nouveau la quesstion : s'ils venaient, utilisserais-tu ton pouvoir contre les mange-vie pour me protéger ?} "

"\emph{Oui} ," siffla Harry. La tristesse et l'horreur que Harry avait repoussées jaillirent à nouveau, et son côté obscur n'avait aucun motif lui permettant de gérer ces émotions. "\emph{Pourquoi, professeur Quirrell, pourquoi êtes-vous ainsi…} "

Le professeur Quirrell sourit. "Ce qui me fait penser. Est-ce que tu m'as trahi ?"

"\emph{Je ne vous ai pas encore trahi.} "

Le professeur Quirrell s'avança jusqu'aux ustensiles et commença à trancher une racine d'une main. Le couteau se déplaçait si vite qu'il n'était presque plus visible et le professeur Quirrell semblait n'exercer aucun effort. Le phénix de Feudeymon vola jusqu'au coin opposé et y demeura. "Étant donné la situation et les incertitudes auxquelles nous faisons face, il semble plus sage d'y aller lentement, comme le ferait un élève de première année," dit le professeur de Défense. "Autant discuter pendant que nous attendons. Tu avais des questions, petit ? J'ai dit que j'y répondrai, alors, pose-les."

