
\chapter{L'Expérience de Prison de Stanford, pt 3}

Le cadavre d'une femme ouvrit les yeux, et les orbes éteints et creusés regardèrent dans le vide.

"Folle," marmonna Bellatrix d'une voix brisée, "il semble que petite Bella devient folle..."

Le professeur Quirrell avait calmement et précisément expliqué à Harry comment agir en présence de Bellatrix ; comment créer le faux-semblant qu'il devrait maintenir dans son esprit.

\emph{Vous avez trouvé opportun, ou peut-être amusant, de rendre Bellatrix amoureuse de vous, de la lier à votre service.} 

Cet amour aurait survécu à Azkaban, avait dit le professeur Quirrell, car pour elle il ne constituait pas une pensée heureuse.

\emph{Elle vous aime totalement, absolument, de tout son être. Vous ne le lui rendez pas, mais vous le trouvez utile. Elle le sait.} 

\emph{Elle était l'arme la plus mortelle que vous possédiez, et vous l'appeliez chère Bella.} 

Harry se souvint de la nuit où le Seigneur des Ténèbres avait tué ses parents : l'amusement froid, le rire méprisant, la voix haute perchée faite de haine mortelle. Il ne semblait pas difficile de deviner ce que le Seigneur des Ténèbres dirait.

"J'espère que vous n'êtes \emph{pas}  folle, chère Bella," dit le murmure glacé. "La folie n'est pas utile."

Les yeux de Bella cillèrent, ils essayèrent de mettre au point sur le vide.

"Mon... seigneur... je me suis rendue là où vous m'avez dit de vous attendre, mais vous n'êtes pas venu... je vous ai cherché mais je n'ai pas pu vous trouver... vous êtes en vie..." Tous ces mots provenaient d'un bas marmonnement, et s'il y avait là la moindre émotion, Harry ne pouvait pas la détecter.

"\emph{Montrez-lui votre vissage,} " siffla le serpent aux pieds de Harry.

Harry rejeta le capuchon de la Cape d'Invisibilité.

La partie de lui-même à laquelle il avait donné le contrôle de ses expressions faciales regarda Bella sans la moindre trace de pitié, avec seulement un intérêt froid et calme (alors qu'au fond de lui-même, il songeait : \emph{je vous sauverai, je vous sauverai quoi qu'il en coûte...} )

"La cicatrice..." marmonna Bellatrix. "Cet enfant..."

"C'est ce qu'il pensent tous," dit la voix de Harry, et il émit un léger gloussement. "Vous m'avez cherché au mauvais endroit, chère Bella."

(Harry avait demandé au professeur Quirrell pourquoi ce ne serait pas lui qui jouerait le rôle du Seigneur des Ténèbres, et le professeur Quirrell avait fait remarquer qu'il n'y avait aucune raison plausible pour laquelle \emph{il}  serait possédé par l'ombre de Celui-Dont-Il-Ne-Faut-Pas-Prononcer-Le-Nom).

Les yeux de Bellatrix restèrent fixés sur Harry, et elle demeura coite.

"\emph{Dites quelque chosse en Fourchelangue,"}  siffla le serpent.

Le visage de Harry se tourna vers le serpent afin de clairement montrer qu'il s'adressait à celui-ci, et il siffla : "\emph{Un deux troiss quatre ccinq ssix ssept huit neuf dix.} "

Il y eut une pause.

"Ceux qui ne craignent pas les ténèbres..." murmura Bellatrix.

Le serpent siffla : "\emph{Sseront conssumés par elles.} "

"Seront consumés par elles," chuchota la voix glaciale. Harry n'avait pas particulièrement envie de réfléchir à la façon dont le professeur Quirrell avait obtenu ce mot de passe. Son cerveau, qui y pensa quand même, suggéra que cela avait nécessité un Mangemort, un lieu silencieux et isolé, et de la Légilimancie à coups de barre de fer.

"Votre baguette," murmura Bellatrix, "je l'ai cachée dans le cimetière, Seigneur, avant de partir... sous la tombe à droite de celle de votre père... me tuerez-vous maintenant, si c'est tout ce que vous vouliez de moi... je pense que j'ai toujours voulu que ce soit vous qui me tuiez... mais je ne peux m'en souvenir à présent, ça devait être une pensée heureuse..."

Le cœur de Harry se tordit dans sa poitrine, c'était insupportable - et il ne pouvait pas pleurer, il ne pouvait pas laisser son Patronus faiblir -

Son visage laissa apparaître un soupçon d'agacement, et sa voix fut tranchante : "Assez d'idioties. Vous allez venir avec moi, chère Bella, à moins que vous ne préfériez la compagnie des Détraqueurs."

Le visage de Bellatrix tressailli sous le coup d'une brève incompréhension, mais les membres ratatinés ne bronchèrent pas.

"\emph{Vous devrez la faire léviter jusqu'à la ssortie} ," siffla Harry au serpent. "\emph{Elle ne peut plus ssonger à ss'enfuir."} 

"\emph{Oui,"}  siffla le serpent, "\emph{mais ne la ssous-esstimez pas. Elle était la plus mortelle des guerrières." } La tête verte s'inclina en signe d'avertissement. "\emph{Il serait sage de me craindre, enfant, même si j'étais affamé et mort aux neufs dixièmes ; méfiez-vous d'elle, ne permettez à aucune faille d'apparaître dans votre jeu."} 

Le serpent vert glissa élégamment jusqu'au couloir.

Et peu après, un homme au teint cireux et à l'air effrayé peint sur son visage barbu entra d'un pas servile dans la pièce, baguette en main.

"Seigneur ?" dit le serviteur d'une voix hésitante.

"Faites ce que l'on vous a dit de faire," dit le Seigneur des Ténèbres de sa voix glaciale, d'autant plus terrible qu'elle émanait du corps d'un enfant. "Et ne laissez pas votre Patronus vaciller. Souvenez-vous que si je ne reviens pas, vous n'aurez pas de récompense, et qu'il ne sera pas mis fin aux souffrances de votre famille avant longtemps."

Après avoir prononcé ces épouvantables paroles, le Seigneur des Ténèbres recouvrit sa tête de la Cape d'Invisibilité et disparut.

Le serviteur ouvrit la porte de la cage de Bellatrix et tira une petite aiguille de ses robes, avec laquelle il piqua le squelette humain. L'unique goutte de sang qui fut ainsi produite fut rapidement absorbée dans une petite poupée qui avait été déposée au sol, et le serviteur commença à psalmodier en chuchotant.

Peu après, un autre squelette vivant se trouvait au sol, immobile. Puis le serviteur sembla hésiter un instant, jusqu'à ce qu'un ordre impatient n'émane du vide. Le serviteur dirigea alors sa baguette vers Bellatrix, prononça un mot, et le squelette vivant sur le lit fut nu, et celui au sol fut vêtu d'une robe passée.

Le serviteur arracha un petit bandeau de tissu de la robe placée sur le faux corps ; et de ses propres robes, l'homme effrayé produit une flasque en verre, vide, avec de petites traces d'un fluide doré encore accrochées à sa paroi interne. La flasque fut cachée dans un coin, le bandeau de tissu déposé au-dessus de celle-ci, la couleur terne de ce dernier se fondant presque avec le mur de métal gris.

Un autre mouvement de la baguette du serviteur fit flotter le squelette humain depuis le lit jusqu'aux airs, et presque du même mouvement, le corps fut recouvert de nouvelles robes noires. Une bouteille de lait au chocolat à l'air ordinaire fut placée dans sa main, et un chuchotement glacial ordonna à Bellatrix de s'en saisir et de la boire, ce qu'elle fit, son visage n'exprimant à présent rien d'autre que de la perplexité.

Puis le serviteur rendit Bellatrix invisible, puis il se rendit lui-même invisible, et ils partirent. La porte se ferma derrière eux et le loquet fit un déclic, plongeant de nouveau le couloir dans les ténèbres, inchangé mis à part une petite flasque placée dans le coin d'une cellule, et un corps frais allongé sur le sol de celle-ci.
\par\noindent\rule{\textwidth}{0.4pt}
Plus tôt, dans le magasin désert, le professeur Quirrell avait dit à Harry qu'ils allaient commettre le crime parfait.

Sans réfléchir, Harry avait commencé à répéter les proverbes standards disant que le crime parfait n'existait pas, avant de vraiment y réfléchir pendant environ deux tiers de seconde, de se souvenir d'un proverbe plus sage, et de fermer sa bouche à mi-phrase.

\emph{Qu'est-ce que tu penses savoir, et comment penses-tu que tu le sais ?} 

Si vous \emph{commettiez}  le crime parfait, personne ne le découvrirait jamais - alors comment quiconque pouvait-il \emph{savoir}  qu'il n'y avait pas de crimes parfaits ?

Dès que vous regardiez les choses sous cet angle, vous vous rendiez compte que des crimes parfaits étaient probablement commis \emph{en permanence} , et que le médecin légiste marquait que la mort avait eu des causes naturelle, ou le journal écrivait que le magasin n'avait jamais fait beaucoup de profit et qu'il avait fini par faire faillite...

Lorsque le corps de Bellatrix fut trouvé dans sa cellule le lendemain, là, dans la prison d'Azkaban, d'où (tout le monde le savait) on ne s'était jamais échappé, personne ne prit la peine de faire une autopsie. Personne n'y songea à deux fois. Ils refermèrent juste la porte du couloir et partirent, et la \emph{Gazette du Sorcier}  le mentionna dans la rubrique nécrologique du lendemain...

... ça, c'était le crime parfait que le professeur Quirrell avait prévu.

Et ce n'est pas lui qui le fit foirer.

