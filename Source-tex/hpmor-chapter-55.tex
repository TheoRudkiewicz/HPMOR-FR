
\chapter{L'Expérience de Prison de Stanford, pt 5}

Dans un couloir en ruine et délabré, éclairé par de faibles lampes à gaz, un garçon progresse lentement, une main tendue en avant, vers le serpent immobile qui était autrefois le corps de son professeur.

Harry n'était qu'à un mètre quand il le sentit, le picotement à la limite de ses sens.

Si faible, la sensation funeste...

Le professeur Quirrell \emph{était}  donc en vie.

La pensée n'engendra aucune joie, seulement une sorte de désespoir vide.

Harry serait bientôt pris, et quelle que soit ses explications, tout cela aurait l'air suspect. Personne n'allait plus jamais lui faire confiance, on allait penser qu'il était le futur Seigneur des Ténèbres, on ne l'aiderait pas lorsque le temps de combattre Lord Voldemort viendrait, Hermione le laisserait tomber, même Dumbledore irait probablement se chercher un autre héros...

... peut-être qu'ils le renverraient juste chez lui, chez ses parents.

Il avait échoué.

Harry regarda le corps ratatiné de l'agent de police qu'il avait étourdi, le sang, déjà en train de sécher, autour de coupures mineures, les parties brûlées des robes rouges aux broderies tortueuses.

Il avait été stupide. Il \emph{n'aurait pas dû}  étourdir l'officier de police, il aurait dû s'en \emph{tenir}  à son histoire de kidnapping par le professeur Quirrell...

\emph{Il n'est peut-être pas trop tard,}  murmura une voix à l'intérieur de lui. \emph{Tu peux peut-être encore réparer ton erreur. L'Auror te voit, il se souvient que tu l'as étourdi... mais s'il était mort, si le professeur Quirrell était mort, si Bellatrix était morte, alors il n'y aurait plus personne pour contredire ton histoire.} 

Lentement, la main de Harry commença à se lever, pointa sa baguette vers l'agent de police, et -

Sa main s'arrêta.

Il avait la vague sensation qu'il se comportait, sans savoir en quoi, de façon fort peu habituelle. Comme s'il avait oublié quelque chose, quelque chose d'important, mais il avait du mal à se souvenir de quoi il s'agissait exactement.

Oh. C'était ça. Il était quelqu'un qui croyait à la valeur de la vie humaine.

Un incompréhension accompagna la pensée : il ne pouvait plus tout à faire se rappeler \emph{pourquoi}  les vies des autres avaient semblé avoir de la valeur...

\emph{Très bien} , dit la partie logique de son esprit, \emph{pourquoi mon esprit a-t-il changé entre avant et maintenant ?} 

Parce qu'il était à Azkaban...

Et qu'il avait oublié de relancer le Patronus...

La moindre action semblait mystérieusement requérir un effort immense, comme si la pensée d'agir était elle-même était un poids trop lourd pour être porté ; mais Harry il était encore capable de craindre les Détraqueurs et relancer le Patronus semblait donc être une bonne idée. Et même s'il ne pouvait pas se souvenir de ce que c'était que d'être heureux, il savait quand même qu'il ne l'était pas.

La main de Harry se leva, plaça sa baguette à l'horizontale devant lui, ses doigts se mirent en position de départ.

Puis il s'interrompit.

Il ne pouvait pas... bien se rappeler... de ce qu'il avait utilisé en guise de pensée heureuse.

C'était bizarre car ça avait semblé être quelque chose de très important, il aurait vraiment dû s'en souvenir... quelque chose en rapport avec la mort ? Mais ça n'était pas joyeux...

Son corps tremblait, Azkaban n'avait pas semblé si froide plus tôt, et avec cette pensée la température sembla chuter de nouveau. Il était trop tard, il avait plongé trop profond, il ne pourrait jamais lancer le Patronus maintenant -

\emph{Peut-être que c'est le Détraquage qui parle et que ce n'est pas une estimation correcte, } observa la partie logique de son être par une habitude qui avait été encodée en lui et ne demandait aucune énergie pour être activée. \emph{Vois la peur des Détraqueurs comme un biais cognitif et essaies de la dépasser tout comme tu dépasserais tout autre biais cognitif. Ton désespoir n'indique peut-être pas que la situation est vraiment désespérée. Il n'indique peut-être rien d'autre que la présence de Détraqueurs. À partir de maintenant, toute émotion négative et toute estimation pessimiste doit être considérée suspecte et fallacieuse jusqu'à avoir été prouvée.} 

(Si vous aviez observé le garçon au milieu de ces pensée, vous auriez vu une moue distante, abstraite et perplexe le long de son visage, en-dessous des lunettes et de la cicatrice en éclair. Sa main était restée en position de départ pour le Patronus et n'avait pas bougé.)

\emph{La présence des Détraqueurs interfère avec la partie de toi consacrée au bonheur. Si tu ne peux pas retrouver ta pensée heureuse par association mnémonique en utilisant la clé "joie", peut-être que tu peux l'atteindre par un autre chemin. Quelle était la dernière fois que tu as parlé du Patronus à quelqu'un ?} 

Harry semblait tout aussi incapable de se souvenir de cela.

Une vague de désespoir se déversa en lui et fut rejetée par la partie logique de son être parce qu'elle était indigne de confiance, externe, pas-Harry, et même si le poids mort continuait de s'appuyer sur lui, son esprit continuait de penser - penser ne requérait pas beaucoup d'effort...

\emph{Quelle était la dernière fois où tu as parlé des Détraqueurs à quelqu'un ?} 

Le professeur Quirrell avait dit qu'il était déjà capable de sentir leur présence, et Harry avait dit au professeur Quirrell... il avait dit au professeur Quirrell...

... de s'accrocher au souvenir des étoiles, à la chute éthérée dans l'espace, comme à une barrière Occlumantique tendue d'un bout à l'autre de son esprit.

Son second cours de Défense de l'année, vendredi, c'est là que le professeur Quirrell lui avait montré les étoiles, et de nouveau à Noël.

Il n'eut alors pas à faire beaucoup d'efforts pour s'en souvenir, les points blancs incandescents sur un noir parfait.

Harry se souvint de la grande traînée nuageuse de la Voie Lactée.

Il se souvint de la paix.

Une partie du froid à l'extérieur de ses membres sembla battre en retraite.

Il avait prononcé des mots, haut et fort, le jour où il avait lancé le Patronus pour la première fois, son esprit pouvait se souvenir des sons et de la voix, mais les sensations semblaient lointaines...

...\emph{Mon rejet absolu de la mort m'a semblé être la réponse évidente.} 

On lance le Véritable Patronus en pensant à la valeur de la vie humaine.

\emph{...Mais il y a d'autres vies qui sont toujours là pour qu'on se batte pour elles. Ta vie, et ma vie, et la vie de Hermione Granger, et toutes les vies de la Terre, et les vies au-delà, qui doivent être défendues et protégées.} 

L'idée de tuer tout le monde... ça n'avait pas été lui, c'était le Détraquage qui avait parlé...

Le désespoir venait de l'influence des Détraqueurs.

\emph{Là où il y a de la vie, il y a de l'espoir. L'Auror est toujours en vie. Le professeur Quirrell est toujours en vie. Bellatrix est toujours en vie. Je suis toujours en vie. Personne n'est encore vraiment mort...} 

Harry pouvait se représenter la Terre à cet instant, au milieu du champ d'étoiles, l'orbe blanc-bleu.

...\emph{et je ne les laisserai pas mourir !} 

"Expecto Patronum !"

Les mots s'échappèrent, un peu hésitants, et lorsque la forme humaine réapparut, elle fut tout d'abord terne, lunaire plutôt que solaire, blanche plutôt qu'argentée.

Mais elle se renforça, lentement, à mesure que Harry respirait à un rythme contrôlé, à mesure qu'il récupérait. Qu'il laissait la lumière repousser les ténèbres de son esprit. Qu'il se souvenait de choses qu'il avait presque oubliées et qu'il les faisait affluer vers le Patronus.

Même lorsque la lumière eut une fois de plus atteint le sommet de sa puissance, qu'elle eut illuminé le couloir avec plus de force que les lampes à gaz, qu'elle eut entièrement banni le froid, les membres de Harry tremblaient encore. C'était passé bien trop près.

Harry prit une profonde inspiration. Très bien. Il était temps de reconsidérer les choses maintenant que ses pensées n'étaient plus artificiellement assombries par des Détraqueurs.

Harry passa la situation en revue.

...toujours assez désespérée, à vrai dire.

Ce n'était pas le désespoir écrasant d'avant mais Harry se sentait encore chancelant, et c'était peu dire. Il n'osait pas devenir obscur mais c'était son côté obscur qui avait la capacité de gérer ce niveau de difficulté avec calme. C'était son côté obscur qui aurait rit avec dédain à l'idée même d'abandonner juste parce qu'il avait perdu le professeur Quirrell et qu'il était perdu dans les profondeurs d'Azkaban et qu'un policier l'avait vu. Le Harry ordinaire n'était pas capable de prendre ce genre de choses calmement.

Mais il n'y avait pas d'autre choix que de continuer d'avancer. Il n'y avait \emph{rien}  de plus absurde que d'abandonner avant d'avoir réellement perdu.

Harry regarda autour de lui.

De faible lampes à gaz éclairaient un couloir de métal gris dont les flancs et le sol et le plafond était tailladés par endroits, comme creusés et fondus, révélant à qui voulait bien regarder qu'une bataille avait eu lieu ici.

Le professeur Quirrell aurait pu réparer cela facilement s'il avait...

Le sentiment de trahison frappa alors Harry à pleine force.

\emph{Pourquoi... pourquoi a-t-il... pourquoi...} 

\emph{Parce qu'il est maléfique} , dirent Gryffondor et Poufsouffle, doucement et tristement. \emph{On te l'a dit.} 

\emph{Non ! } pensa Harry avec désespoir. \emph{Non, ça n'a aucun sens, on allait commettre le crime parfait, l'Auror aurait pu recevoir un sortilège d'Oubliettes, le couloir aurait pu être réparé, il n'était pas trop tard mais il AURAIT été trop tard s'il était mort !} 

\emph{Mais le professeur Quirrell n'avait jamais compté commettre le crime parfait, dit la sombre voix de Serpentard. Il } voulait \emph{que le crime soit remarqué. Il voulait que tout le monde sache que quelqu'un avait tué un Auror et avait fait sortir Bellatrix Black d'Azkaban. Il avait préparé une sorte d'indice, de preuve révélant ton implication, pour l'utiliser comme moyen chantage ; et tu aurais été lié à lui pour toujours.} 

Le Patronus de Harry faillit disparaître.

\emph{Non...}  pensa Harry.

\emph{Oui,}  dirent tristement trois autres parties de lui.

\emph{Non. Ça n'a toujours aucun sens. Le professeur Quirrell devait savoir que je me retournerais contre lui à l'instant où je le verrais tuer un Auror. Que j'aurais fort bien pu risquer d'aller tout confesser à Dumbledore en espérant plaider que j'avais été trompé, ce qui aurait été vrai. Et... en termes de chantage, tuer un Auror contre ma volonté, est-ce } vraiment\emph{ plus grave que de libérer Bellatrix d'Azkaban avec mon consentement et soutien ? Il aurait été plus fourbe de garder une preuve de mon implication dans le crime initial mais de continuer à faire semblant d'être mon allié pendant aussi longtemps que possible en gardant le chantage uniquement pour le jour où cela deviendrait nécessaire...} 

\emph{Rationalisation, } dit Serpentard. \emph{Alors pourquoi le professeur Quirrell } a-t-il \emph{fait cela ?} 

Et Harry pensa, avec une légère teinte de désespoir - en sachant, alors que la pensée se formait, qu'il était motivé en partie par un désir de rejeter la réalité, et que ce n'était pas ainsi que la technique devait être utilisée - il pensa : \emph{je remarque que je suis confus.} 

Il y eut un silence interne. Aucune partie de lui semblait n'avoir quoi que ce soit à ajouter.

Harry devait-il réévaluer la possibilité que Bellatrix soit maléfique ?

... pas pour les besoins de la mission. Il était \emph{acquis}  que Bellatrix était actuellement maléfique. Qu'elle soit une innocente devenue ainsi par le biais de torture, de Légilimancie et de rituels indicibles ou qu'elle l'ait volontairement choisi, cela n'avait pas grande influence sur la situation actuelle. Le fait important était que Bellatrix pensait que Harry était le Seigneur des Ténèbres et qu'elle lui obéirait.

C'était donc une ressource. Mais Bellatrix était affamée et morte aux neuf dixièmes...

'\emph{Oh, je me sens un peu mieux maintenant, comme c'est bizarre...'} 

Bellatrix avait dit cela de sa voix brisée lorsque Harry avait perdu tout contrôle sur son Patronus.

Harry pensa, et il n'aurait pas bien su dire \emph{pourquoi}  il pensait cela, c'était peut-être son esprit qui inventait des choses, mais... il semblait que ce que les Détraqueurs vous avaient pris il y a longtemps était perdu pour toujours. Mais ce que les Détraqueurs vous avaient pris \emph{récemment} , le Véritable Patronus pourrait le rendre. Comme la différence entre une tasse qu'on vide et une tasse inusitée qui disparaît. Bellatrix pourrait alors récupérer ce qu'elle avait perdu pendant la semaine précédente. Pas des souvenirs heureux, ceux-là avaient été dévorés il y a des années. Mais la force et la magie qui lui avait été prises pendant la semaine précédente, peut-être qu'elle avait récupéré cela. Comme l'équivalent d'une semaine de repos, d'une semaine pour réapprovisionner sa magie...

Harry regarda la forme animale du professeur Quirrell.

... peut-être assez pour un \emph{Innerver} .

Si l'idée de réveiller le professeur Quirrell était \emph{vraiment } si bonne que ça.

Une partie de désespoir regagna alors Harry. Il ne pouvait pas faire confiance au professeur Quirrell, il ne pouvait pas être certain qu'il serait sage de le réveiller, pas après ce qui venait de se produire.

\emph{Reprends-toi} , pensa Harry à sa propre intention, et il regarda l'Auror ratatiné.

Bellatrix pourrait \emph{aussi}  sortir un sortilège d'Oubliettes.

Ce serait la première étape de toute façon. On n'en était pas encore à une évasion réussie, et les Aurors \emph{sauraient}  que quelque chose d'étrange s'était produit, ils soupçonneraient sur le corps de Bellatrix et pratiqueraient peut-être une autopsie. Mais c'était une étape.

... et \emph{serait-il}  si difficile que ça de sortir d'Azkaban ? S'ils pouvaient atteindre le niveau supérieur assez vite, avant que l'Auror ne soit censé avoir rendu son rapport, avant que quiconque n'ait remarqué qu'il avait disparu, alors ils pourraient juste s'envoler par le trou que le professeur Quirrell avait fait et s'éloigner suffisamment pour activer le Portoloin que Harry avait déjà en sa possession (Le professeur Quirrell et Harry avaient tous deux des Portoloins et les deux étaient assez puissants pour transporter deux humains et un serpent en option. De même qu'avec leur départ doublement masqué de la chambre de Marie, le professeur Quirrell avait ajouté suffisamment de marges de sécurité à son plan pour réussir à impressionner même Harry.)

Bellatrix pourrait transporter la forme animale du professeur Quirrell, car Harry n'osait ni la toucher ni la faire léviter.

Harry se tourna et avança vivement vers les marches où Bellatrix attendait. Il pouvait sentir son moral remonter un peu. Le plan \emph{commençait}  à avoir l'air bon et il n'y avait pas de temps à perdre.

Que faire du professeur Quirrell, et même de Bellatrix, après que le Portoloin les ait amenés à l'endroit où ils étaient censés remettre cette dernière aux mains du Guérisseur psychiatrique... eh bien, Harry pourrait trouver ça en chemin. Il devrait probablement embobiner le Guérisseur pour lui faire faire quelque chose - ce qui allait demander de sérieux talents d'embobineur, et Harry n'était même pas sûr de ce qu'il \emph{voulait}  que le Guérisseur fasse - mais lui et Bellatrix devaient bouger \emph{maintenant} .

En faisant rapidement défiler la suite des événements dans son imagination, Harry vit que le problème principal surviendrait lorsqu'ils auraient atteint le toit. Le professeur Quirrell avait été celui qui s'était introduit, invisible, avant de jeter un charme de Confusion sur les dispositifs de surveillance capables de détecter des visiteurs dans le territoire aérien d'Azkaban qui avaient alors vu un paysage en boucle pendant quelques minutes. Le professeur Quirrell avait dit qu'il ne pouvait pas désillusionner le Patronus de Harry ; et si ce dernier \emph{éteignait}  le Patronus, les Détraqueurs remarqueraient que Bellatrix était partie et alerteraient les Aurors...

Le fil de pensée de Harry trébucha.

Parfois, 'Oh, merde' était très loin de suffire.
\par\noindent\rule{\textwidth}{0.4pt}
En dépit de l'adrénaline, les mains de Li restèrent stables lorsqu'il débloqua les barreaux de l'armoire à disparaître qui reliait Azkaban à une pièce bien gardée située à l'intérieur du département de la justice magique (une armoire à disparaître à sens unique, bien sûr. La sécurité autorisait quelques moyens rapides d'\emph{entrer}  à Azkaban, tous hautement restreints, mais \emph{aucun}  moyen d'en sortir rapidement).

Li fit un bon pas en arrière, dirigea sa baguette vers l'armoire et prononça l'incantation : "\emph{Harmonia Nectere Passus} ", et même pas une seconde plus tard -

La porte de l'armoire s'ouvrit grand, accompagnée d'un 'boum', et une sorcière costaude à la mâchoire carrée entra dans la pièce. Elle avait des cheveux gris qui formaient un cadre serré autour de sa tête. Elle ne portait ni insigne ni bijoux ni aucun ornement, seules les robes ordinaires d'un Auror lui semblaient digne de la revêtir : le directeur Amélia Bones, directrice du département de la justice magique, réputée être la seule sorcière de tout le DJM à avoir une chance lors d'un combat dans les règles contre Maugrey Fol-Oeil (non pas qu'aucun d'eux ne soient du genre à se battre dans les règles). Li avait entendu des rumeurs disant qu'Amélia pouvait transplaner dans l'enceinte du DJM, et c'était là le genre de chose qui donnait naissance à de telles rumeurs : il avait sonné à l'alarme à peinte cinquante secondes plus tôt.

"Dans les airs, maintenant !" aboya Amélia par-dessus son épaule à l'intention du trio féminin d'Aurors qui la suivait équipé de balais de police, ils avaient dû tous se tenir serrés là en attendant que Li active l'armoire. "Je veux une meilleure couverture aérienne des lieux ! Et assurez-vous de garder vos sortilèges anti-désillusion activés !" Puis elle tourna sa tête vers lui. "Auror Li, au rapport ! Savons-nous déjà comment ils sont entrés ?"

Un autre trio porteur de balais se matérialisa dans l'armoire à disparaître et en sortit à grandes enjambées alors même que Li commençait à parler.

Ils furent suivi par un trio de tireurs d'élite de baguette magique en équipement de combat complet.

Puis un autre trio de tireurs d'élite.

Puis une autre équipe sur balais.
\par\noindent\rule{\textwidth}{0.4pt}
Lorsque Harry parvint aux escaliers, la forme émaciée nommée Bellatrix Black reposait là, immobile, les yeux fermé, et lorsque Harry demanda d'un chuchotement aigu et froid si elle était éveillée, il ne reçut aucune réponse.

Un bref soubresaut de peur fut contré par la pensée que le professeur Quirrell l'avait assommée pour l'empêcher d'entendre le misérable serviteur du Seigneur des Ténèbres soudain devenir un criminel endurci puis un mage de combat expert. C'était mieux ainsi, car elle n'avait donc pas non plus entendu la voix de Harry dire 'Expecto Patronum'.

Harry enleva la capuche de la Cape, pointa sa baguette vers Bellatrix et murmura aussi gentiment qu'il le pouvait : "\emph{Innerver} ".

À en voir la façon dont le corps de Bellatrix s'était légèrement arqué, Harry songea qu'il n'était pas tout à fait parvenu au niveau de douceur recherché.

Les orbites noires s'ouvrirent.

"Chère Bella," dit Harry de sa voix froide et aiguë, "j'ai bien peur que nous ayons un léger problème. As-tu suffisamment récupéré pour pratiquer des sortilèges simples ?"

Il y eut une pause, puis Bellatrix hocha son pâle visage.

"Très bien," dit Harry d'un ton sec. "Je ne vais pas te demander de le faire sans aide, chère Bella, mais j'ai peur qu'il te faille marcher." Il dirigea sa baguette vers elle. "\emph{Wingardium Leviosa} ."

Harry garda le niveau d'énergie du sort à un niveau lui permettant de le maintenir un bon moment, et cela suffit pourtant à soulever environ les deux tiers du poids actuel de cette dernière. Elle était... mince.

Lentement, comme si ça avait été la première fois depuis des années, Bellatrix Black se mit sur pied.
\par\noindent\rule{\textwidth}{0.4pt}
Amélia déambulait dans la salle de garde avec l'Auror Li et son blaireau derrière elle. Elle avait fait faire une pirouette à son Retourneur de Temps dès qu'elle avait entendu l'alarme et elle avait ensuite passé une heure fort tendue à préparer ses forces à leur arrivée. On ne pouvait pas créer de \emph{boucle}  temporelle depuis l'intérieur d'Azkaban, car son futur ne pouvait pas interagir avec son passé, elle n'avait pas donc pas pu arriver avant que le DJM n'ait reçu le message, mais elle aurait dû arriver à temps pour...

Son regard se dirigea immédiatement vers le corps nu et à l'air très mort qui flottait par-delà la fenêtre.

"Où est Bellatrix Black ?" demanda-t-elle d'un ton qui ne tolérait pas de refus. La créature, qui était faite de peur, semblait n'en inspirer aucune à Amélia.

Mais même le sang de celle-ci se figea l'espace d'un instant lorsque le corps écarta ses lèvres et gargouilla : "\emph{Ne sait pas} ."
\par\noindent\rule{\textwidth}{0.4pt}
De nouveau totalement invisible, Harry regarda Bellatrix se pencher lentement, prendre la baguette du professeur Quirrell (que Harry n'osait pas toucher) et se redresser tout aussi lentement.

Puis elle pointa sa baguette vers le serpent et dit, d'une voix précise mais murmurée : "\emph{Innerver} ."

Le serpent ne broncha pas.

"Devrais-je essayer de nouveau, seigneur ?" chuchota-t-elle.

"Non," dit Harry. Il ravala son malaise. Lorsqu'il s'était rendu compte que les Détraqueurs avaient probablement déjà alerté les Aurors, il avait décidé de jouer le tout pour le tout et d'essayer de réanimer le professeur Quirrell. Sa voix haute perchée et glaciale continua, imperturbable : "Te penses-tu capable d'opérer un sortilège d'Oubliettes, chère Bella ?"

Bella marqua une pause puis dit en hésitant : "Je pense que oui, seigneur."

"Élimine la dernière demi-heure de souvenirs de cet Auror," ordonna Harry. Il avait un peu réfléchi à son envie de fournir une explication à cet ordre, à ce qu'il dirait si Bellatrix demandait pourquoi ils ne se contentaient pas de le tuer, auquel cas Harry expliquerait qu'ils prétendaient appartenir à une autre faction puis lui dirait de se la fermer -

Mais Bellatrix pointa simplement sa baguette vers l'Auror, demeura silencieuse pendant un moment, et murmura enfin : "\emph{Oubliettes} ."

Elle chancela alors, mais elle ne tomba pas.

"Très bien, ma chère Bella," dit Harry, et il gloussa légèrement. "Et je vais te demander de porter ce serpent."

À nouveau la femme ne dit rien, ne demanda aucune explication, ne demanda pas pourquoi Harry ou le lanceur de Patronus apparemment invisible ne pouvait pas le faire. Elle se contenta de se traîner jusqu'à l'endroit où le serpent reposait, de s'incliner lentement, de le ramasser, et de l'étendre par-dessus son épaule.

(Une toute petite partie de Harry remarqua qu'il était très \emph{relaxant}  d'avoir un laquais qui se contentait de suivre les ordres d'une façon aussi inconditionnelle, et cette partie alla même jusqu'à penser qu'il pourrait carrément s'habituer à avoir un laquais tel que Bellatrix avant d'être réduite au silence par les cris du reste de son être mortellement offensé).

"Suis," ordonna le garçon à son laquais, et il commença à marcher.
\par\noindent\rule{\textwidth}{0.4pt}
On commençait à se sentir à l'étroit dans la salle de garde, presque trop à l'étroit pour pouvoir respirer, même s'il restait de l'espace autour d'Amélia elle-même ; si pouvoir respirer signifiait que la directrice Bones allait se sentir à l'étroit, il valait mieux ne pas respirer.

Amélia regarda Ora, qui tripotait le miroir de l'Auror McCusker. "Spécialiste Weinbach," aboya-t-elle, ce qui fit sursauter la sorcière. "Une réponse du miroir de Une-Main ?"

"Aucune," dit Ora d'un ton nerveux, "c'est... je veux dire, il doit être brouillé, pas détruit, brouillé avec précaution, car il n'a déclenché aucune des alarmes, mais la ligne est tellement morte que le miroir pourrait être cassé et qu'on ne verrait pas la différence..."

L'expression d'Amélia ne changea pas mais la partie d'elle qui faisait déjà le deuil de Une-Main devint un peu plus triste et un peu plus en colère. Sept mois, il lui avait resté sept mois avant sa retraite, après un siècle entier de service. Elle se souvenait de lui, jeune Auror enthousiaste, il y a si longtemps, et pendant toute sa carrière il avait servi le DJM avait une loyauté parfaite, du moins lorsqu'il s'agissait de choses vraiment importantes...

Quelqu'un allait \emph{brûler}  pour ça.

Le Détraqueur flottait toujours de l'autre côté de la fenêtre et projetait son inutile ombre de malheur sur leur travail en cours ; tout ce que la créature pouvait faire était de régurgiter son ignorance ou de ne pas répondre du tout lorsqu'on lui demandait des choses comme : 'Bellatrix Black s'est-elle échappée ?' et 'Pourquoi ne pouvez-vous pas la trouver ?' et 'Comment est-elle cachée ?' Amélia commençait à s'inquiéter de la possibilité que les criminels soient déjà partis quand -

"Nous avons trouvé un trou dans le toit au-dessus de la spirale C !" cria quelqu'un depuis l'embrasure de la porte. "Encore ouvert, mesures de contournement des barrières encore actives !"

Les lèvres d'Amélia se retroussèrent pour former un sourire qui rappelait un loup ouvrant sa mâchoire pour manger.

Bellatrix Black était toujours à Azkaban.

Et à Azkaban elle resterait pour toujours.

Amélia s'avança vivement vers la fenêtre, ignorant maintenant le Détraqueur, et regarda le ciel au-dessus d'eux pour vérifier de ses propres yeux la présence des balais en patrouille. Elle n'avait pas une vue complète du ciel, mais elle put voir dix balais passer le long d'un itinéraire de patrouille, et celait aurait déjà dû suffire à attraper n'importe qui, mais elle comptait bien mettre le plus de balais possible là-haut. Ses Aurors étaient équipés du balais de course le plus rapide du marché : le Nimbus 2000 ; pas de course-poursuites ratées pour \emph{son}  équipe.

Amélia se détourna de la fenêtre et fit la moue. Ça commençait à être ridicule, la pièce était bien trop petite, et les deux tiers des gens présents n'avaient pas \emph{besoin}  d'être ici, ils avaient juste \emph{envie}  d'être proche du cœur de l'action. S'il y avait une chose qu'Amélia ne pouvait pas tolérer, c'était les gens qui faisaient ce qu'ils voulaient au lieu de faire ce qu'il fallait.

"Très bien, vous tous !" mugit-elle à leur intention. "On arrête de traîner ici et on commence à sécuriser le niveau supérieur de chaque spirale ! C'est ça," répondit-elle aux airs surpris, "les trois spirales ! Ils pourraient percer un tunnel dans le sol ou le plafond pour passer de l'un à l'autre, au cas où vous ne l'auriez pas déjà deviné ! On descend niveau par niveau jusqu'à les avoir attrapés ! Je prends la spirale C, Scrimgeour, tu es sur la B..." elle s'interrompit alors, se rappelant que Maugrey avait prit sa retraite l'année précédente, qui pouvait-elle... "Shacklebolt, tu es sur la A, et prends le plus fort des autres combattants avec toi ! Vérifiez chaque bloc de cellules que vous croisez, regardez sous les couvertures, lancez la panoplie complète des charmes de détection dans chaque couloir ! Personne ne quitte Azkaban jusqu'à ce que les criminels soient attrapés, personne ! Et..." Ils regardèrent Amélia avec surprise alors que sa voix restait en suspens.

\emph{Les criminels avaient inventé un moyen d'empêcher les Détraqueurs de trouver Bellatrix Black.} 

Ça aurait dû être \emph{impossible} ;

Contempler cela lui glaça les sangs. C'était comme...

Amélia prit une profonde inspiration et parla de nouveau, de la voix d'acier d'une commandante. "Et lorsque vous les attrapez, assurez-vous bien que ce sont les vrais criminels, pas des gens de chez nous qu'on aurait obligé à prendre du Polynectar. Si quelqu'un a comportement bizarre, vérifiez qu'ils ne sont pas sous Imperium. Gardez-vous les uns-les autres dans votre champ de vision en permanence. Ne présumez pas qu'un Auror en uniforme est ami si vous ne reconnaissez pas le visage." Elle se tourna vers le spécialiste des communications. "Dites au balais que si l'un d'eux s'en va sans raison, \emph{la moitié restante}  doit le pourchasser pendant que le \emph{reste}  continue de patrouiller. Et changez les harmoniques partout où vous pouvez le faire, ils ont peut-être volé nos clés." Puis de nouveau à l'intention du reste de la pièce. "Aucun Auror n'est au-dessus de tout soupçon à moins qu'il ne lui reste plus aucune famille à menacer."

Elle les vit alors, les regards froids qui gagnaient les visages plus âgés, le tressaillement chez quelques Aurors plus jeunes, et elle sut qu'ils comprenaient.

Mais elle le dit, juste pour être sûre.

"Aujourd'hui, nous participons tous à la vieille Guerre des Sorciers. Ce n'est pas parce que Vous-Savez-Qui est mort que les Mangemorts ont oublié ses tours. Maintenant, \emph{allez}  !"
\par\noindent\rule{\textwidth}{0.4pt}
Harry marchait en silence à travers les couloirs éclairés au gaz, invisible entre Bellatrix et la silhouette d'argent qui les suivait, et il essayait de trouver un meilleur plan.

Au début, lorsqu'il s'était rendu compte que les Aurors étaient probablement alertés, et qu'en plus le professeur Quirrell ne se réveillait pas...

Ça \emph{s'était}  figé là-haut pendant une seconde.

Et c'était resté figé, même après qu'il ait poussé Bellatrix et lui-même à continuer de descendre, afin de gagner le plus de temps possible ; Harry s'était dit que les Aurors commenceraient en haut et descendraient niveau par niveau. Ils pouvaient se permettre de se déplacer lentement et sûrement ; ils savaient que leur proie n'avait aucune échappatoire.

Harry n'avait pas réussi à trouver par où il pourrait sortir.

Jusqu'à ce qu'il se dise : \emph{bon, si c'était juste un jeu de guerre, que ferait le général Chaos ?} 

Et une réponse avait immédiatement suivi.

Puis Harry avait pensé : \emph{mais si c'est si facile que } ça\emph{, pourquoi personne ne s'est-il jamais échappé d'Azkaban ?} 

Et après avoir vu le problème potentiel : \emph{très bien, que ferait le général Chaos pour régler } ça \emph{?} 

Ce sur quoi le général Chaos avait fourni un amendement au plan initial.

C'était...

C'était la chose la plus follement \emph{Gryffondor}  que Harry ait jamais...

Alors il essayait maintenant de trouver un \emph{meilleur}  plan, et la pêche n'était pas bonne.

\emph{On chipote, on chipote} , dit Gryffondor. \emph{Qui se plaignait de ne pas avoir de plan il y a une minute ? Tu devrais être content qu'on ait quelque chose, monsieur Maintenant-On-Est-Foutus.} 

"Seigneur," murmura Bellatrix d'un ton hésitant, alors même qu'elle abordait la prochaine volée de marches descendantes, "vais-je retourner dans ma cellule, seigneur ?"

Le cerveau de Harry était distrait, et il mit donc un bon moment à absorber les mots, puis un autre pour comprendre leur horreur, et pendant ce temps, Bellatrix continua de parler :

"Je préférerais... s'il vous plaît, seigneur, je préférerais vraiment mourir," dit sa voix. Puis, d'une voix plus petite, un chuchotement à peine présent : "mais j'y retournerais si vous me le demandez, seigneur..."

En pilote automatique, la voix de Harry siffla : "Nous ne retournons pas à ta cellule". Aucun de ses sentiments ne fut autorisé à atteindre son visage.

\emph{Euh...}  dit Poufsouffle. \emph{Est-ce que tu viens vraiment de penser : 'Tu devrais travailler pour moi, parce que } moi\emph{, je saurais t'apprécier' ?} 

\emph{Une pierre répondrait à une loyauté pareille, } pensa Harry. \emph{Même ce n'est que par erreur que j'en suis le récipiendaire, je ne peux pas m'empêcher de -} 

\emph{Elle est l'assassin et le tortureur loyal du Seigneur des Ténèbres, et la prétendue raison à sa loyauté est qu'une fille innocente a été brisée en milles morceaux et a été utilisée comme matériau brut pour la fabriquer, } dit Poufsouffle. \emph{Est-ce que tu as oublié ?} 

\emph{Si quelqu'un me montre autant de loyauté, même si c'est par erreur, il y a une partie de moi qui ne peut pas s'empêcher de ressentir quelque chose. Le Seigneur des Ténèbres doit avoir été... } maléfique\emph{ ne semble pas assez fort, il doit avoir été } vide\emph{... pour ne pas apprécier sa loyauté, qu'elle soit artificielle ou non.} 

Les meilleures parties de Harry n'avait pas grand chose à répondre à cela.

Et c'est alors que Harry l'entendit.

Elle était faible et grandissait à chacun de leurs pas.

La voix d'une femme, distante, indistincte.

Automatiquement, les oreilles de Harry s'efforcèrent de déchiffrer les mots.

"...s'il te plaît ne..."

"...ne voulais pas..."

"...meurs pas..."

Puis son cerveau sut \emph{qui}  il entendait, et au même moment comprit \emph{ce}  qu'il entendait.

Parce que le professeur Quirrell n'était plus là pour maintenir le silence et qu'Azkaban n'était en fait pas silencieuse.

La faible voix de la femme. Elle répétait :

"Non, je ne voulais pas, ne meurs pas s'il te plaît !"

"Non, je ne voulais pas, ne meurs pas s'il te plaît !"

Elle grandissait à chacun des pas de Harry, il pouvait entendre l'émotion dans sa voix à présent, l'horreur, le remords, le désespoir de...

"Non, je ne voulais pas, ne meurs pas s'il te plaît !"

...le pire souvenir de la femme, répété encore et encore...

"Non, je ne voulais pas, ne meurs pas s'il te plaît !"

...le meurtre qui l'avait envoyé à Azkaban...

"Non, je ne voulais pas, ne meurs pas s'il te plaît !"

...où elle était condamnée par les Détraqueurs à voir la personne qu'elle avait tué à mourir et à mourir et à mourir encore dans une boucle infinie. Elle avait dû être mise à Azkaban récemment, à en juger par la quantité de vie encore présente dans sa voix.

La pensée vint alors à Harry que, lorsque le professeur Quirrell avait passé ces portes, lorsqu'il avait entendu ces sons, il n'avait pas donné le moindre signe de trouble ; et Harry aurait dit que c'était la marque irrévocable d'une personne mauvaise si ses propres lèvres n'étaient pas restés silencieuses en présence de Bellatrix, si sa respiration n'était pas restée régulière, alors qu'au fond de lui quelque chose hurlait et hurlait et hurlait.

Le Patronus s'intensifia, il n'échappa pas au contrôle de Harry mais il s'intensifia à chacun de ses pas.

Il s'intensifia encore, à mesure que Harry et Bellatrix descendaient les escaliers, et elle trébucha et Harry lui offrit son bras gauche, sortit de la Cape, bravant la sensation funeste née de la proximité du serpent enroulé autour du cou de Bellatrix. Un air surpris apparut sur le visage de cette dernière, mais elle accepta le bras et ne dit rien.

Aider Bellatrix aida Harry, mais ce ne fut pas suffisant.

Pas lorsqu'il vit l'immense porte de métal au centre du couloir de ce niveau.

Pas lorsqu'ils s'approchèrent et que la voix de la femme se tut, parce qu'un Patronus était maintenant là et qu'elle ne revivait plus le pire de ses souvenirs.

\emph{Bien} , dit une voix à l'intérieur de lui. \emph{C'était la première étape} .

Les pas de Harry l'emmenèrent inévitablement vers la porte de métal.

Et...

\emph{Maintenant, déverrouille la porte -} 

...Harry continua de marcher...

\emph{Tu crois que tu es en train de faire quoi ? Reviens et sors-la de là !} 

...continua de marcher...

\emph{Sauve-la ! Qu'est-ce que tu fais ! Elle souffre TU DOIS LA SAUVER !} 

Le Portoloin que portait Harry pouvait transporter deux humains, seulement deux, accompagnés d'un serpent. S'ils avaient aussi eu le Portoloin du professeur Quirrell... mais ils ne l'avaient \emph{pas} , c'est la forme humaine du professeur Quirrell qui le portait, il était impossible de l'obtenir... Harry pouvait sauver une seule personne aujourd'hui, et il y avait une seule personne au niveau le plus bas d'Azkaban, désespérément dans le besoin...

"NE PARS PAS !" De derrière la porte de métal, la voix lui parvint dans un cri. "Non, non, non, ne pars pas, ne l'emmène pas, non non non -"

Il y avait une lumière dans le couloir, et elle s'intensifiait.

"S'il te plaît," sanglota la voix de la femme, "s'il te plaît, je ne sais plus comment mes enfants s'appellent -"

"Assieds-toi, Bella," dit la voix de Harry, et il parvint, sans savoir comment, à prononcer ces mots d'un chuchotement glacé, "je dois m'occuper de cela," le sortilège de lévitation diminua puis se désactiva lorsque Bella s'assit avec obéissance, sa silhouette squelettique formant une tache sombre sur le couloir de plus en plus lumineux.

\emph{Je mourrai} , pensa Harry.

La lumière continua d'augmenter.

Après tout, il n'était pas \emph{certain}  que Harry mourrait.

C'était seulement la probabilité d'une mort, et certaines choses ne valaient-elles pas la possibilité de mourir ?

La lumière continua d'augmenter, un Patronus plus grand commençait à se former, la silhouette humaine brillante devenait indistinct, perdue dans l'air brûlant, et la vie de Harry s'en allait nourrir le feu.

\emph{Si j'anéantis tous les Détraqueurs, alors même si je vis, ils sauront que c'était moi, que c'est moi le responsable... je perdrai mes soutiens, je perdrai la guerre...} 

\emph{Ah ouais ?}  dit la voix intérieur qui le poussait à continuer. \emph{Après que tu ais détruit tous les Détraqueurs d'Azkaban ? Je pense plutôt que ça tendrait plutôt à prouver que tu es qualifié pour être un Seigneur de la Lumière, alors SAUVE LA SAUVE LA TU DOIS LA SAUVER -} 

La forme humanoïde n'était plus discernable.

Le couloir n'était plus visible.

Sous la Cape, le corps de Harry était invisible.

Il n'y avait plus qu'un point de vue décorporé au milieu d'une étendue de lumière d'argent infinie.

Harry pouvait sentir la vie qui le quittait et alimentait le sortilège ; au loin, il pouvait sentir les ombres de la Mort qui commençaient à s'effilocher.

\emph{Je comptais accomplir plus que ça dans ma vie... j'allais combattre le Seigneur des Ténèbres, j'allais unir les mondes Moldu et Magique.} 

Les nobles buts semblaient très distants, très abstraits comparés à une femme suppliant qu'on l'aide, et il n'était pas \emph{certain}  que Harry ferait jamais quelque chose de plus important que cet acte, cet acte unique dont il était capable, ici et maintenant.

Et avec ce qui aurait pu être son dernier souffle, Harry pensa :

\emph{Il y a d'autres Détraqueurs, probablement d'autres Azkabans... si je veux faire ça, je devrais le faire plus près de l'abysse centrale, comme ça, ça me prendra moins de vie, ce qui augmentera la probabilité que je survive et que je détruise d'autres Détraqueurs... même en admettant que c'est le choix optimal, s'il y a un bon moment et un bon endroit pour le faire, ce n'est pas ici et maintenant, CE N'EST PAS ICI ET MAINTENANT !} 

\emph{Quoi ?}  dit l'autre partie de lui d'un ton indigné tout en cherchant un contre-argument qui n'existait pas -

La lumière s'éclipsa lentement, à mesure que Harry se concentrait sur ce fait indiscutable, sur cette vérité évidente qu'ils n'étaient pas à l'endroit optimal, que le bon moment ne \emph{pouvait pas}  être \emph{maintenant} ...

La lumière s'éclipsa lentement.

Une partie de la vie de Harry reflua vers lui.

Une partie avait été perdue sous la forme de radiation.

Mais il lui en restait assez pour rester sur pied et pour maintenir une certaine luminosité dans la silhouette d'argent ; et lorsque son bras armé se leva et que sa voix murmura "Wingardium Leviosa", la magie obéit, s'écoula hors de lui et aida Bellatrix à se remettre sur pied (car ce n'était pas sa magie qu'il avait dépensée, sa magie n'avait jamais été le combustible du Patronus).

\emph{Je jure,}  pensa Harry alors qu'il respirait aussi régulièrement que possible en présence de Bellatrix, alors que des larmes coulaient le long de ses joues invisibles, \emph{je jure sur ma vie et ma magie et mon art de rationaliste, je jure sur tout ce que je tiens pour sacré et sur tous mes souvenirs heureux, je fais le serment qu'un jour je mettrai fin à ce lieu, s'il vous plaît, s'il vous plaît, puissé-je être pardonné...} 

Et ils continuèrent tous deux, tandis que la voix d'une meurtrière hurlait et suppliait quelqu'un de revenir et de la sauver.

Il y aurait dut y avoir plus de temps, il y aurait dut y avoir une cérémonie pour le sacrifice que Harry avait fait de cette partie de lui, mais Bellatrix était à côté, et il ne pouvait que continuer sans s'interrompre, sans rien dire, en respirant à un rythme régulier.

Et il continua donc, laissant une partie de lui en arrière. Elle demeurerait pour toujours en ce temps et en ce lieu, il le savait. Même après qu'il soit un jour revenu avec une compagnie de lanceurs de Véritable Patronus et qu'ils aient détruit tous les Détraqueurs présents. Même s'il faisait fondre le bâtiment triangulaire et qu'il brûlait l'île jusqu'à ce que la mer la recouvre, ne laissant aucune trace pouvant indiquer qu'un tel endroit avait jamais existé. Même alors, il ne la récupérerait pas.
\par\noindent\rule{\textwidth}{0.4pt}
Le troupeau de créatures lumineuses s'arrêta, regarda vers le bas, puis se remit à patrouiller dans les couloirs de métal comme si rien ne s'était passé.

"Comme la dernière fois ?" lâcha la directrice Bones à l'intention de l'Auror Li, et le jeune Auror répondit : "Oui m'dame."

La directrice fit une nouvelle requête afin de voir si les Détraqueurs pouvaient maintenant trouver leur cible, et elle n'eut pas l'air surprise en recevant un réponse négative quelques secondes plus tard.

Emmeline Vance se sentait déchirée entre deux allégeances.

Elle n'était plus un membre de l'Ordre du Phénix, car il s'était dissolu après la fin de la dernière guerre. Et pendant la guerre elle avait su, ils avaient tous su, que le directeur Croupton approuvait leurs batailles secrètes en silence.

Bones n'était pas Croupton.

Mais maintenant, ils traquaient Bellatrix Black, et elle avait été un Mangemort, et c'était certainement des Mangemorts qui étaient en train de la sauver. Leur Patronus se comportaient étrangement - toutes les créatures lumineuses s'arrêtaient et regardaient en bas avant de recommencer à suivre leur maître. Et les Détraqueurs ne pouvaient pas trouver leur cible.

Il lui semblait que le moment était extrêmement opportun pour consulter Albus Dumbledore.

Devait-elle juste \emph{suggérer}  à Bones de contacter Dumbledore ? Mais si elle ne l'avait pas déjà fait...

Emmeline hésita un moment, probablement trop longtemps, puis finit par se décider. \emph{Au diable tout ça, } pensa-t-elle. \emph{On est tous dans le même camp, on doit se serrer les coudes, que Bones apprécie ou pas.} 

Par une pensée, son moineau d'argent voleta jusqu'à son épaule.

"Ralentis pour garder nos arrières," murmura-t-elle doucement, presque sans remuer ses lèvres, "attends jusqu'à ce que personne ne te regarde directement, puis vas voir Albus Dumbledore. S'il n'est pas déjà seul, attends qu'il le soit. Et dis-lui cela : Bellatrix Black s'échappe d'Azkaban et les Détraqueurs ne peuvent pas la trouver."

