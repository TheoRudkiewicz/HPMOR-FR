
\chapter{L'Expérience de Prison de Stanford, Fin}

Minerva leva les yeux vers l'horloge et observa les nombres d'argents, les aiguilles dorées, leur mouvement saccadé. Des Moldus avaient inventé cela et les sorciers ne s'étaient pas souciés de savoir quelle heure il était avant qu'ils ne le fassent. Après sa construction de Poudlard on avait utilisé des cloches déclenchées par des sabliers. C'était une de ces choses dont les Puristes du Sang auraient aimé qu'elles soient fausses, et que Minerva savait donc très bien.

Elle avait reçu une mention d'excellence pour son A.S.P.I.C. en étude des Moldus, ce qui la couvrait aujourd'hui de honte au vu de l'étendue de son ignorance. Plus jeune déjà, elle s'était rendue compte que le cours était une imposture enseignée par un Puriste du Sang sous prétexte que les nés-Moldus n'étaient pas capables de juger de ce que les nés-sorciers avaient besoin d'apprendre sur le sujet, en réalité parce que le conseil d'administration de Poudlard n'appréciait pas du tout les Moldus. Mais elle se souvenait avec tristesse qu'à dix-sept ans, la mention d'excellence avait eu bien plus d'importance que le reste.

\emph{Si Harry Potter et Voldemort combattent au moyen d'armes Moldues, le monde ne sera plus qu'un champ de flammes !} 

Si elle n'arrivait pas à se le représenter, c'est parce qu'elle n'arrivait pas à imaginer Harry Potter se battant contre Vous-Savez-Qui.

Elle avait fait face au Seigneur des Ténèbres quatre fois et avait survécu à chacune, trois fois protégée par Albus et une fois avec Maugrey à ses côtés. Elle se souvenait du visage brisé semblable à celui d'un serpent, des écailles visibles ici ou là sur sa peau, des yeux rouges lumineux et de la voix qui riait d'un sifflement aigu et ne promettait rien d'autre que de la cruauté et de la douleur : un monstre pur, absolu.

Et il lui était si simple de se représenter Harry Potter, le visage radieux d'un jeune garçon qui oscillait entre le plus grand sérieux pour ce qui était dérisoire et la plus grande dérision pour ce qui était sérieux.

Il était trop douloureux de les imaginer face à face, baguettes brandies.

Ils n'avaient pas le droit, ils n'avaient aucun droit de mettre ça sur les épaules d'un garçon de onze ans. Elle savait ce que le directeur venait de décider au sujet de Harry car ça avait été à elle de prendre les disposition nécessaires ; et s'il s'était s'agit d'elle au même age elle serait entrée en rage, elle aurait hurlé et pleuré, elle aurait été inconsolable pendant des semaines, et...

\emph{Il n'est pas un élève de première année comme les autres,}  avait dit Albus. \emph{Il a été marqué comme l'égal du Seigneur des Ténèbres, et il possède un pouvoir que le Seigneur des Ténèbres ignore.} 

Terrible voix creuse émanant avec force de la gorge de Sybille Trelawney, la véritable prophétie originelle résonna une fois de plus dans son esprit. Elle avait l'impression que le sens n'était pas celui que le directeur avait cru y lire mais il était ne parvenait pas à formuler la nuance.

Cela semblait pourtant toujours vrai : s'il existait sur toute la Terre un enfant de onze ans capable de supporter de fardeau, c'était celui qui approchait son bureau à l'instant même. Et si elle disait quoi que ce soit qui s'approche de 'pauvre Harry' devant lui... eh bien, il n'aimerait pas ça.

\emph{Donc maintenant je dois trouver un moyen de tuer un Seigneur des Ténèbres immortel,}  avait-il dit le jour où il l'avait appris. \emph{J'aurais vraiment aimé que vous me disiez ça avant qu'on commence à faire du shopping.} ..

Cela faisait assez longtemps qu'elle était directrice de Gryffondor et elle avait vu assez de ses amis mourir pour savoir que certaines personnes personne ne pouvaient pas être mises à l'abri de leur destinée héroïque.

On frappa à la porte et le professeur McGonagall dit : "Entrez."

Lorsque Harry entra, son visage avait la même expression froide et alerte qu'elle avait vue à chez Marie ; et elle se demanda l'espace d'un instant s'il avait revêtu ce masque, ce soi, pendant toute la journée.

Le jeune garçon s'assit sur la chaise qui faisait face au bureau et dit : "Le moment est-il venu où l'on me dit ce qui se passe ?" Les mots étaient neutres, au contraire du tranchant qui aurait dû accompagner l'expression présente sur son visage.

Les sourcils du professeur McGonagall s'élevèrent sous le coup de la surprise avant qu'elle ne puisse les retenir et elle dit : "Le directeur ne vous a rien dit, M. Potter ?"

Le garçon secoua la tête. "Seulement qu'il avait reçu un avertissement selon lequel je pourrais être en danger mais que j'étais à présent en sécurité."

Minerva avait du mal à croiser son regard. Comment pouvaient-ils lui \emph{faire}  ça, comment pouvaient-ils tout remettre entre les mains d'un garçon de onze ans, cette guerre, ce destin, cette prophétie... et ils ne lui faisaient même pas \emph{confiance} ...

Elle s'obligea à regarder Harry en face et vit que ses yeux verts étaient calmes, braqués sur elle.

"Professeur McGonagall ?" dit le garçon d'une voix basse.

"M. Potter," dit-elle, "j'ai peur que ce ne soit pas à moi de vous l'expliquer, mais si après ce qui va suivre le directeur ne vous dit \emph{toujours}  rien, vous pourrez revenir me revoir et j'irai lui crier dessus pour vous."

Les yeux du garçon s'écarquillèrent, une partie du véritable Harry qui se révélait par une faille avant que le masque composé ne se remette en place.

"Quoi qu'il en soit," dit-elle vivement. "Je suis navrée du dérangement, M. Potter, mais je dois vous demander d'utiliser votre Retourneur de Temps pour revenir six en heures en arrière, à trois heures de l'après-midi afin de donner le message suivant au professeur Flitwick : Argent sur le trois. Demandez au professeur de noter l'heure à laquelle vous lui avez donné le message. Et le directeur souhaite vous voir ensuite, quand cela vous conviendra."

Il y eut un silence.

Puis le garçon dit : "Suis-je donc soupçonné d'avoir fait un usage abusif de mon Retourneur de Temps ?"

"Pas par \emph{moi}  !" dit le professeur McGonagall avait hâte. "Je \emph{suis}  navrée pour le dérangement, M. Potter."

Il y eut un autre silence et le jeune garçon haussa les épaules. "Cela va interférer avec mon rythme de sommeil mais je suppose qu'on ne peut rien y faire. Merci de faire savoir aux elfes de maison qu'ils devront obtempérer si je demande un petit-déjeuner très matinal, disons à trois heures demain matin."

"Bien sûr, M. Potter," dit-elle. "Merci de votre compréhension."

Le garçon se leva de sa chaise et lui offrit un salut formel avant de se glisser hors de la pièce, sa main allant déjà sous sa chemise, là où son Retourneur de Temps l'attendait ; et elle faillit appeler : \emph{Harry !}  ; mais elle n'aurait pas su quoi dire ensuite.

Au lieu de cela elle attendit, les yeux sur l'horloge.

Combien de temps devait-elle attendre avant que Harry Potter ne remonte dans le temps ?

Elle n'avait en fait pas besoin d'attendre du tout ; s'il l'avait fait, alors ça avait déjà eu lieu...

Minerva sut alors qu'elle différait parce qu'elle était nerveuse et fut attristée de s'en rendre compte. Une espièglerie, oui, une espièglerie impensable, ineffable, avec toute la prudence et la prévoyance d'une pierre en chute libre - elle ne savait pas comment le garçon était parvenu à tromper le Choixpeau pour qu'il ne le répartisse pas à Gryffondor, où il aurait évidemment dû être - mais rien de ténébreux, rien de malfaisant, jamais. Sous l'espièglerie se cachait sa bonté, aussi profonde et véritable que celle des jumeaux Weasley, même si un sortilège de Doloris n'aurait pas suffit à faire qu'elle le dise à voix haute.

"\emph{Expecto Patronum} ," dit-elle, puis : "Vas voir le professeur Flitwick et rapporte sa réponse après lui avoir demandé ceci : 'M. Potter t'a-t-il donné un message de moi, quel était ce message, et quand l'as-tu reçu ?'"
\par\noindent\rule{\textwidth}{0.4pt}
Une heure plus tôt, après avoir utilisé la dernière heure de son Retourneur de Temps et avoir enfilé sa Cape d'Invisibilité, Harry remit le sablier sous sa chemise.

Et il se dirigea vers les donjons de Serpentard aussi vite que ses jambes invisibles lui permettaient de le faire sans néanmoins courir. La directrice adjointe était heureusement déjà à un étage de lui...

Quelques escaliers dont les marchent furent descendues deux à deux mais pas trois à trois plus tard, Harry s'arrêta dans un couloir au fond duquel se trouvait l'entrée des dortoirs de Serpentard.

Il prit un morceau de parchemin (et non pas de papier), une Plume à Papote (et non pas un stylo) de sa bourse et dit à la plume : "Écris exactement les lettres suivantes : Z-P-N-S-Y N-E-T-R-A-G-F-H-E-Y-R-G-E-B-V-F."

Il existait deux types de codes cryptographiques : ceux qui empêchaient votre petit frère de lire vos messages et ceux qui empêchaient d'importants gouvernements de les lire ; il s'agissait là du premier type, mais c'était mieux que rien. En théorie personne n'était censé le lire mais si cela arrivait, cette personne ne se souviendrait de rien d'intéressant à moins d'avoir auparavant appris ce qu'était la cryptographie.

Harry mit le message dans une enveloppe en parchemin, et de sa baguette il la cacheta d'un peu de cire verte.

En principe, Harry aurait bien sûr pu faire tout cela des heures plus tôt, mais attendre d'avoir \emph{entendu}  le message des lèvres du professeur McGonagall lui donnait étrangement l'impression de moins Jouer avec le Temps.

Il mit ensuite cette enveloppe dans une seconde enveloppe qui contenait déjà une autre feuille de papier porteuse d'autres instructions ainsi que de cinq Mornilles d'argent.

Il ferma cette enveloppe (à l'extérieur de laquelle un nom était déjà inscrit), la cacheta d'un peu plus de cire verte, et plaça une dernière Mornille sur ce sceau.

Puis il mit \emph{cette}  enveloppe dans une dernière enveloppe, sur laquelle était écrit en grandes lettres : "Merry Tavington".

Puis il se pencha au tournant du couloir pour regarder le portrait grincheux et gardien de la porte des dortoirs de Serpentard, qui attendait ; et comme Harry ne souhaitait pas que le portrait se souvienne de n'avoir pas vu quelqu'un d'invisible, il fit léviter l'enveloppe jusqu'à l'homme grincheux et la pressa contre lui.

L'homme renfrogné baissa les yeux vers l'enveloppe, l'observa à travers un monocle, soupira et sa tourna pour faire face à l'intérieur des dortoirs Serpentard avant d'appeler : "Message pour Merry Tavington !"

L'enveloppe put alors tomber au sol.

Et quelques instants plus tard la porte du portrait s'ouvrit et Merry se saisit de l'enveloppe.

Elle l'ouvrirait et y trouverait une Mornille ainsi qu'une enveloppe destinée à une élève de quatrième année prénommée Margaret Bulstrode.

(Les Serpentard faisaient tout le temps ce genre de choses, et un Mornille était indubitablement le signe d'une commande urgente).

Margaret ouvrirait \emph{son}  enveloppe et y trouverait cinq Mornilles ainsi qu'une enveloppe à déposer dans une salle de classe inusitée...

...\emph{après}  avoir utilisé son Retourneur de Temps pour revenir cinq heures en arrière...

...ce sur quoi, si elle avait été assez rapide, elle trouverait cinq autres Mornilles qui l'attendaient.

Et un Harry Potter invisible attendrait dans cette salle de classe entre trois heures et trois heures et demie de l'après midi, juste au cas où quelqu'un essaierait ce test évident.

Enfin, évident pour le professeur Quirrell en tout cas.

Le professeur Quirrell avait aussi trouvé évident que (a) Margaret Bulstrode avait un Retourneur de Temps et (b) elle en faisait une utilisation assez libre, par exemple en disant à sa jeune sœur des potins juteux "avant" que qui que ce soit d'autre ne les ait entendus.

Une partie du stress le quitta lorsqu'il s'éloigna de la porte du portrait, toujours invisible. Son esprit avait réussi à s'inquiéter de la réussite du plan même en \emph{sachant}  qu'il avait déjà réussi. Il ne restait maintenant plus que la confrontation avec Dumbledore et il en aurait fini pour la journée... il irait à la gargouille du directeur à neuf heures du soir car s'y rendre à huit heures serait suspect. Il pourrait donc prétendre avoir seulement mal compris ce que le professeur McGonagall avait voulu dire par "ensuite"...

Lorsqu'il pensa au professeur McGonagall, une vague douleur serra de nouveau la poitrine de Harry.

Alors il se retira un peu plus profondément dans son côté obscur, qui avait maintenu une expression calme, qui avait masqué la fatigue, qui avait continué de marcher.

Un jour, il aurait à rendre des comptes, mais parfois il fallait emprunter le maximum et laisser les factures à demain.
\par\noindent\rule{\textwidth}{0.4pt}
Lorsque les escaliers en spirales eurent mené Harry jusqu'à la grande porte en chêne qui constituait le dernier obstacle avant le bureau de Dumbledore, même son côté obscur commençait à céder à l'épuisement ; mais puisqu'il était maintenant \emph{légalement}  quatre heures au-delà de son coucher habituel, il serait sans danger de révéler une partie du caractère physique de cette fatigue tout en continuant de masquer son aspect émotionnel.

La porte en chêne s'ouvrit grand -

Les yeux de Harry s'étant déjà mis au point sur le large bureau et le trône situé derrière lui, il lui fallut un moment pour se rendre compte que celui-ci était vide et que le bureau était désert exception faite d'un unique livre relié de cuir ; puis Harry fit dériver son regard et vit le sorcier debout au milieu de ses objets insolites, de ses appareils mystérieux par dizaines. Fumseck et le Choixpeau étaient posés sur leurs perchoirs respectif et l'on pouvait voir deux parapluies ainsi que trois pantoufles rouges pour pied gauche. Tout était à sa place, semblable à l'ordinaire, hormis le vieux sorcier lui-même, droit dans ses robes du noir le plus formel. Voir ces robes sur cet individu constitua un choc visuel pour Harry, comme de voir son père vêtu d'un costume d'affaires.

Albus Dumbledore semblait à la fois très âgé et très triste.

"Bonjour, Harry," dit le vieux sorcier.

Depuis l'intérieur d'un soi alternatif maintenu à la façon d'une barrière Occlumantique, un Harry innocent et ignorant de tout ce qui se passait inclina sa tête froidement et dit : "Monsieur le directeur. Je suppose que vous avez reçu une réponse du professeur McGonagall, et si cela ne vous dérange pas, j'aimerais \emph{vraiment}  savoir ce qui se passe."

"Oui," dit le vieux sorcier, "il est temps, Harry Potter." Le dos se raidit, un peu seulement, car le sorcier se tenait déjà très droit ; mais ce changement parvint néanmoins à donner l'impression qu'il avait gagné trente centimètres de hauteur, de la force sinon de la jeunesse, et qu'il était devenu redoutable sinon dangereux ; l'étendue de son pouvoir l'enveloppait tel une cape. Il parla d'une voix claire : "Aujourd'hui commence ta guerre contre Voldemort."

"Quoi ?" dit le Harry externe, celui qui ne savait rien, tandis que la chose qui observait depuis l'intérieur songea plus ou moins la même chose, à de nombreuses profanités près.

"On a fait sortir Bellatrix Black d'Azkaban ; elle s'est échappée d'une prison inviolable," dit le vieux sorcier. "Si une prouesse porte la signature de Voldemort, c'est bien celle-ci, et elle, son plus fidèle serviteur, est l'un des trois éléments nécessaires à son retour parmi nous dans une nouvelle enveloppe charnelle. Comme il a été prédit, l'ennemi que tu as vaincu revient dix ans plus tard."

Rien en Harry ne trouvait quoi répondre à cela, du moins pas pendant les quelques secondes disponibles avant que le vieux sorcier ne continue :

"Peu de choses devront changer pour toi," dit-il. "J'ai commencé à rassembler l'Ordre du Phénix, qui te servira, et j'ai alerté les quelques âmes qui devraient et vont comprendre : Amélia Bones, Alastor Maugrey, Bartemius Crouch et certains autres. De la prophétie - oui, il y a une prophétie - je ne leur ai rien dit, mais ils savent que Voldemort est de retour et ils savent que ton rôle sera vital. Eux et moi combattrons les prémisses de cette guerre, tandis que tu deviendras plus fort et peut-être plus sage, ici, à Poudlard." Le vieux sorcier leva la main comme s'il avait une requête à soumettre. "Donc pour le moment, il ne va y avoir qu'un seul changement te concernant, et je t'implore de comprendre sa nécessité. Reconnais-tu le livre sur mon bureau, Harry ?"

Alors que la partie interne de Harry hurlait et se frappait la tête contre des murs imaginaires, le Harry externe pivota et regarda en direction de ce qui se révéla être -

Le silence dura un bon moment.

Puis Harry dit : "C'est un exemplaire du \emph{Seigneur des Anneaux}  de J.R.R Tolkien."

"Tu as su reconnaître une citation de ce livre," dit Dumbledore, le regard attentif, "je présume donc que tu t'en souviens bien. Corrige moi si je me trompe."

Harry se contenta de le fixer du regard.

"Il est important que tu comprennes," dit Dumbledore, "que ce livre n'est pas une représentation réaliste d'une guerre de sorciers. John Tolkien n'a jamais combattu Voldemort. Ta guerre ne sera pas celle des livres que tu as lu. La vie ne ressemble pas aux histoires. Comprends-tu, Harry ?"

Assez lentement, Harry acquiesça ; puis il fit non de la tête.

"En particulier," dit Dumbledore, "Gandalf fait une chose particulièrement idiote dans le premier tome. Il en fait beaucoup, c'est le sorcier de Tolkien ; mais cette erreur est la plus impardonnable. La voici : lorsque Gandalf soupçonne, même l'espace d'un instant, que Frodon est porteur de l'Anneau, il devrait envoyer Frodon à Fondcombe \emph{sur le champ} . Peut-être sera-t-il gêné, ce vieux sorcier, si ses soupçons s'avèrent infondés. Peut-être considère-t-il qu'il est impoli de donner ainsi des ordres à Frodon, qui serait alors grandement incommodé et devrait mettre ses projets et ses plaisirs de côté. Mais un peu de gêne, d'impolitesse et d'incommodement ne sont rien comparé à la perte de toute une guerre le jour où les neuf Nazguls balaient la Comté et s'emparent de l'Anneau pendant qu'il lit de vieux parchemins à Minas Tirith. Et ce n'est pas seulement Frodon qui aurait souffert ; toute la Terre du Milieu serait tombée en esclavage. S'il ne s'était \emph{pas } agi que d'une histoire, Harry, alors ils auraient perdu leur guerre. Comprends-tu ce dont je parle ?"

"Euh..." dit Harry, "pas exactement..." Lorsque Dumbledore se comportait ainsi, quelque chose chez ce dernier rendait plus ardue la tâche de rester glacial ; son côté obscur avait du mal avec le bizarre.

"Alors je serai clair," dit le vieux sorcier. Sa voix était dure et ses yeux tristes. "Frodon aurait dû être immédiatement envoyé à Fondcombe par Gandalf - et il n'aurait jamais dû quitter Fondcombe sans escorte. Il n'y aurait pas dû y avoir de nuit de terreur à Bree, de Hauts des Galgals, de Mont Venteux où Frodon fut blessé, il auraient pu perdre toute la guerre à chacune de ces occasions, à cause de la bêtise de Gandalf ! Comprends-tu ce que je te dis, fils de Michael et Pétunia ?"

Et le Harry qui ne savait rien comprenait.

Et le Harry qui ne savait rien voyait que c'était la réponse habile, la réponse sage, la réponse intelligente et sensée.

Et le Harry qui ne savait rien dit exactement ce qu'un Harry innocent \emph{aurait } dit tandis que l'observateur silencieux poussait un hurlement d'agonie et de désespoir.

"Vous voulez dire," dit Harry, la voix tremblantes, tandis que les émotions qui l'habitaient brûlaient, perçaient la couche de calme externe, "que je ne vais pas rentrer chez mes parents pour Pâques."

"Tu les \emph{reverras} ," dit prestement le vieux sorcier. "Je les prierai de venir te voir, et ils seront reçu de la meilleure des façons possibles. Mais tu ne rentreras pas pour Pâques, Harry. Tu ne rentreras pas pendant l'été. Tu ne prendras plus de déjeuners au Chemin de Traverse, même avec le professeur Quirrell pour t'avoir à l'œil. Ton sang est le second ingrédient dont Voldemort a besoin pour revenir. Tu ne quitteras donc plus jamais l'enceinte de Poudlard sans avoir une raison vitale de le faire et un garde suffisamment forte pour repousser des attaques assez longtemps pour pouvoir te mener en lieu sûr."

De l'eau s'amoncelait aux coins des yeux de Harry. "Est-ce une demande ?" dit sa voix chevrotante. "Ou un ordre ?"

"Je suis navré, Harry," dit le vieux sorcier d'une voix douce. "Tes parents en reconnaîtront la nécessité, du moins je l'espère, mais sinon... j'ai peur qu'ils n'aient aucun recours ; la loi, aussi injuste soit-elle, ne les reconnaît pas comme tes tuteurs légaux. Je suis navré, Harry, et je comprendrai que cela te pousse à me mépriser, mais je dois le faire."

Harry pivota et regarda la porte ; il ne pouvait plus regarder Dumbledore car il ne pouvait plus faire confiance à l'expression de son visage.

\emph{C'est ce qu'il t'en coûte} , dit Poufsouffle en son for intérieur, \emph{tout comme tu as imposé des coûts aux autres. Cela changera-t-il ton point de vue sur l'affaire, comme le professeur Quirrell le prédit ?} 

Le masque du Harry innocent dit automatiquement ce qu'il aurait dit : "Mes parents sont-ils en danger ? Doivent-\emph{ils}  être amenés ici ?"

"Non," dit la voix du vieux sorcier. "Je ne pense pas. Vers la fin de la guerre, les Mangemorts ont appris à ne pas attaquer les familles de l'Ordre. Et si Voldemort agit à présent sans l'aide de ses ancien compagnons, il sait toujours que c'est moi qui prend encore les décisions, et il sait aussi que s'il menaçait ta famille, je ne lui donnerais rien. Je lui ai appris que je ne cède pas au chantage, et il n'essaiera donc pas."

Harry se retourna alors et vit qu'un froid s'était installé sur le visage de Dumbledore en écho à son changement de voix. Derrières ses lunettes, ses yeux étaient devenus aussi durs que de l'acier, et s'ils ne correspondaient pas à l'homme, ils correspondaient aux robes noires formelles.

"Dans ce cas, y a-t-il autre chose ?" dit la voix tremblante de Harry. Il y réfléchirait plus tard, il trouverait une contre-mesure pleine d'astuce, il demanderait au professeur Quirrell s'il y avait un moyen de convaincre le directeur qu'il avait tort. Pour le moment, maintenir le masque requérait toute l'attention de Harry.

"Voldemort a utilisé un engin Moldu pour s'échapper d'Azkaban," dit le vieux sorcier. "Il t'observe et il apprend de toi. Un homme du ministère nommé Arthur Weasley rendra bientôt un édit interdisant tout utilisation d'objets Moldus pendants les batailles du professeur de Défense. À l'avenir, lorsque tu auras une bonne idée, garde-la pour toi."

Cela n'avait pas d'importance comparé au reste. Harry se contenta d'acquiescer et répéta : "Y a-t-il autre chose ?"

Il y eut un silence.

"S'il te plaît," dit le vieux sorcier dans un murmure. "Je n'ai pas le droit de demander ton pardon, mais s'il te plaît, Harry James Potter-Evans-Verres, dis moi au moins que tu comprends mes raisons." On pouvait voir de l'eau dans les yeux du vieux mage.

"Je comprends," dit la voix du Harry extérieur, qui comprenait, "enfin... c'est plus ou moins ce que je me disais, de toute façon... je me demandais si j'aurais pu vous convaincre vous et mes parents de me laisser rester à Poudlard pendant l'été, comme les orphelins, pour que je puisse y lire dans la bibliothèque, de toute façon c'est plus intéressant d'être à Poudlard..."

Un bruit guttural émana d'Albus Dumbledore.

Harry se tourna de nouveau vers la porte. Il ne s'en était pas sorti indemne mais il s'en était sorti.

Il fit un pas en avant.

Sa main se tendit vers la poignée.

Un cri perçant fendit les airs -

Harry pivota, et comme si le temps s'était ralenti il vit le phénix, déjà en vol, qui battait des ailes dans sa direction.

Le vrai Harry, celui qui se savait coupable, eut un moment de panique ; il n'y avait pas pensé, il ne l'avait pas anticipé, il avait été prêt à faire face à Dumbledore mais il avait oublié \emph{Fumseck}  -

Un, deux et trois. Les ailes du phénix battirent trois fois, semblables à un feu qui s'éveillait puis s'endormait, et le temps sembla s'écouler bien trop lentement alors que Fumseck survolait les mystérieux appareils et se dirigeait vers Harry.

Et l'oiseau rouge-or flotta devant lui, suspendu par de doux mouvements d'ailes, oscillant comme la flamme d'une bougie.

"Qu'y a-t-il, Fumseck ?" dit le faux Harry d'un ton perplexe en regardant le phénix droit dans les yeux, exactement comme s'il avait été innocent. Le véritable Harry ressentit la même nausée immonde que lorsqu'il avait entendu le professeur McGonagall lui témoigner sa confiance et pensa : \emph{Suis-je devenu méchant aujourd'hui, Fumseck ? Je ne pensais pas que c'était mal... est-ce que tu me détestes maintenant ? Si je suis devenu une chose que les phénix détestent, peut-être que je devrais juste abandonner maintenant, tout abandonner maintenant et avouer -} 

Fumseck cria, le cri le plus terrible que Harry avait jamais entendu, un cri qui fit vibrer tous les appareils de la pièce et éveilla les silhouettes endormies des portraits.

Le cri transperça les défenses de Harry comme une épée chauffée à blanc aurait percé du beurre, il fit s'effondrer ses couches protectrices comme un ballon percé qui aurait éclaté, il réorganisa ses priorités en un éclair et le fit de souvenir de la chose importante entre toutes ; les larmes commencèrent à se déverser des yeux de Harry, le long de ses joues, et sa voix buta à mesure que les mots sortirent sa gorge, comme s'il avait toussé de la lave...

"Fumseck dit," fit la voix de Harry, "qu'il veut que, je fasse, quelque chose, au sujet, des prisonniers, d'Azkaban -"

"Fumseck, \emph{non}  !" dit le vieux sorcier. Dumbledore s'avança et tendit une main suppliante vers le phénix. Sa voix était presque aussi désespérée que le cri du phénix l'avait été. "Tu ne peux pas lui demander cela, Fumseck, il n'est encore qu'un enfant !"

"Vous êtes allé à Azkaban," chuchota Harry, "vous avez emmené Fumseck avec vous, il a vu - \emph{vous}  avez vu - vous étiez \emph{là} , vous avez vu - \emph{POURQUOI N'AVEZ-VOUS RIEN FAIT ? POURQUOI NE LES AVEZ-VOUS PAS LAISSÉS SORTIR ?"} 

Lorsque les instruments eurent fini de vibrer, Harry se rendit compte que Fumseck avait crié en même temps que lui et qu'il volait maintenant à ses côtés, face à Dumbledore, et que la tête rouge-or était à la hauteur de la sienne.

"Peux-tu," murmura le vieux sorcier, "peux-tu vraiment entendre la voix du phénix si distinctement ?"

Harry sanglotait presque trop pour pouvoir parler ; face aux portes de métal qu'ils avaient franchies, face aux voix qu'ils avaient entendues, face aux pires souvenirs, face à la supplication désespérée tandis qu'il s'éloignait ; tout cela avait jailli en lui avec la force d'un brasier lorsque le phénix avait crié, lorsqu'il avait renversé tous les remparts. Harry ignorait s'il pouvait vraiment entendre la voix du phénix, s'il aurait comprit Fumseck sans l'avoir déjà su. Tout ce qu'il savait, c'est qu'il avait une excuse plausible pour dire ce que le professeur Quirrell lui avait dit de ne \emph{jamais}  mentionner dans une conversation à partir de ce jour ; parce que c'était exactement ce qu'un Harry innocent aurait dit, ce qu'il aurait fait s'il \emph{avait}  entendu le phénix assez distinctement. "Ils souffrent - nous devons les aider -"

"Je ne \emph{peux pas}  !" s'écria Albus Dumbledore. "Harry, Fumseck, je ne \emph{peux pas} , je ne peux rien y faire !"

Un autre cri perçant.

"\emph{POURQUOI PAS ? ALLEZ-Y ET FAITES LES SORTIR !} "

Le vieux sorcier arracha ses yeux du phénix et rencontra ceux de Harry. "Harry, dis-le à Fumseck pour moi ! Dis-lui que ce n'est pas aussi simple ! Les phénix ne sont pas que de simples animaux mais ils \emph{sont}  des animaux, Harry, ils ne peuvent pas comprendre -"

"Je ne comprends pas non plus," dit Harry d'une voix tremblante. "Je ne comprends pas pourquoi vous \emph{donnez des gens à manger aux Détraqueurs ! Azkaban n'est pas une prison, c'est une salle de torture où vous torturez les gens jusqu'à la MORT !} "

"Percival," dit le vieux sorcier d'une voix rauque, "Percival Dumbledore, mon propre père, Harry, mon propre père est mort à Azkaban ! Je sais, je sais que c'est horrible ! \emph{Mais que voudrais-tu me voir faire ?}  Entrer à Azkaban de force ? Voudrais-tu me voir déclarer une rébellion contre le ministère ?"

CROA !

Il y eut une silence, puis la voix tremblante de Harry dit : "Fumseck ignore tout des gouvernements, il veut juste que vous - sortiez les prisonniers - de leur cellule - et il vous aidera à combattre - si quelqu'un vous barre la route - et - et moi aussi, monsieur le directeur ! Je viendrai avec vous et je détruirai tous les Détraqueurs qui s'approcheront ! Nous nous inquiéterons des retombées politiques ensuite, je vous parie que vous et moi pourrions nous le permettre -"

"Harry," murmura le vieux sorcier, "les phénix ne comprennent pas la façon dont une bataille gagnée peut faire perdre une guerre." Des larmes coulaient le long des joues du vieux sorcier et gouttaient sur sa barbe d'argent. "Ils ne connaissent que la bataille. Ils sont bons, mais pas sages. C'est pourquoi ils choisissent d'avoir des sorciers pour maîtres."

"Pouvez-vous amener les Détraqueurs à un endroit où je pourrai les atteindre ?" dit Harry d'une voix qui était maintenant implorante. "Amenez-les par groupes de quinze - je pense que je pourrais en détruire autant à la fois sans me faire de mal -"

Le vieux sorcier secoua la tête. "C'était suffisamment difficile de faire accepter la perte de l'un d'entre eux - ils m'en donneraient peut-être un de plus, mais certainement pas deux - ils sont considérés comme la propriété de la nation, Harry, comme des armes à utiliser en cas de guerre -"

La rage de Harry éclata alors comme un incendie, peut-être venait-elle du phénix maintenant posé sur son épaule, peut-être venait-elle de son côté obscur, mais les deux colères se mélangèrent en lui, le chaud et le froid, et c'est une étrange voix qui dit par sa gorge : "Dites-moi quelque chose. Que doit faire un gouvernement, que doivent faire les électeurs d'une démocratie, que doit faire le \emph{peuple}  d'une \emph{pays}  avant qu'il ne me faille décider que je ne suis plus de son côté ?"

Les yeux du vieux sorcier s'écarquillèrent face au garçon sur l'épaule duquel un phénix se tenait. "Harry... s'agit-il là de tes paroles ou de celles du professeur de Défense -"

"Parce qu'il doit y avoir une \emph{limite} , non ? Et si ce n'est pas Azkaban, quelle est-elle ?"

"Harry, s'il te plaît, écoute moi ! Les sorciers ne pourraient pas vivre ensemble s'ils se déclaraient la guerre, chacun contre tous, à chaque fois qu'ils avaient un différend ! Il y aurait toujours \emph{quelque chose}  -"

"\emph{Azkaban n'est pas seulement 'quelque chose' ! C'est le mal !} "

"Oui, même le mal ! Même certains maux, Harry, car les sorciers ne sont pas parfaitement bons ! Mais mieux vaut malgré tout vivre en paix que dans le chaos ; et si nous pénétrions dans Azkaban de force, ce serait le début du \emph{chaos} , en es-tu conscient ?" plaida le vieux sorcier. "Et il est possible de t'opposer à la volonté de ton prochain ouvertement ou en secret sans le \emph{haïr} , sans déclarer qu'il est méchant, que c'est ton ennemi ! Je ne pense pas que les habitants de ce pays méritent que tu les juge ainsi, Harry ! Et même si certains d'entre eux ont - et les enfants, et les élèves de Poudlard, et les bons présents parmi les mauvais ?"

Harry regarda son épaule, là où Fumseck était perché, et vit que le phénix le regardait et que ses yeux flambaient sans briller ; des flammes rouges dans une mer de feu d'or.

\emph{Qu'en penses-tu, Fumseck ?} 

"Croa ?" dit le phénix.

Fumseck n'avait pas comprit la conversation.

Le jeune garçon regarda le vieux sorcier et dit d'une voix forte : "ou peut-être que les phénix sont plus sages que nous et plus intelligents que nous, peut-être qu'ils nous suivent partout en espérant qu'un jour nous les \emph{écouterons} , qu'un jour nous \emph{comprendrons} , qu'un jour nous \emph{ferons}  simplement \emph{sortir}  les prisonniers de leur \emph{cellule}  -"

Harry fit demi-tour, tira la porte en chêne, entra dans la cage d'escalier et claqua la porte derrière lui.

Les escaliers commencèrent à tourner, Harry commença à descendre, il mit son visage entre ses mains et il pleura.

Ce n'est qu'à mi-chemin de la descente qu'il remarqua la différence, la chaleur toujours présente, et il se rendit compte que -

"Fumseck ?" murmura Harry.

- le phénix était toujours sur son épaule, perché comme il l'avait vu plusieurs fois sur celle de Dumbledore.

Il regarda de nouveau dans ses yeux, dans les flammes rouges, dans le feu d'or.

"Tu n'es quand même pas mon phénix maintenant... si ?"

Croa !

"Oh," dit Harry d'une voix un peu tremblante, "je suis heureux de l'entendre, Fumseck, parce que je ne pense pas - le directeur - je ne pense pas qu'il mérite -"

Harry se tut et reprit son souffle.

"Je ne pense pas qu'il le mérite, il essayait juste de bien agir..."

Croa !

"Mais tu es en colère et tu veux qu'il comprenne quelque chose. Je vois."

Le phénix lova sa tête contre l'épaule de Harry et la gargouille de pierre se déplaça d'un mouvement souple afin de laisser Harry retourner dans les couloirs de Poudlard.

