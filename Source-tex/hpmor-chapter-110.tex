
\chapter{Réflexions, partie 2}

L'air sévère de Dumbledore ne dura que le temps de laisser place à l'ahurissement. "Quirinus ? Que…"

Et il y eut un moment de flottement.

"Bon," dit Albus Dumbledore. "Je me sens vraiment stupide."

"J'espère bien," répondit immédiatement le professeur Quirrell ; s'il avait été surpris de se faire avoir, il le cachait. D'un geste nonchalant, il retrouva ses robes de professeur.

Dumbledore eut de nouveau l'air sévère, cette fois beaucoup plus. "Et dire que je me promène à la recherche de ce qui reste de Voldemort sans jamais me rendre compte que le professeur de Défense de Poudlard est une victime malade, à moitié morte, possédée par un esprit bien plus puissant que lui. Si tant d'autres n'étaient pas passé à côté, je me serais déclaré sénile."

"Vous feriez bien," dit le professeur Quirrell. Il leva les sourcils. "Vraiment, suis-je si difficile à reconnaître sans les yeux rouges lumineux ?"

"Oh, absolument," dit Albus Dumbledore d'un ton monocorde. "Tu joué ton rôle parfaitement. Je m'avoue entièrement dupé. Quirinus Quirrell avait l'air… comment dit-on déjà ? Ah oui, voilà. Il avait l'air sain d'esprit."

Le professeur Quirrell gloussa. On aurait pu jurer que les deux hommes avaient une conversation anodine. "Je n'ai jamais été fou, tu sais. Lord Voldemort n'était qu'un jeu de plus, tout comme le professeur Quirrell."

Albus Dumbledore n'avait pas l'air d'apprécier cette petite conversation. "Je me doutais que tu pourrais dirais cela. Je suis désolé de devoir te le dire, Tom, mais toute personne capable de jouer le rôle de Voldemort \emph{est}  Voldemort."

"Ah," dit le professeur Quirrell, tout en levant un doigt en guise d'avertissement. "Il y a une faille dans ce raisonnement, vieil homme. Toute personne capable de jouer le rôle de Voldemort doit être ce que les moralistes appellent 'mauvais', nous sommes d'accord là-dessus. Mais peut-être que le vrai moi est certes totalement, profondément, irrémédiablement mauvais, mais d'une façon différente, plus intéressante que le personnage de Voldemort…"

"Je crois bien," dit Albus Dumbledore entre des dents serrées, "que je m'en fiche."

"Alors tu dois te croire sur le point d'être débarrassé de moi," dit le professeur Quirrell. "Comme c'est intéressant. Je vais perdre mon existence immortelle à moins de découvrir quel piège tu m'as tendu et la façon d'y échapper le plus vite possible." Le professeur Quirrell s'interrompit un instant. "Mais repoussons vainement ce moment et parlons d'autre chose d'abord. Comment as-tu fais pour m'attendre dans le Miroir ? Je te croyais ailleurs."

"J'y suis," dit Albus Dumbledore, "et je suis \emph{aussi}  dans le Miroir, malheureusement pour toi. J'ai toujours été là ; depuis le début."

"Ah," dit le professeur Quirrell, et il soupira. "J'imagine que ma petite diversion a été inutile, alors."

La rage de Dumbledore se déchaîna alors. "\emph{Diversion ?} ", rugit-il, et ses yeux de saphirs s'emplirent de furie. "\emph{Tu as tué Maître Flamel pour créer une diversion ?} "

Le professeur Quirrell eut l'air consterné. "L'injustice de cette accusation me blesse. Je n'ai pas tué la personne que tu connais sous le nom de Flamel. J'ai simplement ordonné à un autre de le faire."

"\emph{Comment as-tu pu ? Même toi, comment as-tu pu ? Il était la bibliothèque de nos savoirs ! De secrets dont tu as à jamais privé le monde magique !} "

Le sourire du professeur Quirrell était devenu presque cruel. "Tu sais, je ne comprends toujours pas comment ton esprit tordu peut accepter que Flamel soit immortel et me juger monstrueux quand j'essaie de le devenir."

"Maître Flamel ne s'est jamais abaissé à être \emph{immortel}  ! Il…" Dumbledore s'étrangla. "Il a seulement veillé passé le crépuscule de sa très longue journée, pour notre bien…"

"Je ne sais pas si tu t'en souviens," dit le professeur Quirrell d'un ton désinvolte, "mais te rappelles-tu de ce jour dans ton bureau, avec Tom Jedusor ? Le jour où je t'ai supplié, où je me suis mis à genoux et je t'ai supplié de me présenter Nicholas Flamel pour que je puisse devenir son apprenti, pour que je puisse un jour me fabriquer une Pierre Philosophale ? C'est la dernière fois que j'ai essayé d'être quelqu'un de bien, si tu veux savoir. Tu m'as dit non et tu m'as donné une leçon sur l'immoralité de la peur de la mort. J'ai quitté ton bureau amer et furieux. Je me suis dit que si mon désir de ne pas mourir me ferait passer pour mauvais, alors autant l'être vraiment ; et un mois plus tard, j'ai tué Abigail Myrtle afin d'obtenir l'immortalité autrement. Même lorsque j'en ai su plus sur Flamel, ton hypocrisie a continué de m'énerver, et c'est pour cela que je vous ai tourmentés, toi et les tiens, plus que je ne l'aurais fait dans d'autres circonstances. J'ai souvent songé que tu devrais le savoir, mais nous n'avons jamais eu une chance de parler franchement."

"Je refuse," dit Albus Dumbledore, son regard inflexible. "Je n'accepte pas la moindre once de responsabilité pour ce que tu es devenu. C'était entièrement, pleinement toi et tes décisions."

"Ça ne m'étonne pas de t'entendre dire ça," dit le professeur Quirrell. "Eh bien, je me demande bien quelles responsabilités tu acceptes. Tu as accès à un pouvoir de Divination inhabituel, cela, je l'ai compris il y a longtemps. Tu agissais de façon trop absurde, et la façon dont les choses jouaient en ta faveur étaient trop invraisemblable. Alors dis-moi. Savais-tu ce qui aurait lieu, cette veillée de Toussaint où j'ai été temporairement défait ?"

"Je le savais," dit Albus Dumbledore d'une voix sombre et froide. "J'en accepte la responsabilité, ce qui est une chose que tu ne comprendras jamais."

"Tu t'es arrangé pour que Severus Rogue entende la prophétie dont il m'a fait part."

"J'ai permis que cela se produise," dit Albus Dumbledore.

"Et dire que j'étais tout excité d'avoir enfin un peu de cette prescience." Le professeur Quirrell secoua la tête avec un air triste. "Donc le grand héros Dumbledore a sacrifié Lily et James Potter, ses pions involontaires, juste pour me bannir pendant quelques années."

Les yeux d'Albus Dumbledore ressemblaient à des pierres. "S'ils avaient su, James et Lily auraient accepté leur mort."

"Et le petit bébé ?" dit le professeur Quirrell. "Je doute que les Potter auraient été empressés de le placer sur la route de Tu-Savez-Qui."

Le tressaillement fut à peine visible. "Le Survivant s'en est plutôt bien sorti. Tu as essayé de le transformer en toi, hein ? Au lieu de ça, tu es devenu un cadavre et Harry Potter est devenu le sorcier que tu aurais dû être." Un peu de Dumbledore était réapparu derrière les lunettes en croissant de lune, un peu du pétillement de ses yeux. "Tout le génie glacé de Tom Jedusor, dompté par la chaleur de l'amour de James et Lily. Je me demande ce que tu as ressenti en voyant ce que Tom Jedusor aurait pu devenir s'il avait grandi dans une famille aimante ?"

Les lèvres du professeur Quirrell se tordirent. "J'ai été surpris et même choqué lorsque j'ai découvert les profondeurs abyssales de la naïveté de M. Potter."

"Je suppose que la blague t'est passée au-dessus de la tête." C'est alors que Dumbledore sourit enfin. "Comme j'ai ri, quand j'ai compris ! Quand j'ai vu que tu avais créé un bon Voldemort pour qu'il s'oppose au mauvais ! Ah, qu'est-ce que j'ai ri ! Je n'ai jamais été assez dur pour remplir ma fonction, mais Harry Potter fera plus que m'égaler le jour où il accèdera au pouvoir." Le sourire d'Albus Dumbledore disparut. "Même si j'imagine que Harry devra trouver quelque autre Seigneur des Ténèbres à vaincre, puisque tu ne seras plus là."

"Ah oui. À ce propos." Le professeur Quirrell commença à s'écarter du Miroir et sembla être interrompu juste avant de quitter la zone où il aurait été réfléchi par le Miroir si le miroir avait réfléchi quoi que ce soit. "Intéressant."

Le sourire de Dumbledore était devenu plus froid. "Non, Tom. Tu n'iras nulle part."

Le professeur Quirrell hocha la tête. "Qu'as-tu fait, exactement ?"

"Tu as refusé la mort," dit Dumbledore, "et si je détruisais ton corps, ton esprit reviendrait comme une bête stupide qui ne comprend pas qu'on ne veut plus d'elle. Alors je t'envoie hors du Temps, dans un instant figé d'où ni moi ni personne ne pourra te tirer. Peut-être Harry Potter y parviendra-t-il un jour, si une prophétie dit vrai. Il voudra certainement avoir ton avis sur l'identité du responsable de la mort de ses parents. Pour toi, cela ne durera qu'un instant - si jamais tu reviens. Quoi qu'il en soit, Tom : tous mes vœux."

"Hm," dit le professeur Quirrell. Il était repassé devant un Harry muet et horrifié pour ensuite s'arrêter à l'autre frontière créée par le Miroir. "Comme je le pensais. Tu utilises le vieux scellé de Merlin, ce que l'histoire de Topherius Chang appelle le Processus de l'Intemporel. Si la légende dit vrai, il a été actif trop longtemps pour que quiconque puisse l'arrêter, même toi."

"En effet," dit Albus Dumbledore. Ses yeux devinrent soudain méfiants.

Juste à côté de la porte, Harry attendait en silence, contrôlant sa terreur. Il put la sentir dans les airs ; il put sentir la \emph{présence}  s'amonceler dans le champ du Miroir. Quelque chose de plus étrange que de la magie, tout chez elle incompréhensible hormis son étrangeté et la certitude de son pouvoir. Elle était venue lentement, mais à présent, elle grandissait plus vite.

"Tu pourrais encore inverser l'effet, si le témoignage de Chang est véridique," dit le professeur Quirrell. "Selon les légendes, la plupart des pouvoirs du Miroir marchent dans les deux sens. Tu pourrais donc bannir ce qui est de l'autre côté du Miroir. T'envoyer toi-même dans cet instant figé. Si tu le voulais, bien sûr."

"Et pourquoi est-ce que je ferais ça ?", dit Albus Dumbledore d'une voix tendue. "J'imagine que tu vas me dire que tu as des otages ? C'est inutile, Tom ! \emph{Idiot}  ! Sombre \emph{idiot}  ! Tu aurais dû savoir que je ne te donnerais rien en échange des otages que tu as pris."

"Tu as toujours eu un temps de retard," dit le professeur Quirrell. "Permets-moi de te présenter mon otage."

Une autre présence envahit les airs autour de Harry, comme un fourmillement sur sa chair à mesure que la magie de Tom Jedusor passait tout près de sa peau. La Cape d'Invisibilité lui fut arrachée et s'envola, noire et chatoyante.

Le professeur Quirrell la saisit et s'en vêtit promptement ; en moins d'une seconde, avait il disparut sous la Cape.

Albus Dumbledore vacilla, comme s'il venait de perdre un important soutien.

"Harry Potter," souffla le directeur. "\emph{Que fais-tu ici ?} "

Harry regarda l'image d'Albus Dumbledore. La surprise et la consternation se disputaient les traits de ce dernier.

La culpabilité et la honte frappèrent Harry d'un coup, insupportables, et il put sentir l'incompréhensible présence atteindre un nouveau point culminant. Il sut instinctivement qu'il n'avait plus de temps, que c'en était fini de lui.

"C'est ma faute," dit-il d'une petite voix, dit la partie de lui qui avait pris le contrôle de sa gorge en cet ultime instant. "J'ai été stupide. J'ai toujours été stupide. Vous ne devez pas me sauver. Adieu."

"Regardez-moi ça," chantonna la voix sans corps apparent du professeur Quirrell, "on dirait bien que je n'ai plus de reflet."

"Non," dit Albus Dumbledore. "Non, non, \emph{NON !} "

Une longue baguette gris sombre jaillit d'une manche d'Albus Dumbledore et fut saisie par sa main. Dans l'autre, comme surgit de nulle part, apparut un court bâton de pierre sombre.

Albus Dumbledore les jeta violemment de côté au moment même où la sensation de pouvoir atteignait une intensité insupportable et disparaissait soudain.

Le Miroir se remit à montrer le reflet ordinaire de la salle de pierre blanche éclairée d'or. Il n'y avait plus aucune trace du lieu où Albus Dumbledore avait été.

