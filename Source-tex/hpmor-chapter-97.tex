
\chapter{Rôles, partie 8}
Harry Potter et les Méthodes de la Rationalité Chapitre 97 : Rôles, partie 8
\emph{(Note de l'auteur : Je vous suggère de relire les derniers paragraphes du chapitre 96 avant de commencer celui-ci pour vous souvenir de ce qui se passe.)} 
\par\noindent\rule{\textwidth}{0.4pt}
Pour la seconde fois de la journée, les yeux de Harry s'emplirent de larmes. Sans prêter attention aux regards interloqués des Serdaigle présents dans la salle commune, il s'avança vers la créature d'argent que Draco Malfoy avait envoyée et la berça dans ses bras comme si elle était vivante ; puis il s'orienta vers sa chambre en chancelant, comme à moitié aveugle, et se dirigea vers le fond de sa malle, le serpent d'argent silencieux entre ses bras.
\par\noindent\rule{\textwidth}{0.4pt}
\emph{La cinquième rencontre : dimanche 19 avril à 10h12.} 

La rencontre que Lord Malfoy avait exigée de son débiteur - qui lui devait 58.203 Gallions - se déroulait, selon les lois Britanniques, à la banque centrale de Gringotts.

L'Enchanteur-en-chef Dumbledore y avait quelque peu résisté et avait essayé d'empêcher Harry Potter de quitter la sécurité de Poudlard (un mot auquel Harry Potter avait réagit en levant les mains et en mimant des guillemets). Quant à lui, le Survivant avait semblé méditer en silence avant de consentir à la rencontre, et il avait été étrange qu'il accepte ainsi l'exigence de son ennemi.

Le directeur de Poudlard, qui du point de vue de l'Angleterre magique était le tuteur légal de Harry Potter, avait rejeté sa décision.

Le Comité d'Examen des Dettes du Magenmagot avait rejeté la décision du directeur de Poudlard.

L'Enchanteur-en-Chef avait rejeté la décision du Comité d'Examen des Dettes.

Le Magenmagot avait rejeté la décision de l'Enchanteur-en-chef.

Et ainsi le Survivant était parti pour la banque centrale de Gringotts sous la garde renforcée de Maugrey Fol Œil et d'un trio d'Aurors ; l'œil bleu vif de Maugrey pivotait follement en tous sens comme pour annoncer à un attaquant potentiel qu'il était Sur Ses Gardes, Constamment Vigilant, et qu'il incinérerait avec joie les reins du premier qui éternuerait en direction du Survivant.

Harry Potter traversa les portes grandes ouvertes de Gringotts que surplombait la devise \emph{Fortius Quo Fidelius}  en observant son environnement avec plus d'attention que les fois précédentes. Lors de ses trois dernières visites à Gringotts il avait simplement admiré les piliers de marbre, les torches couleur or et l'architecture pas à tout à fait semblable à celle créée par les humains d'Angleterre magique. Entre-temps s'étaient produit l'Incident à Azkaban et d'autres choses ; et à présent, pour sa quatrième visite, Harry songeait à la Rébellion des Gobelins, au ressentiment constant de ces derniers concernant l'interdiction qui leur avait été faite d'avoir leur propre baguette, et à certains faits qui n'avaient pas été indiqués dans le livre de première année mais que Harry avait sus deviner par reconnaissance de motif et que le professeur Flitwick avait confirmés d'une toute petite voix. Lord Voldemort avait tué des gobelins et des sorciers - une décision incroyablement stupide de sa part, à moins que Harry ait raté quelque chose - mais Harry ignorait entièrement ce que les gobelins pensaient du Survivant. Ils avaient la réputation de payer ce qu'ils devaient et de prendre ce qu'ils pensaient leur être dû, ainsi que celle d'interpréter ces questions de façons quelque peu partielle.

Aujourd'hui, les gardes qui se tenaient bien droits dans leur armures, placés à intervalles réguliers dans la banque, regardaient le Survivant avec un visage sans expression et observaient Maugrey et les Aurors d'un air furieux ponctué de moments de mépris amer. Aux comptoirs de l'entrée de la banque, des employés gobelins observaient avec tout autant de mépris les sorciers dont ils remplissaient les mains de Gallions ; l'un d'eux offrit un sourire dur et révéla ses dents à une sorcière qui semblait être à la fois désespérée et en colère.

\emph{Si je comprends bien la nature humaine - et si j'ai raison de penser que toutes les espèces magiques humanoïdes sont génétiquement humains mais dotés d'un effet magique héréditaire - alors vous ne deviendrez probablement pas ami avec un sorcier juste parce que je serai poli envers vous ou parce que je dirai que je vous comprend. Mais je me demande si vous soutiendrez le Survivant dans sa tentative de renverser le ministère, si je vous promettais de révoquer ensuite la Loi de la Baguette… ou si je vous donnais discrètement des baguettes et des livres de sortilèges en échange de votre soutient… est-ce pour ça que le secret de la fabrication des baguettes est réservée aux gens comme Ollivander ? Quoi que, si vous êtes vraiment humains, et seulement humains, alors la nation gobeline a probablement ses propres horreurs internes, sa propre Azkaban, car ça aussi, c'est la nature humaine ; auquel cas tôt ou tard je devrai aussi renverser ou réformer votre gouvernement. Hmm.} 

Un Gobelin âgé apparut devant eux et Harry inclina la tête avec une courtoisie gracieuse, un geste auquel le gobelin âgé répondit d'un semi hochement de tête abrupte. Il n'y eut pas de montagnes russes ; le gobelin âgé se contenta de les introduire dans un bref couloir qui s'achevait sur une petite salle d'attente où se trouvaient trois bancs à taille de gobelin et une chaise pour sorcier où personne ne s'assit.

"Ne signe rien que Lucius Malfoy te tendra," dit Maugrey Fol Œil. "\emph{Rien} , c'est compris mon gars ? Si Malfoy te tend une copie des \emph{Merveilleuses Aventures du Survivant}  et te demande ton autographe, dis-lui que tu t'es foulé un doigt. N'effleure même pas une plume tant que tu seras à Gringotts. Si quelqu'un t'en tend une, casse-la et ensuite casse-toi les doigts. Est-ce que j'ai besoin de plus t'expliquer, petit ?"

"Pas particulièrement," dit Harry. "On a aussi des avocats en Angleterre Moldue et les nôtres penseraient que les vôtres sont très mignons."

Peu de temps après, Harry Potter remit sa baguette à un garde gobelin en armure qui le fouilla au moyen de sondes toutes plus intéressantes les unes que les autres avant de laisser à Maugrey la garde de sa bourse.

Puis Harry passa par une autre porte et sous une brève cascade du voleur qui s'évapora dès qu'il en sortit.

De l'autre côté de la porte, une pièce plus large, richement décorée, dont toute la longueur était occupée par une grande table en or ; deux immenses chaises de cuir d'un côté de la table et un petit tabouret de bois de l'autre, le perchoir du débiteur. Deux gobelins en armure complète, munis d'oreillettes et de lunettes ornées, se tenaient de chaque côté de la pièce. Aucun des deux camps n'aurait de baguette ou autre ustensile magique, et les gobelins attaqueraient immédiatement si quiconque osait faire usage de magie sans baguette lors de cette rencontre pacifique supervisé par la banque de Gringotts. Les oreillettes ornées empêcheraient les gardes gobelins d'entendre la conversation, à moins que l'on ne s'adresse directement à eux, et les lunettes rendraient flous les visages des sorciers. En bref, il s'agissait quasiment d'un \emph{véritable}  système de sécurité - du moment que l'on était Occlumens.

Harry monta sur son tabouret de bois peu confortable en songeant \emph{que c'est subtil}  avec sarcasme et attendit ses créanciers.

L'intervalle de temps avant que Lucius n'entre dans la pièce fut très court, bien plus court que le temps maximum pendant lequel on pouvait légalement faire attendre un débiteur. Il s'assit dans sa chaise de cuir avec une aisance forgée par la pratique. Sa cane à tête de serpent n'était plus dans ses mains, sa longue chevelure flottait à sa suite comme elle l'avait toujours fait ; son visage était insondable.

Un jeune garçon aux cheveux blond-blanc le suivait sans faire de bruit, vêtu de robes plus fines que n'importe quel uniforme de Poudlard, et il marchait après son père avec l'air de se contrôler. Un garçon qui était aussi créancier de Harry à hauteur de quarante Gallions, qui appartenait aussi à la maison Malfoy et qui, techniquement, était donc concerné par la résolution du Magenmagot à l'origine de cette rencontre.

\emph{Draco.}  Harry ne parla pas à voix haute, il ne laissa pas l'expression de son visage changer. Il ne savait quoi dire. Même \emph{Pardon}  ne semblait pas convenable. Il n'avait pas non plus osé dire cela au Patronus de Draco, lorsqu'ils avaient organisé cette rencontre en quelques échanges brefs ; et pas seulement parce que Lucius risquait d'écouter. Apprendre que la pensée heureuse de Draco l'était toujours et qu'il pouvait encore désirer que Harry le sache avait suffit à satisfaire ce dernier.

Lucius Malfoy parla d'abord, voix neutre, visage décidé. "Je ne comprends pas ce qui se passe à Poudlard, Harry Potter. Pourriez-vous me l'expliquer ?"

"Je l'ignore," dit Harry. "Si je comprenais ces événements, je ne les aurais pas laissés se produire, Lord Malfoy."

"Alors répondez à cette question. \emph{Qui}  êtes-vous ?"

Harry contempla avec calme le visage de son créancier. "Je ne suis pas Vous-Savez-Qui, comme vous pensiez que je l'étais," dit-il. N'étant pas \emph{complètement}  idiot, il avait finit par comprendre à qui Lucius pensait s'adresser devant tout le Magenmagot. "Je ne suis évidemment pas un garçon normal. De façon toute aussi évidente, cela a \emph{probablement}  quelque chose à voir avec ce qui a fait de moi le Survivant. Mais j'en ignore les causes et les raisons autant que vous. J'ai posé la question au Choixpeau et il les ignore aussi."

Lucius Malfoy hocha la tête avec un regard distant. "Je n'ai pu trouver de raison qui vous pousserait à payer cent mille Gallions pour sauver la vie d'une Sang-de-Bourbe. Pas de raison, excepté une qui expliquerait son pouvoir autant que sa soif de sang ; mais alors elle est morte aux mains d'un troll et vous vivez toujours. Mon \emph{fils}  m'a aussi dit de \emph{nombreuses}  choses sur vous, Harry Potter, des choses qui \emph{n'avaient pas le moindre sens} , j'ai entendu les délires des fous de Ste Mangouste et ils étaient beaucoup plus raisonnable que les événements auxquels, à en croire les dire de mon \emph{fils}  pendant qu'il était sous \emph{Veritaserum} , vous avez participé, et je souhaiterais que vous m'expliquiez, sur-le-champ, cette \emph{démence absolue}  dont vous êtes \emph{l'auteur} ."

Harry se retourna vers Draco dont le visage alternait entre la contrôle de soi-même et une grande nervosité.

"Moi aussi," dit Draco d'une voix aiguë et vacillante, "j'aimerais comprendre, Potter."

Harry ferma les yeux et parla sans regarder. "Un garçon élevé par des Moldus qui se croyait malin. Tu m'as vu, Draco, et tu as songé qu'il serait éminemment utile de se lier d'amitié avec le Survivant, plus qu'avec tous les autres enfants de ton année, afin de lui montrer le monde tel qu'il est. Et j'ai pensé la même chose à ton sujet. Sauf que toi et moi avions des idées différentes de ce qui est vrai. Je ne dis pas qu'il y a plusieurs vérités, je veux dire qu'il y a différentes croyances et une seule réalité, un seul univers capable de rendre ces croyances vraies ou fausses…"

"Tu m'as menti."

Harry ouvrit les yeux et regarda Draco. "Je préférerais dire," répondit Harry d'une voix pas tout à fait assurée, "que les choses que je t'ai dites étaient vraies d'un certain point de vue."

"\emph{Un certain point de vue ?} " La colère de Draco Malfoy semblait aussi justifiée que celle de Luke Skywalker, et lui non plus n'était pas d'humeur à accepter les excuses d'Obi-Wan. "Il y a un mot pour les choses vraies d'un certain point de vue. On les appelle des \emph{mensonges !} "

"Ou des ruses," dit Harry d'un ton neutre. "Des affirmations techniquement vraies mais qui trompent l'auditeur, qui le poussent à former des croyances fausses. Je pense que la distinction mérite d'être faite. Ce que je t'ai dit était une prophétie auto-réalisatrice ; tu as cru que tu ne pouvais pas te tromper toi-même et tu n'as donc pas essayé de le faire. Tes nouvelles capacités sont réelles et il te serait très néfaste de commencer à lutter intérieurement contre elles. Les gens ne peuvent pas se forcer à croire que bleu est vert par la seule force de leur volonté, mais ils \emph{croient}  qu'ils en sont capables, et ça peut être presque aussi grave."

"Tu m'as \emph{utilisé} ," dit Draco Malfoy.

"Seulement d'une façon qui t'a rendu plus fort. C'est ça, être utilisé par un ami."

"\emph{Même moi je sais que ça n'est pas ça, l'amitié !} "

Lucius Malfoy parla de nouveau. "Dans quel but ? À quelle fin ?" Même la voix de Malfoy senior n'était pas très stable. "\emph{Pourquoi ?} "

Harry l'observa un moment puis se tourna vers Draco. "Ton père ne va probablement pas y croire," dit Harry. "Mais toi, Draco, tu devrais être capable de voir que tout ce qui s'est produit est compatible avec cette hypothèse. Et qu'une hypothèse plus cynique n'expliquerait pas pourquoi je n'ai pas plus insisté quand tu pensais que j'avais l'avantage sur toi, ni pourquoi je t'ai tant appris. Je pensais que l'héritier de la maison Malfoy, que l'on avait vu attraper une née-Moldue pour l'empêcher de tomber du toit de Poudlard, serait un bon candidat du compromis, capable de diriger l'Angleterre magique après la réformation."

"Vous souhaitez donc me faire croire," dit Lucius Malfoy d'un filet de voix, "que vous prétendez être fou. Bien, mettons cela de côté. Dites-moi qui a lâché le troll dans Poudlard."

"Je l'ignore," dit Harry.

"Dites-moi qui vous \emph{soupçonnez} , Harry Potter."

"J'ai quatre suspects. L'un d'eux est le professeur Rogue…"

"\emph{Rogue ?} " laissa échapper Draco.

"Le deuxième est bien sûr le professeur de Défense de Poudlard, juste parce qu'il est professeur de Défense." Harry aurait voulu éviter de le mentionner car il ne souhaitait pas porter le professeur Quirrell à l'attention des Malfoy au cas où il serait innocent, mais Draco aurait pu lui faire remarquer cette omission. "Le troisième, vous ne me croiriez pas. Le quatrième est une catégorie fourre-tout que j'appelle Le Reste." \emph{Et je pense que je ne devrais pas mentionner devant vous le cinquième, Lord Voldemort.} 

Le visage de Lucius Malfoy mima un grognement. "Me croyez-vous incapable de ne pas reconnaître l'appât sur votre hameçon ? Parlez-moi de cette troisième possibilité, Potter, celle que vous me souhaitez croire être la \emph{bonne} , et laissons là ces jeux."

Harry regarda Lucius Malfoy sans broncher. "J'ai un jour lu un livre que je n'était pas censé lire, et voilà ce que j'y ai trouvé : toute communication a lieu entre deux égaux. Les employés mentent à leur patron et le patron s'attend à ce qu'on lui mente. Je ne me fais pas désirer, je remarque qu'il est simplement impossible, dans la situation actuelle, que je vous parle de ce troisième suspect et que vous pensiez que ce soupçon est autre chose qu'une ruse de ma part."

Draco parla alors. "C'est Père, c'est ça ?"

Harry jeta un regard surpris à Draco.

Draco parla d'un ton neutre. "Tu soupçonnes Père d'avoir envoyé le troll dans Poudlard pour se venger de Granger, c'est ça ? C'est ce que tu penses, hein ?"

Harry ouvrit la bouche pour dire \emph{À vrai dire, pas du tout} , puis pour une fois dans sa vie parvint à réfléchir et à se taire.

"Je vois…" dit-il lentement. "Alors c'est de \emph{ça}  qu'il s'agit. Lucius Malfoy annonce publiquement que Hermione ne s'en sortira pas comme ça, et, oh surprise, un troll la tue." Harry sourit alors et découvrit ses dents. "Et si je nie cela ici, alors Draco, qui n'est pas Occlumens, pourra témoigner sous Veritaserum que le Survivant ne soupçonne \emph{pas}  Lucius Malfoy d'avoir envoyé un troll à Poudlard tuer Hermione Granger, assermentée à la maison Potter, dont la dette de sang avait récemment été payée au prix de cent mille Gallions etcetera." Harry s'inclina légèrement bien que son tabouret de bois n'ait aucun dossier lui permettant de le faire confortablement. "Mais maintenant que vous l'avez porté à mon attention, je vois que c'est tout à fait plausible. Bien sûr que \emph{vous}  avez tué Hermione Granger, tout comme vous avez menacé de le faire devant tout le Magenmagot."

"Absolument pas," dit Lucius Malfoy, son visage à nouveau insondable.

Harry découvrit une fois de plus ses dents par ce non-sourire. "Eh bien dans \emph{ce}  cas, quelqu'un \emph{d'autre}  doit avoir tué Hermione après s'être joué du système de sécurité de Poudlard, la même personne qui a essayé \emph{plus tôt}  de \emph{faire accuser Hermione du meurtre de Draco Malfoy} . Soit vous avez tué Hermione Granger après reçu un paiement en échange de sa vie, soit vous avez accusé une fille innocente de tentative de meurtre sur votre fils avant de prendre tout l'argent de ma famille pour de faux prétextes ; l'une de ces deux possibilités doit être vraie."

"Peut-être que \emph{vous}  l'avez tuée dans l'espoir de récupérer votre argent." Lucius Malfoy s'était penché en avant et regardait Harry avec sévérité.

"Alors \emph{je}  n'aurais pas donné tout mon argent pour la sauver en premier lieu. Comme vous le savez déjà. N'insultez pas mon intelligence, Lucius Malfoy… non, attendez, désolé, vous deviez \emph{dire}  ça au cas où Draco devrait pouvoir en témoigner, oubliez ça."

Lucius Malfoy se rassit dans sa chaise et contempla Harry.

"J'ai essayé de vous le dire, Père," dit Draco dans un souffle, "mais personne ne peut imaginer à quoi ressemble Harry Potter avant de l'avoir \emph{rencontré} …"

Harry tapota sa joue d'un doigt. "Donc les gens commencent à pouvoir appréhender des évidences absolues ? À vrai dire, je suis surpris. Je n'aurais pas prévu ça." À ce stade, Harry avait saisit le rythme du cynisme du professeur Quirrell et était devenu capable de le générer lui-même. "Je ne pensais pas qu'un journal serait capable mentionner un concept tel que 'Soit X est vrai, soit Y est vrai, mais nous ignorons lequel." Je me serais attendu à ce que les journalistes écrivent des articles ne contenant que des propositions atomiques, comme 'X est vrai', ou 'Y est faux', ou 'X est vrai et Y est faux'. Mais pas des connecteurs logiques plus complexes comme 'Si X est vrai alors Y est vrai mais nous ne savons pas si X est vrai.' Et tous ceux qui vous soutiennent risqueraient de passer rapidement de 'On ne peut pas prouver que Lucius Malfoy a tué Granger, quelqu'un d'autre a pu faire le coup', à 'On ne peut pas prouver qu'il y avait quelqu'un d'autre pour faire accuser Granger' tant qu'ils ne seraient pas certains de devoir essayer d'avoir raison sur les deux tableaux… attendez, est-ce que vous ne \emph{possédez}  pas la Gazette du Sorcier ?"

"La Gazette du Sorcier," dit Lucius Malfoy d'une voix mince, "que je ne possède certainement pas, est bien trop respectable pour publier de telles inepties calomnieuses. Malheureusement, tous les sorciers influents ne sont pas aussi raisonnables."

"Ouais. Compris." Harry hocha la tête.

Lucius jeta un coup d'œil à Draco. "Le reste de ce qu'il a dit - est ce que c'était important ?"

"Non, Père, pas du tout."

"Merci, fils." Lucius regarda Harry de nouveau. Sa voix, lorsqu'il parla, se rapprochait plus de son ton traînant, froid et confiant habituel. "Vous pourriez parvenir à me persuader de vous accorder quelque faveur si vous admettiez devant tout le Magenmagot ce qu'il est clair que vous savez, à savoir que je ne suis pas responsable de ce méfait. Je serais prêt à réduire la dette qui vous lie encore à la maison Malfoy de façon importante ou même à en modifier les termes afin de permettre un paiement ultérieur."

Harry regard Lucius Malfoy sans broncher. "Lucius Malfoy. Vous êtes maintenant parfaitement conscient du fait que Hermione Granger a effectivement été piégée, que votre fils a servi d'appât, qu'elle a été victime d'un sortilège de faux souvenirs ou pire et qu'avant ces événements, la maison Potter n'avait aucun grief à votre égard. Ma contre-proposition est la suivante : rendez-moi l'argent de ma famille. J'annoncerai alors devant tout le Magenmagot que la maison Potter n'a aucune animosité envers la maison Malfoy et nous présenterons un front soudé contre les responsables de tout ceci. Nous pouvons rejeter les rôles que nous sommes censés jouer et nous allier au lieu de combattre. Ce pourrait être la seule chose que l'ennemi ne s'attend pas à ce que nous fassions."

Il y eut un bref silence dans la pièce, exception faite des deux gardes gobelins qui continuèrent de respirer comme si de rien n'était.

"Vous \emph{êtes}  fou," dit Lucius Malfoy avec froideur.

"Ce ne serait que justice, Lucius Malfoy. Vous ne pouvez certainement attendre de moi que je coopère tandis que vous contrôlez la richesse de la maison Potter pour des prétextes que vous savez maintenant être faux. Je comprend que les apparences aient pu auparavant être trompeuses, mais vous n'êtes maintenant plus dupe."

"Vous n'avez rien à m'offrir qui vaille cent mille Gallions."

"Vraiment ?" dit Harry d'une voix distraite. "Je me le demande. Je pense qu'il est assez probable que vous vous préoccupiez plus du bien-être à long terme de la maison Malfoy que de la marotte politique du Seigneur des Ténèbres raté de la génération précédente." Harry jeta un regard lourd de sens vers Draco. "La prochaine génération dessine ses propres lignes de front et forme de nouvelles alliances. Votre fils peut en être exclu, ou alors il peut grimper directement au sommet. Cela n'a-t-il pas plus de valeur pour vous que quarante mille Gallions auxquels vous ne vous attendiez pas particulièrement et dont vous n'avez pas particulièrement besoin ?" Harry eut un fin sourire. "Quarante mille Gallions. Deux millions de livres sterling moldues. Votre fils sait quelque chose de la taille de l'économie moldue ; elle pourrait vous surprendre. Les Moldus trouveraient amusant que le destin d'un pays dépende de deux millions de livres sterling. Ils trouveraient ça mignon. Et je le trouve aussi, Lucius Malfoy. Ce n'est pas que je suis au désespoir. C'est que que vous méritez une chance d'agir de façon juste."

"Oh ?" dit Lucius Malfoy. "Et si je refuse cette chance que je mérite ?"

Harry haussa les épaules. "Cela dépend du genre de gouvernement de coalition qui se forme en l'absence des Malfoy. Si le gouvernement peut être réformé pacifiquement et qu'agir autrement entraverait cette paix, je vous paierait la somme risible que je vous dois. Ou peut-être que les Mangemorts seront à nouveau jugés pour leurs crimes passés puis exécutés en toute justice suite à un procès, légal évidemment."

"Vous êtes vraiment fou," dit doucement Lucius Malfoy. "Vous n'avez ni pouvoir ni fortune, et pourtant vous parlez ainsi devant moi."

"Oui, c'est bête d'imaginer que je pourrais vous faire peur. Après tout, vous n'êtes pas un Détraqueur."

Et Harry continua de sourire. Il avait fait des recherches, et un bézoard \emph{soignerait}  apparemment presque n'importe quel poison si on le fourrait assez vite dans la bouche de la victime. Peut-être qu'il ne réparerait pas les dommages causés par les radiations de polonium métamorphosé, mais peut-être que si. Harry avait alors consulté les points de congélations de plusieurs acides et il s'était avéré que l'acide sulfurique gelait à pile dix degrés, ce qui signifiait que Harry pourrait acheter un litre de cet acide au supermarché moldu, le faire geler puis le métamorphoser en un tout petit glaçon à lancer dans la bouche de quelqu'un. Aucun bézoard ne pourrait combattre ça une fois que la métamorphose se serait estompée. Harry n'avait aucune intention de le dire à voix haute, bien sûr, mais maintenant qu'il avait entièrement échoué à empêcher que quiconque meure pendant le déroulement de sa quête, il n'avait aucune intention de laisser ses actes être restreints, ni par la loi ni même par le code de Batman.

\emph{Dernière chance de vivre, Lucius. À un niveau éthique, ta vie a cessé de t'appartenir le jour où tu as commis ta première atrocité pour le compte des Mangemorts. Tu es encore humain et ta vie a encore une valeur intrinsèque, mais tu ne possèdes plus la protection déontologique dont jouissent les innocents. Toute personne gentille a le droit de te tuer si elle pense que sur le long terme, cela sauvera des vies ; et c'est ce que je penserai de toi si tu commences à te mettre en travers de on chemin. Celui qui a lancé le troll après Hermione devra t'avoir aussi pris pour cible et t'avoir frappé d'une malédiction qui transforme les anciens Mangemorts en un tas de boue. Que c'est triste.} 

"Père," dit Draco d'une petite voix. "Je pense que vous devriez réfléchir à cette offre, père."

Lucius Malfoy regarda son fils. "Tu veux rire."

"Il dit vrai. Je ne pense pas que Harry Potter a inventé ses livres, personne n'aurait pu écrire tout ça et j'y ai lu des choses que j'ai pu moi-même vérifier. Et si seulement la moitié de ce qui s'y trouve est vraie, il a raison, cent mille Gallions ne représenteront pas grand chose. Si nous les lui donnons, il sera réellement de nouveau ami avec la maison Malfoy - de la façon dont \emph{il}  voit l'amitié, en tout cas. Et sinon, il sera notre ennemi, que ce soit dans son intérêt ou pas, il en aura après nous. Il réfléchit vraiment comme ça. Pour lui, il ne s'agit pas d'argent, il s'agit de ce qu'il appelle l'honneur."

Harry Potter inclina la tête. Il souriait toujours.

"Mais soyons clair sur un point," dit Draco en le regardant droit dans les yeux. Une lueur féroce brillait dans ses yeux. "\emph{Tu m'as trompé. } Et\_ tu es mon débiteur.\_"

"Je l'admets," dit doucement Harry. "Mais cela dépend bien sûr du reste."

Lucius Malfoy ouvrit la bouche pour dire quelque chose et la referma. "Fou," répéta-t-il.

Il y eut un long débat entre le père et le fils pendant lequel Harry demeura silencieux.

Puis les yeux de Lucius Malfoy se tournèrent enfin vers Harry. "Et vous croyez," dit Lucius Malfoy, "que vous pouvez persuader Londubat et Bones de vous suivre même si Dumbledore s'y oppose."

Harry hocha la tête. "Ils vous soupçonneront d'être responsable, bien sûr. Mais je leur dirai que c'était mon plan depuis le début, cela devrait aider."

"J'imagine," dit Lucius Malfoy au bout d'un moment, "que je pourrais faire rédiger un contrat vous absolvant de \emph{presque}  tout le reste de votre dette, si je venais à adhérer à cette idée folle. J'aurai bien sûr besoin de plus de garanties…"

Harry fouilla promptement dans ses robes et en sortit un parchemin qu'il déplia et étala sur la table d'or. "À vrai dire, j'ai pris la liberté de le faire," dit Harry. Il avait passé quelques heures studieuses dans la bibliothèque de Poudlard avec les livres de droit disponibles. Heureusement, Harry avait cru constater que les lois d'Angleterre magique étaient d'une charmante simplicité comparées à leurs équivalentes Moldues. Écrire que la dette de sang originelle et le paiement étaient annulés, que la richesse des Potter et tous les autres objets de sa chambre forte lui seraient rendus, que le reste de la dette était annulé, et que les Malfoy n'avaient commis aucune faute, tout cela n'avait pris que quelques lignes de plus que l'équivalent oral. "J'ai dû promettre à mes gardiens de ne rien signer qui vienne de vous. Je me suis donc assuré de rédiger ceci moi-même et de le signer avant de partir."

Draco s'étrangla dans un petit rire.

Lucius lit le contrat et sourit sans gaieté. "C'est d'une simplicité charmante."

"J'ai aussi promis de ne pas toucher de plume tant que je serai à Gringotts," dit Harry. Il fouilla dans ses robes et en sortit un stylo Moldu et une feuille de papier normale. "Cette formulation vous conviendra-t-elle ?" Harry griffonna rapidement une note telle qu'un avocat aurait pu l'écrire qui disait que la maison Potter ne tenait pas la maison Malfoy pour responsable de du meurtre de Hermione Granger de quelque façon que ce soit et ne croyait pas qu'elle avait quoi que ce soit à voir avec celui-ci, puis il souleva la feuille afin que Lucius Malfoy puisse la lire.

Lucius Malfoy regarda la feuille, leva quelque peu les yeux au ciel puis dit : "J'imagine que cela suffira. Même si le mot correct serait \emph{absous}  plutôt que \emph{disculpe} …"

"Bien essayé, mais non. Je sais exactement ce que ce mot signifie, Lord Malfoy." Harry prit son parchemin et commença à recopier sa formulation initiale plus soigneusement.

Lorsqu'il eut fini, Lord Malfoy tendit son bras au-dessus de la table, saisit le stylo et le regarda pensivement. "L'un de vos ustensiles moldus, je suppose ? Que fait celui-ci, fils ?"

"Il écrit sans encrier," répondit Draco.

"Je peux le constater. J'imagine que certains pourraient voir là un amusant colifichet." Lucius aplanit le contrat à la surface de la table, plaça sa main sous la ligne destinée à recevoir sa signature et tapota pensivement le stylo contre le début de celle-ci.

Harry s'arracha à ce spectacle et regarda le visage Lucius Malfoy. Il se força à respirer de façon régulière mais ne parvint pas tout à fait à empêcher ses muscles de se tendre.

"Notre bon ami, Severus Rogue," dit Lucius Malfoy en tapotant toujours son stylo sur la ligne qui attendait sa signature. "Le professeur de Défense, qui se fait appeler Quirrell. Je vous le demande à nouveau : qui est votre troisième suspect, Harry Potter ?"

"Si vous comptez de toute façon signer, je vous conseille fortement de commencer le faire tout de suite, Lord Malfoy. Vous profiterez plus de cette information si vous ne croyez pas que j'essaie de vous persuader de quelque chose."

Un autre sourire sans joie. "Je prendrai le risque. Parlez, si vous souhaitez que cette affaire se poursuive."

Harry hésita puis il dit d'un ton neutre : "Mon troisième suspect est Albus Dumbledore."

Le stylo se figea sur le parchemin. "Un étrange allégation," dit Lucius d'une voix traînante. "Dumbledore a grandement perdu la face quand une élève de l'école qu'il dirige est morte. Pensez-vous que je croirais n'importe quoi le concernant, simplement parce qu'il est mon ennemi ?"

"Il n'est qu'un suspect parmi plusieurs, Lord Malfoy, et pas forcément le plus plausible. Mais si j'ai été capable de tuer un troll des montagnes adulte, c'est parce que j'avais une arme que Dumbledore m'avait offerte au début de l'année. Ce n'est pas une preuve, mais c'est suspect. Et si vous pensez que ce n'est pas le style de Dumbledore de tuer l'une de ses élèves, eh bien, la même pensée m'est venue."

"Ce n'est \emph{pas}  son style ?" dit Draco Malfoy.

Lucius Malfoy secoua la tête d'un mouvement mesuré, prudent. "Pas tout à fait, mon fils. Dumbledore a une certaine façon de faire le mal." Lord Malfoy se renversa dans sa chaise puis devint particulièrement immobile. "Parlez-moi de cette arme."

"Je pense qu'il y a des détails que je ne devrais pas dévoiler en votre présence, Lord Malfoy." Harry inspira. "Laissez-\emph{moi}  être clair sur ce point. Je n'essaie pas de vous faire gober l'idée que Dumbledore est derrière cela, je ne fais que mettre en avant la possibilité…"

Draco Malfoy parla alors. "L'appareil que Dumbledore t'a donné - est-ce que c'était fait pour tuer les trolls ? Je veux dire \emph{juste}  les trolls ? Peux-tu nous dire cela ?"

Lucius tourna la tête et regarda son fils avec une certaine surprise.

"Non…" dit lentement Harry. "Ce n'était pas une épée spécialement faite pour pourfendre les trolls ou quelque chose comme ça."

Les yeux de Draco étaient braqués sur lui. "L'appareil aurait-il fonctionné contre un assassin ?"

\emph{Pas si ses boucliers avaient été levés.}  "Non."

"Lors d'un combat contre une brute ?"

\emph{Un rocher en expansion logé dans la gorge est fondamentalement mortel.}  "Non. Je ne pense pas qu'il était censé être utilisé contre des humains."

Draco hocha la tête. "Donc seulement des créatures magiques. Est-ce qu'il aurait constitué une bonne arme contre un hippogriffe en colère ou quelque chose comme ça ?"

"Est-ce que le sortilège d'étourdissement fonctionne sur les hippogriffes ?" dit lentement Harry.

"Je ne sais pas," dit Draco.

"Oui," dit Lucius Malfoy.

\emph{Comparé à essayer de bien ajuster un Wingardium Leviosa et un Finite Incantatem…}  "Alors le sortilège d'étourdissement serait un meilleur moyen de se débarrasser d'un hippogriffe." Vu sous cet angle, il semblait de plus en plus plausible qu'un rocher métamorphosé n'était une arme optimale \emph{que}  contre une créature magique faite de chair et de sang mais dotée d'une peau résistante à la magie. "Mais… enfin, la chose aurait pu ne pas être \emph{censée}  servir d'arme, je l'ai utilisé d'une façon étrange, ça aurait pu n'être qu'une de ses lubie…"

"Non," dit lentement Lucius Malfoy. "\emph{Pas}  une lubie. Pas une coïncidence. Pas Dumbledore."

"Alors, c'est lui," dit Draco. Ses yeux se plissèrent lentement et il eut un hochement de tête mauvais. "C'est lui \emph{depuis le début} . Le Legilimens de la cour a \emph{dit}  que quelqu'un avait utilisé la Légilimancie sur Granger. Dumbledore a \emph{admit}  que c'était lui. Et je parie que le système de sécurité \emph{s'est}  activé quand Granger m'a lancé le sort mais que Dumbledore l'a juste \emph{ignoré} ."

"Mais…" dit Harry. Il regarda Lucius en se demandant si c'était vraiment à son avantage de remettre cette idée en question. "Quel serait son \emph{motif}  ? Allons-nous dire qu'il est méchant et en rester là ?"

Draco Malfoy sauta de sa chaise et commença à déambuler à travers la pièce. Ses robes noires flottaient derrière lui et les gardes gobelins le regardèrent, assez surpris, à travers leurs lunettes enchantées. "Pour comprendre un plan tordu, il faut regarder ce qui se passe et se demander qui en profite. Sauf que Dumbledore ne s'attendait pas à ce que tu essaies de sauver Granger au procès, il a essayé de t'en empêcher. Qu'est-ce qui se serait passé si Granger \emph{était}  allée à Azkaban ? Les maisons Malfoy et Potter se seraient détestées pour toujours. De tous les suspects, le seul qui désire \emph{ça} , c'est Dumbledore. Donc ça colle. \emph{Tout}  colle. Celui qui a vraiment commis ce meurtre est… Albus Dumbledore !"

"Euh," dit Harry. "Mais pourquoi me donner un arme anti-troll à \emph{moi}  ? J'ai dit que c'était suspect mais je n'ai pas dit que ça avait le moindre sens."

Draco hocha pensivement la tête. "Peut-être que Dumbledore pensait que tu arrêterais le troll avant qu'il n'arrive jusqu'à Granger, et qu'alors il aurait pu blâmer Père. Beaucoup de gens seraient très en colère s'ils pensaient que père avait ne serait-ce qu'\emph{essayé}  de faire une chose pareille dans l'enceinte Poudlard. Comme Père a dit, Dumbledore a du perdre la face quand les gens ont découvert qu'une élève était vraiment morte à l'école ; Poudlard est connue pour sa sécurité. Donc ça n'était probablement pas censé se produire."

L'esprit de Harry revint involontairement à l'horreur dans les yeux de Dumbledore lorsque ce dernier avait vu le corps de Hermione Granger.

\emph{Serais-je arrivé à temps si les jumeaux Weasley ne s'étaient pas fait voler leur carte magique ? Cela aurait-il pu être son plan ? Et alors quelqu'un a volé leur carte sans que Dumbledore le sache et je suis arrivé trop tard… mais non, ça n'a pas vraiment de sens, j'ai découvert ça trop tard, et puis comment Dumbledore aurait-il pu deviner que j'allais utiliser un balai… eh bien il savait que j'en avais un…} 

Un tel plan n'aurait jamais pu marcher.

Et il n'avait pas marché.

Mais quelqu'un à la limite de la sénilité aurait pu \emph{s'attendre}  à ce qu'il marche et un phénix était peut-être incapable de faire la différence.

"Ou," continua Draco en déambulant toujours, "peut-être que Dumbledore a un troll enchanté sous le coude et qu'il s'attendait à ce que tu le vainques une autre fois, lors d'un autre plan, mais qu'il a fini par décider d'utiliser le troll contre Granger. Je ne peux pas imaginer que Dumbledore avait prévu \emph{tout}  ça depuis notre première semaine d'école…"

"Je peux l'imaginer," dit Lucius Malfoy d'un ton bas. "J'ai déjà vu chose semblable, venant de Dumbledore."

Draco eut un hochement de tête appuyé. "Alors je n'ai jamais été \emph{censé}  mourir lors du premier plan. Dumbledore savait que le professeur Quirrell me surveillait, ou bien il comptait s'arranger pour que quelqu'un me découvre à temps - ma mort lui aurait fait perdre la face et m'aurait empêché de témoigner contre Hermione Granger. Mais me faire partir pour que je ne puisse pas prendre la tête de Serpentard l'arrangeait bien. Et la fois suivante, Harry était censé arrêter le troll avant qu'il n'atteigne Granger, car alors tout le monde vous aurait accusé vous, Père, sauf que cette fois les choses ne se sont pas passées comme il l'avait prévu."

Lucius Malfoy leva ses yeux verts après avoir regardé son fils avec une surprise non dissimulée. "Si c'est vrai… mais je me demande si Harry Potter ne fait que simuler sa réticence à y croire."

"Peut-être," dit Draco. "Mais je suis à peu près sûr que non."

"Alors, si c'est vrai…" la voix de Lucius Malfoy demeura en suspens. Une lente furie s'allumait dans ses yeux.

"Que \emph{ferions-nous} , exactement ?" dit Harry.

"Cela aussi me semble clair," dit Draco. Il fit un demi-tour et leva un doigt en l'air. "Nous trouverons la preuve de la culpabilité de Dumbledore et le présenterons à la justice !"

Harry Potter et Lucius Malfoy se regardèrent.

Aucun des deux ne savait tout à fait quoi répondre.

"Mon fils," dit Lucius Malfoy après un moment, "tu as vraiment beaucoup accompli aujourd'hui."

"Merci, Père !"

"Cela dit, nous ne sommes pas dans une pièce de théâtre, nous ne sommes pas de Aurors, et nous ne nous en remettons pas au bon vouloir d'un procès."

Les yeux de Draco se ternirent un peu. "Oh."

"Je, ah, j'ai un certain faible pour les procès," lança Harry. \emph{Je n'arrive pas à croire que je suis vraiment en train de parler de ça} . Il avait besoin de rentrer chez lui puis de prendre un papier et un crayon pour voir si le raisonnement de Draco se tenait \emph{vraiment} . "Et pour les preuves."

Lucius Malfoy se retourna alors vers Harry Potter ; ses yeux étincelaient d'une furie grise.

"Si vous m'avez trompé," dit Lucius Malfoy d'un ton de colère sourde, "si tout ceci est un mensonge, alors je ne vous pardonnerai pas. Mais si ce n'en est pas une… amenez-moi la preuve nécessaire à condamner Dumbledore devant le Magenmagot, ou la preuve nécessaire à le renverser, et il n'y a rien que la maison Malfoy ne fera pour vous, Harry Potter. Rien."

Harry inspira profondément. Il avait besoin de tout reprendre et de calculer les vraies probabilités mais il n'en avait pas le \emph{temps} . "Si \emph{c'est}  Dumbledore, alors le retirer de l'échiquier créera un immense trou dans la structure du pouvoir d'Angleterre."

"Effectivement," dit Lucius Malfoy avec un sombre sourire. "Aviez-vous l'ambition de le combler vous-même, Harry Potter ?"

"Une partie de vos opposants pourrait ne pas apprécier. Ils pourraient vous combattre."

"Ils perdront," dit Lucius Malfoy ; et son visage avait à présent la dureté de l'acier.

"Alors voilà ce que je voudrais que la maison Malfoy fasse pour moi, Lord Malfoy, si Dumbledore se faisait destituer à cause de moi. Quand l'opposition sera le plus effrayée - c'est là qu'on leur offrira un compromis de dernière minute destiné à éviter la guerre civile. Ce ne sera peut-être pas ce que certains de vos alliés auraient préféré, mais de nombreux partis neutres seront heureux de cette perspective de stabilité. Le marché sera qu'au lieu de vous donner le pouvoir à vous, tout de suite, ils le donneront à Draco Malfoy lorsqu'il sera assez âgé."

"\emph{Quoi ?} " dit Draco.

"Draco a témoigné sous Veritaserum qu'il de sa tentative d'aider Hermione Granger. Je parie qu'il y aura de nombreux opposants qui choisiront de tenter leur chance avec lui plutôt que de se battre. Je ne sais pas exactement comment ce sera mis en vigueur - un Serment Inviolable, un contrat de Gringotts ou quelque chose comme ça - mais il y aura une sorte d'accord disant que le pouvoir reviendra à Draco après ses études à Poudlard. J'encouragerai tous les partisans que peut avoir le Survivant à ratifier ce marché. J'essaierai de persuader Londubat, Bones, et les autres. Notre premier plan ouvrira la voie au reste, si vous prenez garde de traiter de façon honorable avec Londubat et Bones pour cette fois."

"Père, je \emph{jure}  que je n'ai pas…"

Le visage de Lucius se tordit en un sinistre sourire. "Je sais que tu n'avais pas prévu ça, mon fils. Bien." L'homme aux cheveux blancs regarda Harry Potter, assis de l'autre côté de la table. "J'accepte ces termes. Mais faites défaut à une partie de notre accord, que ce soit notre premier marché ou le second, et il y aura des conséquences pour vous, Harry Potter. Aucun argument ingénieux ne saura l'empêcher."

Et Lucius Malfoy signa le parchemin.
\par\noindent\rule{\textwidth}{0.4pt}
Maugrey Fol Œil avait regardé la porte en bronze de la salle de réunion de Gringotts pendant des heures dans la mesure où un homme capable de voir dans toutes les directions à la fois pouvait regarder quelque chose.

Maugrey songeait que le problème, quand on essayait de soupçonner un homme comme Lucius Malfoy, c'était qu'on pouvait passer la journée à penser à tout ce qu'il pourrait fomenter et n'en avoir toujours pas fini.

La porte s'ouvrit grand et Harry Potter sortit à pas lourds, des gouttes de sueur sur le front.

"Est-ce que tu as signé quelque chose ?" demanda immédiatement Maugrey.

Harry Potter le regarda en silence puis fouilla dans ses robes et en sortit un parchemin plié. "Les gobelins mettent déjà celui-ci en œuvre," dit Harry Potter. "Ils en ont fait trois copies avant que je parte."

"MERLIN SOIT MAUDIT, PETIT…" Maugrey s'arrêta quand son Œil perçut la seconde moitié du document que Harry Potter commençait à lentement dérouler, comme avec réticence. Un regard lui suffit à lire les paragraphes écrits d'une main soigneuse et à voir la signature de Lucius Malfoy en-dessous de celle de Harry Potter. Maugrey explosa alors quand la partie supérieure du document entra dans son champ de Vision. "Tu \emph{disculpes la maison Malfoy de toute lien avec la mort de Hermione Granger ?}  Est-ce que tu as la moindre idée de ce que tu as fait, petit idiot ? Nom de Merlin, pourquoi est-ce que tu ferais une chose OH BORDEL…"

