
\chapter{Réciprocité}

Waoh. Un porte-parole de l'agent de Rowling a dit qu'elle accepte l'existence de fanfictions tant que personne ne les fait payer et que tout le monde est d'accord avec le fait que les copyrights originaux lui appartiennent ? C'est vraiment sympa de sa part. Je ne savais pas. Donc merci, JKR, et vôtre soit le royaume !
\par\noindent\rule{\textwidth}{0.4pt}
Je ressens le besoin de prévenir que je n'utilise pas ce chapitre pour me "défouler". Je n'ai aucune rancœur contre qui que ce soit, c'est juste que l'histoire s'écrit elle-même, et quand on commence à lâcher des enclumes sur la tête d'un personnage, c'est dur de s'arrêter.

Quelques critiques ont demandé si les informations scientifiques de cette histoire étaient vraies ou inventées. Oui, elles sont vraies, et si vous allez voir mon profil, vous verrez un lien vers un certain site d'information qui vous enseignera à peu près tout ce que Harry James Potter-Evans-Verres sait \emph{et un peu plus encore} .

Merci à \emph{tous}  mes critiques. (Particulièrement à Darkandus on Viridan Dreams pour son commentaire étonnamment inspirant "Les poumons et le thé ne sont pas faits pour interagir.")
\par\noindent\rule{\textwidth}{0.4pt}
"\emph{Ton père est presque aussi génial que mon père."} 
\par\noindent\rule{\textwidth}{0.4pt}
Les lèvres de Pétunia Evans-Verres tremblaient et ses yeux larmoyaient pendant que Harry étreignait son buste sur le quai numéro neuf de la gare de King's Cross. "Tu es sûr que tu ne veux pas que je vienne avec toi, Harry ?"

Harry leva les yeux vers elle. Il jeta un coup d'œil à son père, Michael Verres-Evans, qui avait un air de dur-mais-fier typique, puis à nouveau à sa mère, qui avait l'air d'avoir plutôt... perdu son quant-à-soi. "Maman, je sais que tu n'aimes pas beaucoup le monde magique. Tu n'es pas obligée de venir. Vraiment pas."

Pétunia grimaça. "Harry, tu ne devrais pas t'en faire pour moi, je suis ta mère et si tu as besoin que quelqu'un soit avec toi -"

"Maman, je serai seul à Poudlard pendant des \emph{mois}  et des \emph{mois} . Si je n'arrive pas à gérer un quai seul, mieux vaut l'apprendre plus tôt que plus tard et pouvoir encore tout annuler." Il baissa le volume de sa voix jusqu'à atteindre celui d'un murmure. "Et puis Maman, ils m'aiment tous là-bas. Si j'ai le moindre problème, tout ce que j'aurai à faire c'est d'enlever mon bandeau," Harry tapota le bandeau de sport qui recouvrait sa cicatrice, "et j'aurai alors \emph{beaucoup}  plus d'aide que je ne pourrais en désirer."

"Oh, Harry," murmura Pétunia. Elle s'agenouilla et le serra fort dans ses bras, face à lui, leurs joues l'une contre l'autre. Harry pouvait sentir sa respiration saccadée, puis il entendit un sanglot s'échapper de ses lèvres, étouffé et masqué, mais présent. "Oh, Harry, je t'aime, souviens-toi toujours de ça."

\emph{C'est comme si elle avait peur de ne plus jamais me revoir,}  la pensée surgit brutalement dans la tête de Harry. Il savait que la pensée était vraie mais il ne savait pas pourquoi Maman avait si peur.

Il essaya de deviner. "Maman, tu sais que je ne vais pas devenir comme ta sœur juste parce que j'apprends la magie ? Je ferai toute la magie que tu voudras - enfin, si j'en suis capable - et si tu veux que je n'utilise \emph{aucune}  magie dans la maison, je le ferais aussi, je te promets que je ne laisserai jamais la magie nous séparer.

Un câlin écrasant lui coupa le souffle. "Tu as bon cœur," lui murmura sa mère à l'oreille. "Très bon cœur, mon fils."

Et alors Harry s'étrangla un peu lui aussi.

Sa mère le relâcha et se leva. Elle sortit un mouchoir de sa poche et, d'une main tremblante, tamponna ses yeux et son maquillage qui coulait.

Aucun question ne fut posée sur la possibilité que son père puisse l'accompagner du côté magique de la gare de King's Cross. Papa avait du mal à ne serait-ce que regarder la malle de Harry. La magie courait de familles en familles, et Michael Verres-Evans ne pouvait même pas marcher.

Au lieu de ça son père s'éclaircit la gorge. "Bonne chance à l'école, Harry," dit-il. "Penses-tu que j'ai acheté assez de livres ?"

Harry avait expliqué à son père qu'il pensait que c'était sa chance de faire quelque chose de vraiment révolutionnaire et important, et le Professeur Verres-Evans avait hoché la tête et avait annulé son emploi du temps très chargé pour deux jours pleins afin d'organiser le Plus Grand Raid de Librairies d'Occasion Jamais Fait, qui avait couvert quatre villes et produit \emph{trente}  boîtes de livres scientifiques attendant à présent à l'étage caverne de la malle de Harry. La plupart des livres s'étaient vendus pour une livre ou deux, mais certains s'étaient vendus pour certainement \emph{plus}  que ça, comme le tout dernier \emph{Manuel de Chimie et Physique}  ou l'ensemble complet de l'\emph{Encyclopaedia Britannica}  1972. Son père avait essayé d'empêcher Harry de voir les prix, mais Harry avait estimé que son père avait dépensé \emph{au moins}  mille livres. Harry avait dit à son père qu'il le rembourserait dès qu'il aurait appris à convertir l'or des sorciers en argent Moldu, et son père lui avait dit d'aller se jeter dans un lac.

Puis son père lui avait demandé : \emph{Penses-tu que j'ai acheté assez de livres ?.}  La réponse que Papa attendait était très claire.

Bizarrement, la gorge de Harry était enrouée. "On ne peut jamais avoir assez de livres," dit-il, récitant la devise de la famille Verres, et son père s'agenouilla et lui donna un câlin bref et résolu. "Mais tu as \emph{certainement}  essayé," dit Harry, et il se sentit s'étrangler à nouveau. "C'était un très, très, \emph{très}  bon essai."

Son père se redressa. "Alors..." dit-il. "Vois-\emph{tu}  un quai neuf trois-quarts ?"

La gare de King's Cross était énorme et bondée, les murs et le sol couverts d'ordinaires carreaux couleur terre, plein de gens ordinaires se dépêchant vers leurs affaires ordinaires et ayant des conversations ordinaires qui généraient beaucoup de bruit ordinaire. La gare de King's Cross avait un quai numéro neuf (sur lequel ils se tenaient) et un quai numéro dix (juste à côté) mais il n'y avait absolument rien entre le quai numéro neuf et le quai numéro dix mis à part un mince mur-écran peu prometteur. Une immense ouverture en direction du ciel laissait entrer plus qu'assez de clarté pour illuminer l'absence totale de quoi que ce soit entre le quai neuf et le quai dix.

Harry regarda autour de lui sans ciller jusqu'à ce que ses yeux pleurent, pensant, \emph{allez, vue-de-mage, allez, vue-de-mage} , mais absolument rien ne lui apparut. Il pensa à sortir sa baguette et à l'agiter, mais McGonagall l'avait mis en garde contre l'usage de sa baguette. Et puis si il y avait une nouvelle douche d'étincelles multicolores cela pourrait les mener à une arrestation pour allumage de feux d'artifice dans une gare. Et encore, cela partait du principe que sa baguette ne déciderait pas de faire autre chose, comme par exemple de faire exploser King's Cross. Harry n'avait que survolé ses livres scolaires (et ça avait été un survol des plus étranges) dans un effort rapide destiné à déterminer quel genre de livres scientifiques il devrait acheter durant les 48 heures suivantes.

Eh bien, il avait - Harry jeta un coup d'œil à sa montre - une heure entière pour trouver une solution, puis qu'il était censé être à bord du train à onze heures. Peut-être que c'était l'équivalent d'un test de QI et que les enfants stupides ne pouvaient pas devenir sorciers. (Et le temps supplémentaires que vous vous octroyiez déterminait si vous étiez consciencieux, ce qui était le second facteur le plus important de la réussite scolaire.)

"Je trouverai un moyen," dit Harry à ses parents, qui attendaient. "C'est probablement un espèce de test."

Le père de Harry grimaça. "Hm... essaie peut-être de chercher des traces de pas au sol menant à un endroit absurde -"

"\emph{Papa !} " dit Harry. "Arrête ça ! Je n'ai même pas encore \emph{essayé}  de trouver la solution tout seul !" En plus, c'était une très bonne suggestion, ce qui était pire.

"Désolé," s'excusa son père.

"Ah..." dit la mère de Harry. "Je ne pense pas qu'il feraient ça à un étudiant, si ? Tu es sûr que le Professeur McGonagall ne t'a rien dit ?"

"Peut-être qu'elle était distraite," dit Harry sans vraiment réfléchir.

"\emph{Harry !} " sifflèrent son père et sa mère à l'unisson. "\emph{Qu'as-tu fait ?} "

"J'ai, euh -" Harry avala sa salive. "Écoutez, on n'a pas le temps pour ça."

"\emph{Harry}  !"

"Je suis sérieux ! On n'a pas le temps maintenant ! Parce que c'est vraiment une longue histoire et que je dois trouver comment aller à l'école !"

La mère de Harry se plaquait une main sur le visage. "C'était grave à quel point ?"

"Je, ah," \emph{Je ne peux pas en parler pour des raisons de Sécurité Nationale} , "à peu près moitié aussi grave que l'Incident du projet à la Foire Scientifique ?"

"\emph{Harry !} "

"J'ai, euh, oh regardez il y a des gens avec une chouette je vais leur demander comment aller sur le quai !" et Harry couru loin de ses parents vers une famille aux flamboyants cheveux roux, sa malle ondulant automatiquement derrière lui.

Une femme dodue leva les yeux vers lui alors qu'il approchait. "Bonjour mon cher, première fois à Poudlard ? Ron est nouveau, lui aussi -" et elle se figea. Elle le dévisagea très attentivement. "\emph{Harry Potter}  ?"

Quatre garçons et une fille aux cheveux roux et une chouette pivotèrent et se figèrent sur place aux aussi.

"Oh, \emph{non mais franchement}  !" protesta Harry. Il avait prévu de se faire appeler M. Verres au moins jusqu'à Poudlard. "J'ai acheté un bandeau, et tout ! Comment savez-vous qui je suis ?"

"Oui," dit le père de Harry, arrivant de derrière lui en longues enjambées faciles, "comment \emph{savez-vous}  qui il est ?" Sa voix comportait une note d'effroi.

"Ta photo était dans les journaux," dit un des deux vrais jumeaux.

"\emph{HARRY !"} 

"Papa ! Ce n'est pas ce que tu crois ! C'est parce que j'ai vaincu le Seigneur des Ténèbres Tu-Sais-Qui quand j'avais un an !"

"\emph{QUOI } ?"

"Maman peut t'expliquer."

"\emph{QUOI ?"} 

"Ah... Michael, très cher, il y a certaines choses avec lesquelles j'ai pensé qu'il serait mieux de ne pas t'embêter jusqu'à maintenant -"

"Excusez-moi," dit Harry à la famille rousse dont tous les membres le fixaient, "mais ce serait plutôt extrêmement utile si vous pouviez me dire comment je peux me rendre sur le quai neuf trois quart \emph{tout de suite} ."

"Ahhh..." dit la femme. Elle leva une baguette et pointa en direction du mur entre les quais. "Marche juste tout droit vers la barrière entre les quais neuf et dix. Ne t'arrête pas et n'ai pas peur de t'écraser dedans, c'est très important. Mieux vaut y aller au petit trot si tu es nerveux."

"Et quoi que tu fasses, ne pense pas à un éléphant."

"\emph{George}  ! Ignore-le, cher Harry, il n'y a aucune raison de ne pas penser à un éléphant."

"Je suis Fred, Maman, pas George -"

"Merci !" dit Harry, et il partit en courant vers la barrière.

Attends une minute, ça ne marcherait pas \emph{sauf si il y croyait}  ?

C'était dans les moments comme ceux-là que Harry haïssait le fait que son cerveau fonctionne assez vite pour se rendre compte qu'il se trouvait dans une situation où le "doute résonnant" s'appliquait, c'est à dire que si il avait commencé par penser qu'il traverserait la barrière alors tout se serait bien passé, mais maintenant il était inquiet de ne pas assez \emph{croire}  au fait qu'il traverserait la barrière, ce qui voulait dire qu'il \emph{était}  vraiment inquiet de s'écraser dessus -

"\emph{Harry ! Reviens ici, j'attends des explications !} " C'était son Père.

Harry ferma les yeux et ignora tout ce qu'il savait au sujet des croyances justifiées et essaya juste de croire \emph{très fort}  qu'il allait traverser la barrière et -

- les sons l'entourant changèrent.

Harry ouvrit les yeux et s'arrêta net. Il se sentait vaguement sale d'avoir fait un effort délibéré pour croire en quelque chose.

Il se tenait sur un quai illuminé, à l'air libre, à côté d'un unique train gigantesque, long de quatorze voitures, précédé par une immense locomotive à vapeur en métal écarlate avec une cheminée qui promettait "mort" à la qualité de l'air. Le quai était déjà légèrement bondé (bien que Harry eut une bonne heure d'avance) et des douzaines d'enfants ainsi que leurs parents fourmillaient autour des bancs, des tables, et de divers camelots et vendeurs.

Il était inutile de mentionner qu'il n'y avait pas d'endroit comme celui ci à la gare de King's Cross, ni d'espace pour l'y cacher.

\emph{Ok, donc soit (a) Je me suis téléporté dans un tout autre lieu (b) ils savent plier l'espace comme ce n'est pas permis ou (c) ils ignorent simplement les règles.} 

Il y avait un son de glissement derrière lui, et Harry se retourna pour confirmer que sa malle l'avait bien suivi sur ses petites tentacules griffues. Apparemment, pour des raisons magiques, son bagage était parvenu à croire avec assez de force pour passer à travers la barrière. C'était à vrai dire assez troublant, si on y réfléchissait.

Un moment plus tard, le garçon roux qui semblait être le plus jeune traversa l'arcade d'acier (arcade d'acier ?) en courant, tirant sa malle derrière lui avec une laisse et s'écrasant presque sur Harry. Harry se sentit stupide de n'avoir pas bougé et commença à s'écarter rapidement de la zone d'atterrissage, et le garçon roux le suivi en tirant avec force sur la laisse de sa malle pour rester à hauteur. Un moment plus tard, une chouette blanche voleta à travers l'arcade et vint se poser sur l'épaule du garçon.

"Cor," dit le garçon roux, "es-tu \emph{vraiment}  Harry Potter ?"

\emph{Pas ça encore} . "Je n'ai aucune méthode logique me permettant d'en être certain. Mes parents m'ont éduqué de façon à ce que je \emph{crois}  être Harry Potter, et beaucoup de gens ici m'ont dit que je \emph{ressemblais}  à mes parents, je veux dire mes autres parents, mais," Harry fronça les sourcils, se rendant compte que : "pour tout ce que \emph{j'} en sais, il pourrait tout à fait y avoir des sorts permettant de polymorpher un enfant en une autre apparence spécifique -"

"Euh, quoi mon gars ?"

\emph{Pas parti pour Serdaigle on dirait} . "Oui, je suis Harry Potter."

"Je suis Ron Weasley," dit le grand gamin aux taches de rousseurs et au grand nez, et il exhiba une main tendue que Harry serra poliment tandis qu'ils marchaient. La chouette donna à Harry un hululement étrangement mesuré et courtois (à vrai dire plutôt une sorte de "eehhhhh", ce qui surprit Harry).

C'est alors que Harry se rendit compte de la possibilité d'une catastrophe imminente et élabora un moyen de l'empêcher. "Juste un instant," dit-il à Ron, et il ouvrit l'un des tiroirs de sa malle, celui qui, si il se souvenait bien, était pour les Vêtements d'Hiver - c'était le cas - et il trouva, sous son manteau d'hiver, l'écharpe la plus légère en sa possession. Harry enleva son bandeau et tout aussi rapidement déplia l'écharpe et l'enroula autour de son visage. Ça lui donnait un peu chaud, particulièrement en été, mais Harry pourrait survivre.

Puis il ferma le tiroir (contenant maintenant son bandeau inutile, bien qu'il n'appartienne pas vraiment à ce compartiment) et tira un autre tiroir dont il extirpa ses robes noires de sorcier, qu'il se fourra par-dessus la tête, puisqu'il était maintenant hors du territoire Moldu.

"Voilà," dit Harry, satisfait. Le son ne fut que légèrement étouffé par l'écharpe sur son visage. Il se tourna vers Ron. "De quoi j'ai l'air ? Stupide, je sais, mais peut-on m'identifier comme étant Harry Potter ?"

"Euh," dit Ron. Il ferma sa bouche, qui avait été grande ouverte. "Pas vraiment, Harry."

"Très bien," dit Harry. "Cependant, et afin de ne pas déjouer le but de cet exercice, tu t'adressera dorénavant à moi par," Verres risquait de ne plus fonctionner, "M. Spoo."

"D'accord Harry," dit Ron avec incertitude.

\emph{La Force n'est pas très puissante chez celui-ci} . "Appelle... moi... Monsieur... Spoo."

"D'accord, Monsieur Spoo -" Ron s'interrompit. "Je ne peux pas faire ça, ça me fait me sentir stupide."

\emph{Ce n'est pas qu'une sensation} . "D'accord. \emph{Toi} , choisis un nom."

"M. Canon," dit Ron immédiatement. "comme les Canons de Chudley."

"Ah..." Harry avait une affreuse appréhension lui disant qu'il allait horriblement regretter d'avoir posé cette question : "Qui ou que sont les Canons de Chudley ?"

"\emph{Qui sont les Canons de Chudley ?}  Une des meilleures équipes de toute l'Histoire du Quidditch ! Bien sûr ils ont finit derniers de la ligue l'année dernière mais -"

"Qu'est-ce que le Quidditch ?"

Poser cette question fut aussi une erreur.

"Donc laisse moi résumer," dit Harry lorsqu'il sembla que l'explication de Ron (associée de maints gestes) s'épuisait. "Attraper le Vif vaut \emph{cent cinquante points}  ?"

"Ouais -"

"Combien de buts à dix points marque-t-on généralement \emph{sans}  compter le Vif ?"

"Euh, peut-être quinze ou vingt dans les parties professionnelles -"

"C'est juste stupide. Ça viole toutes les règles possibles de la conception de jeux. Écoute, le reste de ce jeu a l'air plus ou moins sensé, grosso modo, pour un sport en tout cas, mais tu es en train de me dire qu'attraper le Vif écrase presque tout autre écart de point. Les deux Attrapeurs sont là à voler dans les airs à la recherche du Vif et n'interagissent généralement avec personne, repérer le Vif en premier sera généralement une affaire de chance -"

"Ce n'est pas de la chance !" protesta Ron. "Tu dois garder tes yeux en mouvement avec la bonne technique -"

"Ce n'est pas \emph{interactif} , il n'y a pas de va-et-vient avec l'autre joueur, et puis à quel point est-ce amusant de regarder quelqu'un d'incroyablement doué pour bouger ses yeux ? Au bout d'un moment l'Attrapeur qui a un coup de chance se précipite, il attrape le Vif et rend tout le travail des autres inutile. C'est comme si quelqu'un avait prit un vrai jeu et y avait greffé ce poste supplémentaire inutile juste pour que quelqu'un puisse être Le Joueur Le Plus Important sans vraiment avoir besoin de participer ni d'apprendre le reste du jeu. Qui était le premier Attrapeur, le fils idiot du Roi qui voulait jouer au Quidditch mais ne pouvait pas comprendre les règles ?" En fait, maintenant que Harry y réfléchissait, ça semblait être une hypothèse étonnamment bonne. Mettez-le sur un balai et dites-lui d'attraper le truc brillant...

Le visage de Ron se renfrogna. "Si tu n'aimes pas le Quidditch, tu n'as pas à t'en moquer !"

"Si on ne peut pas critiquer, on ne peut pas améliorer. Je suggère des façons d'\emph{améliorer le jeu} . Et c'est très simple. Virez le Vif."

"Ils ne vont pas changer le jeu juste parce \emph{tu}  leur dit de le faire !"

"Je \emph{suis}  le Survivant, tu sais. Les gens m'écouteront. Et peut-être que si j'arrive à les persuader de changer le jeu à Poudlard, l'innovation se répandra."

Un air d'horreur absolue se répandait sur le visage de Ron. "Mais, mais, si tu enlèves le Vif, comment qui que ce soit saura que le jeu est finit ?"

"\emph{Achetez... une... horloge.}  Ce serait beaucoup plus équitable que d'avoir des parties se terminant parfois au bout de dix minutes, parfois pas après plusieurs heures, et l'organisation serait aussi beaucoup plus prévisible pour les spectateurs." Harry soupira. "Oh, arrête de me donner cet air d'horreur absolue, je ne vais probablement pas \emph{vraiment}  prendre le temps de détruire cette chose pathétique que vous appelez sport national, et de le rebâtir plus fort et plus intelligent ; à mon image. J'ai des choses beaucoup, \emph{beaucoup}  plus importantes dont je dois me préoccuper." Harry eut l'air pensif. "Mais cela dit, ça ne \emph{prendrait}  pas beaucoup de temps d'écrire les 95 thèses de la Réforme Sans Vif et de les clouer à la porte d'une église -"

"Potter," traîna la voix d'un jeune garçon, "\emph{qu'est} -ce que tu as sur le visage \emph{qu'est-ce}  qui se tient à côté de toi ?"

L'air horrifié de Ron fut remplacé par de la haine absolue. "\emph{Toi !} "

Harry tourna la tête ; et c'était bien Draco Malfoy, qui avait peut-être été forcé de revêtir les robes règlementaires de l'école mais se rattrapait avec une malle à l'air au moins aussi magique et bien plus élégante que celle de Harry, décorée d'argent et d'émeraudes et portant ce que Harry devina être les armoiries de la famille Malfoy, un magnifique serpents à crocs surmontant des baguettes d'ivoire.

"Draco !" dit Harry. "Euh, ou Malfoy si tu préfères, même si je trouve que ça fait un peu penser à Lucius. Je suis content de voir que tu vas si bien après notre dernière, euh, notre dernière rencontre. Voici Ron Weasley. J'essaie de rester incognito, alors appelle moi, euh," Harry regarda ses robes, "Monsieur Black."

"\emph{Harry}  !" siffla Ron. "Tu ne peux pas utiliser \emph{ce}  nom !"

Harry cligna des yeux. "Pourquoi pas ?" Ça \emph{sonnait}  joliment sombre, comme un homme mystérieux international -

"Je dirais que c'est un \emph{excellent}  nom," dit Draco, "mais la Noble et Ancienne Maison des Black pourrait y trouver à redire. Que penses-tu de M. Argent ?"

"Éloigne-\emph{toi}  de... de M. Or," dit Ron froidement, et il s'avança d'un pas. "Il n'a pas besoin de parler aux gens comme toi !"

Harry leva une main apaisante. "Je me ferai appeler M. Bronze, merci pour le schéma d'appellation. Et Ron, euh," Harry lutta pour trouver une façon agréable de dire : "Je suis heureux que tu sois si...enthousiaste à l'idée de me protéger, mais ça ne me dérange pas particulièrement de discuter avec Draco -"

Ce fut apparemment un coup fatal pour Ron, qui se tourna vers Harry avec des yeux à présent enflammés par l'outrage. "\emph{Quoi}  ? \emph{Sais} -tu qui il est ?"

"Oui, Ron," dit Harry, "tu te souviens peut-être que je l'ai appelé Draco sans qu'il ait besoin de se présenter."

Draco ricana. Puis ses yeux s'éclairèrent lorsqu'il vit la chouette blanche posée sur l'épaule de Ron. "Oh, qu'est ce que c'est que \emph{ça}  ?" dit Draco avec un second ricanement plein de malveillance. "Où est le fameux rat de la famille Weasley ?"

"Enterré dans le jardin," dit Ron froidement.

"Oh, comme c'est triste. Pot... ah, M. Bronze, je devrais mentionner qu'il est couramment accepté que la famille Weasley jouit de \emph{la meilleure histoire d'animal de compagnie jamais entendue} . Voudrais-tu la raconter, Weasley ?"

Le visage de Ron se contorsionna. "Tu ne trouverais pas ça drôle si ça arrivait à \emph{ta}  famille !"

"Oh," ronronna Draco, "mais ça n'\emph{arriverait}  jamais aux Malfoys."

Les mains de Ron devinrent des poings -

"C'est assez," dit Harry, mettant autant d'autorité tranquille dans sa voix qu'il en était capable. Il était certain que, quel qu'en soit le contenu, c'était un souvenir douloureux pour le garçon roux. "Si Ron ne veut pas en parler, il n'y est pas obligé, et je te demanderai de ne pas en parler non plus."

Draco jeta un regard surprit à Harry, et Ron acquiesça. "C'est ça Harry ! Je veux dire M. Bronze ! Tu vois le genre de personne qu'il est ? Maintenant dis-lui de s'en aller !"

Harry compta mentalement jusqu'à dix, ce qui pour lui fut un rapide \emph{12345678910}  - une vieille habitude conservée depuis l'âge de cinq ans où sa mère lui avait pour la première fois donné l'instruction de le faire, et Harry s'était dit que sa façon à lui était plus rapide et tout aussi efficace. "Ron," dit Harry calmement, "Je ne vais pas lui dire de s'en aller. Il peut me parler si il le veut."

"Eh bien je n'ai pas l'intention de traîner avec quelqu'un qui traîne avec Draco Malfoy," annonça Ron froidement.

Harry haussa les épaules. "Ça te regarde. \emph{Je}  ne compte pas laisser qui que ce soit me dire avec qui je peux et ne peux pas passer du temps." Et il chantait silencieusement \emph{va-t-en s'il te plaît, va-t-en s'il te plaît} .

Sous le coup de la surprise, le visage de Ron se vida de toute expression, comme si il s'était vraiment attendu à ce que sa réplique fasse effet. Puis il fit demi-tour, tira la laisse de son bagage et partit précipitamment du quai.

"Si tu ne l'aimais pas," dit Draco avec curiosité, "pourquoi n'es-tu pas simplement parti ?"

"Euh... sa mère m'a aidé à comprendre comment aller sur ce quai depuis la gare de King's Cross, donc c'était un peu difficile de lui dire d'aller se faire voir. Et puis ce n'est pas que je le \emph{déteste} ," dit Harry, "c'est juste que je, que je..." Harry chercha ses mots.

"...ne vois aucune raison justifiant son existence ?" proposa Draco.

"À peu près."

"Quoi qu'il en soit, Potter... si tu a vraiment été éduqué par des Moldus -" Draco s'interrompit, comme s'il attendait une dénégation, mais Harry ne dit rien "- alors tu ne te rends peut-être pas compte de ce que c'est que d'être connu. Les gens vont vouloir te prendre \emph{tout}  ton temps. Tu \emph{dois}  apprendre à dire non."

Harry acquiesça, et prit un air pensif. "Ça a l'air d'être un très bon conseil."

"Si tu décides d'être gentil avec eux, ça veut juste dire que tu finiras par passer le plus clair de ton temps avec les plus insistants. Décide de ceux avec qui tu \emph{veux}  passer du temps et dis à tous les autres de s'en aller. Les gens \emph{vont}  te juger en fonction de ceux avec qui ils te voient traîner, et tu ne veux pas être vu avec des gens comme Ron Weasley."

Harry acquiesça à nouveau. "Si ça ne te dérange pas que je te le demande, comment m'as-tu reconnu ?"

"\emph{M. Bronze} ," lâcha Draco, "Je t'\emph{ai}  rencontré, souviens-toi. Je t'ai même très bien rencontré. J'ai vu quelqu'un se promenant avec une écharpe enroulée autour de sa tête et à l'air \emph{complètement ridicule} . Alors j'ai fait une \emph{folle supposition} ."

Harry inclina la tête, acceptant le compliment. "Je suis \emph{profondément}  désolé, à ce propos," dit Harry. "Je veux dire, notre première rencontre. Je ne voulais pas t'embarrasser devant Lucius."

Draco rejeta l'excuse d'un mouvement de la main tout en regardant Harry d'une étrange façon. "J'aurais juste aimé que Père soit arrivé pendant que \emph{tu}  me flattais \emph{moi}  -" rit Draco. "Mais \emph{merci}  d'avoir dit ce que tu as dit à Père. Sans ça, j'aurais eu beaucoup plus de mal à tout expliquer."

Harry fit une révérence encore plus poussée. "Merci à \emph{toi}  d'avoir fait la même chose en retour avec le Professeur McGonagall."

"De rien. Mais l'une des assistante doit avoir fait jurer le secret absolu à l'un de ses amis, car Père dit qu'il y a d'\emph{étranges rumeurs}  qui circulent, comme quoi toi et moi nous serions battus ou quelque chose comme ça."

"Aïe," dit Harry en grimaçant. "Je suis \emph{vraiment } désolé -"

"Non, on a l'habitude, Merlin sait qu'il y a déjà d'étranges rumeurs au sujet de la famille Malfoy."

Harry hocha la tête. "Heureux t'entendre que tu es hors du pétrin -"

Draco sourit. "Père a, hum, un sens de l'humour assez \emph{raffiné} , mais il \emph{comprend}  ce que c'est que d'avoir des amis. Il le comprend \emph{très}  bien. En fait, il m'a fait répéter ceci chaque soir avant d'aller au lit pendant tout le mois dernier : 'Je me ferai des amis à Poudlard.' Lorsque je lui ai tout expliqué et qu'il a vu que c'était ce que j'avais essayé de faire, il s'est non seulement excusé auprès de moi mais il m'a offert une glace."

La mâchoire de Harry se décrocha. "\emph{Tu as réussi à transformer ça en une glace } ?"

Draco acquiesça, et il avait l'air aussi fier de lui que cet exploit le méritait. "Eh bien, père \emph{savait}  ce que je faisais, bien sûr, mais c'est lui qui m'avait apprit \emph{comment}  le faire, et si je souris comme il faut \emph{pendant}  que je le fais, ça devient une blague père-fils et alors il \emph{doit}  m'acheter une glace ou sinon je lui donne un regard triste, comme si je pensais l'avoir déçu."

Harry observa Draco d'un air calculateur, sentant qu'il était en présence d'un autre maître. "Tu as reçu des \emph{leçons}  sur la façon de manipuler les gens ?"

"Depuis aussi longtemps que je me souvienne," dit Draco fièrement. "Père m'a payé des précepteurs."

"Wow," dit Harry. Avoir lu \emph{Influence et Manipulation}  de Robert Cialdini n'était probablement pas à la hauteur, comparé à ça (même si c'était quand même un sacré livre). "Ton père est presque aussi génial que mon père."

Les sourcils de Draco s'élevèrent noblement. "Oh ? Et qu'est ce que \emph{ton}  père fait ?"

"Il m'achète des livres."

Draco considéra l'affirmation. "Ça n'a pas l'air très impressionnant."

"Il faut y être pour comprendre. En tout cas, je suis content d'entendre ça. Vu la façon dont Lucius te regardait, j'ai cru qu'il allait te c-crucifier."

"Mon père m'aime vraiment," dit Draco avec fermeté. "Il ne ferait certainement pas une chose pareille."

"Euh..." dit Harry. Il se souvint de la figure parfaite en robe noire et aux cheveux blancs qui était entrée chez Madame Malkin, maniant sa magnifique cane à poignée d'argent. C'était juste tellement difficile de visualiser ce tueur parfait sous les traits d'un papa gâteau. "Ne le prends pas mal, mais comment \emph{sais} -tu ça ?"

"Hein ?" Il était clair que ce n'était pas une question que Draco se posait très souvent.

"Je pose la question fondamentale de la rationalité : Pourquoi crois-tu ce que tu crois ? Que penses-tu savoir et comment penses-tu que tu le sais ? Qu'as-tu \emph{vu}  qui te fasse penser que Lucius ne te sacrifierait pas comme il sacrifierait toute autre pièce de son jeu ?"

Draco jeta un nouveau regard à Harry. "Que sais-\emph{tu}  au juste de Père ?"

"Hm... siège au Magenmagot, siège au Conseil des Gouverneurs de Poudlard, incroyablement riche, a l'attention du ministre Fudge, a la confiance du ministre Fudge, a probablement des photos hautement embarrassantes du ministre Fudge, plus grand de tous les Puristes du Sang depuis que le Seigneur des Ténèbres est parti, ancien membre du cercle intérieur des Mangemorts, reconnu comme porteur de la Marque des Ténèbres mais s'en est sorti en disant qu'il était sous un sort d'Impérium, ce qui était ridiculement peu vraisemblable et à peu près tout le monde le savait... méchant avec un 'M' capital et tueur né... je crois que c'est tout."

Les yeux de Draco devinrent si étroits qu'on aurait dit des fentes. "Je vois que McGonagall t'a tout dit."

"Non, elle a refusé de me dire \emph{quoi que ce soit}  au sujet de Lucius, à part que je devais me tenir à l'écart de sa personne. Donc, durant l'Incident au Magasin de Potions, pendant que le Professeur McGonagall était occupée à discuter avec le propriétaire et à essayer de tout garder sous contrôle, j'ai attrapé l'un des clients et je l'ai interrogé \emph{lui}  au sujet de Lucius."

Les yeux de Draco s'agrandirent à nouveau. "Tu as \emph{vraiment}  fait ça ?"

Harry jeta un regard perplexe à Draco. "Si j'ai menti la première fois, je ne vais pas te dire la vérité juste parce que tu me poses encore la question."

Il y eut une pause, tandis que Draco absorbait cette information.

"Tu vas tellement aller à Serpentard."

"Je vais tellement aller à Serdaigle, merci bien. Je veux le pouvoir juste pour avoir les livres."

Draco gloussa. "Ouais, bien sûr. Enfin bref... pour répondre à ta question..." Draco prit une profonde inspiration, et son visage devint sérieux. "Père a manqué un vote du Magenmagot pour moi. J'étais sur un balai et je suis tombé et je me suis brisé de nombreuses côtes. Ça faisait vraiment mal. Je n'avais jamais eu aussi mal et je pensais que j'allais mourir. Alors Père a raté ce vote très important, parce qu'il était là à côté de mon lit à Ste Mangouste, me tenant la main et me promettant que tout irait bien."

Harry regarda ailleurs, mal à l'aise, puis, avec effort, se força à regarder à nouveau Draco. "Pourquoi me dis-tu \emph{ça}  ? Ça semble assez... intime..."

Draco regarda Harry avec grand sérieux. "L'un de mes précepteurs m'a dit un jour que les gens forment des amitiés fortes en connaissant des choses intimes l'un sur l'autre, et la raison pour laquelle la plupart des gens n'ont pas d'amis proches est qu'ils sont trop gênés pour partager quoi que ce soit de vraiment intime." Draco ouvrit ses paumes d'un air invitant. "A ton tour ?"

Harry observa que savoir que l'expression pleine d'espoir de Draco lui avait été instillée par des mois de pratique ne la rendait pas moins efficace. En fait, si, ça la \emph{rendait moins}  efficace, mais malheureusement pas \emph{sans effet} . On pouvait dire la même chose au sujet de la façon intelligente dont Draco poussait à la réciprocité en offrant un cadeau non sollicité, une technique que Harry avait découverte dans ses livres de psychologie sociale (une expérience avait montré qu'un cadeau inconditionnel de 5 \$ était deux fois plus efficace qu'une offre conditionnelle de 50 \$ à pousser les gens à répondre à des questionnaires). Draco avait fait cadeau d'une confidence non sollicitée à Harry, et il l'invitait maintenant à offrir une confidence en retour... et le truc, c'était que Harry \emph{se sentait}  poussé à le faire. Un refus, Harry en était certain, se heurterait à un air triste regard déçu, et peut être à une petite quantité de mépris indiquant que Harry avait perdu des points.

"Draco," dit Harry, "il faut que tu sache que je sais exactement ce que tu es en train de faire. Mes livres appellent ça \emph{réciprocité}  et ils parlent du fait qu'il a été prouvé que, pour obtenir de quelqu'un qu'il fasse quelque chose, il était deux fois plus efficace de simplement lui donner deux Mornilles que de lui en promettre vingt..." Harry laissa sa phrase en suspens.

Draco avait l'air triste et déçu. "Ce n'était pas censé être un piège, Harry. C'est une véritable technique pour devenir amis."

Harry leva une main. "Je n'ai pas dit que je n'allais pas te répondre. J'ai juste besoin de temps pour choisir quelque chose d'intime mais d'inoffensif. Disons juste... que je voulais que tu saches qu'on ne peut pas me brusquer à faire quelque chose." Un moment de réflexion permettait de désamorcer de nombreuses techniques de conformisation, une fois que vous saviez les reconnaître.

"Très bien," dit Draco. "J'attendrai pendant que tu trouves quelque chose. Oh, et s'il te plaît, enlève ton écharpe pendant que tu me le dis."

\emph{Simple mais efficace} .

Et Harry ne pouvait s'empêcher de remarquer à quel point, comparé à Draco, il avait été maladroit, embarrassé et sans grâce dans ses tentatives de résister à la manipulation / de sauver la face / de frimer. \emph{J'ai besoin de ces précepteurs} .

"Très bien," dit Harry après un moment. "Voilà mon histoire." Il jeta un coup d'oeil aux alentours puis enroula l'écharpe autrement autour de son visage, dévoilant tout sauf la cicatrice. "Euh... il semble que tu puisses vraiment compter sur ton père. Je veux dire... si tu lui parles sérieusement, il va toujours t'écouter et te prendre au sérieux."

Draco hocha la tête.

"Parfois," dit Harry, et il avala sa salive. C'était étonnamment difficile, mais c'était censé l'être. "Parfois je souhaite que Papa soit plus comme le tien." Les yeux de Harry fuirent ceux de Draco plus ou moins automatiquement, et Harry se força à le regarder à nouveau.

Puis Harry fut frappé par \emph{l'énormité de ce qu'il venait de dire} , et il ajouta hâtivement, "Non pas que je souhaite que mon Papa soit un instrument de mort parfait comme Lucius, je voulais juste dire que je voudrais qu'il me prenne au sérieux -"

"Je comprends," dit Draco avec un sourire. "Et voilà... maintenant il semble qu'on s'est un peu rapproché du statut d'amis, non ?"

Harry hocha la tête. "Ouais. En effet. Euh... sans vouloir t'offenser je pense que je vais remettre mon déguisement, je ne veux \emph{vraiment}  pas avoir à gérer -"

"Je comprends."

Harry enroula à nouveau l'écharpe tout autour de son visage.

"Mon père prend tous ses alliés au sérieux," dit Draco. "C'est pourquoi il a beaucoup d'alliés. Tu devrais peut-être le rencontrer."

"J'y penserai," dit Harry d'une voix neutre. Il secoua la tête avec incrédulité. "Alors comme ça tu es vraiment son seul point faible. Heh."

Maintenant Draco jetait un regard \emph{vraiment}  bizarre à Harry. "Tu veux aller boire quelque chose, ou trouver un endroit où nous asseoir ?"

Harry se rendit compte qu'il était resté debout au même endroit trop longtemps et s'étira, essayant de faire craquer son dos. "Certainement."

Le quai commençait maintenant à se remplir, mais il restait une zone plus tranquille du côté le plus éloigné de la locomotive à vapeur rouge. Sur le chemin ils croisèrent un vendeur, un homme chauve mais barbu, avec un petit chariot portant des journaux et des bandes dessinées ainsi que des canettes vert néon empilées.

Le vendeur, à vrai dire, était penché en arrière, et buvait depuis l'une des canettes vert néon au moment même où il repéra l'élégant et raffiné Draco Malfoy s'approcher aux côtés d'un garçon mystérieux à l'air incroyablement stupide avec une écharpe attachée autour de la tête, ce qui poussa le vendeur à subir une quinte de toux soudaine au milieu d'une gorgée et à faire dégouliner une grande quantité de liquide vert néon sur sa barbe.

"Excusez moi," dit Harry, "mais qu'\emph{est}  ce que c'est que ça exactement ?"

"De l'Hilari-Thé," dit le vendeur, "Si vous en buvez, quelque chose de surprenant aura lieu et ça vous fera renverser du thé sur vous ou quelqu'un d'autre. Mais il est enchanté pour se dissiper quelques secondes plus tard -" La tache sur sa barbe disparaissait en effet déjà.

"Que c'est drôle," dit Draco. "Que c'est bien drôle. Venez, M. Bronze, allons trouver un autre -"

"Attends," dit Harry.

"\emph{Oh allez !}  C'est juste, juste \emph{puéril}  !"

"Non Draco, je suis navré, je \emph{dois}  étudier ça. Qu'est ce qui se passe si je bois de l'Hilari-Thé tout en faisant de mon mieux pour garder la conversation complètement sérieuse ?"

Le vendeur sourit et haussa les épaules avec un air mystérieux. "Qui sait ? Vous verrez un ami passer par là dans un costume de grenouille ? \emph{Quelque chose}  d'amusant et d'inattendu aura lieu, d'une façon ou d'une autre -"

"Non. Je suis navré. Je n'y crois pas. Ça viole ma suspension de l'incrédulité (déjà abusée) de tant de façons que je n'ai même pas les mots pour le décrire. Il est, il est juste \emph{hors de question}  qu'une satanée \emph{boisson}  puisse manipuler la réalité pour produire des \emph{situations comiques} , ou je vais abandonner et prendre ma retraite aux Bahamas -"

Draco grogna. "Allons-nous \emph{vraiment}  faire ça ?"

"Cinq Noises la canette," dit le vendeur.

"\emph{Cinq Noises ?}  Vous pouvez vendre des soda manipulateurs de réalité à \emph{cinq Noises la canette ?} " Harry mit la main dans sa bourse, dit "quatre Mornilles, quatre Noises," et les abattit sur le comptoir. "Deux douzaines de canettes s'il vous plaît."

"J'en prendrai une aussi," soupira Draco, et il commença à tendre la main vers ses poches.

Harry secoua rapidement la tête. "Non, c'est pour moi, et ça ne compte pas comme une faveur non plus, je veux voir si ça marche sur toi aussi." Il jeta une canette à Draco et commença à nourrir sa bourse, dont l'Ouverture Élargissante mangea les canettes, accompagnant l'opération de petits bruits de rot, ce qui ne restaura pas vraiment la foi de Harry dans le fait qu'un jour il trouverait une explication raisonnable à tout ça.

Vingt-deux rots plus tard, Harry avait la dernière canette achetée dans sa main. Draco le regardait, dans l'expectative, et ils décapsulèrent leurs canettes au même instant.

Harry souleva son écharpe pour exposer sa bouche, et ils penchèrent leurs têtes en arrière et burent l'Hilari-Thé. Étrangement, ça avait un \emph{goût}  vert néon - extra-pétillant, et plus citronné que du citron.

Rien ne se passa.

Harry regarda le vendeur, qui les regardait avec bienveillance.

\emph{Très bien, si ce type vient de profiter d'un accident pour me vendre vingt-quatre canettes de soda vert, je vais applaudir son esprit d'entrepreneuriat et ensuite je le tuerai.} 

"Ça n'arrive pas toujours immédiatement," dit le vendeur. "Mais ça aura lieu une fois par canette, garanti ou remboursé."

Harry prit une autre longue gorgée.

Une fois de plus, rien ne se passa.

\emph{Peut être que je devrais juste boire la chose aussi vite que possible...et espérer que mon estomac n'explose pas avec tout ce dioxyde de carbone, ou que je ne rote pas pendant que je le bois...} 

Non, il pouvait se permettre d'être \emph{un peu}  patient. Mais honnêtement, Harry ne voyait pas comment ça allait fonctionner. Vous ne pouviez pas vous approcher de quelqu'un et lui dire "Maintenant je vais vous surprendre" ou "Et maintenant je vais vous dire la fin d'une blague, et ça va être vraiment drôle." Ça détruisait l'impact du choc. Dans l'état de préparation mentale de Harry, Lucius Malfoy aurait pu passer devant eux habillé en ballerine que ça ne l'aurait pas fait s'étouffer. Quelle sorte de manigance tarée l'univers allait-il cracher \emph{cette fois}  ?

"Bon, asseyons-nous," dit Harry. Il se prépara à prendre une nouvelle gorgée et regarda en direction des bancs, plus loin, ce qui le mit exactement au bon angle pour rabattre son regard et voir la portion de l'étal du vendeur consacrée à un journal nommé Le Chicaneur, qui portait le gros titre suivant :


\begin{center}\emph{DRACO MALFOY TOMBE} \\\emph{} \emph{ENCEINTE DU SURVIVANT} \end{center}


"\emph{Gah}  !" cria Draco alors que du liquide vert fluo était pulvérisé depuis Harry jusqu'à sa personne. Draco se tourna vers Harry, les yeux en feu, et il serra sa canette. "Fils de sang-de-bourbe ! Voyons comment \emph{tu}  aimes qu'on te crache dessus !" Draco prit délibérément une lampée juste alors que ses yeux tombaient sur le gros titre.

Par réflexe, Harry essaya de protéger son visage alors que le spray de liquide volait dans sa direction. Malheureusement, il bloqua en utilisant la main qui contenait l'Hilari-Thé, envoyant le reste du liquide vert éclabousser son épaule.

Harry regarda la canette dans sa main tout en s'étouffant et en postillonnant, et la couleur verte commença à disparaître des robes de Draco.

Puis il leva à nouveau les yeux et fixa le gros titre du journal.


\begin{center}\emph{DRACO MALFOY TOMBE} \\\emph{} \emph{ENCEINTE DU SURVIVANT} \end{center}


Les lèvres de Harry s'ouvrirent et dirent : "buh-bluh-buh-buh"

Trop d'objections à la fois, c'était ça le problème. Chaque fois que Harry essayait de dire "Mais nous n'avons que onze ans !" l'objection "Mais les hommes ne peuvent pas tomber enceinte !" exigeait la priorité et se faisait ensuite rouler dessus par "Mais il n'y a rien entre nous, vraiment !"

Puis Harry regarda à nouveau sa canette.

Il ressentait un profond désir de courir en criant à s'en vider les poumons jusqu'à ce qu'il tombe enfin à cause du manque d'oxygène, et la seule chose qui l'empêchait de faire ça était qu'il avait lu un jour que la panique complète était la marque d'un problème scientifique \emph{vraiment}  important.

Harry grogna, jeta violemment la canette dans une poubelle proche, et revint rôder près du vendeur. "Un exemplaire du \emph{Chicaneur}  s'il vous plaît." Il paya quatre Noises de plus, récupéra une nouvelle canette d'Hilari-Thé de sa bourse, puis rôda vers la zone de pique-nique où Draco fixait sa propre canette de soda avec une expression de franche admiration.

"Je retire ce que j'ai dit," dit Draco, "c'était plutôt sympa."

"Hey, Draco, tu sais ce qui est encore mieux pour devenir amis qu'échanger des secrets ? Commettre un meurtre."

"J'ai un précepteur qui dit ça," accorda Draco. Il passa sa main sous ses robes et se gratta d'un mouvement simple et naturel. "Qui as-tu en tête ?"

Harry abattit \emph{Le Chicaneur}  sur la table de pique-nique. "Celui qui a inventé cette manchette."

Draco grogna. "Pas un homme. Une fille. Une fille \emph{de dix ans} , si tu peux y croire. Elle est devenue folle après la mort de sa mère, et son père, qui possède le journal, est \emph{convaincu}  que c'est une voyante, donc quand il ne sait pas, il demande à Luna Lovegood et croit \emph{tout}  ce qu'elle dit."

Sans vraiment y penser, Harry décapsula une nouvelle canette d'Hilari-Thé et se prépara à boire. "Tu veux rire ? C'est encore pire que le journalisme Moldu, ce qui semble physiquement impossible."

Draco grogna à nouveau. "Elle a une espèce d'obsession perverse au sujet des Malfoys en plus, et son père nous est politiquement opposé, donc il imprime tout ce qu'elle dit à notre sujet. Dès que je serai assez vieux je vais la violer."

Le liquide vert gicla hors des narines de Harry, imprégnant l'écharpe qui couvrait toujours cette zone. L'Hilari-Thé et les poumons n'étaient pas faits pour interagir, et Harry passa les quelques secondes suivantes à tousser frénétiquement.

Draco regarda durement Harry. "Quelque chose ne va pas ?"

C'est à ce moment que Harry réalisa soudainement que (a) les sons venant du reste du quai s'étaient transformés en un bruit continu et flou à peu près au moment où Draco s'était gratté sous ses robes, et que (b) lorsqu'il avait précédemment parlé du meurtre comme méthode permettant de créer des liens, il y avait eu exactement une personne dans la conversation qui avait cru qu'ils blaguaient tous les deux.

\emph{C'est ça. Parce qu'il avait l'air d'un enfant si normal. Et il } est\emph{ un enfant normal, il est exactement ce que vous pourriez attendre d'un enfant mâle de base } \emph{si il avait été élevé par le serviteur le plus effrayant du Seigneur des Ténèbres et/ou son papa gâteau.} 

"Oh, c'est juste," toussa Harry, oh dieu comment allait il sortir la conversation de ce cul-de-sac, "que j'étais surpris par la façon dont tu étais prêt à en discuter si ouvertement, tu n'avais pas l'air d'avoir peur d'être pris."

Draco renifla. "Tu plaisantes ? La parole de \emph{Luna Lovegood}  contre la mienne ?"

Bon sang de bonsoir. "J'imagine qu'il n'y a pas de détecteur de mensonge magique ?" \emph{Ou de test ADN... pas encore.} 

Draco regarda aux alentours. Ses yeux se rétrécirent. "C'est vrai, tu ne sais rien. Écoute, je vais t'expliquer certaines choses, je veux dire la façon dont les choses fonctionnent vraiment, comme si tu étais à Serpentard et que tu me posais cette même question. Mais tu dois me promettre de ne rien en dire à personne."

"Je peux parler de ce sujet, mais pas dire que c'est \emph{toi}  qui m'en a parlé, c'est ça ? C'est à dire que si un autre Serpentard me pose la même question un jour où l'autre..."

Draco marqua une pause. "Répète ça."

Harry obtempéra.

"D'accord, ça n'a pas l'air d'être une ruse, alors j'accepte. Garde seulement à l'esprit que je peux tout nier. Jure."

"Je le jure," dit Harry.

"La cour utilise le Veritaserum, mais c'est vraiment une blague, tu peux juste t'Oublietter avant de témoigner puis dire que l'autre personne a reçu un sortilège de Mémoire et a un faux souvenir. Si tu as une Pensine, et on en a une, tu peux même récupérer le souvenir plus tard. Généralement la cour favorise la théorie de l'Oubliette plutôt que celle d'un sort de Mémoire plus complexe. Mais la cour a un grand pouvoir discrétionnaire. Et si \emph{je } suis lié à une histoire qui affecte l'honneur d'une Maison Noble, alors ça remonte jusqu'au Magenmagot, où Père contrôle les votes. Après que j'ai été reconnu non coupable la famille Lovegood devrait payer des réparations pour avoir terni mon honneur. Et comme ils sauraient depuis le départ que ça se passerait comme ça, ils garderaient leur bouche cousue."

Un frisson glacé montait en Harry, un frisson accompagné d'instructions qui lui disaient de garder un visage et une voix normaux. \emph{Note à moi-même : Renverser le gouvernement de l'Angleterre magique à la première occasion.} 

Harry toussa à nouveau pour s'éclaircir la gorge. "Draco, s'il te plaît s'il te plaît \emph{s'il te plaît}  ne le prends pas mal, je n'ai qu'une parole, mais comme tu l'as dit je pourrais être à Serpentard et je veux vraiment te poser cette question par pure curiosité, que se passerait-il \emph{théoriquement parlant}  si \emph{je}  témoignais t'avoir entendu planifier ça ?"

"Alors si j'étais n'importe quelle personne sauf un Malfoy, je serais dans le pétrin," répondit Draco avec suffisance. "Mais puisque je \emph{suis}  un Malfoy... Père a les votes. Et après ça il t'écraserait... eh bien, pas facilement je suppose, puisque tu \emph{es}  le Survivant, mais Père est plutôt doué pour ces choses là." Draco fronça les sourcils. "Au fait, \emph{tu}  étais prêt à discuter de son meurtre, alors pourquoi n'étais-tu pas inquiet de \emph{me}  voir témoigner le jour où on la retrouverait morte ? Je ne suis pas aussi connu que toi mais si tu fais quelque chose de mal, tes, ahem, supporters seront beaucoup moins susceptibles de rester de ton côté. Et un meurtre, avec un corps et tout, c'est beaucoup plus sérieux qu'un viol."

Lorsque la conversation ne peut avancer ni reculer, fais-la partir de coté. "C'est un truc de Moldu, dans l'Angleterre Moldue il y a une sacrée différence politique entre esquiver une condamnation pour meurtre et esquiver une condamnation pour le viol d'une petite fille."

"Vraiment ? Bizarre. Pourquoi le meurtre n'est-il pas pire ? Donc ça veut dire que si c'est toi qui la viole, ça rend la chose exceptionnelle de ton point de vue ? Parce que je te laisserai la première place avec plaisir si c'est le cas. Mec, imagine Lovegood L'allumée essayant de prétendre qu'elle a été violée par Draco Malfoy \emph{et}  par le Survivant, même \emph{Dumbledore}  ne la croirait pas."

Heureusement que Harry ne buvait \emph{pas } d'Hilari-Thé à ce moment précis. \emph{Comment, oh comment ma journée a-t-elle pu déraper à ce point ?}  L'esprit de Harry faisait des calculs désespérés et trouva un autre moyen de décaler la conversation.

"A vrai dire, je préférerai que tu te tiennes à l'écart pour un moment. Après avoir découvert que cette manchette provenait d'une fille d'un an ma cadette, je ne pensais plus vraiment au meurtre \emph{ni}  au viol."

"Uh ? Dis moi donc," dit Draco, et il commença à boire une autre lampée de son Hilari-Thé.

Harry ne savait pas si l'enchantement fonctionnait plus d'une fois par canette, mais il \emph{savait}  qu'il pouvait éviter la responsabilité de ce qui allait se produire si il choisissait parfaitement son moment :

"Je pensais : \emph{un jour, je vais épouser cette femme.} "

Draco fit un horrible bruit d'éclaboussure et laissa couler du fluide vert par les coins de sa bouche comme un radiateur de voiture cassé. "\emph{Tu es dingue ?} "

"Bien au contraire, je suis tellement sain d'esprit que ça brûle comme de la glace."

Draco gloussa d'un bruit aigu et juvénile. "Tu as des goûts encore plus bizarres que ceux d'un Lestrange. Mais tu pourrais quand même la violer. Elle est probablement assez folle pour aimer ça et j'ai entendu dire que beaucoup de mariages commençaient comme ça. Et sinon tu pourrais toujours lui jeter Oubliettes et recommencer la semaine suivante."

\emph{Je vais désassembler tes pathétiques restes magiques du Moyen-Âge en des pièces plus petites que les atomes qui les constituent. } "Ça t'embêterait de \emph{me}  laisser me soucier de ça ? Si tu considérais sérieusement l'idée de la violer je pourrais toujours te devoir une faveur -"

Draco agita la main. "Non, c'est cadeau."

Harry regarda la canette dans sa main, la froideur s'installant dans son sang. Charmant, heureux, généreux dans ses faveurs à ses amis, Draco n'était pas un psychopathe. C'était la partie triste et terrible : connaître assez de psychologie humaine pour \emph{savoir}  que Draco n'était \emph{pas}  un monstre. Il y avait eu dix mille sociétés durant le cours de l'Histoire du monde où cette conversation aurait pu avoir lieu. Non, le monde aurait été certainement très différent si il y avait eu besoin d'un \emph{mutant maléfique}  pour dire ce que Draco avait dit. C'était très simple, très humain, c'était ce qui se passait par défaut en l'absence d'intervention extérieure. Pour Draco, ses ennemis n'étaient pas des gens.

Et dans le temps ralenti de ce pays ralenti, ici et là, comme dans les ténèbres-avant-l'aurore qui avaient précédées l'Âge de Raison, le fils d'un noble suffisamment puissant pouvait tenir pour acquis qu'il était au-dessus de la loi. Du moins quand il s'agissait d'un petit viol par ci par là.

Il y avait des endroits en terre Moldue où les choses fonctionnaient encore de cette façon, des pays où ce genre de noblesse existait encore et pensait encore cela, et d'autres terres encore plus sinistres où ce n'était pas réservé à la noblesse. C'était ainsi dans tous les temps et tous les lieux qui ne descendaient pas directement des Lumières. Une descendance qui, semblait-il, n'incluait pas l'Angleterre magique, puisqu'ici les seules contaminations interculturelles avaient été des choses comme les canettes de soda.

\emph{Et si Draco ne change pas d'avis sur son envie de vengeance et que je ne gâche pas ma chance d'être heureux dans la vie en épousant une pauvre fille folle, alors tout ce que j'ai fait c'est de gagner du temps, et pas beaucoup...} 

Pour une fille. Pas pour toutes.

\emph{Je me demande à quel point ce serait difficile de juste faire une liste des plus grands Puristes du Sang et de les tuer.} 

C'était exactement ce qu'ils avaient essayé lors de la Révolution Française, plus ou moins - faire une liste de tous les ennemis du Progrès et enlever tout ce qui était au-dessus du cou - et ça n'avait pas très bien marché, du peu que s'en souvenait Harry. Peut-être qu'il avait besoin de dépoussiérer quelques uns des livres d'Histoire que son père lui avait acheté et de voir si ce qui avait mal tourné lors de la Révolution Française était facile à corriger.

Harry fixa le ciel, ainsi que la pâle forme de la Lune, visible ce matin à travers l'air sans nuages.

\emph{Le monde est cassé et imparfait et fou et cruel et sanglant et noir. C'est une nouvelle ? Tu l'avais de toute façon toujours su...} 

"Tu as l'air bien sérieux," dit Draco. "Laisse moi deviner, tes parents Moldus t'ont dit que ce genre de choses est mal."

Harry hocha la tête, ne faisant pas vraiment confiance à sa voix.

"Eh bien, comme dit Père, il y a peut-être quatre maisons, mais à la fin tout le monde appartient soit à Serpentard soit à Poufsouffle. Et franchement, tu n'es pas du genre Poufsouffle. Si tu décides de t'allier secrètement aux Malfoys... notre pouvoir et notre réputation... te permettraient des choses que même \emph{moi}  je ne peux pas faire. Tu veux \emph{essayer}  pendant quelques temps ? Voir comment c'est ?

\emph{Ne voilà-t-il pas un intelligent petit serpent. Onze ans et déjà à amadouer ta proie hors de sa cachette. Est-il trop tard pour te sauver, Draco ?} 

Harry réfléchit, étudia, et choisit son arme. "Draco, pourrais-tu m'expliquer toute cette histoire de pureté du sang ? C'est assez nouveau pour moi."

Un grand sourire s'étira sur le visage de Draco. "Tu devrais vraiment rencontrer Père et \emph{lui}  demander, c'est notre chef."

"Donne moi juste le discours de l'ascenseur. La version qui tient en trente secondes, je veux dire."

"D'accord," dit Draco. Il prit une profonde inspiration, et sa voix devint légèrement plus grave, et prit une cadence. "Nos pouvoirs ont faibli génération après génération alors que la souillure Sang-de-Bourbe grandit. Là où Salazar et Godric et Rowena et Helga ont un jour érigé Poudlard grâce à leurs pouvoirs, créant le Médaillon et l'Épée et le Diadème et la Coupe et le Choixpeau, aucun sorcier moderne n'a jamais tenté de faire mieux. Nous disparaissons, nous nous transformons en Moldus en nous croisant avec leur engeance et en laissant nos Cracmols vivre. Si la souillure n'est pas arrêtée, bientôt nos baguettes se briseront et notre art cessera, la lignée de Merlin s'achèvera et le sang d'Atlantis échouera. Nos enfants devront gratter la terre pour survivre comme de simples Moldus et la Ténèbre recouvrira le monde entier pour toujours." Draco prit une autre lampée de sa canette, l'air satisfait. Ça semblait être l'argument final en ce qui le concernait.

"Persuasif," dit Harry, utilisant le mot de façon descriptive plutôt que normative. Classique, classique modèle. La Chute après la Grâce, le besoin de protéger ce qui restait de la pureté contre la contamination, le passé en courbe ascendante et le futur en courbe uniquement descendante. Et le modèle avait aussi un \emph{contre} ... "Je dois cependant te corriger sur un fait. Ton information au sujet des Moldus est un peu obsolète. Nous ne grattons plus vraiment la terre."

La tête de Draco fit un mouvement sec vers Harry. "\emph{Quoi}  ? Qu'est ce que tu veux dire, \emph{nous}  ?"

"Nous. Les scientifiques. La lignée de Francis Bacon et le sang des Lumières. Les Moldus ne sont pas restés assis à pleurer parce qu'ils n'avaient pas de baguettes, nous avons nos \emph{propres}  pouvoirs maintenant, avec ou sans magie. Si tous vos pouvoirs échouent alors nous aurons perdu quelque chose de très précieux, car votre magie est la seule chose qui nous donne un indice sur la façon dont l'univers doit \emph{vraiment}  fonctionner - mais vous ne vous retrouverez pas à gratter la terre. Vos maisons seront toujours fraîches en été et chaudes en hiver, il y aura toujours des docteurs et de la médecine. La science peut vous maintenir en vie si la magie échoue. Ce serait une tragédie et nous devrions tous vouloir l'empêcher, mais ce ne serait pas littéralement la fin de toute la lumière du monde. Je dis ça comme ça."

Draco avait reculé d'un bon mètre et son visage était un mélange entre la peur et l'incrédulité. "\emph{Par Merlin, mais de quoi parles-tu, Potter ?} "

"Eh, j'ai écouté \emph{ton}  histoire, tu ne veux pas écouter la mienne ?" \emph{Maladroit} , se semonça Harry, mais Draco arrêta bien de reculer et sembla écouter.

"Bien," dit Harry, "Je dis que tu n'as pas l'air d'avoir prêté attention à ce qui se passe dans le monde Moldu." Probablement parce que tout le monde magique semblait considérer le reste de la Terre comme étant un bidonville méritant autant d'attention que le \emph{Financial Times}  n'en accordait à la misère quotidienne du Burundi. "Bon. Rapide vérification. Les sorciers sont-ils jamais allés sur la Lune ? Tu sais, ce truc ?" Harry pointa du doigt en direction de l'énorme globe lointain.

"\emph{Quoi}  ?" dit Draco. Il était assez clair que cette pensée ne s'était jamais présentée au garçon. "\emph{Aller}  sur la - c'est juste un -" Son doigt pointa en direction de la petite chose pâle dans le ciel. "On ne peut pas Transplaner à un endroit où on est jamais \emph{allé}  et comment qui que ce soit irait sur la Lune la \emph{première}  fois ?"

"Attends," dit Harry à Draco, "Je voudrais te montrer un livre que j'ai apporté avec moi, je crois me souvenir dans quelle boîte il se trouve." Harry se leva et s'agenouilla et sortit les escaliers qui menaient au niveau caverne de sa malle, puis descendit les escaliers à la cavalcade, souleva une boîte qui était posée sur une autre boîte, s'approcha périlleusement du moment où il traiterait ses livres avec irrespect, arracha le couvercle de la boîte et, avec rapidité mais précaution, extirpa une pile de livres -

(Harry avait hérité de la capacité quasi-magique des Verres à se souvenir d'où tous ses livres se trouvaient, même après ne les avoir vus qu'une fois, ce qui était assez mystérieux étant donné l'absence de lien génétique.)

Et Harry courut en haut des escaliers et fourra la cage d'escalier dans la malle d'un coup de talon, puis, haletant, tourna les pages de son livre jusqu'à ce qu'il ait trouvé l'image qu'il voulait montrer à Draco.

Celle avec le terrain blanc, sec et couvert de cratères, et les gens en combinaison, et le globe blanc-bleu suspendu au-dessus.

Cette image.

\emph{L'} image, si une seule image devait jamais survivre.

"\emph{Ça} ", dit Harry, sa voix tremblante parce qu'il ne pouvait contenir sa fierté, "c'est à ça que la Terre ressemble depuis la Lune."

Draco se pencha lentement. Il y avait une étrange expression sur son visage. "Si c'est une \emph{vraie}  image, pourquoi ne bouge-t-elle pas ?"

\emph{Bouger}  ? Oh. "Les Moldus peuvent faire des images qui bougent mais ils ont besoin d'une plus grande boite pour les montrer, ils ne peuvent pas encore les faire tenir sur des pages de livres."

Le doigt de Draco se posa sur une des combinaisons. "Qu'est ce que c'est ?" Sa voix commençait à vaciller.

"Ce sont des humains. Ils portent des combinaisons qui recouvrent tout leur corps afin de leur donner de l'air, car il n'y a pas d'air sur la Lune."

"C'est impossible," murmura Draco. Il y avait de la terreur dans ses yeux, ainsi qu'une confusion absolue. "Aucun Moldu ne pourrait jamais faire ça. \emph{Comment...} "

Harry reprit le livre, tourna les pages jusqu'à ce qu'il ait trouvé ce qu'il cherchait. "C'est une fusée qui s'élève. Le feu la pousse toujours plus haut, jusqu'à ce qu'elle arrive à la Lune." Tourna d'autres pages. "C'est une fusée au sol. Cette petite poussière à coté est une personne." Draco s'étouffa. "Aller sur la lune coûte l'équivalent de... probablement autour de deux mille millions de Gallions." Draco s'étrangla. "Et ça a demandé les efforts de... probablement plus de personnes que le nombre total d'habitants de l'Angleterre magique." \emph{Et lorsqu'ils arrivèrent, ils laissèrent une plaque disant : 'Nous venons en paix, pour toute l'humanité.' Tu n'es pas encore prêt à entendre ces mots, Draco, mais j'espère que tu le seras un jour...} 

"Tu dis la vérité," dit lentement Draco. "Tu ne fabriquerais pas un livre entier juste pour me raconter ça - et puis je peux l'entendre dans ta voix. Mais... mais..."

"Comment, sans baguettes ni magie ? C'est une longue histoire, Draco. La science ne fonctionne pas en agitant des baguettes et en chantonnant des sortilèges, elle fonctionne en sachant comment l'univers fonctionne à un niveau si profond que l'on sait exactement comment faire faire à l'univers ce qu'on veut qu'il fasse. Si la magie consiste à jeter un Impero sur quelqu'un pour lui faire faire ce que l'on veut, alors la science consiste à les connaître si bien que l'on sait exactement quoi leur dire pour leur faire croire que c'était leur idée depuis le début. C'est beaucoup plus difficile que d'agiter une baguette, mais ça marche là où les baguettes échouent, exactement comme si, si l'Impero échouait, on pourrait toujours essayer de persuader la personne. Et la Science se construit de génération en génération. Il faut vraiment \emph{savoir}  ce qu'on fait quand on fait de la science - et quand on comprend vraiment quelque chose, on peut l'expliquer à quelqu'un d'autre. Les plus grands scientifiques du siècle dernier, les plus grands noms qu'on prononce avec révérence aujourd'hui encore... leurs pouvoirs ne sont \emph{rien}  comparé aux plus grands scientifiques d'aujourd'hui. Il n'y a pas d'équivalent scientifique aux arts perdus qui ont érigé Poudlard. Les pouvoirs de la science ne font que croître d'année en année. Et nous commençons à comprendre et à démêler les secrets de la vie et de l'hérédité. Nous serons capables d'observer le sang dont tu as parlé et de voir ce qui fait de toi un sorcier, et dans une ou deux génération, nous pourrons aussi persuader ce sang de faire de vos enfants de puissants sorciers. Donc tu vois, ton problème n'est pas aussi grave qu'il en a l'air, parce que dans quelques décennies la science pourra le résoudre pour toi."

"Mais..." dit Draco. Sa voix tremblait. "Si les \emph{Moldus}  ont ce genre de pouvoir... alors... que sommes \emph{nous}  ?"

"Non Draco, tu n'y es pas. Ne vois-tu pas ? La science exploite le pouvoir de la compréhension humaine pour observer le monde et comprendre comment il fonctionne. Elle ne peut échouer sans que l'humanité n'échoue. Ta haïrais voir ta magie s'éteindre, mais tu serais toujours \emph{toi} . Tu serais toujours là pour le regretter. Puisque la science se repose sur mon intelligence humaine, c'est le pouvoir qui ne peut m'être enlevé sans m'enlever \emph{moi} . Même si les lois de l'univers changent, que tout mon savoir devient nul et non avenu, je n'aurai qu'à découvrir les nouvelles lois, comme ça a été fait auparavant. Ce n'est pas propre aux \emph{Moldus} , c'est propre aux \emph{humains} , ça ne fait qu'affiner le pouvoir que tu utilises à chaque fois que tu regardes quelque chose que tu ne comprends pas et que tu demandes 'Pourquoi ?'. Tu es un Serpentard, Draco, ne vois-tu pas l'implication ?"

Draco leva les yeux du livre et regarda Harry. Son visage exhibait une compréhension naissante. "Les sorciers peuvent apprendre à utiliser ce pouvoir."

Et maintenant, précautionneusement... l'appât est placé, maintenant l'hameçon... "Si tu peux te voir comme un \emph{humain}  plutôt que comme un \emph{sorcier}  alors tu peux entraîner et affiner tes pouvoirs d'humain."

Et si \emph{cette}  consigne n'était pas dans \emph{tous}  les curriculums scientifiques, Draco n'avait pas besoin de le savoir, n'est-ce pas ?

Les yeux de Draco semblaient profondément pensifs. "As-tu...déjà fait ça ?"

"Jusqu'à un certain point," accorda Harry. "Ma formation n'est pas pas complète. Pas à onze ans. Mais - mon père m'a \emph{aussi}  offert des précepteurs, vois tu." D'accord, c'étaient des étudiants en doctorat affamés, et c'était parce que Harry avait un cycle de sommeil de 26 heures - qu'est ce que le Professeur McGonagall allait \emph{faire}  à ce sujet ? - mais laissons cela de côté pour le moment...

Lentement, Draco hocha la tête. "Penses-tu que tu puisses maîtriser les \emph{deux}  arts, ajouter leurs pouvoirs, et..." Draco fixa Harry. "Devenir le Seigneur des deux mondes ?"

Harry eut un rire maléfique, à ce point de la conversation ça semblait venir naturellement. "Draco, il faut que tu te rendes compte que l'intégralité du monde que tu connais, toute l'Angleterre magique, n'est qu'une case d'un plateau de jeu bien plus grand. Le plateau de jeu inclut des endroits comme la Lune, et les étoiles dans le ciel nocturne, qui sont des lumières comme le Soleil seulement inimaginablement plus lointaines, et d'autres choses comme des galaxies, qui sont considérablement plus immenses que la Terre et le Soleil, des choses si grandes que seuls les scientifiques peuvent les voir et que tu ne sais même pas qu'elles existent. Mais je \emph{suis}  vraiment un Serdaigle, tu sais, pas un Serpentard. Je ne veux pas diriger l'univers. Je pense juste qu'il pourrait être mieux organisé."

Le visage de Draco témoignait d'une crainte révérentielle. "Pourquoi \emph{me}  dis tu ça ?"

"Oh... il n'y a pas beaucoup de gens qui savent faire de la \emph{vraie}  science - comprendre quelque chose pour la première fois, même si ça les rend incroyablement confus. Ça m'aiderait d'avoir de l'aide."

Draco fixa Harry la bouche ouverte.

"Mais ne t'y trompes pas Draco, la vraie science \emph{n'est pas}  comme de la magie, tu ne peux pas la pratiquer et t'en aller inchangé, comme quand tu apprends les mots d'un nouveau sort. Le pouvoir a un prix, un prix si élevé que la plupart des gens refusent de le payer."

Draco hocha la tête, comme si finalement, il entendait quelque chose qu'il pouvait comprendre. "A quel prix ?"

"Apprendre à admettre qu'on a tort."

"Euh," dit Draco, après que la pause dramatique se soit étiré durant quelques instants. "Tu vas expliquer ça ?"

"En essayant de comprendre comment quelque chose fonctionne à un niveau aussi profond, les quatre-vingt dix neuf premières explications auxquelles tu vas arriver seront fausses. La centième sera juste. Tu dois donc apprendre à admettre que tu as tort, encore et encore et encore. Ça n'a pas l'air d'être grand chose, mais c'est si difficile que la plupart des gens ne peuvent pas pratiquer la science correctement. Toujours se remettre en question, toujours jeter un nouveau regard aux choses qu'on avait toujours considérées comme acquises," comme d'avoir un Vif dans le Quidditch, "et chaque fois que tu changes d'avis, tu changes qui tu es. Mais je mets la charrue avant les bœufs. Bien avant les bœufs. Je veux juste que tu saches... que je t'offre de partager un peu de mon savoir. Si tu le veux. Il y a juste une condition."

"Oh oh," dit Draco. "Tu sais, Père dit que quand quelqu'un te dit ça, ce n'est jamais, jamais un bon signe."

Harry acquiesça. "Maintenant ne te méprends pas en t'imaginant que j'essaie d'ériger une barrière entre toi et ton père. Ce n'est pas ce dont il s'agit. C'est juste que je veux pouvoir avoir affaire à quelqu'un de mon âge, plutôt que ce soit une histoire entre moi et Lucius. Je pense que ton père serait d'accord avec ça, il sait que tu dois grandir un jour. Mais tes coups dans ta partie doivent être les tiens. C'est ma condition - que j'ai affaire à toi, Draco, pas à ton père."

"Assez," dit Draco. Il se leva. "Beaucoup trop d'un coup. Je dois y aller et réfléchir à tout ça. Sans parler du fait qu'il est grand temps de monter dans le train."

"Prends ton temps," dit Harry. "Souviens-toi juste que ce n'est pas une offre exclusive, même si tu l'acceptes. La vraie science nécessite parfois plus d'une personne."

Les sons du quai gagnèrent en netteté alors que Draco s'éloignait.

Harry regarda la montre à son poignet, un simple modèle mécanique que son père lui avait acheté dans l'espoir qu'elle continue de fonctionner en présence de magie. Elle tictaquait toujours, et si c'était la bonne heure, alors il n'était pas encore tout à fait onze heures. Il aurait probablement dû monter dans le train et commencer à chercher c'était-quoi-son-nom-déjà, mais il sembla utile de prendre quelques minutes avant cela afin de faire quelques exercices de respiration et de voir si son sang se réchauffait.

Mais lorsque Harry releva les yeux de sa montre, il vit deux silhouettes à l'air totalement ridicule approcher, leurs visages masqués par des écharpes d'hiver.

"Bonjour, M. Bronze," dit l'une des figures masquées. "Pourrions-nous vous convaincre de vous joindre à l'Ordre du Chaos ?"
\par\noindent\rule{\textwidth}{0.4pt}
\emph{Après-coup} 

Peu de temps après ça, une fois l'agitation de la journée tassée, Draco se pencha au-dessus d'un bureau la plume à la main. Il avait une chambre privée dans les donjons de Serpentard, avec son propre bureau et son propre feu - malheureusement même \emph{lui}  ne méritait pas une connexion au réseau de cheminées, mais au moins Serpentard ne croyait pas à cette \emph{ineptie } complète qui consistait à faire dormir \emph{tout le monde}  dans des dortoirs. Il n'y avait pas beaucoup de chambres privées, vous deviez être parmi les meilleurs \emph{des meilleurs}  de la meilleure des Maisons, et au moins \emph{cela}  pouvait être considéré comme évident avec la maison Malfoy.

\emph{Cher Père,}  écrit Draco.

Et il s'arrêta.

De l'encre coula lentement de sa plume, tachant le parchemin, non loin des mots.

Draco n'était pas stupide. Il était jeune, mais ses précepteurs lui avaient appris à reconnaître certaines choses par simple reconnaissance de formes. Draco savait que Potter se sentait probablement beaucoup plus proche de la faction de Dumbledore qu'il ne le laissait entendre... bien que Draco pensait tout de même que Potter pouvait être tenté. Mais il était clair comme du cristal que Potter essayait de tenter Draco tout autant que Draco essayait de le tenter.

Et il était tout aussi clair que Potter était brillant, et bien plus que légèrement fou, et jouait un grand jeu qu'il ne comprenait pas lui-même, improvisant à toute vitesse avec la subtilité d'un Nundu déchaîné. Mais Potter était parvenu à choisir une tactique que Draco ne pouvait tout simplement pas refuser. Il avait offert à Draco une partie de son propre pouvoir, pariant sur le fait que Draco ne pourrait pas l'utiliser sans devenir comme lui. Son père lui avait dit que c'était une technique très avancée, et avait prévenu Draco qu'elle échouait souvent.

Draco savait qu'il n'avait pas tout compris de ce qui s'était passé... mais Potter \emph{lui}  avait offert une chance de jouer et cette chance était maintenant \emph{sienne} . Et si il crachait tout maintenant, ça deviendrait la chance de père.

C'était aussi simple que ça, au final. Les techniques moindres nécessitaient l'ignorance de la cible, ou du moins leur incertitude. La flatterie devait être déguisée en admiration de façon plausible ("Tu aurais dû aller à Serpentard" était un vieux classique, très efficace sur un type de personne qui ne s'y attendait généralement pas, et si ça fonctionnait vous pouviez le réutiliser.) Mais lorsqu'on trouvait le levier ultime d'une personne, ça n'avait plus d'importance qu'ils sachent que vous saviez. Potter, dans sa folle précipitation, avait deviné l'une des clés de l'âme de Draco. Et si Draco savait que Potter le savait - même si ça avait été plutôt évident -, ça ne changeait rien.

Donc cette fois, pour la première fois de sa vie, il avait de vrais secrets à garder. Il jouait sa propre partie. Ce fait comportait une sourde douleur, mais il savait que Père serait fier, et ça voulait dire que tout allait bien.

Laissant les taches d'encre en place - il y avait là un message, un message que son père comprendrait, car ils avaient joué au jeu des subtilités bien plus d'une fois - Draco écrit la seule question qui l'avait vraiment rongé dans toute cette affaire, la partie qu'il, il le pensait, \emph{aurait dû}  comprendre, mais qu'il ne comprenait pas, pas du tout.

\emph{Cher Père,} 

\emph{Suppose que je te dise que j'ai rencontré un étudiant à Poudlard, pas encore membre de notre cercle de connaissances, qui t'a appelé un 'instrument de mort parfait' et a dit que j'étais ton 'seul point faible'. Qu'aurais-tu à dire sur lui ?} 

Il fallut peu de temps pour qu'une chouette apporte la réponse à Draco.

\emph{Mon fils bien-aimé,} 

\emph{Je dirais que tu as eu la chance de rencontrer quelqu'un qui savoure les confidences intimes de Severus Rogue, notre ami et allié de valeur.} 

Draco fixa la lettre pendant un moment, et la jeta finalement dans le feu.

