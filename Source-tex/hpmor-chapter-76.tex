
\chapter{Interlude avec le confesseur   coûts ir}

Rianne Felthorne descendit les escaliers de pierre rugueuse et de mortier rudimentaire en maintenant entre les candélabres un \emph{Lumos}  allumé du bout de sa baguette dressée.

Elle arriva à la caverne de pierre vide percée de nombreuses ouvertures sombres, éclairée par une torche d'apparence ancienne qui s'embrasa lorsqu'elle entra.

Personne d'autre ne se trouvait encore ici, et après de longues minutes à rester nerveusement debout elle entama un sortilège destiné à métamorphoser un canapé assez grand pour que deux personnes puissent s'y asseoir ou peut-être même s'y allonger. Un simple tabouret de bois aurait été plus simple, elle aurait pu le faire en quinze seconde, mais... eh bien...

Même après que le canapé ait été entièrement Rogue invoqué, le professeur Rogue n'était toujours pas arrivé. Elle s'assit sur la gauche du canapé, son pouls remontant le long de sa gorge comme une enclume. Étrangement, à mesure que le délai s'allongeait, sa nervosité augmentait au lieu de diminuer.

Elle savait que ce serait la dernière fois.

La dernière fois avant que tous ses souvenirs ne disparaissent et qu'elle ne se retrouve dans une mystérieuse caverne à se demander ce qui se passait.

Il y avait dans cette idée quelque chose qui ressemblait à la mort.

Les livres disaient qu'un sortilège d'Oubliettes bien fait n'était pas nocif, que les gens oubliaient des choses tout le temps. Ils rêvaient puis se réveillaient sans se souvenir de leurs rêves. Le sortilège d'Oubliettes ne créait même pas tant de discontinuité que cela, seulement un bref instant de désorientation ; comme d'être distrait par un grand bruit et de perdre le fil d'une pensée impossible à retrouver ensuite. C'était ce que les livres disaient et c'était pour cela que les sortilèges d'oubli étaient entièrement approuvés par le ministère à toute fin gouvernementale officielle.

Mais quand même, \emph{ces pensées} , ces pensées qu'elle avait en ce moment même ; bientôt plus personne ne les aurait. Lorsqu'elle regardait le futur, il n'y avait personne pour compléter ces pensées qu'elles n'avait pas fini de former. Même si elle parvenait à tout régler dans son esprit pendant la prochaine minute, il n'en resterait ensuite plus rien. N'était-ce pas exactement le genre de chose que vous vous diriez si vous étiez à une minute de votre mort ?

Puis vint le bruit de pas étouffés...

Severus Rogue entra dans la caverne.

Ses yeux allèrent jusqu'à elle, assise sur le canapé, et il eut étrange regard, étrange parce qu'il n'était ni sardonique, ni énervé ni froid.

"Merci, Mlle Felthorne," dit doucement Rogue, "c'est très attentionné." Le maître des potions sortit sa baguette et lança les charmes d'isolement usuels, puis il se rapprocha et s'assit lourdement sur le canapé, à côté d'elle.

Son pouls battit alors pour une toute autre raison.

Elle se tourna lentement pour regarder le professeur Rogue et vit que sa tête reposait contre le dossier du canapé et que ses yeux étaient clos. Il ne dormait cependant pas. Son visage était tendu, nerveux, plein de douleur.

Elle sut - elle en fut soudain certaine - que cette vision ne lui avait été accordée que parce qu'elle ne s'en souviendrait ensuite pas ; et que personne avant elle n'avait jamais eu le droit de le voir.

La conversation haletante qui se déroulait à cet instant dans l'esprit de Rianne Felthorne ressemblait quelque peu à ceci : \emph{Je pourrais juste me pencher en avant et l'embrasser, tu as complètement perdu ce qui te sert de boule, ses yeux sont fermés je parie qu'il ne m'arrêterait pas à temps, je parie que personne ne retrouverait ton corps avant plusieurs années -} 

Mais le professeur Rogue ouvrit les yeux (à son grand dam et soulagement) et dit d'une voix plus normale : "Votre paiement, Mlle Felthorne." De ses robes il prit un rubis, taillé selon le standard Gringotts, et le lui tendit. "Cinquante facettes. Je comprendrais si vous les comptiez."

Elle tendit une main tremblante en espérant que Rogue placerait le rubis dans la main, qu'elle sentirait le contact de sa peau chaude contre la sienne -

Mais au lieu de cela, Rogue éleva légèrement sa main et fit tomber le rubis dans celle de Rianne Felthorne, puis il s'adossa de nouveau contre le canapé. "Vous vous souviendrez l'avoir trouvé par terre, sur le sol de cette caverne, que vous serez venue explorer," dit Rogue. "Et puisque personne sauf vous n'y croirait, vous vous souviendrez vous être dit qu'il serait moins problématique de déposer l'argent dans un compte séparé à Gringotts."

Pendant un moment, il n'y eut que les légers craquements de la torche.

"Pourquoi -" dit Rianne Felthorne. \emph{Il sait que je ne m'en souviendrai pas} . "Pourquoi \emph{l'avez-vous}  fait ? Je veux dire - vous m'avez dit de vous dire où seraient les brutes et combien il y en aurait mais pas si Granger serait là. Et je sais que le fonctionnement du Retourneur de Temps nécessite que, si vous voulez \emph{faire}  que Granger soit là, il ne faut pas qu'on vous dise si ça a déjà eu lieu ou non. Donc j'ai deviné que c'était \emph{nous}  qui lui disions où aller. C'était nous, n'est-ce pas ?"

Rogue hocha la tête sans parler. Ses yeux étaient de nouveau fermés.

"Mais," dit Rianne. "Je ne comprends pas \emph{pourquoi}  vous l'aidiez. Et maintenant - après ce que vous avez fait à Granger dans la grande salle - je ne comprends plus rien du tout." Rianne ne s'était jamais considéré comme quelqu'un de particulièrement gentil. Elle avait à peine remarqué la controverse au sujet du général Soleil. Mais quelque chose dans l'idée \emph{d'aider}  Granger à combattre les brutes... eh bien, elle s'était habituée à voir cela comme le côté des gentils et à se voir \emph{elle}  comme appartenant à ce camp. Et elle avait découvert que cela lui plaisait. C'était difficile de simplement abandonner cette idée. "Pourquoi avez-vous fait cela, professeur Rogue ?"

Rogue secoua la tête et son visage se pinça.

"C'est -" dit Rianne d'une voix hésitante. "Enfin - tant que nous sommes là - y a-t-il quoi que ce soit dont vous voudriez me parler ?" Il y avait quelque chose qu'\emph{elle}  aurait voulu dire mais elle n'arrivait pas à faire passer les mots au travers de ses lèvres.

"Je peux penser à autre chose," dit Rogue après un silence. "Si vous êtes intéressée, Mlle Felthorne."

Les yeux de Rogue étaient fermés et elle ne pouvait donc se contenter de hocher la tête. Sa voix de brisa presque lorsqu'elle se força à dire : "Oui."

"Il y a un garçon dans votre classe qui vous aime bien, Mlle Felthorne," dit Rogue, les yeux toujours fermés. "Je ne vous dirai pas son nom. Mais il vous regarde à chaque fois que vous traversez la pièce, quand il pense que vous ne regardez pas. Il rêve de vous et désire vous posséder, mais il n'a jamais demandé ne serait-ce qu'un baiser."

Le martèlement de son cœur s'intensifia.

"Répondez-moi honnêtement, Mlle Felthorne. Que pensez-vous de ce garçon ?"

"Eh bien..." dit-elle. Elle butait sur les mots. "Je pense - que ne même pas demander un baiser - serait -"

\emph{Triste.} 

\emph{Lamentable.} 

"Faible," dit-elle d'une voix tremblante.

"Je suis d'accord," dit Rogue. "Imaginez cependant que ce garçon vous ait aidé. Considéreriez-vous que vous lui devez un baiser, s'il le demandait ?"

Elle inspira d'un coup sec.

"Ou penseriez-vous," continua Rogue, ses yeux toujours fermés, "qu'il ne fait que vous ennuyer ?"

Les mots la poignardèrent comme un couteau et elle ne put s'empêcher d'avoir un halètement soudain.

Les yeux de Rogue s'ouvrir grand et son regard croisa le sien.

Puis le maître des potions commença à rire, un rire petit et glousseur.

"Non, pas \emph{vous} , Mlle Felthorne !" dit Rogue. "Pas \emph{vous}  ! Nous parlons \emph{vraiment}  d'un garçon. Un qui suit votre cours de potions, à vrai dire."

"Oh," dit-elle. Elle essaya de se souvenir de ce que Rogue avait dit plus tôt, à présent assez agacée par la pensée d'un garçon qui l'aurait toujours observée en silence. "Eh bien, euh, dans ce cas. Ça fiche plutôt la \emph{trouille} . Qui est-ce ?"

Le maître de potions secoua la tête. "Cela n'a pas d'importance," dit Rogue. "Par curiosité, que penseriez-vous si ce garçon était encore amoureux de vous des années plus tard ?"

"Euh," dit-elle, légèrement perdue, "ce serait carrément pathétique ?"

Un petit craquement de la torche résonna dans la caverne.

"C'est étrange," dit doucement Rogue. "J'ai eu deux mentors dans ma vie. Ils étaient tous les deux incroyablement perspicaces mais aucun ne m'a jamais dit les choses que je ne savais pas voir. La raison du silence du premier est assez claire, mais pour le second..." Le visage de Rogue se pinça. "Je suppose qu'il serait naïf de ma part de demander pourquoi il s'est tu."

Le silence s'étira tandis que Rianne cherchait désespérément quelque chose à dire.

"Il est curieux," dit Rogue d'une voix encore plus douce, "de regarder en arrière après trente-deux ans et de se demander quand sa vie a été détruite au-delà de tout espoir de récupération. Cela a-t-il été déterminé lorsque le Choixpeau a crié 'Serpentard !' ? Cela semble injuste puisqu'on ne m'a offert aucun choix ; le Choixpeau a parlé au moment où il a touché ma tête. Pourtant je ne puis dire qu'il m'ait faussement nommé. Je n'ai jamais chéri le savoir en tant que tel. Je n'ai pas été loyal envers la seule personne que je considérais comme une amie. Je n'ai jamais été enclin à de saintes colères, alors comme aujourd'hui. Courage ? Il n'y a aucune bravoure à risquer une vie déjà détruite. Mes petits peurs m'ont toujours dominées et je ne me suis jamais détourné d'aucun des chemins que j'ai emprunté, à cause de ces petites peurs. Non, le Choixpeau n'aurait jamais pu me mettre dans sa maison à elle. Peut-être ma perte était-elle alors déjà écrite. Même si le Choixpeau fait bien son œuvre, je me demande si c'est juste. Est-ce juste que certains enfants possèdent plus de courage que d'autres et que la vie d'un homme soit ainsi jugée ?"

Rianne Felthorne commençait à se rendre compte qu'elle n'avait pas la moindre idée de qui le maître des potions était à l'intérieur et que malheureusement, toutes ces sombres profondeurs cachées ne l'aidaient en rien pour son problème à elle.

"Mais non," dit Rogue. "Je sais quand les choses ont mal tourné pour la dernière fois. Je pourrais indiquer le jour exact, l'heure à laquelle j'ai manqué ma dernière chance. Mlle Felthorne, le Choixpeau vous a-t-il proposé Serdaigle ?"

"O-oui," dit-elle sans réfléchir.

"Avez-vous jamais été douée pour les énigmes ?"

"Oui," dit-elle à nouveau, parce que quoi que le professeur Rogue soit sur le point de dire, elle ne l'entendrait pas si elle répondait \emph{non} .

"Je suis nul en énigmes," dit Rogue d'une voix lointaine. "On m'en a un jour donné une à résoudre, et je n'en ai même pas compris la partie la plus simple avant qu'il ne soit trop tard. Je n'ai même pas compris que l'énigme était pour \emph{moi}  avant qu'il ne soit trop tard. Je croyais l'avoir simplement entendue par hasard, alors qu'en réalité, c'était moi qu'on avait entendu par hasard. J'ai donc vendu mon énigme à un autre, et c'est là que le naufrage de ma vie a dépassé tout espoir de sauvetage." La voix de Rogue était toujours lointaine, plus absente que peinée. "Et même maintenant, je n'ai rien compris d'important. Dites moi, Mlle Felthorne, imaginez qu'un homme ait un couteau en main, qu'il trébuche sur un bébé et se poignarde lui-même. Diriez-vous que le bébé avait," la voix de Rogue baissa d'un ton, comme s'il imitait une autre voix, plus grave, "\emph{LE POUVOIR DE LE VAINCRE}  ?"

"Euh... non ?" dit-elle d'un ton hésitant.

"Alors \emph{qu'est-ce}  qu'avoir le pouvoir de vaincre quelqu'un ?"

Rianne étudia l'énigme. (En souhaitant, et ce n'était pas la première fois, avoir choisi Serdaigle sans égard pour la désapprobation parentale. Mais le Choixpeau ne lui avait jamais offert Gryffondor). "Eh bien..." dit Rianne. Elle avait du mal à formuler ses pensées. "Cela veut dire qu'on a la \emph{capacité}  de le faire mais qu'on y est pas \emph{obligé} . Cela veut dire qu'on pourrait le faire si on essayait -"

"Une choix," dit le maître des potions de sa même voix lointaine, comme s'il ne s'adressait pas du tout à elle. "Il y aura un choix. C'est ce que l'énigme semble sous-entendre. Et ce choix n'est pas déjà écrit, car l'énigme ne dit pas \emph{le vaincra}  mais \emph{le pouvoir de le vaincre} . Comment un homme adulte marquerait-il un bébé comme son égal ?"

"Quoi ?" dit Rianne. Elle n'avait rien compris.

"Il est simple de \emph{marquer}  un bébé. N'importe quel sortilège maléfique puissant occasionnerait une cicatrice durable. Mais on pourrait le faire sur n'importe quel enfant. Quelle marque signifierait que le bébé est son \emph{égal}  ?"

Elle répondit par la première chose qui lui vint à l'esprit. "Si vous signez un contrat de fiançailles, ça veut dire que vous serez son égal un jour lorsqu'il grandira et que vous vous marierez."

"C'est..." dit Rogue. "Ce n'est probablement pas ça, Mlle Felthorne, mais merci d'avoir essayé." Les longs doigts délicats et affûtés par la préparation de potions à des degrés de précision inimaginables s'élevèrent et massèrent les tempes de l'homme. "Le poids de mots si fragiles suffit à me rendre fou. Un pouvoir qu'il ignore... ça \emph{doit}  être plus qu'un sortilège inconnu quelconque. Pas quelque chose qu'\emph{il}  pourrait acquérir par la pratique et l'étude. Quelque talent inné ? Personne ne peut apprendre à être un métamorphomage... et pourtant ce n'est certainement pas là un pouvoir qu'il \emph{ignore} . Je ne peux pas non plus voir comment \emph{les deux}  pourraient détruire l'autre jusqu'à ce qu'il n'en reste qu'un vestige. Je peux le concevoir dans un sens, mais pas dans l'autre..." Le maître des potions soupira. "Rien de cela n'a de sens pour vous Mlle Felthorne, n'est-ce pas ? Les mots ne sont rien. Ce sont des ombres. C'est \emph{l'intonation}  qu'elle a utilisé qui en portait le sens, et c'est quelque chose que je n'ai jamais pu..."

Le maître des potions laissa sa voix en suspens tandis que Rianne le regardait fixement.

"Une \emph{prophétie}  ?" dit Rianne d'un couinement haut perché. "Vous avez entendu une \emph{prophétie}  ?" Elle avait pris des cours de divination pendant deux mois avant d'abandonner par dégoût et c'est là tout ce qu'elle avait jamais appris sur le fonctionnement de ce domaine.

"J'essaierai une dernière chose," dit Rogue. "Quelque chose que je n'ai pas encore essayé. Mlle Felthorne, écoutez le \emph{son}  de ma voix, la \emph{façon}  dont je le dis, pas les mots eux-mêmes, et dites moi ce vous pensez qu'ils signifient. Pouvez-vous faire cela ? Bien," dit Rogue lorsqu'elle acquiesça docilement, même s'il elle n'était pas du tout sûre de ce qu'elle devait faire.

Et Severus Rogue inspira et entonna : "\emph{CAR CES DEUX DIFFERENTS ESFIS NE PEUVENT EXISTER DANS LE MÊME FONGUE.} "

Son échine en fut parcourue de frissons, et savoir que les mots caverneux avaient été prononcés en imitant une véritable prophétie ne fit qu'aggraver les choses. Sur les nerfs, elle dit la première chose qui lui vint à l'esprit, peut-être influencée par celui avec qui elle se trouvait. "Ces deux différents ingrédients ne peuvent exister dans le même chaudron ?"

"Mais pourquoi \emph{pas} , Mlle Felthorne ? Quel est le \emph{sens}  d'une telle phrase ? Qu'est-on vraiment en train de nous dire ?"

"Ah..." hasarda-t-elle. "Si les deux ingrédients se mélangent, ils prendront feu et brûleront le chaudron ?"

L'expression de Rogue ne bougea pas d'un millimètre.

"Peut-être," dit-il enfin après qu'ils soient resté assis sur le canapé le temps d'un silence horrible qui sembla durer plusieurs minutes, "cela expliquerait-il le mot \emph{devra} . Merci, Mlle Felthorne. Vous avez de nouveau été d'une grande aide."

"Je..." dit-elle. "J'ai été heureuse de..." et les mots se bloquèrent dans sa gorge. Le maître des potions l'avait remerciée d'un ton irrévocable et elle savait que le temps de la Rianne Felthorne qui se rappelait ses instants approchait de sa fin. "J'aimerais ne pas avoir à oublier cela, professeur Rogue !"

"J'aimerais," dit Severus Rogue d'un chuchotement si bas qu'elle put à peine l'entendre, "que tout se soit déroulé autrement..."

Le maître des potions se leva du canapé et le poids de sa présence disparut du flanc de Rianne. Il se tourna, sortit sa baguette de ses robes et la dirigea vers elle.

"Attendez..." dit-elle. "Avant cela..."

C'était, de façon curieuse, incroyablement difficile de faire le premier pas pour passer du fantasme à la réalité, de l'imagination à l'action. Même si ce n'était qu'un pas et que cela n'irait jamais plus loin. L'écart à franchir s'allongea, devint la distance séparant deux montagnes.

Le Choixpeau ne lui avait jamais offert Gryffondor...

...était-il juste que la vie d'une femme soit ainsi jugée ?

\emph{Si tu ne peux pas le dire maintenant, alors que tu ne t'en souviendras ensuite plus - alors que rien de présent ne se prolongera jamais, comme si tu étais sur le point de mourir - alors quand le feras-tu, quand le diras-tu à qui que ce soit ?} 

"Pourriez-vous m'embrasser avant ?" dit Rianne Felthorne.

Les yeux noirs de Rogue l'étudièrent si intensément que sa rougeur descendit jusqu'à sa poitrine et elle se demanda s'il savait très bien qu'elle agissait encore en faible, que ce n'était pas un baiser qu'elle avait vraiment voulu.

"Pourquoi pas," dit le maître de potions d'une voix basse, puis il se pencha au-dessus du canapé et l'embrassa.

Ce n'était pas ce qu'elle avait imaginé. Dans ses fantasmes, les baisers de Rogue avaient été fougueux, volés, mais c'était - c'était juste \emph{gênant} , en fait. Les lèvres de Rogue, pressées contre les siennes, trop fort, les appuyant sur ses dents, sous un mauvais angle, leurs nez plus ou moins tordus, les lèvres de Rogue trop \emph{serrées}  et -

Ce n'est que lorsque le maître des potions se redressa et qu'il recommença à lever sa baguette qu'elle comprit.

"Ce n'était pas..." dit-elle d'une voix curieuse en levant les yeux vers lui ."Ce n'était pas... est-ce que c'était... votre premier..."

Rianne Felthorne cligna des yeux face à la caverne de pierre qu'elle venait de découvrir, avec toujours en main l'extraordinaire rubis qu'elle avait trouvé caché dans la poussière d'un coin. C'était une aubaine incroyable, et elle ne savait pas pourquoi regarder ce rubis la faisait se sentir si triste, comme si elle avait oublié quelque chose, quelque chose qui avait été précieux pour elle.

