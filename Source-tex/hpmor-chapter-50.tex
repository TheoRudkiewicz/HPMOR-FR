
\chapter{Égocentrisme}

Padma Patil avait finit de dîner un peu tard, non loin de sept heures et demie, et sortait maintenant de la Grande Salle à bonne allure, en chemin vers le dortoir Serdaigle et les salles d'étude. Les ragots l'amusaient et la destruction de la réputation de Granger l'amusait encore plus, mais tout cela pouvait la distraire de son travail scolaire. Elle avait remis à plus tard une rédaction d'une copie sur le bois de \emph{Iomillialor}  à rendre pour le cours de Botanique de la semaine suivante, et elle n'avait à présent que cette nuit pour le terminer.

C'est en passant le long d'un couloir de pierre étroit et tordu qu'elle entendit le murmure, comme s'il était venu directement de son dos.

"\emph{Padma Patil...} "

Elle pivota aussi vite que l'éclair, sa baguette, déjà tirée d'une poche de ses robes, atterrissant dans ses mains ; si Harry Potter pensait qu'il pouvait se faufiler derrière elle et \emph{lui}  faire peur aussi facilement -

Il n'y avait personne.

Elle se retourna immédiatement et regarda dans l'autre sens, si cela avait été un sortilège de Ventriloquie -

Il n'y avait personne derrière non plus.

Le murmure revint, doux et dangereux, accompagné d'un léger sifflement.

"\emph{Padma Patil, jeune Serpentard..."} 

"Harry Potter, jeune Serpentard," dit-elle haut et fort.

Elle avait combattu Potter et sa Légion du Chaos des dizaines de fois, et elle \emph{savait}  que, d'une façon ou d'une autre, Harry Potter était à l'œuvre...

...même si le sortilège de Ventriloquie ne fonctionnait que si le lanceur était à portée de vue, et dans ce corridor tortueux elle pouvait facilement voir jusqu'au tournant de chaque extrémité ; il n'y avait personne ici...

...aucune importance. Elle connaissait son ennemi.

Il y eut un gloussement chuchoté, qui venait maintenant de derrière elle, et elle pivota et pointa sa baguette vers le chuchotement puis s'écria : "\emph{Luminos !} "

L'éclair de lumière rouge partit et frappa le mur, et celui-ci s'illumina d'un halo cramoisi avant de s'éteindre bien vite.

Elle ne s'était pas vraiment attendue à ce que cela fonctionne. Harry Potter ne \emph{pouvait } pas être invisible, pas pour de vrai, c'était une magie inaccessible à la plupart des \emph{adultes} , et elle n'avait jamais cru aux neufs dixièmes des histoires qui circulaient à son sujet.

La voix chuchotante rit de nouveau, à présent depuis son flanc.

"Harry Potter se tient au bord du précipice," murmura la voix comme si elle était maintenant très proche de son oreille, "il vacille, mais vous, vous êtes déjà en train de tomber, jeune Serpentard..."

"Le Choixpeau ne m'a jamais déclaré Serpentard, \emph{moi}  !" Elle recula jusqu'au mur afin de ne pas avoir à regarder derrière elle et leva sa baguette en position d'attaque.

Encore le doux rire. "Harry Potter est dans la salle commune de Serdaigle depuis une demi-heure, il aide Kevin Sifflebranche et Michael Corner à apprendre des recettes de potions. Mais cela n'a pas d'importance. Je suis ici pour te faire part d'un avertissement, Padma Patil, et si tu fais le choix de l'ignorer, cela te regarde."

"Très bien," dit-elle froidement. "Vas-y et préviens moi, Potter, je n'ai pas peur de toi."

"Serpentard fut une grande Maison, jadis," dit le murmure ; il semblait être devenu plus triste. "Serpentard était une Maison que vous auriez été fière de choisir, Padma Patil. Mais quelque chose s'est produit, quelque chose a tourné à l'aigre ; sais-tu ce qui est arrivé à la maison Serpentard, Padma Patil ?"

"Non, et je m'en fiche !"

"Mais tu devrais t'en préoccuper," dit le murmure comme s'il était venu juste de derrière sa tête appuyée au mur. "Car tu es toujours cette fille à qui le Choixpeau a offert ce choix. Penses-tu que choisir Serdaigle signifie que tu n'es pas Pansy Parkinson, que tu ne seras jamais Pansy Parkinson, quel que soit ton comportement ?"

En dépit de tout, de petits frissons de peur se répandaient maintenant depuis sa colonne vertébrale jusqu'à la surface de sa peau. Elle avait aussi entendu \emph{ces}  histoires-là au sujet de Harry Potter ; qu'il était secrètement Legilimens. Mais elle se tint droite, et elle mit tout le mordant qu'elle put dans sa voix lorsqu'elle dit : "Les Serpentard sont devenus mauvais afin d'obtenir du pouvoir, tout comme \emph{toi} , Potter. Et \emph{je}  ne le ferai pas. Jamais."

"Mais tu répandrais volontiers des rumeurs au sujet d'une fille innocente," chuchota la voix, "même si cela ne t'aide en rien à atteindre ne serait-ce qu'une seule de tes ambitions, et sans même considérer le fait qu'elle a des alliés puissants qui pourraient s'en trouver offensés. Ce n'est pas là le fier Serpentard des jours anciens, Padma Patil, ce n'est pas la fierté de Salazar, c'est un Serpentard en décomposition, Padma Parkinson au lieu de Padma Malfoy..."

Elle commençait à être plus terrorisée qu'elle ne l'avait jamais été au cours de sa vie, et la possibilité qu'il s'agisse \emph{vraiment}  d'un fantôme commençait à s'insinuer dans son esprit. Elle n'avait jamais entendu dire que les fantômes pouvaient se dissimuler ainsi, mais peut-être qu'ils n'avaient tout simplement pas l'habitude de le faire - sans parler du fait que la plupart des fantômes n'étaient pas aussi \emph{angoissants} , ils n'étaient que des gens morts après tout - "Qui \emph{êtes} -vous ? Le Baron Sanglant ?"

"Lorsque Harry Potter s'est fait brutaliser," chuchota la voix, "il a ordonné à ses alliés de refréner toute envie de vengeance ; vous souvenez-vous de cela, Padma Patil ? Car Harry Potter vacille, mais il n'est pas encore perdu ; il lutte, il se sait en proie au péril. Mais Hermione Granger n'a fait aucune requête similaire auprès de ses propres alliés. Harry Potter est maintenant en colère contre vous, Padma Patil, plus en colère qu'il ne l'aurait jamais été s'il s'était agi de lui... et \emph{il}  a lui-même des alliés."

Un frisson de peur la parcourut, elle sut qu'il avait été visible, et cela la fit se détester elle-même.

"Oh, n'ai pas peur," respira la voix. "Je ne te ferai pas de mal. Car vois-tu, Padma Patil, Hermione Granger est réellement innocente. \emph{Elle}  ne se tient pas au bord du précipice, \emph{elle}  ne tombe pas. Elle n'a pas demandé à ses alliés de s'empêcher de te faire du mal parce que la pensée qu'une telle chose est possible ne lui viendra jamais à l'esprit. Et Harry Potter sait très bien que s'il te faisait du mal, ou s'il faisait en sorte qu'il t'arrive quelque chose, pour le bien de Hermione Granger, alors elle ne lui adresserait plus la parole, et ce jusqu'à ce que le Soleil ait fini de brûler et que la dernière étoile du ciel se soit éteinte." La voix était à présent très triste. "Elle est une fille véritablement bienveillante, une personne telle que je peux seulement rêver de l'être..."

"Granger ne peut pas lancer le Patronus !" dit Padma. "Si elle était réellement aussi gentille qu'elle prétend l'être -"

"Peux-\emph{tu}  lancer le Patronus, Padma Patil ? Tu n'as même pas osé t'y essayer, car tu craignais le résultat."

"Ce n'est pas \emph{vrai}  ! Je n'avais pas le temps, c'est tout !"

Le murmure continua. "Mais Hermione Granger a essayé, ouvertement, devant ses amis, et lorsque sa magie a échoué, elle a été consternée et surprise. Car le sortilège du Patronus a des secrets que peu ont jamais connu, que peut-être personne hormis moi ne connaît aujourd'hui." Un gloussement doux, sans timbre. "Qu'il soit dit que ce n'est pas une souillure de son esprit qui empêche sa lumière de sortir. Hermione Granger ne peut lancer le Patronus pour la même raison que Godric Gryffondor, qui éleva ces murs, ne le put jamais."

Le couloir \emph{devenait}  plus froid, elle en était certaine, comme si quelqu'un utilisait le sortilège de rafraîchissement.

"Et Harry Potter n'est pas le seul allié de Hermione Granger." Il y avait maintenant une note d'amusement sec dans le chuchotement, et dans un instant d'effroi cela lui fit soudain penser au professeur Quirrell. "Il me semble que Filius Flitwick et Minerva McGonagall sont très attachés à elle. Vous est-il venu à l'esprit que si ces deux-là apprenaient ce que vous faites à Hermione Granger, ils pourrait devenir moins attachés à vous ? Ils n'interviendront peut-être pas ouvertement ; mais peut-être traîneront-ils un peu plus avant de vous décerner des points, peut-être seront-ils un peu plus lents à faire dériver les opportunités vers vous -"

"Potter m'a \emph{balancée}  ?"

Un gloussement fantomatique, un hé-hé-hé sec. "Pensez-vous que ces deux-là sont stupides, sourds et aveugles ?" Dans un murmure plus triste : "Pensez-vous que Hermione Granger ne leur est pas chère, qu'ils ne verront pas qu'elle souffre ? Tout comme ils ont peut-être un jour été attaché à vous, leur jeune et intelligente Padma Patil, mais vous êtes en train de tout gâcher..."

La gorge de Padma était sèche. Elle n'y avait pas pensé, pas du tout.

"Je me demande combien de gens se soucieront de vous, Padma Patil, au bout de ce chemin que vous empruntez. Vous distinguer de votre sœur a-t-il tant d'importance que cela ? Voulez-vous être l'ombre de la lumière de Parvati ? Votre peur la plus profonde a toujours été d'être en harmonie avec elle, de \emph{redevenir}  harmonie avec elle, devrais-je dire ; mais cela mérite-t-il de faire souffrir une fille innocente, uniquement pour marquer votre différence à la mesure de ce geste ? Devez-vous être la jumelle \emph{maléfique} , Padma Patil, ne pouvez-vous pas trouver un autre bien à poursuivre ?"

Son cœur battait à toutes forces contre sa poitrine. Elle, elle n'avait jamais parlé de cela à \emph{personne}  -

"Je me suis toujours demandé comment les élèves se brutalisent les uns les autres," soupira la voix. "Comment les enfants rendent leur vie plus difficile, comment ils transforment leur école en prison, de leurs propres mains. Pourquoi les humains se rendent-ils la vie si pénible ? Je peux vous donner une partie de la réponse, Padma Patil. C'est parce que s'ils n'imaginent pas qu'ils pourraient aussi souffrir à cause de leurs méfaits, les gens ne s'arrêtent pas pour réfléchir avant d'infliger de la douleur. Mais souffrir, oh oui, Padma Patil, vous allez souffrir, si vous restez sur cette voie. Vous souffrirez de la même solitude, de la même peur des autres envers vous, du même manque de confiance que vous infligez à présent à Hermione Granger. Seulement pour vous, ce sera mérité."

Sa baguette tremblait.

"Vous n'avez pas choisi votre camp lorsque vous êtes allée à Serdaigle, jeune fille. Vous choisissez votre camp par la façon dont vous vivez votre vie, par ce que vous faites aux autres et par ce que vous vous faites à vous-même. Illuminerez-vous la vie des autres, ou l'assombrirez-vous ? C'est cela, le choix entre la Lumière et les Ténèbres, pas un mot crié par le Choixpeau. Et la partie difficile, Padma Patil, n'est pas de rester dans la 'Lumière', la partie difficile est de savoir où elle se trouve, et de s'avouer à soi-même qu'on a emprunté le mauvais chemin."

Il y eut un silence. Il continua un moment, et Padma se rendit compte qu'elle avait été congédiée.

Elle faillit faire tomber sa baguette en essayant de la remettre dans sa poche. Elle faillit tomber en faisant un pas loin du mur, en se détournant pour partir -

"Je n'ai pas toujours fait le bon choix entre la Lumière et les Ténèbres," dit le chuchotement, maintenant fort et dur, directement dans son oreille. "Ne prenez pas ma sagesse pour une conclusion irrévocable, n'ayez pas peur de la remettre en question, car, bien que j'ai essayé, j'ai parfois échoué, oh oui, j'ai échoué. Mais vous faites du mal à une innocente véritable, vous n'accomplirez aucune de vos ambitions en le faisant, et cela ne fait partie d'aucun de vos plans malin. Vous infligez de la douleur uniquement pour plaisir que cela vous apporte. Je n'ai pas toujours fait le bon choix entre la Lumière et les Ténèbres, mais je sais que cela appartient au ténèbres, j'en suis certain. Vous faites souffrir une jeune fille innocente et échappez au châtiment seulement parce qu'elle est trop bonne pour tolérer que ses alliés n'agissent à votre encontre. Je ne peux vous faire de mal pour cela, alors sachez seulement que cela ne m'inspire aucun respect. Vous n'êtes pas de la trempe de Serpentard ; allez, et faites vos devoirs de Botanique, jeune Serdaigle !"

Le dernier chuchotement s'échappa dans un sifflement plus fort qui ressemblait presque à celui d'un serpent, et Padma fuit, elle fuit le long des couloirs comme si des Moremplis l'avaient poursuivie, elle courut sans considération pour les règles sur les cavalcades dans les couloirs, même lorsqu'elle dépassa d'autres élèves qui la regardèrent d'un air surpris, elle ne s'arrêta pas, elle courut jusqu'aux dortoirs Serdaigle, son pouls battait le long de son cou, la porte lui demanda "Pourquoi le Soleil brille-t-il le jour et non la nuit ?" et il lui fallu trois essais avant qu'elle ne puisse formuler une réponse cohérente, et la porte s'ouvrit alors et elle vit -

- quelques filles et quelques garçons qui la regardaient tous, et dans un coin de la table pentagonale, Harry Potter et Michael Corner et Kevin Sifflebranche qui relevaient la tête de leur manuel.

"Doux Merlin !" s'exclama Pénélope Deauclaire, se levant de son canapé. "Qu'est-ce qui t'es arrivé, Padma ?"

"Je," bégaya-t-elle, "j'ai, j'ai entendu - un fantôme -"

"Ce n'était pas le Baron Sanglant ?" dit Deauclaire. Elle leva sa baguette et un instant plus tard elle tenait une tasse, puis un \emph{Aguamenti}  remplit la tasse d'eau. "Voilà, bois ça, assieds-toi -"

Padma avançait déjà vers la table pentagonale. Elle regarda Harry Potter, qui la regardait en retour, de son regard calme, grave et un peu triste.

"\emph{Tu}  as fait ça !" dit Padma. "Comment - tu - comment oses-tu !"

Il y eut un silence soudain dans le dortoir Serdaigle.

Harry se contenta de la regarder.

Et il dit : "Y a-t-il quelque chose que je puisse faire pour toi?"

"Ne le nies pas," dit Padma d'une voix tremblante, "\emph{tu}  as mis ce fantôme à mes trousses, il a \emph{dit}  -"

"Je suis sérieux," dit Harry. "Est-ce que je peux t'aider ? Te trouver de la nourriture, ou aller te chercher un soda, ou t'aider avec tes devoirs, ou quelque chose comme ça ?"

Tout le monde les regardait.

"Pourquoi ?" dit Padma. Elle ne savait pas quoi dire d'autre, elle ne comprenait pas.

"Parce que certains d'entre nous se tiennent au bord du précipice," dit Harry. "Et la différence se joue dans ce qu'on fait pour les autres. Me laisseras-tu t'aider, Padma, s'il te plaît ?"

Elle le regarda et sut à cet instant qu'il avait reçu son avertissement, le même qu'elle.

"Je..." dit-elle. "Je dois écrire une copie sur \emph{Iomillialor } -"

"Laisse moi courir jusqu'à ma chambre et prendre mes affaires de Botanique," dit Harry. Il se leva de la table pentagonale, regarda Sifflebranche et Corner. "Désolé les gars, je vous verrai plus tard."

Ils ne dirent rien, ils se contentèrent de la regarder, comme tous les autres dans le dortoir, alors que Harry Potter marchait jusqu'aux escaliers.

Et alors qu'il commençait à monter, il dit : "Et personne ne la harcèlera de questions à moins qu'\emph{elle}  ne veuille en parler, j'espère que tout le monde a \emph{compris}  ?"

"Compris," dirent la plupart des première année et quelques uns des élèves plus âgés, certains d'un ton assez effrayé.
\par\noindent\rule{\textwidth}{0.4pt}
Et elle parla de nombreuses choses avec Harry Potter, de choses autres que le bois de \emph{Iomillialor}  - même sa peur de redevenir en harmonie avec Parvati, dont elle n'avait jamais parlé à \emph{personne} , mais après tout l'allié fantomatique de Harry savait déjà. Et Harry plongea dans sa bourse et en tira des livres \emph{bizarres} , et il les lui prêta à condition d'un secret absolu, en disant que si elle pouvait comprendre ces livres ils changeraient suffisamment ses motifs de pensée pour qu'elle ne redevienne jamais en harmonie avec Parvati...

À neuf heures, quand Harry dit qu'il devait partir, la rédaction n'était finie qu'à moitié.

Et lorsque Harry marqua une pause et la regarda en s'éloignant, et qu'il dit que \emph{lui}  pensait qu'elle était de la trempe de Serpentard, cela la fit se sentir bien pendant une minute entière avant qu'elle ne se rende compte de ce qu'on venait de lui dire et de qui le lui avait dit.
\par\noindent\rule{\textwidth}{0.4pt}
Lorsque Padma descendit au petit déjeuner ce matin, elle vit Mandy l'apercevoir et murmurer quelque chose à l'oreille de la fille assise à côté d'elle, à la table Serdaigle.

Elle vit la fille se lever du banc et marcher vers elle.

La nuit dernière, Padma avait été heureuse que la fille soit logée dans l'autre dortoir ; mais maintenant qu'elle y pensait, c'était pire, maintenant il fallait qu'elle le fasse devant \emph{tout le monde} .

Mais même si elle transpirait, elle savait qu'elle devait le faire.

La fille s'approcha -

"Je suis désolée."

"Quoi ?" dit Padma. C'était \emph{sa}  réplique.

"Je suis désolée," répéta Hermione Granger. Sa voix était forte, afin que tout le monde puisse entendre. "Je... je n'ai pas demandé à Harry de faire ça, et j'étais en colère quand je l'ai appris, et je lui ai fait promettre de ne pas recommencer, avec \emph{personne} , et je ne vais pas lui parler pendant une semaine... je suis vraiment, \emph{vraiment } désolée, Mlle. Patil."

Le dos de Hermione Granger était raide, son visage était raide, on pouvait voir de la sueur sur son visage.

"Euh," dit Padma. Ses pensées étaient complètement brouillées à présent...

Son regard bondit vers la table Serdaigle, où un garçon les regardait avec des yeux plissés et ses mains serrées sur ses jambes.
\par\noindent\rule{\textwidth}{0.4pt}
\emph{Plus tôt :} 

"Je t'ai dit d'être plus \emph{gentil}  !" couina Hermione.

Harry commençait à suer. Il n'avait encore jamais entendu Hermione lui crier dessus, et dans la salle vide, c'était assez bruyant.

"Je - mais - mais j'ai \emph{été}  gentil !" protesta Harry. "Je l'ai presque \emph{sauvée} , Padma avançait sur la mauvaise voie et je l'en ai détournée ! J'ai probablement changé sa vie entière, en mieux ! Et puis, tu aurais dû entendre la version \emph{originale}  de ce que le professeur Quirrell a proposé que je fasse -" c'est là que Harry se rendit compte de ce qu'il disait et qu'il ferma la bouche une seconde trop tard.

Hermione serra ses boucles amande, un geste qu'il ne l'avait jamais vue faire. "Qu'est-ce qu'\emph{il } a dit de faire ? De la \emph{tuer}  ?"

Le professeur de Défense avait suggéré que Harry identifie tous les élèves influents, tous les élèves clés dans son année et dans les autres afin de prendre le contrôle de tout le moulin à rumeurs de Poudlard, remarquant que c'était un défi globalement utile et amusant pour tout vrai Serpentard faisant ses études à Poudlard.

"Rien de \emph{tel} ," dit Harry, "il a juste dit que d'une façon générale je devrais gagner de l'influence auprès des gens qui démarrent les rumeurs, et \emph{j'ai}  décidé que la version \emph{gentille}  de cela serait juste de directement informer Padma du sens de ses actes et de leurs conséquences possibles au lieu d'essayer de la menacer ou quoi que ce soit du genre -"

"\emph{Tu appelles ça ne pas menacer quelqu'un ?} " les mains de Hermione tiraient maintenant ses propres cheveux.

"Euh..." dit Harry. "J'imagine qu'elle a pu se sentir \emph{un peu}  menacée, mais Hermione, les gens font tout ce qu'ils pensent pouvoir faire en toute impunité, la douleur qu'ils infligent ne leur importe pas tant qu'ils ne souffrent pas eux-mêmes, si Padma pense qu'il n'y a \emph{pas}  de conséquence lorsqu'on répand des mensonges à ton sujet alors \emph{bien sûr}  qu'elle va continuer à le faire -"

"Et tu pense qu'il n'y aura pas de conséquences après ce que \emph{tu}  as fait ?"

Soudain, Harry se sentit malade.

Hermione avait l'air plus en colère qu'elle ne l'avait jamais été. "Qu'est-ce que tu crois que les autres élèves pensent de toi maintenant, Harry ? De \emph{moi } ? Si Harry n'aime pas la façon dont vous parlez de Hermione, on vous lâchera des fantômes dessus, est-ce que c'est ce que tu veux qu'ils pensent ?"

Harry ouvrit la bouche et aucun mot ne sortit, il n'y avait juste... pas réfléchi ainsi, à vrai dire...

Hermione s'abaissa pour reprendre les livres de la table où elle les avait abattus. "Je ne vais pas te parler pendant une semaine, et je vais \emph{dire}  à tout le monde que je ne te parlerai pas pendant une semaine, et je leur dirai \emph{pourquoi} , et \emph{peut-être}  que cela défera une partie de ce que tu as fait. Et après cette semaine, je - je déciderai de ce que je ferai, je suppose -"

"\emph{Hermione}  !" La voix de Harry s'éleva pour devenir un hurlement désespéré. "\emph{J'essayais d'aider !} "

La fille se retourna et le regarda, alors qu'elle ouvrait la porte de la salle de classe.

"Harry," dit-elle, et sa voix tremblait un peu en-dessous de la colère, "le professeur Quirrell t'aspire vers les ténèbres, vraiment Harry, je suis sérieuse."

"Ce... n'était pas lui, ce n'était pas ce qu'il avait dit de faire, c'était seulement \emph{moi}  -"

La voix de Hermione n'était maintenant presque plus qu'un murmure. "Un jour tu vas aller déjeuner avec lui, et ce sera ton côté obscur qui reviendra, ou peut-être même que tu ne reviendras pas du tout."

"Je te promets," dit Harry, "que je \emph{vais}  revenir du déjeuner."

Il n'avait pas réfléchi en le disant.

Et Hermione se tourna de nouveau et sortit à grande enjambées et claqua la porte derrière elle.

\emph{Super invocation des lois de l'ironique dramatique, abruti, } nota le Critique Interne de Harry. \emph{Maintenant tu vas mourir samedi, tes derniers mots auront été 'Je suis désolé, Hermione,' et elle regrettera toujours que son dernier acte ait été de claquer la porte -} 

\emph{Oh tais-toi.} 
\par\noindent\rule{\textwidth}{0.4pt}
Lorsque Padma s'assit à côté de Hermione pour le petit déjeuner et lui dit d'une voix assez forte pour que les autres entendent que le fantôme lui avait dit des choses qu'elle avait eu besoin d'entendre et que Harry Potter avait eu raison de le faire, il y en eut certains qui furent moins effrayés, et d'autres qui le furent encore plus.

Et après cela, les gens \emph{dirent}  moins de choses méchantes au sujet de Hermione, du moins en première année, du moins en public, là où Harry Potter risquerait de l'entendre.

Lorsque le professeur Flitwick demanda à Harry s'il était responsable de ce qui était arrivé à Padma, et que Harry dit oui, le professeur Flitwick lui dit qu'il serait retenu pendant deux jours. Même si cela avait seulement été un fantôme et que Padma n'avait pas souffert, quand même, ce n'était pas un comportement acceptable venant d'un élève Serdaigle. Harry hocha la tête et dit qu'il comprenait pourquoi le professeur devait faire ça et qu'il ne protesterait pas ; mais étant donné que cela \emph{semblait } avoir \emph{bel et bien}  changé Padma, le professeur Flitwick pensait-il vraiment, en toute confidence, que Harry avait eu tort ? Et le professeur Flitwick marqua une pause, sembla y réfléchir vraiment, et il dit alors à Harry, d'une voix solennellement couinante, qu'il lui fallait apprendre à établir des rapports normaux avec les autres élèves.

Et Harry ne put s'empêcher de penser que c'était un conseil que le professeur Quirrell ne lui donnerait jamais.

Harry ne pouvait s'empêcher de penser que s'il avait fait les choses à la façon du professeur Quirrell, le façon \emph{Serpentard}  normale, un mélange de conditionnement positif et négatif pour amener Padma et les autres pipelettes sous son contrôle explicite, alors Padma n'en aurait pas parlé, et Hermione ne l'aurait jamais découvert...

...auquel cas Padma n'aurait pas été sauvée, elle serait restée sur la mauvaise voie, et elle aurait finit par en souffrir. Ce n'était pas comme si Harry avait \emph{menti } à Padma de quelque façon que ce soit lorsqu'il était revenu dans le passé, était devenu invisible et avait utilisé le sortilège de Ventriloquie.

Harry n'était pas toujours sûr d'avoir fait le bon choix, ni même \emph{un}  bon choix, et Hermione n'avait pas faiblit de sa résolution de ne pas lui parler - mais elle parlait beaucoup avec Padma. Étudier seul : cela faisait plus mal que ce à quoi Harry s'était attendu ; comme si son cerveau commençait déjà à oublier son aptitude à la solitude longuement travaillée.

Les jours jusqu'au déjeuner avec le professeur Quirrell avançaient très, très lentement.

