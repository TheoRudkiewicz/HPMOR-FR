
\chapter{Contrôle des pulsions}

ph'nglui mglw'nafh J. K. Rowling wgah'nagl fhtagn
\par\noindent\rule{\textwidth}{0.4pt}
"\emph{Je me demande ce qui cloche chez } lui\emph{"} 
\par\noindent\rule{\textwidth}{0.4pt}
"Turpin, Lisa !"

Chuchotement chuchotement chuchotement harry potter chuchotement chuchotement serpentard chuchotement chuchotement non sérieusement c'est quoi cette histoire chuchotement chuchotement

"SERDAIGLE !"

Harry se joignit aux applaudissements accueillaient la jeune fille qui, les bordures de ses robes maintenant bleu foncé, marchait timidement vers la table de Serdaigle. Lisa Turpin avait l'air partagée entre son désir de s'asseoir aussi loin de Harry Potter que possible et son désir de courir vers lui, de s'insérer de force à coté de lui, et de commencer à lui arracher des réponses.

Être au centre d'un événement extraordinaire et curieux pour être ensuite Trié à la Maison Serdaigle se rapprochait beaucoup d'être trempé dans de la sauce barbecue et jeté dans une fosse pleine de chatons affamés.

"J'ai promis au Choixpeau Magique de ne pas en parler," chuchota Harry pour la énième fois.

"Oui, vraiment."

"Non, j'ai vraiment promis au Choixpeau Magique de ne pas en parler."

"Très bien, j'ai promis au Choixpeau Magique de ne pas parler de \emph{presque tout}  et le reste est \emph{privé}  tout comme ça l'était pour \emph{toi}  alors \emph{arrête de me poser la question} ."

"Tu veux savoir ce qui s'est passé ? Très bien ! Voilà une partie de ce qui s'est passé ! J'ai dit au Choixpeau que le Professeur McGonagall le menaçait de lui mettre le feu et il a dit au Professeur McGonagall qu'elle était une jeune impudente et qu'elle devrait déguerpir de sa pelouse !"

"Si tu ne vas pas croire ce que je dis alors \emph{pourquoi est-ce que tu me poses la question}  !"

"Non, je ne sais pas non plus comment j'ai vaincu le Seigneur des Ténèbres ! Préviens-moi si tu découvres comment !"

"\emph{Silence !"}  cria le Professeur McGonagall depuis le podium de la table d'honneur. "\emph{Pas de discussions avant la fin de la Cérémonie de Triage !"} 

Le volume sonore s'estompa brièvement pendant que chacun attendait de voir si elle allait faire des menaces crédibles, puis les chuchotements reprirent à nouveau.

Dumbledore se leva, souriant chaleureusement.

Silence instantané. Quelqu'un donna des coups de coude frénétiques à Harry tandis qu'il continuait de chuchoter, et Harry se coupa à mi-phrase.

Dumbledore se rassit.

\emph{Note à moi-même : On ne plaisante pas avec Dumbledore.} 

Harry essayait encore de digérer tout ce qui avait eu lieu durant l'Incident avec le Choixpeau Magique. Et le moindre de ces mystères n'était pas ce qui s'était passé à l'instant où Harry avait enlevé le Choixpeau de sa tête ; il avait alors entendu un léger murmure qui ne semblait venir de nulle part, quelque chose qui sonnait étrangement Anglais tout en étant un sifflement, quelque chose qui avait dit : \emph{"Ssalutations de Sserpentard à Sserpentard : si tu veux chercher mes ssecrets, parle à mon sserpent."} 

Harry devinait vaguement que ce n'était pas censé faire partie du processus de Triage officiel. Et que c'était un bout de magie supplémentaire mis en place par Salazar Serpentard pendant la fabrication du Choixpeau. Et que le Choixpeau lui-même n'était pas au courant. Et que ça avait été déclenché quand le Choixpeau avait dit "SERPENTARD !", et peut-être que d'autres conditions avaient été satisfaites. Et qu'un Serdaigle tel que lui n'était \emph{vraiment, vraiment pas censé l'avoir entendu.}  Et que si il pouvait trouver une façon sûre de faire jurer le secret à Draco pour qu'il puisse l'interroger à ce sujet, ce serait un moment parfait pour boire de l'Hilari-Thé.

\emph{Jeune garçon, tu te résous à ne pas suivre le chemin d'un Seigneur des Ténèbres} 

\emph{et l'univers commence à jouer avec toi à l'instant où le Choixpeau quitte ta tête. Il y a des jours où ça ne paie pas de se battre contre le destin. Peut-être que j'attendrai jusqu'à demain avant de mettre en pratique ma résolution de ne pas devenir un Seigneur des Ténèbres.} 

"GRYFFONDOR !"

Ron Weasley reçu \emph{beaucoup}  d'applaudissements, et pas seulement de Gryffondor. La famille Weasley était apparemment très aimée par ici. Après un moment, Harry sourit et commença à applaudir avec les autres.

Mais après tout, il n'y avait pas de meilleur jour qu'aujourd'hui pour se détourner du Côté Obscur.

Que l'univers et le destin aillent se faire voir. Il en ferait voir au Choixpeau.

"Zabini, Blaise !"

Pause.

"SERPENTARD !" cria le chapeau.

Harry applaudit aussi Zabini, ignorant les étranges coups d'œil qu'il recevait de la part de tout le monde, y compris de Zabini.

Aucun autre nom ne fut appelé après ça, et Harry se rendit compte que "Zabini, Blaise" avait l'air proche de la fin de l'alphabet. Génial, maintenant il avait applaudit \emph{seulement}  Zabini... oh, tant pis.

Dumbledore se leva à nouveau et commença à se diriger vers le podium. Apparemment ils allaient se voir offrir un discours -

Et Harry fut frappé par l'inspiration, celle d'un test expérimental \emph{brillant} .

Hermione avait dit que Dumbledore était le plus puissant des sorciers, non ?

Harry mit sa main dans sa bourse et chuchota : "Hilari-Thé".

Pour que l'Hilari-Thé fonctionne, il faudrait qu'il fasse dire à Dumbledore quelque chose de \emph{tellement}  ridicule que même dans l'état de préparation mentale de Harry il s'étranglerait \emph{malgré tout} . Du genre : aucun étudiant de Poudlard ne pourrait porter de vêtement de l'année, ou sinon tout le monde serait transformé en chats.

Mais après tout si \emph{quelqu'un dans ce monde}  pouvait résister au pouvoir de l'Hilari-Thé, ce serait Dumbledore. Donc si ça marchait, l'Hilari-Thé était littéralement \emph{invincible} .

Harry décapsula l'Hilari-Thé sous la table, voulant agir discrètement. La canette fit un petit son de sifflement. Quelques tête se détournèrent vers lui, mais se retournèrent bientôt vers -

"Bienvenue ! Bienvenue à Poudlard en cette nouvelle année !" dit Dumbledore, rayonnant sur les étudiants avec ses bras grands ouverts, comme si rien n'aurait pu lui faire plus plaisir que de tous les voir ici.

Harry prit une première gorgée d'Hilari-Thé et abaissa la canette. Il avalerait le soda petit à petit et essaierait de ne pas s'étrangler \emph{quoi que dise Dumbledore}  -

"Avant que nous ne commencions notre banquet, je voudrais dire quelques mots. Et les voici : Content content boum boum marécage marécage marécage ! Merci !"

Tout le monde applaudit et acclama, et Dumbledore se rassit.

Harry était assis, figé, tandis que le soda ruisselait le long des coins de sa bouche. Il avait au moins réussi à s'étrangler \emph{discrètement} .

Il n'aurait vraiment vraiment \emph{vraiment}  pas dû faire ça. Incroyable à quel point ça devenait bien \emph{plus évident} , \emph{une seconde}  après qu'il soit \emph{trop tard} .

Rétrospectivement, il aurait probablement dû remarquer que quelque chose n'allait pas lorsqu'il avait pensé à la possibilité que tout le monde soit transformé en chats... ou même avant ça, se rappelant sa note mentale disant qu'il ne fallait pas plaisanter avant Dumbledore... ou sa résolution d'accorder plus de considération aux autres... ou peut-être si il avait eu \emph{un seul fragment de sens commun} ...

C'était sans espoir. Il était corrompu jusqu'au cœur. Gloire au Seigneur des Ténèbres Harry. On ne pouvait pas combattre le destin.

Quelqu'un demanda à Harry si il allait bien. (Les autres commençaient à se servir de nourriture, qui était magiquement apparue sur la table, wahou.)

"Je vais bien," dit Harry. "Excuse moi. Euh. Était-ce un... discours \emph{normal } pour le directeur ? Vous n'aviez...pas l'air...très surpris..."

"Oh, Dumbledore est clairement dément" dit un Serdaigle à l'air plus agé qui s'était assis à coté de lui et s'était présenté avec un quelconque prénom que Harry n'allait certainement pas se rappeler. "Très amusant, incroyablement puissant, mais complètement cinglé." Il marqua une pause. "Plus tard j'aimerais te demander pourquoi un fluide vert est sorti de tes lèvres et a ensuite disparu, même si je m'attends à ce que tu aie promis au Choixpeau Magique de ne pas parler de ça non plus."

Avec un grand effort, Harry s'empêcha de baisser les yeux vers l'incriminante canette d'Hilari-Thé qu'il avait en main.

Après tout, l'Hilari-Thé n'avait pas arbitrairement \emph{matérialisé}  une gros titre du Chicaneur au sujet de Draco et lui. Draco l'avait expliqué d'une façon qui donnait l'impression que tout avait eu lieu... naturellement ? Comme si l'Hilari-Thé avait \emph{altéré l'histoire pour que tout concorde ?} 

Harry s'imagina se frapper le front contre la table. \emph{Bam, bam, bam}  faisait sa tête dans son esprit.

Un autre étudiant baissa sa voix jusqu'au niveau d'un chuchotement. "J'ai entendu dire que Dumbledore était secrètement un cerveau génial qui contrôlait beaucoup de choses et qu'il utilisait sa folie comme couverture pour que personne ne puisse le soupçonner."

"J'ai entendu ça aussi," chuchota un troisième étudiant, et il y eut des hochements de tête furtifs tout autour de la table.

Ça ne pouvait qu'attirer l'attention de Harry.

"Je vois," chuchota Harry, baissa sa voix à son tour. "Donc tout le monde sait que Dumbledore est secrètement un cerveau."

La plupart des étudiants acquiescèrent. Un ou deux semblèrent soudain pensifs, y compris l'étudiant plus âgé assis à coté de Harry.

\emph{Êtes vous certains que c'est la table des Serdaigle ?}  Harry parvint à ne pas poser cette question tout haut.

"Brillant !" chuchota Harry. "Si tout le monde le sait, personne ne soupçonnera que c'est un secret !"

"Exactement," chuchota un étudiant, puis il fronça les sourcils. "Attends, ça n'a pas l'air de coller -"

\emph{Note à moi-même : Le 75ème centile des étudiants de Poudlard, c'est à dire la Maison Serdaigle, n'est pas le programme pour enfants surdoués le plus exclusif au monde.} 

Mais au moins il avait apprit un fait important aujourd'hui. L'Hilari-Thé était omnipotent. Et \emph{ça}  voulait dire...

Harry cligna des yeux de surprise alors que son esprit faisait le lien évident.

...\emph{ça}  voulait dire que dès qu'il aurait appris un sort permettant d'altérer temporairement son sens de l'humour, il pouvait faire survenir \emph{n'importe quoi}  en faisant en sorte de ne trouver \emph{qu'une seule chose}  suffisamment surprenante pour s'étrangler en la voyant avoir lieu, puis en buvant une canette d'Hilari-Thé.

\emph{Eh bien c'était un court voyage vers la divinité. Même moi je m'attendais à ce que ça prenne plus longtemps que mon premier jour d'école} .

Maintenant qu'il y pensait, il avait aussi complètement saccagé Poudlard en dix minutes de Triage.

Harry ressentait un certain regret à cette pensée - Merlin seul savait ce qu'un Directeur fou allait faire à ses sept prochaines années de scolarité - mais il ne pouvait à la fois \emph{s'empêcher}  de ressentir un tiraillement de fierté.

Demain. Pas plus tard que demain, au plus tard, allait-il s'arrêter d'avancer sur le chemin qui menait à Seigneur des Ténèbres Harry. Une perspective qui semblait plus effrayante minute après minute.

Et pourtant, étrangement, de plus en plus attrayante. Une partie de son esprit visualisait déjà les uniformes de laquais.

"Mange," grogna l'étudiant plus âgé assis à coté de lui, et il le frappa dans les côtes. "Ne pense pas. Mange."

Harry chargea automatiquement son assiette avec ce qui se trouvait en face de lui, des saucisses bleues avec des petits morceaux brillants, wahou.

"A quoi pensais-tu, le Triage -" commença Padma Patil, l'un des Serdaigles de première année.

"On n'importune pas pendant les repas !" dit un chorus d'au moins trois personnes. "Règle de Maison !" ajouta un autre. "Autrement nous mourrions tous de faim."

Harry découvrit qu'il espérait vraiment, \emph{vraiment vraiment}  que sa nouvelle idée astucieuse ne fonctionne pas \emph{vraiment} . Et que l'Hilari-Thé fonctionne d'une autre façon et qu'il n'ait pas \emph{vraiment}  l'omnipotent pouvoir d'altérer la réalité. Non pas qu'il ne \emph{veuille}  pas devenir omnipotent. C'était juste qu'il ne pouvait supporter l'idée de vivre dans un univers qui fonctionnait vraiment comme ça. Il y avait quelqu'un chose d'indigne dans le fait de s'élever grâce à l'utilisation intelligente d'une boisson gazeuse.

Mais il \emph{allait}  le vérifier expérimentalement.

"Tu sais," dit d'un ton aimable l'étudiant plus âgé assis à coté de lui, "nous avons un système pour forcer les gens comme toi à manger, veux-tu découvrir de quoi il s'agit ?"

Harry laissa tomber et commença à manger sa saucisse bleue. C'était plutôt bon, surtout les morceaux brillants.

Le dîner s'acheva avec une rapidité surprenante. Harry essaya de conserver un échantillon d'au moins une petite partie de toutes les nouvelles nourritures étranges qu'il avait vues. Sa curiosité ne supportait pas l'idée de \emph{ne pas savoir}  quel était le goût de quelque chose. Dieu merci ce n'était pas un restaurant ou vous deviez commander un seul plat et où vous ne sauriez jamais le goût de toutes les autres choses qui étaient sur le menu. Harry \emph{détestait}  ça, c'était comme une chambre de torture destinée à quiconque avait une étincelle de curiosité : \emph{Découvres un seul des mystères sur la liste, ha ha ha !} 

Puis ce fut l'heure du dessert, pour lequel Harry avait complètement oublié de laisser de la place. Il abandonna après avoir échantillonné un petit morceau de tarte à la mélasse. Toutes ces choses allaient certainement repasser au moins une fois avant la fin de l'année.

Qu'y avait-il sur sa liste de choses à faire, mis à part les activités scolaire habituelles ?

\emph{A faire 1. Fais des recherches sur les sorts d'altération de l'esprit pour que tu puisses tester l'Hilari-Thé et voir si tu as vraiment trouvé un chemin menant à l'omnipotence. En fait, fais des recherches sur tous les types de magie de l'esprit que tu pourras trouver. L'esprit est la fondation de notre pouvoir en tant qu'humains, donc toute magie l'affectant est la plus importante des magies.} 

\emph{A faire 2. En fait, c'est 'A faire 1' et l'autre est 'A faire 2'. Parcours les bibliothèques de Poudlard et de Serdaigle, familiarise-toi avec le système et assure toi que tu as au moins lu tous les titres de livres. Deuxième passage : lire toutes les tables des matières. Coordonne-toi avec Hermione qui a une mémoire bien meilleure que la tienne. Vois si il y a un système d'emprunt inter-bibliothèque à Poudlard et vois si vous pouvez tous les deux, surtout Hermione, visiter aussi ces bibliothèques. Si d'autres Maisons ont des bibliothèques privées, découvre comment y accéder légalement ou comment t'y introduire.} 

\emph{Option 3a : Fais jurer le secret à Hermione et commence les recherches sur 'De Serpentard à Serpentard : si tu veux chercher mes secrets, parle à mon serpent.' Problème : ça a l'air hautement confidentiel, et ça pourrait prendre un moment avant de tomber par hasard sur un livre contenant un indice.} 

\emph{A faire 0 : Cherche quelles sortes de sorts de recherche-et-obtention-d'information existent ; si il y en a. La magie de bibliothèques n'est pas aussi importante que la magie de l'esprit mais elle a une priorité bien plus élevée.} 

\emph{Option 3b : Chercher un sort pouvant lier Draco à un secret, ou vérifier de façon magique la sincérité de la promesse de Draco de garder un secret (Veritaserum ?), puis l'interroger sur le message de Serpentard...} 

A vrai dire... Harry avait un mauvais pressentiment au sujet de l'option 3b.

Maintenant qu'il y réfléchissait, il ne sentait pas trop l'option 3a non plus.

Les pensées de Harry revinrent au pire moment de sa vie jusqu'à ce jour, ces longues secondes d'horreur glaceuses de sang sous le Choixpeau, quand il pensait avoir déjà échoué. Il avait souhaité revenir de quelques minutes dans le passé et changer quelque chose, n'importe quoi avant qu'il ne soit trop tard...

Et il s'était révélé qu'il n'était après tout pas trop tard.

Vœu exaucé.

On ne pouvait pas changer l'histoire. Mais on pouvait la réussir du premier coup. Faire quelque chose de différent au \emph{premier}  essai.

Toute cette affaire avec Serpentard, rechercher ses secrets... ça ressemblait horriblement au genre de chose qu'on se rappelerait des années plus tard et au sujet de laquelle on dirait : 'Et c'est \emph{là}  que les choses ont mal tournées.'

Et il souhaiterai déséspérément avoir la capacité de revenir dans le temps et de faire un autre choix...

Vœu exaucé. Maintenant quoi ?

Harry sourit lentement.

C'était une pensée plutôt \emph{contre-intuitive} ... mais...

Mais il \emph{pourrait} , il n'y avait aucune règle disant qu'il ne pouvait pas, il \emph{pourrait}  prétendre n'avoir jamais entendu ce petit murmure. Laissons l'univers continuer exactement comme il l'aurait fait si ce moment crucial ne s'était jamais produit. Vingt ans plus tard, c'est exactement ce qu'il souhaiterait. Et il se trouvait que vingt ans avant vingt ans plus tard, c'était maintenant. Modifier le passé lointain était facile du moment qu'on y pensait au bon moment.

Ou... c'était encore \emph{plus}  contre-intuitif... il pourrait même en informer, oh, disons \emph{le Professeur McGonagall} , au lieu de Draco \emph{ou}  Hermione. Et il pourrait réunir quelques personnes choisies et enlever ce petit sort supplémentaire du Choixpeau.

Mais oui. Ça semblait être une idée \emph{remarquablement } bonne maintenant que Harry y avait \emph{pensé} .

Tellement évidente rétrospectivement, et pourtant, Harry était parvenu à ne pas penser aux Options 3c et 3d.

Harry se décerna +1 point dans son programme anti-Seigneur-des-Ténèbres-Harry.

Ça avait été un tour terriblement cruel que le Choixpeau lui avait joué, mais on ne pouvait en discuter les résultats en termes conséquentialistes. Cela dit, ça lui avait certainement donné une meilleure idée de ce à quoi pouvait ressembler la perspective d'une victime.

A faire 4 : s'excuser auprès de Neville Londubat.

Ok, il était parti maintenant, il n'avait qu'à continuer comme ça. \emph{Et chaque jour, à chaque moment, je deviens plus Pur et plus Pur...} 

Les gens autour de Harry avaient presque fini de manger à présent, et les plats à desserts commencèrent à disparaître, et les assiettes sales aussi.

Lorsque toutes les assiettes eurent disparues, Dumbledore se leva à nouveau de son siège.

Harry ne put s'empêcher de ressentir le besoin de boire un autre Hilari-Thé.

\emph{Tu veux RIRE} , pensa Harry à l'intention de cette partie de lui-même.

Mais l'expérience ne comptait pas si elle n'était pas reproduite, n'est-ce pas ? Et les dégâts avaient déjà été causés, non ? Ne voulait-il pas voir ce qui allait se produire \emph{cette}  fois-ci ? N'était-il pas \emph{curieux}  ? Et si il obtenait un différent résultat ?

\emph{Eh, je parie que tu es la partie de mon cerveau qui m'a poussé à jouer ce tour à Neville Londubat} .

Euh, peut-être ?

\emph{Et n'est-ce pas } immanquablement\emph{ évident que si je fais ça je vais le regretter une seconde après qu'il soit trop tard ?} 

Ben...

\emph{Ouais. Donc, NON.} 

"Ahem," dit Dumbledore depuis le podium, se passant la main dans sa longue barbe d'argent. "Juste un mot de plus maintenant que nous sommes tous nourris et étanchés. J'ai quelques informations de début de trimestre à vous donner."

"Les premières années devraient noter que la forêt de ces terres est interdite à tous les élèves. C'est pourquoi elle est appelée la Foret Interdite. Si son accès était autorisé elle ne serait pas appelée la Forêt Interdite."

Clair et direct. \emph{Note à moi-même : La Forêt Interdite est interdite.} 

"M. Rusard, le concierge, m'a demandé de vous rappeler qu'aucune magie ne peut être utilisée entre les classes dans les couloirs. Malheureusement nous savons tous que ce qui \emph{devrait être}  et ce qui \emph{est}  sont deux choses différentes. Merci de garder cela à l'esprit."

Euh...

"Les essais de Quidditch auront lieu durant la deuxième semaine du trimestre. Toute personne désirant jouer pour l'équipe de leur Maison devrait contacter Madame Bibine. Toute personne désirant reformuler l'intégralité du Quidditch devrait contacter Harry Potter."

Harry inhala sa propre salive et se lança dans une quinte de toux tandis que tous les yeux se tournaient vers lui. Mais comment \emph{diable}  ! Il n'avait jamais croisé les yeux de Dumbledore... du moins il le \emph{pensait} . Il n'avait alors certainement pas pensé au Quidditch ! Il n'en avait parlé à personne hormis à Ron Weasley et il ne \emph{pensait}  pas que Ron le dirait quelqu'un d'autre... ou Ron avait-il couru se plaindre auprès d'un professeur ? \emph{Mais comment...} 

"De plus, je dois vous dire que cette année, le corridor du troisième étage coté droit est hors limites pour quiconque ne souhaiterait pas mourir d'une mort très douloureuse. Il est gardé par une série élaborée de pièges dangereux et potentiellement mortels, et il est impossible que vous les franchissiez tous, en particulier si vous êtes en première année."

Harry ne ressentait plus rien à ce stade.

"Et finalement je présente mes plus profonds remerciements au Professeur Quirinus Quirrell pour avoir héroïquement accepté d'assumer la charge de Professeur de Défense contre les Forces du Mal de Poudlard." Dumbledore balaya les étudiants d'un regard scrutateur. "J'espère que tous les étudiants présenteront au professeur Quirrell la plus grande des courtoisies ainsi que la plus grande \emph{tolérance}  due au service extraordinaire qu'il rend à vous ainsi qu'à cette école, et que vous\emph{ ne nous importunerez pas } de \emph{plaintes tâtillonnes}  le concernant, à moins que \emph{vous}  ne vouliez essayer de faire son travail."

De \emph{quoi}  pouvait-il bien parler ?

"Je cède maintenant la place au Professeur Quirrell, qui souhaiterait dire quelques mots."

Le jeune homme mince et nerveux que Harry avait rencontré au Chaudron Baveur progressait lentement jusqu'au podium, jetant des regards apeurés dans toutes les directions. Harry entrevit l'arrière de sa tête, et il semblait que le Professeur Quirrell devenait déjà chauve en dépit de sa jeunesse apparente.

"Je me demande ce qui cloche chez \emph{lui} ," murmura l'étudiant à l'air plus âgé assis à coté de Harry. D'autres commentaires similaires furent discrètement échangés le long de la table.

Le Professeur Quirrell progressa jusqu'au podium et se tint là, clignant des yeux. "Ah..." dit-il. "Ah..." puis son courage sembla l'avoir totalement abandonné, et il se tint là silencieux, pris d'un tremblement occasionnel.

"Oh, génial," chuchota l'étudiant plus âgé, "on dirait que voilà une \emph{longue}  année de cours de Défense -"

"Salutations, mes jeunes apprentis," dit le Professeur Quirrell d'un ton sec et assuré. "Nous savons tous que Poudlard a une certaine tendance à \emph{l'infortune}  dans ses choix pour ce poste, et nul doute que nombreux sont ceux qui parmi vous se demandent déjà quelle malédiction s'abattra sur moi cette année. Je vous assure qu'aucune malédiction ne m'incapacitera." Il sourit avec finesse. "Croyez-le ou non, j'ai depuis longtemps désiré m'essayer au poste de Professeur de Défense contre les Forces du Mal, ici à l'école des Sorciers de Poudlard. Le premier à donner ce cours était Salazar Serpentard lui-même, et il était de coutume jusqu'au quatorzième siècle que les plus grands sorciers de combat de toutes persuasions s'essaient à enseigner ici. Parmi les anciens Professeurs de Défense se trouvent non seulement le légendaire héros vagabond Harold Shea mais aussi citation l'impérissable fin de citation Baba Yaga, oui, je vois certains d'entre vous frissonner à l'évocation de son nom bien qu'elle soit morte depuis six cent ans. Ça devait être intéressant que d'être alors élève à Poudlard, ne pensez-vous pas ?"

Harry avala péniblement sa salive, essayant de contenir la soudaine montée d'émotion qui l'avait dominée lorsque le Professeur Quirrell avait commencé à parler. Les tons précis de sa voix lui rappelaient un conférencier d'Oxford, et Harry commença à vraiment appréhender le fait qu'il n'allait pas revoir sa Maman ou son Papa avant Noël.

"Vous êtes habitués à voir le poste de Défense tenu par des incompétents, des vauriens et des malchanceux. Pour quiconque doté d'un sens de l'Histoire, sa réputation est tout autre. Tous ceux qui ont enseigné ici ne faisaient pas partie des meilleurs, mais les meilleurs ont tous enseigné à Poudlard. En telle auguste compagnie, et après tant de temps à anticiper ce jour, j'aurais honte de me donner un standard autre que la perfection. Et j'ai donc bien l'intention que chacun de vous se souvienne toujours de cette année comme de celle du \emph{meilleur}  cours de Défense que vous ayez jamais eu. Ce que vous apprendrez cette année vous servira à jamais et sera une fondation solide pour vos arts de Défense, qui qu'aient été vos enseignants passés et futurs.

L'expression du Professeur Quirrell devint sérieuse. "Nous avons \emph{beaucoup}  de terrain à rattraper, et peu de temps pour le parcourir. J'ai par conséquent l'intention de m'éloigner des conventions d'enseignement de Poudlard de plusieurs façons, ainsi que d'introduire des activités du soir." Il marqua une pause. "Si ce n'est pas suffisant, je pourrais peut-être trouver de nouvelles façons de vous motiver. Vous êtes mes étudiants depuis longtemps attendus, et vous \emph{donnerez}  le meilleur de vous-mêmes dans mon cours de Défense depuis longtemps attendu. J'ajouterais bien une terrible menace, comme 'Ou sinon vous souffrirez horriblement', mais ce serait tellement cliché, ne trouvez-vous pas ? Je m'enorgueillis d'être plus créatif que cela. Merci."

Et la vigueur et la confiance semblèrent s'écouler hors du Professeur Quirrell. Sa bouche s'ouvrit toute grande, comme si il s'était soudain trouvé face à un public inattendu, et il se retourna vers son siège dans un tressaillement convulsif, puis traîna les pieds jusqu'à celui-ci, voûté comme si il était sur le point de s'effondrer sur lui-même et d'imploser.

"Il a l'air un peu bizarre," chuchota Harry.

"Bah," dit l'étudiant à l'air plus âgé. "T'as encore rien vu."

Dumbledore revint au podium.

"Et maintenant," dit Dumbledore, "avant que nous n'allions au lit, chantons la chanson de l'école ! Choisissez tous votre air favori et vos paroles favorites, et c'est parti !"

