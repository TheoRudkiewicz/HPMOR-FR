
\chapter{Tout ce que je crois est faux}

\#include "stddisclaimer.h"
\par\noindent\rule{\textwidth}{0.4pt}
"\emph{Bien sûr que c'était ma faute. Il n'y a personne d'autre ici qui pourrait être responsable de quoi que ce soit."} 
\par\noindent\rule{\textwidth}{0.4pt}
"Donc, juste pour être bien clair, " dit Harry, "Papa, si le professeur te fait vraiment léviter, alors que tu sais que tu n'as été attaché à aucun fil, ce sera une preuve suffisante. Tu ne vas pas changer d'avis et dire que c'était un truc de magicien. Ce ne serait pas jouer franc jeu. Si tu penses que tu pourrais réagir comme ça, tu devrais le dire \emph{maintenant} , et nous pourrons trouver une autre expérience qui remplacera celle-ci."

Le père de Harry, le Professeur Michael Verres-Evans, leva les yeux au ciel. "Oui, Harry."

"Et toi, Maman, ta théorie dit que le professeur devrait en être capable, et si ça n'est pas le cas, tu admettras avoir eu tort. Pas d'excuses comme quoi la magie ne fonctionne pas lorsque les gens n'y croient pas, ou quoi que ce soit du genre."

Madame la Directrice Adjointe Minerva McGonagall regardait Harry avec perplexité. Elle avait bien l'air d'une sorcière avec ses robes noires et son chapeau pointu, mais lorsqu'elle parlait, c'était d'un ton formel et Écossais qui n'allait pas du tout avec son apparence. Au premier abord, elle ressemblait à quelqu'un qui devrait ricaner et mettre des bébés dans des chaudrons, mais l'effet était gâché à la seconde où elle ouvrait la bouche. "Est-ce assez, M. Potter ?", dit-elle. "Puis-je à présent opérer la démonstration ?"

"\emph{Assez}  ? Probablement pas, " dit Harry. "mais au moins ça \emph{aidera} . Allez-y, Madame la Directrice Adjointe."

"Professeur suffira,\emph{"}  dit-elle, puis, "\emph{Wingardium Leviosa."} 

Harry regarda son père.

"Huh, " dit Harry.

Son père le regarda à son tour. "Huh, " dit-il en écho.

Puis le Professeur Verres-Evans regarda le Professeur McGonagall. "Très bien, vous pouvez me faire descendre maintenant."

Son père fut précautionneusement descendu jusqu'au sol.

Harry s'ébouriffa les cheveux. Peut-être était-ce à cause de l'étrange partie de lui qui avait \emph{déjà}  été convaincue, mais... "Voilà qui est un peu décevant, " dit Harry. "Vous penseriez qu'il y aurait une sorte d'événement mental spectaculaire au moment d'une mise à jour au sujet d'une probabilité infinitésimale -" Harry s'interrompit. Maman, McGonagall, et même Papa lui jetaient à nouveau \emph{ce regard} . "Je veux dire au moment où on se rend compte que tout ce qu'on croit est faux."

Sérieusement, ça aurait dû être plus spectaculaire. Son cerveau aurait dû être en train d'évacuer à grande eau tout son stock d'hypothèses sur l'univers, aucune d'entre elle ne permettant à cette lévitation d'avoir eu lieu. Au lieu de ça, son cerveau semblait dire : \emph{Très bien, j'ai vu le professeur de Poudlard agiter sa main et faire voler ton père dans les airs. Et maintenant ?} 

La femme sorcière leur souriait et semblait s'amuser. "Souhaiteriez-vous une démonstration supplémentaire, M. Potter ?"

"Vous n'avez pas à faire ça, " dit Harry. "Nous venons de réaliser une expérience décisive. Mais..." Harry hésita. Il ne pouvait pas s'en empêcher. En fait, dans ces circonstances, il ne \emph{devrait pas}  s'en empêcher. Il était convenable et justifié d'être curieux. "Que pouvez vous \emph{faire}  d'autre ?"

Le Professeur McGonagall se transforma en chat.

Harry eut un mouvement de recul involontaire, si vite qu'il trébucha sur une pile de livres abandonnés et fit un dur atterrissage sur son arrière-train dans un bruit de claquement. Ses mains descendirent pour le retenir sans tout à fait atteindre leur but et il y eu un élancement d'avertissement dans son épaule alors que son poids terminait sa chute libre.

Le petit chat tigré redevint immédiatement une femme en robes. "Je suis navrée, M. Potter, " dit McGonagall, l'air sincère, bien que ses lèvres s'étiraient en un sourire. "J'aurais dû vous prévenir."

Harry avait le souffle court. Sa voix sortit étouffée. "\emph{Vous ne POUVEZ pas faire ça !"} 

"Ce n'est qu'une Métamorphose,\emph{"}  dit McGonagall. "Une transformation en Animagus, pour être exacte."

"Vous vous êtes transformée en chat ! Un PETIT chat ! Vous avez violé la Conservation de l'énergie ! Ce n'est pas qu'une règle arbitraire, c'est sous-jacent à la forme de l'opérateur quantique Hamiltonien ! Le rejeter détruit l'unitarité et vous vous retrouvez avec des signaux supraluminiques ! Et les chats sont COMPLIQUÉS ! Un esprit humain ne peut visualiser l'anatomie entière d'un chat, et toute sa biochimie, et qu'en est-il de sa \emph{neurologie } ? Comment pouvez vous continuer à \emph{penser}  avec un cerveau de la taille de celui d'un chat ?"

Les lèvres de McGonagall s'étiraient de plus en plus à présent. "Par magie."

"La magie \emph{ne suffit pas}  à faire ça ! Il vous faudrait être un dieu !"

McGonagall cligna des yeux. "C'est bien la premier fois qu'on me compare à \emph{ça} ."

La vue de Harry se brouillait tandis que son cerveau commençait à comprendre ce qui venait de se briser. Tout le concept d'un univers unifié par des lois mathématiques régulières venait d'être évacué ; la notion même de \emph{physique}  avec. Trois mille ans à résoudre des gros problèmes en les divisant en petits éléments, à découvrir que la musique des planètes avait la même mélodie qu'un pomme qui tombe, à découvrir que les vraies lois étaient parfaitement universelles et n'avaient d'exception nulle part et prenaient la forme de simples mathématiques gouvernant les parties infinitésimales des choses ; \emph{sans parler du fait}  que l'esprit était le cerveau et que le cerveau était fait de neurones, qu'un cerveau était une personne, \emph{était}  -

Puis une femme s'était transformée en chat, et tant pis pour le reste.

Cent questions se battaient pour être la première à franchir les lèvres de Harry. La gagnante se fit entendre : "Et à quel type d'incantation appartient \emph{Wingardium Leviosa}  ? Qui invente les mots pour ces sorts, des enfants en maternelle ?"

"C'est assez, M. Potter", dit McGonagall avec fraîcheur, bien que ses yeux brillaient d'un amusement contenu. "Si vous souhaitez apprendre la magie, je suggère que nous mettions la dernière main à cette paperasserie afin que vous puissiez être inscrit à Poudlard."

"Bien, " dit Harry, quelque peu étourdi. Il rassembla ses pensées. La Marche de la Raison devrait juste recommencer, voilà tout ; il avait toujours la méthode expérimentale et c'était ça le plus important. "Comment puis-je me rendre à Poudlard ?"

Un rire étouffé s'échappa de McGonagall, comme si on le lui avait arraché avec une pince à épiler.

"Un instant, Harry, " dit son père. "Tu te souviens de la raison pour laquelle tu n'as pas été à l'école jusqu'à maintenant ? Tu te souviens de ta situation ?"

McGonagall pivota et fit face à Michael. "Sa situation ? De quoi s'agit-il ?"

"Je ne dors pas comme il faut, " dit Harry. Il fit un geste d'impuissance. "Mon cycle de sommeil est de vingt-six heures. Je dois me coucher deux heures plus tard tous les jours. Je ne peux pas m'endormir plus tôt, et le lendemain je dois me coucher deux heures plus \emph{tard} . 22h, minuit, 2h, 4h, jusqu'à faire un tour d'horloge. Même si j'essaie de me lever tôt, ça ne change rien et je suis une loque toute la journée. C'est pour ça que je n'ai pas été à l'école jusqu'à maintenant."

"C'est une des raisons, " dit sa mère. Harry lui jeta un long regard.

McGonagall fit un long \emph{hmmmmm} . "Je ne me souviens pas avoir entendu parler d'un cas pareil auparavant..." dit-elle lentement. "Je vérifierai avec Madame Pomfrey si elle connaît un remède." Puis son visage s'éclaircit. "Non, je suis sûre qu'il n'y aura pas de problème - Je trouverai une solution d'une façon ou d'une autre. Maintenant, " et son regard devint dur à nouveau, "quelles sont ces \emph{autres}  raisons ?"

Harry jeta à nouveau un long regard à ses parents. "Je suis un objecteur de conscience à la scolarisation infantile, au motif que je ne devrais pas avoir à souffrir de l'abjecte incapacité d'un système scolaire perpétuellement défaillante à me fournir des enseignants ou du matériel d'étude d'une qualité ne serait-ce que minimalement adéquate."

Les deux parents hurlèrent de rire, comme si ils pensaient que ce n'était qu'une bonne blague. "Oh,\emph{"}  dit le père de Harry, les yeux brillant, "est-ce pour \emph{ça}  que tu as mordu un professeur de mathématique en CE2 ?"

"\emph{Elle ne savait pas ce qu'est un logarithme !"} 

"Bien sûr,\emph{"}  répondit la mère de Harry. "La mordre était une réponse très mature."

Le père de Harry hocha la tête. "Une politique mûrement réfléchie et destinée à résoudre le problème fort répandu des enseignants qui ne comprennent pas les logarithmes."

"J'avais \emph{sept ans}  ! Combien de temps allez vous continuer à ressasser cette histoire ?"

"Je sais,\emph{"}  dit sa mère avec compassion, "tu mords \emph{un } professeur de mathématiques et ils ne te laissent jamais l'oublier, c'est ça ?"

Harry se tourna vers McGonagall. "Et voilà ! Vous voyez ce que je dois endurer ?"

"Excusez moi,\emph{"}  dit Pétunia, et elle fuit à travers la porte vers le porche, d'où ses hurlements de rires restaient tout à fait audibles.

"Ahem, eh bien, voyons,\emph{"}  dit McGonagall, qui pour une raison ou une autre semblait avoir du mal à parler, "personne ne mordra de professeurs à Poudlard, est-ce bien clair, M. Potter ?"

Harry fit une mine renfrognée. "D'accord, je ne mordrai personne qui ne m'aura pas mordu d'abord."

Le Professeur Michael Verres-Evans dut lui aussi quitter la pièce lorsqu'il entendit ça.

"Bon,\emph{"}  soupira McGonagall après que les parents de Harry eurent retrouvé leur maîtrise d'eux-mêmes et furent revenus. "Bon, je pense que, dans ces circonstances, je devrais éviter de vous faire acheter votre matériel d'étude jusqu'à un jour ou deux avant le début des classes."

"Quoi ? Pourquoi ? Les autres enfants connaissent déjà la magie, non ? Je dois commencer à les rattraper tout de suite !"

"Soyez rassuré, M. Potter, " répondit McGonagall, "Poudlard est tout à fait capable d'enseigner les fondamentaux. Et je soupçonne, M. Potter, que si je vous laisse seul durant deux mois avec vos livres scolaires, même sans baguette, je reviendrai en cette maison et trouverais à la place un cratère bouillonnant d'une fumée violette entouré d'une ville dépeuplée ainsi qu'un fléau de zèbres en feu terrorisant ce qui reste de l'Angleterre."

La mère et le père de Harry hochèrent la tête à l'unisson.

"\emph{Maman ! Papa !"} 

