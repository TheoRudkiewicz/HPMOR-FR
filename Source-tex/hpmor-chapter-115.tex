
\chapter{Tais-toi et accomplis l'impossible, pt 2}

Harry était dans une sorte d'état d'absence. La concentration absolue était en partiellement dissipée et partiellement présente. Des parties de son esprit avaient été engourdies, peut-être délibérément, peut-être par une partie de lui assez intelligente pour prédire ce qui se passerait sinon. Ce qu'il venait de faire…

La pensée fut refermée et libéra de l'espace mental.

Harry se tenait au milieu d'un cimetière décrépi, des stèles éparpillés autour de lui.

Éclairées par la Lune et les étoiles, on pouvait voir les robes qui jonchaient le sol, entourées de textures différentes de celle de la terre, entourées d'une humidité teintée de rouge et de reflets lunaires. Certaines têtes s'étaient détachées des capuchons des robes et révélaient des cheveux longs ou courts, sombres ou clairs, et c'était tout ce que la Lune permettait de voir. Les masques d'argents étaient restés, donnaient l'impression que les cheveux poussaient de crânes et pas de têtes humaines.

La pensée fut refermée et libéra de l'espace mental.

Un fille dans un uniforme de Poudlard à bordures rouges dormait sur l'autel. Les affaires de Harry étaient empilées non loin.

Au sol gisait un homme pâle et trop grand au visage inhumain des moignons de poignets duquel coulait du sang.

\emph{Dès que le Seigneur des Ténèbres se réveillera, il détruira tout ce que tu aimes. Dumbledore n'est plus là pour l'arrêter.} 

\emph{Il ne peut être emprisonné, car il peut abandonner son corps à tout instant.} 

\emph{Il ne peut être définitivement tué sans détruire plus de cent Horcruxes, et l'un d'eux est la plaque de Pioneer.} 

\emph{Matériel : Une baguette. Cette fois tu peux la brandir et parler.} 

\emph{Tu as cinq minutes.} 

\emph{Résous.} 

Harry chancela vers l'autel, s'y agenouilla et ramassa sa bourse.

Il marcha vers l'endroit où gisait Voldemort.

La sensation funeste avait diminué après que Voldemort se fut évanoui. Maintenant, à mesure que Harry s'approchait, elle atteignait de terrifiants sommets, s'éveillait dans sa cicatrice sous forme de douleur.

Harry ignora un glapissement intérieur. C'était le dernier souvenir de Tom Jedusor, gravé dans son cerveau, le dernier motif mental à avoir été transféré au nourrisson avant que Tom Jedusor n'explose : un sentiment d'horreur et de consternation croissant associé à la résonnance devenue incontrôlable. Harry connaissait maintenant le sens de ce sentiment d'appréhension, et cela le rendait plus simple à ignorer. Il avait parié que l'effet de la résonnance toucherait principalement le lanceur de sort, proportionnellement à sa puissance, et il avait gagné.

Harry regarda le corps de Voldemort et inspira profondément - par la bouche, car les relents de cuivre auquel Harry ne pensaient pas seraient entrés par ses narines.

Harry s'agenouilla à côté de Voldemort, sortit son kit médical de sa bourse et plaça un garrot auto-serrant autour du poignet gauche, puis un autre autour du poignet droit.

Une telle préoccupation envers Voldemort semblait \emph{mauvaise} . Au plus profond de son esprit, une partie de lui savait que certaines personnes venaient de vivre quelque chose d'extrêmement négatif. Si Harry avait fait subir le même sort à Voldemort sans hésiter, voilà qui aurait été équilibré, voilà qui aurait été juste. Ce que Harry faisait à présent lui rappelait un Batman plus soucieux du Joker que des victimes de ce dernier ; lui rappelait un comic book dont les auteurs se torturaient quant à la moralité d'abattre les Célèbres Méchants pendant que des innocents continuaient de mourir en arrière-plan. Montrer plus de sollicitude envers le chef des méchants qu'envers ses laquais, faire \emph{plus attention}  à son sort qu'à celui de ses partisans d'importance moindre : c'était là un défaut de la nature humaine.

Harry se sentit donc mal en se relevant près du corps, les garrots serrés autour des poignets de Voldemort ; il avait l'impression d'être éthiquement monstrueux.

Pourtant, toute pensée stratégie sensée \emph{exigeait}  que le corps de Voldemort ne meure pas. L'âme qu'il s'était créée devait être attachée à son cerveau, empêchée de partir.

Harry fit un pas en arrière, s'éloigna du corps inconscient de Voldemort, et respira profondément par la bouche. Il alla vers ses affaires, reprit ses robes et d'autres objets, en commençant par le Retourneur de Temps qu'il remit autour de son cou, afin d'être prêt à fuir et à revenir si nécessaire…

Plus de cent Horcruxes.

C'était démentiel, il n'y avait pas d'autre mot, un signe du rapport maladif que Voldemort entretenait avec la mort. Un expert en sécurité Moldue aurait appelé cela de la sécurité par clôture, en référence à la construction d'une clôture de plus de cent mètres de haut au milieu du désert. Seul un ennemi très serviable tenterait de passer par-dessus. N'importe qui d'autre ferait simplement le tour, et rendre la clôture plus haute n'y changerait rien.

Lorsqu'on ne se laissait plus effrayer par l'apparente impossibilité du problème, il cessait d'être difficile, surtout comparé au précédent.

Par exemple, les parents de Neville avaient été torturés jusqu'à en devenir définitivement fous. Deux cent Horcruxes améliorés n'empêcheraient pas cette folie ; ils ne seraient plus que des échos du même esprit abîmé.

Si cela avait été la seule façon d'arrêter définitivement Voldemort, cette utilisation d'Endoloris aurait été justifiée. Elle aurait été juste et aurait montré que la vie du Joker ne valait pas plus que celle que le dernier de ses hommes de main.

Harry avait seulement besoin de créer un Patronus, de l'envoyer à… Maugrey Fol-Œil ?… et de lui dire de venir. Enfin, non, il était assez probable que le Patronus ne fonctionnerait pas si il était lancé dans \emph{ce}  but. Il pouvait peut-être se promettre de dire ça à Maugrey puis utiliser son Retourneur de Temps une fois hors de portée des barrières mises en place par Voldemort.

Et Voldemort pourrait alors subir Endoloris jusqu'à en devenir fou.

Ce n'était même pas la pire des fins possibles. Si sa baguette demeurait liée à sa vie et à sa magie où qu'il essaie de fuir, Harry aurait pu jeter la baguette de Voldemort dans la fosse d'Azkaban.

Il se retourna pour et l'observa. Il s'avança et continua de contrôler sa respiration, ignora la sensation de brûlure dans sa gorge. Il avait beau avoir changé de corps, une partie de Harry savait que Voldemort était \emph{aussi}  le professeur Quirrell. Même si le changement de personnalité avait été parfait, même si le professeur Quirrell n'avait donc été qu'un masque de plus…

Voldemort n'avait néanmoins pas prévu de faire souffrir Harry avant de le tuer. Quand Harry l'avait agacé, il n'avait pas pensé à ordonner à ses serviteurs de le torturer. Quand on faisait face à Voldemort, c'était lourd de sens. En fin de compte, peut-être qu'il avait eu un reste de sympathie pour l'autre Tom Jedusor.

…Harry n'aurait pas dû prendre cela en compte.

Si ?

Harry releva les yeux vers les étoiles. Ici, sous l'atmosphère, les étoiles scintillaient, elles étaient incrustée dans le faux dôme du ciel nocturne, répandues sur la trace de la voie Lactée - elle-même lumineuse comme un long ruban - l'air d'être si proches qu'un vol en balai aurait permis de les atteindre.

En ce moment critique, qu'auraient-ils voulu qu'il fasse, les enfants des enfants des enfants ?

La réponse à question semblait aussi évidente - à moins qu'elle n'ait émané de la partie de Harry qui se souciait encore du professeur Quirrell.

Les actes précédents de Harry avaient été justifiés car ils avaient empêché de bien plus grands maux. Il n'aurait pas pu arrêter Voldemort si les Mangemort lui avaient tiré dessus. Mais cet acte ne pouvait être rectifié par une tragédie inutile à l'encontre d'un autre être sentient, et ce même si cet être était Voldemort. Ce ne serait un jour qu'un chagrin de plus parmi ceux de l'ancienne Terre.

Le passé était passé. Après avoir fait le nécessaire, inutile de faire une once de mal de plus. Pas besoin d'équilibrer les choses, de les rendre bien symétrique.

Les enfants des enfants des enfants ne voudraient pas que Voldemort ait à mourir en plus de ses serviteurs. Ils ne voudraient pas le voir souffrir si cela n'accomplissait rien d'autre.

Harry inspira profondément et se délesta - pas de sa haine - pas tout à fait de sa haine - même à la fin, il n'avait pas réussi à haïr son créateur - mais quand bien même, Harry se délesta de \emph{quelque chose} . Du sentiment de \emph{devoir}  haïr Voldemort, de l'idée qu'il avait l'obligation de ressentir cette haine en réponse à la liste infinie de crimes que Voldemort avait commis sans raison particulière, pas même pour son propre bonheur…

\emph{Tu peux,}  lui murmurèrent les étoiles. \emph{Tu peux ne pas le haïr. Ça ne t'empêche pas d'être quelqu'un de bien.} 

En définitive, une seule solution était envisageable, et puisqu'il le savait déjà, il lui était inutile de se torturer. Seul le temps dirait si c'était la meilleure possible.

Il inspira profondément et accumula de la magie en lui. Le sortilège qu'il allait lancer n'aurait pas besoin d'être \emph{précis}  mais c'était quand même l'un des plus puissants qu'il ait jamais appris.

Il repensa à l'injustice qu'il y avait à voir Voldemort survivre à ses partisans. Il sentit la légère trace de froideur dans son sang qui accompagnait son envie d'être impitoyable. Puis il s'en délesta, s'en débarrassa sous la lumière stellaire, parce que son côté obscur n'avait jamais été rien d'autre que des motifs cognitifs hérités, qu'une autre mauvaise habitude de pensée à briser.

Il préféra regarder Hermione qui respirait, allongée sur l'autel, et il laissa enfin les larmes quitter ses yeux. Ce qu'il adviendrait d'elle à présent, le chemin qu'elle choisirait, Harry l'ignorait ; mais elle \emph{existerait}  pour ce choix, et leur amitié n'aurait pas détruit son existence. Ce n'est qu'après avoir remarqué sa surprise, lorsque son espoir s'était réalisé, que Harry avait compris à quel point cet espoir avait été précaire. Les choses se passaient parfois mieux que prévu.

Et Harry prit cette pensée pour alimenter la magie qu'il accumulait.

Le pouvoir qu'il emmagasinait vibrait en lui ; on aurait dit que tout son corps était sa baguette ; soit ses yeux s'étaient brouillés, soit un tremblement blanc lumineux s'était mis à le houx. Et Harry imagina la forme du sortilège qu'il allait lancer, une forme grossière, au motif simple, il fallait seulement inclure…

\emph{Tout, oubliez tout, Tom Jedusor, le professeur Quirrell, toute votre vie, votre mémoire épisodique, les déceptions, l'amertume, les mauvaises décisions, oubliez Voldemort…} 

Et au dernier moment avant de lancer le sort, il eut une dernière pensée, une touche finale…

\emph{Mais si vous avez jamais eu des souvenirs heureux, pas de cruauté ou de rire face à la douleur d'autrui, mais la chaude sensation d'aider l'autre ou d'être aidé, il y en aura peu, peut-être venus de votre enfance, si vous avez jamais eu des souvenirs heureux, alors gardez-les…} 

Quelque chose en lui se déplia alors et il sut qu'il avait fait le bon choix ; il envoya cela aussi dans sa baguette…

\emph{"OUBLIETTES"} 

Et tout se déversa, de Harry vers le sortilège.

Il chuta de côté, laissa tomber sa baguette. Des cris gutturaux sortirent de sa gorge, sa main toucha sa cicatrice en vain, puis l'explosion de douleur dans son crâne se dissipa. Ce n'est que vaguement qu'il aperçut les flocons de neige qui flottaient autour de lui, les grains de lumière argentée à la dérive, comme des poussières de Patronus.

La lumière d'argent ne dura qu'un instant, puis disparut.

Le professeur Quirrell avait disparu.

Il n'en restait qu'un vestige.

Et cet esprit, ou ce qu'il en restait, ne serait plus si différent de celui de Harry.

La prophétie était accomplie.

Chacun avait refait l'autre à son image.

Harry se mit à sangloter, recroquevillé sur la terre.

Il pleura un moment.

Puis il finit par se remettre sur pied, difficilement, et par ramasser sa baguette, parce qu'il lui restait du travail à faire.
\par\noindent\rule{\textwidth}{0.4pt}
Harry posa sa baguette directement sur le moignon de poignet de Voldemort ; sa cicatrice se gonfla de douleur, mais ni l'un ni l'autre n'explosa.

Et Harry entama une Métamorphose.

Lentement - mais plus vite qu'il avait Métamorphosé Hermione - le corps étourdi de l'homme-serpent changea, se reforma. À mesure que la Métamorphose progressait, et surtout lorsque la tête de l'homme-serpent devint vitreuse et réduite, la douleur dans la cicatrice de Harry s'estompa.

Le sortilège devrait être maintenu qu'il soit éveillé ou assoupi ; et plus tard, quand il serait plus âgé et plus puissant, et peut-être avec de l'aide, il annulerait la Métamorphose du Tom Jedusor à l'esprit vidé et soignerait son corps grâce au pouvoir de la Pierre. Ce serait \emph{après}  que le futur Harry ait trouvé quoi faire d'un sorcier presque entièrement amnésique, toujours doté de mauvaises habitudes de pensées et de motifs émotionnels très négatifs - on aurait presque pu appeler cela un côté obscur - en plus de beaucoup de connaissances et de savoir-faire magiques de haut niveau. Il avait fait de son mieux pour ne \emph{pas}  effacer cela, car il en aurait peut-être besoin un jour.

Pendant ce temps, tout comme la magie n'avait pas considéré qu'une licorne Métamorphosée était morte et n'avait donc pas déclenché d'alarmes, les Horcruxes de Voldemort ne considéreraient pas qu'un Voldemort Métamorphosé était mort et n'essaieraient donc pas de le ramener.

En tout cas, c'était l'idée.

La cicatrice de Harry le chatouilla une dernière fois quand l'anneau d'acier alla se glisser sur son petit doigt, muni d'une petite émeraude verte en contact avec sa peau. Puis sa cicatrice se calma.

Un rocher surélevé servit de chaise à Harry, et il s'y rendit en chancelant avant de s'y effondrer, de s'y reposer presque, de repousser l'épuisement qui menaçait de pénétrer son esprit. \emph{Je n'ai pas fini, j'ai encore du travail.} 

Harry prit une autre profonde inspiration, toujours par la bouche, dit "\emph{Lumos} " et observa le cimetière.

Des robes noires et des masques de crânes détachés, entourés de mares de sang…

Hermione Granger, endormie sur un autel.

Les robes vides de Voldemort, ses mains ensanglantées, là où il était tombé.

Quirinus Quirrell et ses robes déchiquetées, désarticulé, là où le sortilège de la Mort l'avait frappé.

Harry imagina quelqu'un d'autre regardant la scène et essayant de la comprendre. Il secoua la tête : ça n'allait pas, ça n'allait pas du tout.

Puis il se releva vivement du rocher en grimaçant. Son esprit protesta, et son corps aussi. Il n'avait pas été particulièrement blessé ni frappé, mais son corps ressentait directement l'impact du stress de la journée.

Il chancela vers ce qui restait de Voldemort et ramassa sa main gauche par terre.

Même sur la main gauche, on pouvait voir la fine trace d'écailles de serpent ; c'était très Voldemort. Tant mieux.

Harry alla jusqu'à l'autel où gisait une Hermione endormie et plaça doucement la main amputée autour du cou de Hermione, plaça précautionneusement les doigts pour qu'ils enserrent sa gorge. C'était difficile : Hermione semblait si paisible, si innocente dans son sommeil, et la main amputée de Voldemort semblait si laide ; Harry ignora violemment la partie de son esprit qui pensait à ça, puisque la pensée était absurde dans ce contexte.

Quelques petits sortilèges de Découpe permirent de gâcher la marque quasiment parfaite des nanofibres, ce qui était d'une importance critique : il ne fallait pas que la blessure de la main ressemble à celle des cous. Plusieurs \emph{Diffindos}  répandirent de petits bouts du poignet de Voldemort sur la chemise de Hermione, et Harry dut se rappeler que cela faisait partie de son plan.

Il répéta l'opération avec la main droite en la plaçant de façon symétrique.

Il utilisa ensuite \emph{Inflammare}  pour roussir les robes de Voldemort avant de placer les robes roussies autour de Hermione.

Le pistolet et la baguette de Voldemort entrèrent dans la bourse de Harry. Il mit la Pierre de Permanence dans une poche ordinaire, car il ignorait ce que la Pierre ferait à la bourse.

Le tas d'objets situés sous la robe de Quirrell, située elle aussi près de l'autel, lui fournit la baguette que le professeur de Défense avait utilisé sous le nom de Quirrell. Harry s'approcha de Quirrell, redressa son corps du mieux qu'il put et plaça sa baguette dans sa main. De façon prévisible, il se mit à pleurer, et essuya ses larmes de sa manche.

Il prit une autre inspiration, toujours par la bouche, répéta "\emph{Lumos} " et inspecta le cimetière une fois de plus.

Des robes noires, des masques de crâne détachés, et une Hermione Granger gisant sur un autel avec les mains tranchées de Voldemort agrippées à sa gorge, et ses habits roussis éparpillés autour d'elle. Quirinus Quirrell mort, ses vêtements arrachés et déchiquetés, sa baguette en main.

Ça irait.

Restait le problème d'attirer l'attention sur la scène.

Harry n'avait presque plus de magie. Mais il en avait toujours assez pour Métamorphoser une feuille en un ballon-sonde dégonflé de trois mètres de diamètre.

Sa bourse lui fournit une bouteille d'oxyacétylène, un bâton de dynamite et un rouleau de mèche. \emph{Soyez prêts, c'est la rengaine des Scouts, soyez prêts pour une vie pleine de trolls et qui sait quoi…} 

Harry gonfla le ballon avec l'oxyacétylène. Lorsqu'il exploserait, la surpression serait peut-être aussi bruyante qu'un bang supersonique.

Il attacha le bâton de dynamite - c'était plus que nécessaire comme source d'explosion, mais ça ferait l'affaire.

Il attacha une mèche de 60 secondes au bâton mais ne l'alluma pas tout de suite.

Il mit sa Cape d'Invisibilité, qui s'était trouvée avec tous les objets, à côté de l'autel.

Il récupéra son balai dans sa bourse et grimpa dessus.

Il lança un sortilège de silence autour de Hermione Granger - ça n'assourdirait pas \emph{tout}  le bruit, loin de là, et ce n'était de toute façon pas comme si elle serait définitivement handicapée si jamais ses tympans crevaient, mais c'était quand même poli.

Et ce fut tout. Après le sortilège de silence, il ne resta plus du tout de magie à Harry. Il était probablement vidé pour au moins une heure.

Il grimpa sur le balai et monta lentement, le ballon-sonde remplit d'oxyacétylène derrière lui. Lorsqu'il dépassa les arbres, le château de Poudlard devint visible, scintillant sous la Lune quelques kilomètres plus loin ; et il fit de son mieux pour estimer la distance et l'angle d'observation depuis Poudlard.

Une fois loin au-dessus de la forêt, il utilisa un briquet pour mettre le feu à la mèche de la dynamite attachée au ballon-sonde rempli d'oxyacétylène. Puis Harry fit pivoter le balai et partit en flèche - mais pas directement vers le château, car cela aurait risqué de l'amener trop près du chemin emprunté par le Harry-passé et le professeur Quirrell. Pas question de laisser le professeur ressentir la présence d'un autre Harry…

Il sentit une lame de tristesse s'enfoncer en lui et la repoussa.

\emph{Trente et un mille, trente-deux mille, trente-trois mille…} 

Lorsqu'il atteint quarante, désireux de préserver ses tympans, il consulta sa montre, nota l'heure exacte, et renversa une fois son Retourneur de Temps.
\par\noindent\rule{\textwidth}{0.4pt}
[NdT : Le prochain chapitre (116) paraîtra \textbf{vendredi 17 avril à partir de 20h}  (à 22h au plus tard).]

