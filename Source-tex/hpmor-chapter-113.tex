
\chapter{Examen Final}

La lune gibbeuse levée dans le ciel sans nuage, la majesté des étoiles et de la Voie Lactée visibles dans les ténèbres : voilà ce qui illuminait trente-sept masques de crânes scintillant au-dessus de robes noires et un Lord Voldemort, encore plus sombre, aux yeux rouges et lumineux.

"Bienvenue, mes Mangemorts," dit la voix de Lord Voldemort, suave, aigüe et terrible. "Non, ne me regardez pas, idiots ! Regardez le petit Potter ! Dix ans, voilà dix ans que nous nous sommes vus. Et pourtant vous répondez à mon appel comme si c'était hier…" Le Seigneur des Ténèbres s'approcha d'une silhouette encapuchonnée et toucha son masque. "Dans une \emph{parodie}  hâtivement métamorphosée de la véritable armure d'un Mangemort, avec un sortilège puéril pour déformer votre voix. Expliquez-vous, M. Honneur."

"Nos vieux masques et nos vieilles robes…" dit la robe dont le Seigneur des Ténèbres avait touché le masque. Même à travers le timbre de voix déformé, la peur était audible. "Nous… après votre départ, nous ne les utilisions plus pour nous battre… je n'ai donc pas maintenu leurs enchantements… et quand vous m'avez ordonné de venir, masqué, j'ai… j'ai toujours eu foi en vous, Maître, mais j'ignorais que vous reviendriez aujourd'hui… je suis véritablement navré de vous avoir déplu…"

"Assez." Le Seigneur des Ténèbres s'approcha d'une autre silhouette qui semblait trembler, alors même que le masque faisait face au Survivant et que sa baguette était fermement pointée vers ce dernier. "J'aurais jugé une telle négligence moins sévèrement si vous aviez continué mon programme autrement… M. Conseil. Mais je reviens pour trouver… quoi ? Un pays conquis en mon nom ?" La voix aigüe monta encore. "Non ! Je vois que jouez le jeu politique habituel du Magenmagot ! Je vois vos frères toujours abandonnés à Azkaban ! Voilà qui me déçoit… je m'avoue déçu… vous avez cru que j'étais parti, que la Marque des Ténèbres était morte, et vous avez abandonné mes buts. N'est-ce pas, M. Conseil ?"

"Non, Maître !" s'écria la silhouette masquée. "Nous savions que vous reviendriez - mais nous ne pouvions pas combattre Dumbledore sans vous…"

"\emph{Crucio.} "

Un cri horrible s'échappa hors du masque, déchira la nuit, et continua pendant de longues, longues secondes.

"Levez-vous," dit le Seigneur des Ténèbres à la silhouette qui s'était effondrée au sol. "Gardez votre baguette sur Harry Potter. \emph{Ne me mentez plus.} "

"Oui, Maître," sanglota la silhouette en se relevant.

Voldemort se remit à faire les cent pas derrières les silhouettes en robes noires. "J'imagine que vous vous demandez aussi ce que Harry Potter fait là… pourquoi il est invité à ma fête de renaissance."

"Je sais, Maître !" dit l'une des robes. "Vous comptez prouver votre puissance en le tuant devant nous, pour que chacun sache parfaitement qui est le plus fort ! Pour montrer que votre sortilège de la Mort peut aussi tuer ce soi-disant Survivant !"

Il y eut un silence. Aucune des personnes dissimulées n'osa parler.

Lentement, le Seigneur des Ténèbres, dans sa chemise à col montant et ses robes noires, se tourna pour faire face au Mangemort qui avait osé parler.

"Voilà," murmura Voldemort d'une voix meurtrière, "qui est un peu trop insensé pour que j'y croie, M. Saule. Ayant entendu la théorie sur ma mort, vous tentez de me pousser à répéter mon erreur ?" Lord Voldemort flottait dans les airs, s'éloignait du sol. "J'imagine que vous en êtes venu à préférer votre paresse à mon emprise, \emph{Macnair}  ?"

Le Mangemort qui avait parlé fut soudain entouré d'un halo bleu. Il pivota, brandit sa baguette vers le Seigneur des Ténèbres et s'écria : "\emph{Avada Kedavra !} "

Voldemort se pencha simplement de côté et évita le rayon vert.

"\emph{Avada Kedavra !} " répéta le Mangemort. Sa main libre faisait d'autres gestes et d'autres couleurs s'ajoutaient au halo protecteur à la fin de chacun. "Aidez-moi, mes frères ! Si nous nous unissons…"

Le Mangemort s'effondra en sept morceaux brûlants, sept bouts de chairs dont les faces cautérisées rougeoyaient encore.

"Yeux et baguettes braqués sur Harry Potter, vous tous," répéta Voldemort d'un ton sourd et dangereux. "Et Macnair vient de se comporter d'une façon profondément stupide, car je suis maître de vos Marques, aujourd'hui et pour toujours. \emph{Je suis immortel.} "

"Maître," dit une autre robe. "La fille sur l'autel… nous servira-t-elle pour un Sombre Délice ? Elle est indigne d'une grande occasion telle que celle-ci. Je pourrais trouver mieux, Maître, si vous me laissiez quelques instants…"

"Non, M. Amical," dit Voldemort, apparemment très amusé. "La petite sorcière que vous voyez sur cet autel n'est autre que Hermione Granger…"

"Quoi ?" s'écria l'une des robes, puis : "Pardon Maître, pardon, j'implore votre…"

"\emph{Crucio.} " Les cris ne durèrent que quelques secondes, et Voldemort semblait avoir principalement agi pour la forme. Il reprit ensuite un ton amusé. "J'ai ressuscité cette Sang-de-Bourbe par la plus noire des magies et à des fins qui me concernent. Aucun d'entre vous ne lui causera le moindre mal. Je préfère vous voir morts que d'apprendre que ma petite expérience a souffert par votre faute. Cet ordre est absolu, indépendant des circonstances - par exemple, même si elle venait à s'échapper." Un rire froid et flûté, comme en réponse à une blague que personne d'autre n'avait comprise.

"Maître," dit l'une des robes d'une voix vacillante et déformée par le masque de crâne. "Maître, je vous prie… je ne vous défierais jamais, je vous obéis… mais Maître, je vous en prie, laissez-moi repartir pour mieux vous servir plus tard… je suis venu ici en hâte, au mépris de… Maître, nous sommes si nombreux à être partis, les autres s'interrogeront, ils remarqueront les absences, les noms des disparus. Bientôt je n'aurai plus d'alibi crédible à offrir."

Un rire froid et aigu. "Ah, M. Blanc, le plus délinquant de mes serviteurs. Je n'ai pas encore décidé si vous allez survivre à votre punition. J'ai moins besoin de vous qu'auparavant, M. Blanc. Dans deux jours, les Mangemorts marcheront à la vue de tous. Mes pouvoirs ont augmenté, et je me suis débarrassé de Dumbledore plus tôt dans la journée." Des hoquets de surprise s'élevèrent chez les Mangemorts, ce à quoi Voldemort ne prêta aucune attention. "Demain je tuerai Bones, Crouch, Maugrey et Scrimgeour, à moins qu'ils n'aient fui. Vous autres irez au ministère et au Magenmagot, où vous lancerez des Imperius selon mes ordres. Nous avons \emph{fini}  d'attendre. Demain soir, je me serai déclaré grand dirigeant d'Angleterre !"

Les masques assemblés respiraient bruyamment, mais une silhouette riait.

"Vous me trouvez amusant, M. Sinistre ?"

"Mes excuses, Maître," dit la silhouette en robes qui avait ri, sa baguette parfaitement stable, pointée sur Harry. "J'ai été ravi d'entendre que vous avez abattu Dumbledore. J'ai fui l'Angleterre comme un lâche parce qu'il m'effrayait et que j'avais perdu tout espoir de vous voir revenir."

Le gloussement de Voldemort résonna dans le cimetière. "Votre candeur vous vaut ma clémence, M. Sinistre. J'ai été surpris de vous voir ici, ce soir ; vous êtes plus compétent que je ne le pensais. Mais avant de nous atteler à ces joyeux projets, une certaine affaire requiert notre attention. Dites-moi, M. Sinistre, si le Survivant vous faisait une promesse, un serment, lui feriez-vous confiance ?"

"Maître… je ne comprends pas…" dit M. Sinistre. Un ou deux Mangemorts tournèrent leur masque vers Voldemort avant de rapidement rediriger leur crâne vers Harry.

"Répondez," siffla Voldemort. "Ce n'est pas un piège, M. Sinistre, et si vous ne me répondez pas sincèrement, vous en subirez les conséquences. Vous connaissiez les parents du garçon, n'est-ce pas ? Des gens francs et honnêtes, non ? Si le garçon choisissait librement de vous faire une promesse, alors même qu'il vous sait être un Mangemort, y croiriez-vous ? Répondez-moi !" La voix de Voldemort était devenue un glapissement.

"Je… oui, Maître, j'imagine que oui…"

"Bien," continua froidement Voldemort. "La possibilité d'une confiance doit exister pour pouvoir être sacrifiée. Quant au lieur du Serment Inviolable… lequel d'entre vous sacrifiera-t-il sa magie ? Ce sera un serment plutôt long… bien plus long que d'habitude… il faudra beaucoup de magie…" Voldemort eut son horrible sourire. "M. Blanc fera l'affaire."

"Non, s'il vous plaît ! \emph{Maître, je vous en supplie !}  Je vous ai mieux servi que tout… de mon mieux…"

"\emph{Crucio} ," dit Voldemort, et M. Blanc poussa un hurlement déformé par son masque pendant une bonne minute. "Réjouissez-vous que je vous laisse la vie ! Maintenant, messieurs Sinistre et Blanc, approchez-vous du petit. Par derrière, idiots ! Ne bloquez pas les baguettes des autres ! Et vous autres, tirez si Harry Potter essaie de fuir, même au risque de toucher vos collègues Mangemorts."

M. Blanc prit son temps et ses robes noires semblaient trembler, tandis que M. Sinistre prit place avec nonchalance.

"Quel est le Serment, Maître ?" dit la voix de M. Sinistre.

"Ah, oui," dit Voldemort. Le Seigneur des Ténèbres continua de faire les cent pas derrière le demi-cercle de Mangemorts. "Aujourd'hui - mais je ne m'attends pas à ce que vous me croyiez - aujourd'hui nous accomplissons l'œuvre de Merlin, mes Mangemorts. Oui ! Devant nous se tient un grand danger, dont la prophétie dit que l'insouciance crasse sera à l'origine de destructions que même moi, j'ai peine à imaginer. Le Survivant ! L'enfant qui fait peur aux \emph{Détraqueurs}  ! Les ruminants qui se croient maître de ce monde auraient dû avoir peur en le découvrant ! Inutiles, les uns autant que les autres !"

"Pardonnez-moi…" dit une robe noire d'une voix hésitante. "Maître… certainement, si c'est le cas… Maître, pourquoi ne pas simplement le tuer ?"

Voldemort rit, un étrange rire amer. Lorsqu'il parla de nouveau, sa voix aigüe était précise : "Voici l'intention derrière le serment, M. Sinistre, M. Blanc, Harry Potter. Écoutez bien et comprenez le Serment qui doit être pris, car son intention vous lie aussi, et sa compréhension doit être partagée. Vous jurerez, Harry Potter, de ne pas détruire le monde, de ne prendre aucun risque lorsqu'il s'agira de ne pas le détruire. Ce Serment ne vous \emph{force}  à aucune action dans le domaine, il ne vous pousse pas à agir de façon stupide. Comprenez-vous cela, M. Sinistre, M. Blanc ? Nous avons affaire à une prophétie de destruction. Une \emph{prophétie !}  Elles peuvent s'accomplir d'étranges façons. Nous devons nous assurer que ce Serment ne précipite pas lui-même la prophétie. Nous ne pouvons permettre à ce Serment de forcer Harry Potter à l'inaction suite à quelque désastre né de sa main, sous prétexte qu'il lui faudrait alors prendre un moindre risque pour y mettre fin. Le Serment ne doit pas non plus le forcer à choisir un risque d'immense destruction au profit de la certitude de ravages moindres. Mais toute la \emph{bêtise}  de Harry Potter," sa voix monta d'un cran, "toute son \emph{inconscience} , tous ses \emph{plans grandioses}  et ses \emph{bonnes intentions}  - il ne les laissera pas nous mener au désastre ! Il ne jouera pas le destin de la Terre aux dés ! Pas de recherches pouvant mener à la catastrophe ! Pas de sceaux descellés, pas de portes ouvertes !" Sa voix redescendit. "À moins que ce vœu lui-même ne risque la destruction du monde, auquel cas, Harry Potter, vous l'ignorerez. Vous ne vous ferez \emph{pas}  confiance dans ce domaine et devrez vous confier sincèrement et totalement à votre amie de confiance afin d'avoir son opinion. Voilà le sens et l'intention de ce Serment. Il ne provoque que des actions que Harry Potter aurait pu lui-même prendre après avoir appris qu'il était l'instrument d'une destruction prophétisée. Car la capacité à choisir doit aussi exister avant d'être sacrifiée. Comprenez-vous, M. Blanc ?"

"Je… je pense… oh Maître, \emph{s'il vous plaît} , faites que le Vœu soit plus court…"

"Silence, idiot, vous vous rendez plus utile aujourd'hui que jamais auparavant dans votre vie. M. Sinistre ?"

"Je pense, Maître, qu'il va falloir me le répéter."

Voldemort eut son sourire trop large et se répéta avec d'autres mots.

"Et maintenant," dit-il froidement, "Harry Potter, vous garderez votre baguette baissée, et vous allez laisser M. Sinistre laisser sa baguette toucher la vôtre ; et vous répéterez les mots que je vous dicte. Si Harry Potter dit un seul autre mot, abattez-le, vous autres."

"Oui, Maître," répondirent trente-quatre voix en chœur.

Harry était transi de froid, il frissonnait, et ce n'était pas seulement parce qu'il était nu au beau milieu de la nuit. Il ne comprenait pas pourquoi Voldemort ne se contentait \emph{pas}  de le tuer. Il ne semblait y avoir qu'un chemin vers l'avenir, celui choisi par Voldemort, et Harry ignorait ce qui viendrait après.

"M. Blanc," dit Voldemort. "Touchez la main de Harry Potter avec votre baguette, et répétez : Magie qui coule en moi, scelle ce Serment."

M. Blanc répéta. Même à travers la distorsion du masque, on croyait pouvoir entendre son cœur se briser.

Derrière Voldemort, les obélisques chantèrent d'un langage que Harry ignorait ; ils répétèrent les mots trois fois puis se turent.

"M. Sinistre," dit Voldemort, "pensez aux raisons pour lesquelles vous auriez pu faire confiance à ce garçon s'il avait librement prêté serment. Pensez à la possibilité d'une confiance et \emph{sacrifiez-la}  à mesure que vous parlez…"

"Par la confiance que j'ai pour vous," dit , "soyez tenu."

Et ce fut au tour de Harry Potter de répéter les paroles de Lord Voldemort, ce qu'il fit :

"Je fais le serment…" dit Harry. Sa voix vacilla, mais il continua : "De ne jamais… par aucun de mes actes… détruire le monde… je ne prendrai aucun risque… lorsqu'il s'agira de ne pas le détruire… si ma main est forcée… je choisirai la voie… de moindre destruction… à moins qu'il ne me semble que ce Serment lui-même… ne mène à la fin du monde… et que l'amie… à qui je me serais honnêtement confié… ne soit d'accord. Je fais le choix…" Harry put les sentir, lorsque le rituel s'accomplit : les brillantes cordes de pouvoir qui s'enroulaient autour de sa baguette et de celle de M. Sinistre, qui s'enroulaient autour de sa main là où M. Blanc l'avait touché, qui s'enroulait autour de \emph{lui}  d'une façon abstraite, déroutante. Il put se sentir \emph{invoquer}  son libre arbitre et il sut que ses prochaines paroles le \emph{sacrifieraient} , que c'était sa dernière chance de faire marche arrière.

"…qu'il en soit ainsi," dit la voix froide et précise de Lord Voldemort.

"…qu'il en soit ainsi," répéta Harry, et il sut à cet instant qu'il n'était plus libre de suivre ou de rejeter le contenu du Serment ; c'était simplement ainsi que son esprit et son corps bougeraient. Ce n'était pas un serment qu'il pourrait briser, même en sacrifiant sa vie. Comme l'eau coulait, comme les calculettes additionnaient, c'était simplement ce-que-Harry-Potter-ferait.

"Le Serment a-t-il pris, M. Blanc ?"

M. Blanc avait l'air de sangloter. "Oui, Maître… j'ai tant perdu, s'il vous plaît, j'ai été assez puni."

"Retournez à vos places," dit Voldemort. "Bien. Regardez tous le petit Potter, préparez-vous à tirer dès qu'il essaiera de fuir, de lever sa baguette ou de parler…" Le Seigneur des Ténèbres lévita, sa silhouette entourée de noir domina le cimetière. Un pistolet se trouvait à nouveau dans sa main gauche, et une baguette était toujours dans sa main droite. "C'est mieux. \emph{Maintenant} , nous tuons le Survivant."

M. Blanc vacilla. M. Sinistre rit à nouveau, et d'autres firent de même.

"Je n'avais pas l'intention d'être drôle," dit froidement Voldemort. "Nous avons affaire à une \emph{prophétie} , idiots. Nous tranchons les fils du destin un par un ; prudemment, très prudemment, sans savoir où nous rencontrerons la première résistance. Voici comment les choses procéderont. Harry Potter sera d'abord étourdi, puis il sera démembré et les blessures seront cautérisées. M. Amical et M. Honneur l'inspecteront à la recherche de toute magie inhabituelle. L'un d'entre vous lui tirera dessus de nombreuses fois avec mon arme Moldue, puis vous serez aussi nombreux que possible à lui lancer le sortilège de la Mort. Alors seulement M. Sinistre brisera son crâne et son cerveau au moyen d'une stèle ordinaire. Je vérifierai son corps, qui sera ensuite brûlé dans un Feudeymon, et nous exorciseront la zone au cas où il aurait laissé un fantôme. Je monterai moi-même la garde pendant six heures, car je ne fais pas entièrement confiances aux protections que j'ai mises en place contre les Retourneurs de Temps ; et quatre d'entre vous inspecteront les environs à la recherche de quoi que ce soit d'intéressant. Même ensuite, vous resterez vigilant quant à son retour, au cas où Dumbledore aurait joué quelque tour insoupçonné. Si vous pouvez imaginer une façon d'assurer que la menace que représente Harry Potter n'est plus, un méthode à laquelle je n'aurais pas pensé, parlez maintenant et je vous récompenserai généreusement… parlez maintenant, par Merlin !"

Un silence ahuri emplit le cimetière ; personne n'ouvrit la bouche.

"Inutiles, tous autant que vous êtes," dit Voldemort avec un mépris amer. "Maintenant, je vais poser une dernière question à Harry Potter, et il ne répondra qu'à moi, en Fourchelangue. Abattez-le à l'instant où il répond par autre chose que des sifflements, à l'instant où il tente de prononcer un mot en langue humaine." Puis Voldemort siffla : "\emph{Pouvoir que j'ignore, il fut dit que tu aurais. J'ai maintenant appris les Arts Moldus grâce à toi, et je les étudie déjà. Tu prétends que ton pouvoir ssur les Détraqueurs doit être compris seul. Si tu posssèdes tout autre pouvoir que je pourrais apprendre, dis-le moi maintenant. Ssinon, je compte tourmenter certains de ceux que tu aimes. Je t'ai déjà promis certaines vies, mais pas toutes. Les sserviteurs Ssang-de-Bourbe de ta petite armée. Tes précieux parents. Ils souffriront tous pendant ce qui leur semblera être des éternités ; et je les enverrai alors, brissés, dans la prisson des mange-vie, pour qu'ils ss'en ssouviennent, jussqu'à ce qu'ils ss'ussent et qu'ils meurent. Pour chaque pouvoir inconnu que tu m'apprends à maîtriser ou pour tout autre secret que j'aurais aimé connaître, tu pourras nommer l'un de ceux qui sseront protégé et honoré ssous mon règne. C'est ausssi une promessse que j'ai l'intention de tenir.} " Le sourire de Voldemort rappelait maintenant les crocs déployés d'un serpent, et le sens qu'ils avaient chez eux : la promesse de consumeraient celui qui leur faisait face. "\emph{Ssi tu te ssoucies de ceux que tu aimes, ne perds pas de temps à chercher à t'échapper. Tu as ssoixante ssecondes pour commencer à m'intéressser, puis ta mort commencera.} "
\par\noindent\rule{\textwidth}{0.4pt}
[NdT : Lors de la parution originale, l'auteur a soumis la communauté de ses lecteurs à un examen final. Ils avaient 60 heures pour inventer ensemble une solution permettant à Harry d'échapper à une mort immédiate, nu, armé seulement de sa baguette, face à 36 Mangemort et à un Voldemort ressuscité. Les solutions devaient être écrites sous forme de reviews.

En cas de réussite, l'histoire continuait. En cas d'échec, l'auteur aurait publié une fin courte et triste.

Les lecteurs ont réussi l'examen.

Pour le plaisir, je publierai le prochain chapitre dans 60 heures : le lundi 13 avril à 22h. Si vous souhaitez essayer le challenge, voici les règles :]


\begin{center}\emph{1. Harry doit réussir seul. La cavalerie n'arrive pas.} \\\emph{Tous ceux qui pourraient vouloir l'aider pensent qu'il regarde un match de Quidditch. } \end{center}



\begin{center}\emph{2. Harry peut seulement faire usage de capacités déjà constatées dans l'histoire ;} \\\emph{Il ne peut pas devenir capable de Légilimancie muette sans baguette en 60 secondes. } \end{center}



\begin{center}\emph{3. Voldemort est méchant et rien ne le persuadera d'être gentil ;} \\\emph{La fonction d'utilité du Seigneur des Ténèbres ne sera pas modifiée par la parole. } \end{center}



\begin{center}\emph{4. Si Harry lève sa baguette ou dit quoi que ce soit dans une langue autre que le Fourchelangue,} \\\emph{les Mangemort lui tireront immédiatement dessus. } \end{center}



\begin{center}\emph{5. Si la ligne temporelle la plus simple est celle où Harry meurt -} \\\emph{c'est à dire si Harry ne peut pas atteindre son Retourneur de Temps sans aide venue d'un Retourneur de Temps -} \\\emph{alors le Retourneur de Temps ne peut être utilisé. } \end{center}



\begin{center}\emph{6. Il est impossible de mentir en Fourchelangue. } \end{center}



\begin{center}\emph{Ces contraintes satisfaites,} \\\emph{Harry peut devenir un rationaliste accompli,} \\\emph{maintenant ou jamais,} \\\emph{indépendemment de ses défauts passés. } \end{center}



\begin{center}\emph{Bien sûr, la 'solution rationnelle',} \\\emph{si on utilise le mot 'rationnel' correctement,} \\\emph{est une façon inutilement pompeuse de dire 'la meilleure solution',} \\\emph{ou 'la solution que j'aime', ou 'la solution qui, selon moi, devrait être utilisée',} \\\emph{et il vaut en général mieux dire ça.} \\\emph{(Le mot 'rationnel' n'a besoin d'être utilisé que pour parler de méthodes de pensée,} \\\emph{indépendemment de solutions particulières). } \end{center}



\begin{center}\emph{Et par le principe de Vinge,} \\\emph{si vous savez déjà ce qu'un esprit intelligent ferait,} \\\emph{vous devez être au moins aussi intelligent.} \\\emph{Demander à quelqu'un ce qu'un joueur optimal ferait} \\\emph{ne devrait pas produire de meilleure réponse que : "Quel est le coup idéal à jouer ?" } \end{center}



\begin{center}\emph{Donc en pratique, ce que je veux dire,} \\\emph{quand je dis que Harry peut revenir un rationaliste accompli,} \\\emph{c'est que Harry peut résoudre ce problème} \\\emph{comme VOUS l'auriez résolu.} \\\emph{Si vous pouvez me dire exactement comment faire quelque chose,} \\\emph{Harry a le droit d'y penser. } \end{center}



\begin{center}\emph{Mais dire par exemple :} \\\emph{"Harry devrait persuader Voldemort de le laisser sortir de la boîte"} \\\emph{ne constitue pas une solution} \\\emph{tant que vous ne savez pas comment il s'y prendrait. } \end{center}


[ NdT : D'autres commentaires de l'auteur sont disponibles à la fin de la page située ici : hpmor point com /chapter/113. ]

